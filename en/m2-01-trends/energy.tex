\section{Power and energy trends}

\begin{frame}[t]{Power and energy: Primary concerns}
\begin{enumerate}
  \item \textgood{Maximum power}
    \begin{itemize}
      \item Processor draws more power than supplied $\Rightarrow$ \textmark{Voltage drop}
            $\Rightarrow$ \textbad{Malfunction}.
      \item Voltage indexing to allow slow down to regulate voltage.
    \end{itemize}

  \mode<presentation>{\vfill\pause}
  \item \textgood{Thermal Design Power (TDP)}
    \begin{itemize}
      \item Power consumption under the maximum theoretical load.
      \item It determines cooling requirement.
      \item It can be exceeded in short periods.
      \item What if TDP is exceeded for longer term?
        \begin{itemize}
          \item Lower clock rate.
          \item Power down the chip.
        \end{itemize}
    \end{itemize}   

  \mode<presentation>{\vfill\pause}
  \item \textgood{Energy}
    \begin{itemize}
      \item Power is energy per unit time.
      \item 1 Watt = 1 Joule / 1 second.
    \end{itemize}
\end{enumerate}
\end{frame}

\begin{frame}[t]{Power and Energy: Example}
\begin{itemize}
  \item Two different systems (\textmark{A} and \textmark{B}).
    \begin{itemize}
      \item \textmark{A} consumes 20\% more power than \textmark{B}.
      \item \textmark{A} runs a task in 70\% of \textmark{B} time.
      \item Which has a lower cost?
    \end{itemize}

  \mode<presentation>{\pause\vfill}
  \item The adequate metric for comparison is \textgood{Energy}.
    \begin{itemize}
      \item $E(B) = P(B) \cdot t(B)$
      \item $E(A) = 1.2 \cdot P(B) \cdot 0.7 \cdot t(B) = 0.84 \cdot E(B)$
      \item System \textmark{A} uses 84\% of \textmark{B} energy.
    \end{itemize}
\end{itemize}
\end{frame}

\begin{frame}[t]{Energy and power in microprocessors}
\begin{itemize}
  \item In CMOS technology, \textmark{energy consumption} is derived from
        \textmark{transistors switching}.

  \mode<presentation>{\pause\vfill}
  \item \textgood{Dynamic energy}:
    \begin{itemize}
      \item Amount of energy needed to switch.
        \begin{itemize}
          \item $0 \rightarrow 1$ or $1 \rightarrow 0$.
          \item $E_d \approx \frac{1}{2} \cdot X_c \cdot V^2$
        \end{itemize}
    \end{itemize}

  \mode<presentation>{\pause\vfill}
  \item \textgood{Dynamic power}:
    \begin{itemize}
      \item Depends on switching frequency.
        \begin{itemize}
          \item $P_d \approx \frac{1}{2} \cdot X_c \cdot V^2 \cdot f$
        \end{itemize}
    \end{itemize}
\end{itemize}
\begin{block}{Note}
\begin{small}
  $X_c$: Capacitive load\qquad
  $V$: Voltage \qquad
  $f$: Frequency
\end{small}
\end{block}
\end{frame}

\begin{frame}[t]{Example}
\begin{itemize}
  \item If a 15\% voltage reduction implies a 15\% frequency reduction:
    \begin{itemize}
      \item Which is the effect on dynamic energy and dynamic power?
    \end{itemize}
\end{itemize}

\mode<presentation>{\pause\vfill}
\begin{block}{Solution}
\begin{displaymath}
\frac{E_{new}}{E_{old}} =
\frac
{(V \cdot 0.85)^2}
{V^2} =
0.85^2 =
0.72
\end{displaymath}
\pause
\begin{displaymath}
\frac{P_{new}}{P_{old}} =
\frac
{(V \cdot 0.85)^2 \cdot (f \cdot 0.85)}
{V^2 \cdot f} =
0.85^3 =
0.61
\end{displaymath}
\end{block}
\end{frame}

\begin{frame}[t]{Evolution}
\begin{itemize}
  \item Trends over time:
    \begin{itemize}
      \item Increasing the \textgood{number of transistors} switching.
      \item Increasing \textgood{switch frequency}.
      \item \textmark{More relevant} than decrease in capacitive load and voltage.
      \item \textemph{Net effect}: Growth in power consumption and energy.
    \end{itemize}

  \mode<presentation>{\vfill\pause}
  \item \textgood{Power Consumption}
    \begin{itemize}
      \item Intel 80386 (1986): 2W.
      \item Intel Core i7-6700K (2015): 91W.
      \item Intel Core i7-12700K (2021): 125W.
      \item \textbad{Approaching limits of cooling}.
    \end{itemize}

  \mode<presentation>{\vfill\pause}
  \item \textemph{Impact on clock frequency}
    \begin{itemize}
      \item Intel 80386 (1986): 33 MHz.
      \item Intel Core i7-6700K (2015): 4.00 -- 4.20 GHz.
      \item Intel Core i7-12700K (2021): 3.60 -- 4.90 GHz.
    \end{itemize}
\end{itemize}
\end{frame}

\begin{frame}[t]{Energy efficiency}
\begin{itemize}
 \item \textgood{Techniques}:
   \mode<presentation>{\vfill}
   \begin{enumerate}
     \item Turn off clock for inactive modules.
       \begin{itemize}
         \item No floating point executing $\Rightarrow$ Switch off FP unit.
         \item Disabled unused cores.
       \end{itemize}

     \mode<presentation>{\vfill}
     \item Dynamic Voltage-Frequency Scaling (DVFS).
       \begin{itemize}
         \item Multiple operating frequency $\Rightarrow$ Power savings.
       \end{itemize}

     \mode<presentation>{\vfill}
     \item Low power modes.
       \begin{itemize}
         \item Requires reactivation mechanism.
       \end{itemize}
     \mode<presentation>{\vfill}

     \item Automatic overclocking.
       \begin{itemize}
         \item Enabled when it is safe.
         \item Example: Core i7 3.3 GHz may run busts at 3.6 GHz.
       \end{itemize}
    \end{enumerate}
\end{itemize}
\end{frame}

\begin{frame}[t]{Static power effect}
\begin{itemize}
  \item Static power increasingly important.
    \begin{itemize}
      \item Current flows even if transistor is off.

\pause
\begin{block}{Static power}
\begin{displaymath}
P_s \approx current_{static} \times Voltage
\end{displaymath}
\end{block}

      \pause
      \item Static power is proportional to number of devices.
    \end{itemize}

  \mode<presentation>{\vfill\pause}
  \item \textmark{Impact}:
    \begin{itemize}
      \item Increasing transistor count $\Rightarrow$ Increases power.
        \begin{itemize}
          \item \textbad{Even if idle!}
        \end{itemize}

      \pause
      \item \textmark{Power gating}: Turn off power supply to inactive modules.

      \pause
      \item High performance chips may leak up to 50\%.
        \begin{itemize}
          \item Large cache memories using SRAM (static RAM).
        \end{itemize}
    \end{itemize}
\end{itemize}
\end{frame}
