\section{Threads in OpenMP}

\begin{frame}[t,fragile]{\emph{Fork-join} parallelism}
\begin{itemize}
  \item Sequential application with parallel sections:
    \begin{itemize}
      \item \textgood{Master thread}: Started with main program.
      \item A parallel section starts a thread set.
      \item Parallelism can be \textmark{nested}. 
    \end{itemize}

  \mode<presentation>{\vfill\pause}
  \item A parallel region is a block marked with the \cppkey{parallel} directive.
\begin{lstlisting}
#pragma omp parallel
\end{lstlisting}
\end{itemize}
\end{frame}

\begin{frame}[t,fragile]{Selecting the number of threads}
\begin{columns}[T]

\column{.5\textwidth}
\begin{block}{Library function}
\begin{lstlisting}[basicstyle=\tiny]
omp_set_num_threads(4);
#pragma omp parallel
{
  // Parallel region
}
\end{lstlisting}
\end{block}

\mode<presentation>{\vfill\pause}
\column{.5\textwidth}
\begin{block}{OpenMP Directive}
\begin{lstlisting}[basicstyle=\tiny]
#pragma omp parallel num_threads(4)
{
  // Parallel region
}
\end{lstlisting}
\end{block}
\end{columns}

\mode<presentation>{\vfill\pause}
\begin{columns}

\column{.25\textwidth}

\column{.5\textwidth}

\begin{block}{Environment variable}
\begin{lstlisting}[style=terminal,basicstyle=\tiny]
OMP_NUM_THREADS=8 program
\end{lstlisting}
\end{block}

\column{.25\textwidth}
\end{columns}
\end{frame}

\begin{frame}[t,shrink=20]{Exercise 2: Computing $\pi$}
\begin{itemize}

  \item Computing $\pi$.
\[
\int_{0}^{1} \frac{1}{1+x^2} dx =
\arctan{1} - \arctan{0} =
\frac{\pi}{4}
\]

  \mode<presentation>{\vfill\pause}
  \item Approximation:
\[
\pi \approx 4 \cdot \sum_{i=0}^{N} f(x_{i} + \frac{\Delta x}{2}) \Delta x =
4 \cdot \Delta x \sum_{i=0}^{N} \frac{1}{1 + (x_i + \frac{\Delta x}{2} )^2}
\]

  \mode<presentation>{\vfill\pause}
  \item Adding the area of N rectangles:
    \begin{itemize}
      \item \textmark{Base}: $\Delta x = \frac{1}{N}$.
      \item \textmark{Points}: $x_i = \Delta x \cdot i$
      \item \textmark{Height}: $f(x_i + \frac{\Delta x}{2})$ = $f(\Delta x \cdot i + \frac{\Delta x}{2})$
            $f(\Delta x (i + \frac{1}{2})$.
  \end{itemize}

\[
\pi \approx 
4 \cdot \Delta x \sum_{i=0}^{N} \frac{1}{1 + (\Delta x \cdot (i + \frac{1}{2} ))^2}
\]

\end{itemize}
\end{frame}

\begin{frame}[t]{Exercise 2: Sequential version}
\mode<presentation>{\vspace{-1em}}
\begin{block}{Computing $\pi$}
\lstinputlisting{lab/04-omp/pi/piseq.cpp}
\end{block}
\end{frame}

\begin{frame}[t,fragile]{Measuring time in C++11}
\begin{itemize}

\item \textmark{include} files:
\begin{lstlisting}
#include <chrono>
\end{lstlisting}

\item Clock type:
\begin{lstlisting}
using clk = chrono::high_resolution_clock;
\end{lstlisting}

\item Get a time point:
\begin{lstlisting}
auto t1 = clk::now();
\end{lstlisting}

\item Time difference (time unit can be specified).
\begin{lstlisting}
auto diff = duration_cast<microseconds>(t2-t1);
\end{lstlisting}

\item Get difference value.
\begin{lstlisting}
cout << diff.count();
\end{lstlisting}

\end{itemize}
\end{frame}

\begin{frame}[t,fragile]{Time measurement example}
\mode<presentation>{\vspace{-1em}}
\begin{block}{Example}
\begin{lstlisting}
#include <chrono>

void f() {
  using namespace std::chrono;

  using clk = chrono::high_resolution_clock;

  auto t1 = clk::now();

  g();

  auto t2 = clk::now();
  auto diff = duration_cast<microseconds>(t2-t1);

  std::cout << "Time= " << diff.count << "microseconds\n";
}
\end{lstlisting}
\end{block}
\end{frame}

\begin{frame}[t,fragile]{Time measurement in OpenMP}
\begin{itemize}

\item Time point:
\begin{lstlisting}
double t1 = omp_get_wtime();
\end{lstlisting}

\item Time difference:
\begin{lstlisting}
double t1 = omp_get_wtime();
double t2 = omp_get_wtime();
double diff = t2-t1;
\end{lstlisting}

\item Time difference between two successive ticks:
\begin{lstlisting}
double tick = omp_get_wtick();
\end{lstlisting}

\end{itemize}
\end{frame}

\begin{frame}[t]{Measuring time in $\pi$ computation}
\begin{block}{piseqtime.cpp}
\lstinputlisting[lastline=12]{lab/04-omp/pi/piseqtime.cpp}
\end{block}
\end{frame}

\begin{frame}[t]{Measuring time in $\pi$ computation}
\begin{block}{piseqtime.cpp}
\lstinputlisting[firstline=14]{lab/04-omp/pi/piseqtime.cpp}
\end{block}
\end{frame}


\begin{frame}[t]{Exercise 2}
\begin{itemize}
\item Create a parallel version from the $\pi$ sequential version
      using a \cppkey{parallel} clause.

\item \textgood{Observations}:
  \begin{itemize}
    \item Include time measurements.
    \item Print the number of threads in use.
    \item Take special care with shared variables.
    \item \textmark{Idea}: 
           Use an array and accumulate partial sum for each thread in the parallel region.
  \end{itemize}
\end{itemize}
\end{frame}
