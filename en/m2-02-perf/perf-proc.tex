\subsection{Processor performance}

\begin{frame}[t]{Execution time}
\begin{itemize}
  \item A processor executes each instruction during several clock cycles.
\end{itemize}
\begin{block}{Time consumed by CPU}
\begin{displaymath}
time_{CPU} = 
\frac{cycles_{CPU}}{\text{clock frequency}}
\end{displaymath}
\end{block}
\end{frame}

\begin{frame}[t]{CPI: Cycles per instruction}
  \begin{itemize}
    \item Average speed may be expressed as cycles per instruction (CPI) using:
      \begin{itemize}
         \item Total number of consumed cycles, and
         \item number of executed instructions or instruction count (IC).
      \end{itemize}
  \end{itemize}
\begin{block}{CPI}
\begin{displaymath}
CPI =
\frac{cycles_{CPU}}{IC}
\end{displaymath}
\end{block}
\end{frame}

\begin{frame}[t]{Factors in execution time}
\begin{block}{CPI and CPU time}
\begin{displaymath}
CPI=\frac{cycles_{CPU}}{IC}
\end{displaymath}
\pause
\begin{displaymath}
time_{CPU}=
\frac{cycles_{CPU}}{f} \pause =
\frac{CPI \times IC}{f} \pause =
CPI \times IC \times T
\end{displaymath}
\end{block}
\mode<article>{
\begin{itemize}
  \item Symbols used:
    \begin{itemize}
      \item \textbf{f}: Clock frequency.
      \item \textbf{T}: Clock period.
    \end{itemize}
\end{itemize}
}
\mode<presentation>{\pause\vfill}
\begin{itemize}
  \item If any of the 3 factors is reduced by 10\%
        the total execution time is reduced by 10\%.
    \begin{itemize}
      \item \alert{But the 3 factors are related}.
    \end{itemize}
\end{itemize}
\end{frame}

\begin{frame}[t]{Instructions classes}
\begin{itemize}
  \item Different instruction classes have different IC and CPI.
\end{itemize}
\begin{columns}
\begin{column}{.7\textwidth}
\begin{block}{Global CPI}
\mode<presentation>{\begin{small}}
\begin{displaymath}
cycles_{CPU}=\sum_{i=1}^n IC_i \times CPI
\end{displaymath}
\pause
\begin{displaymath}
time_{CPU} =
\big( \sum_{i=1}^n IC_i \times CPI_i \big) \times T
\end{displaymath}
\pause
\begin{displaymath}
CPI_{global} = 
\frac
{\sum_{i=1}^n IC_i \times CPI_i}
{IC}
\pause =
\sum_{i=1}^n \frac{IC_i}{IC} \times CPI_i
\end{displaymath}
\mode<presentation>{\end{small}}
\end{block}
\end{column}
\begin{column}{.3\textwidth}
\pause
\alert{Impact of instructions relative frequency in program execution}.
\end{column}
\end{columns}
\end{frame}

\begin{frame}[t]{Example}
\begin{itemize}
  \item In a program's execution we have observed that:
    \begin{itemize}
      \item \textmark{Floating point operation}: 25\% (4.0 CPI on average).
      \item \textmark{Operation FPSQR} (\emph{square root}): 2\% (20.0 CPI).
        \begin{itemize}
          \item \alert{Included in floating point}.
        \end{itemize}
      \item \textmark{Rest of instructions}: 1.33 CPI.
    \end{itemize}

  \mode<presentation>{\pause\vfill}
  \item Choose among design alternatives:
    \begin{enumerate}[a]
      \item Decrease FPSQR CPI to 2.
      \item Decrease all floating point instructions CPI to 2.5.
    \end{enumerate}
\end{itemize}
\end{frame}

\begin{frame}[t]{Solution}
\begin{math}
CPI = 0.25 \times 4 + 1.33 \times 0.75 = \textbf{1.9975}
\end{math}
\mode<presentation>{\pause\vfill}
\begin{math}
0.25 \times CPI_{FP} = 0.23 \times CPI_{otherFP} + 0.02 \times CPI_{FPSQR}
\end{math}
\pause
\begin{math}
0.25 \times 4 = 0.23 \times CPI_{otherFP} + 0.02 \times 20
\end{math}
\pause
\framebox{
\begin{math}
CPI_{otherFP} =
\frac{0.24 \times 4 - 0.02 \times 20}{0.23} =
2.6087
\end{math}
}
\mode<presentation>{\pause\vfill}
\begin{math}
CPI_{nuevoFPSQR} = 0.23 \times 2.6087 + 0.02 \times \alert{2} + 0.75 \times 1.33 = \textbf{1.6375}
\end{math}
\pause
\begin{math}
CPI_{newFP} = 0.25 \times \alert{2.5} + 0.75 \times 1.33 = \textbf{1.6225}
\end{math}
\mode<article>{
\begin{itemize}
  \item \alert{Observation}: Although reducing one instruction (FPSQR)
        from 20 cycles down to 2 cycles may result appealing, a similar
        improvement can be achieved with a more modest reduction (from 4 to 2 cycles)
        for a family of instructions.
\end{itemize}
}
\end{frame}
