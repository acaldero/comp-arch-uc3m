\subsection{The free store}

\begin{frame}[t]{Free store memory}
\begin{itemize}
  \item The \textgood{free store} holds the memory that can be
        acquired and released.

  \vfill
  \item \textmark{IMPORTANT}:
        C++ is not a language with automated resource management. 
    \begin{itemize}
      \item If you acquire a resource you must free it.
      \item Acquired memory must be released.
    \end{itemize}
\end{itemize}
\vfill
\begin{quote}
C++ is my favorite garbage collected language
because it generates so little garbage.
\\
\textbf{Bjarne Stroustrup}
\end{quote}
\end{frame}

\begin{frame}[t,fragile]{Memory allocation operator}
\begin{itemize}
  \item Operator \cppkey{new} allows allocating memory from the free store.
\begin{lstlisting}
int * p = new int; // Allocates memory for an int
char * q = new char[10]; // Allocates memory for 10 char
\end{lstlisting}

  \vfill
  \item \emph{Effect}:
    \begin{itemize}
      \item Operator \cppkey{new} returns a pointer to start of allocated memory.
      \item An expression \cppkey{new} \cppid{T} returns a value of type \cppid{T*}.
      \item An expression \cppkey{new} \cppid{T[sz]} returns a value of type \cppid{T*}.
    \end{itemize}
\end{itemize}
\end{frame}


\begin{frame}[fragile]{Access problems}
\begin{itemize}
  \item A pointer variable is not automatically initialized to a value.
    \begin{itemize}
      \item Dereferencing an uninitialized pointer leads to undefined behavior.
    \end{itemize}
\begin{lstlisting}
int * p;
*p = 42; // Undefined behavior
p[0] = 42; // Undefined behavior
\end{lstlisting}

  \mode<presentation>{\vfill\pause}
  \item A pointer variable initialized to a sequence can only be accessed within
        the sequence limits.
\begin{lstlisting}
int * v = new int[10];
v[0] = 42; // OK
x = v[-1]; // Undefined
x = v[15]; // Undefined
v[10] = 0; // Undefined
\end{lstlisting}
\end{itemize}
\end{frame}

\begin{frame}[t,fragile]{Null pointer}
\begin{itemize}
  \item A pointer can be initialized to the \emph{null pointer value} to signal
        that points to no object.
    \begin{itemize}
      \item \cppkey{nullptr} literal.
    \end{itemize}
\begin{lstlisting}
int * p = nullptr;
char * q = nullptr;
if (p != nullptr) { /* ... */ }
if (q == nullptr) { /* ... */ }
\end{lstlisting}
\end{itemize}
\end{frame}

\begin{frame}[t,fragile]{Memory allocation and initialization}
\begin{itemize}
  \item Operator \cppkey{new} does not initialize the allocated object.
\begin{lstlisting}
int * p = new int;
x = *p; // x has an unknown value
\end{lstlisting}
    \begin{itemize}
      \item Initial value can be specified in curly braces.
\begin{lstlisting}
p = new int{42}; // *p == 42
p = new int{}; // *p == 0
\end{lstlisting}
    \end{itemize}

  \vfill\pause
  \item If a sequence is allocated with \cppkey{new}, no object is initialized.
\begin{lstlisting}
int * v = new int[10];
\end{lstlisting}
    \begin{itemize}
      \item Initial values can be specified in curly braces.
    \end{itemize}
\begin{lstlisting}
v = new int[4]{1,2,3,4}; // v[0] = 1, v[1] = 2, v[2] = 3, v[3] = 4
v = new int[4]{1, 2};    // v[0] = 1, v[1] = 2, v[2] = 0, v[3] = 0
v = new int[4]{};        // v[0] = 0, v[1] = 0, v[2] = 0, v[3] = 0
v = new int[4]{1,2,3,4,5}; // Error. Too many initializers.
\end{lstlisting}
\end{itemize}
\end{frame}

\begin{frame}[fragile]{Memory deallocation operator}
\begin{itemize}
  \item Operator \cppkey{delete} allows to release memory and mark it
        as non-allocated.
  \item It can only be applied to:
    \begin{itemize}
      \item Memory returned by a call to operator \cppkey{new} and currently allocated.
      \item A null pointer value.
    \end{itemize}
\begin{lstlisting}
int * p = new int{10};
*p = 20;
delete p; // Release p
\end{lstlisting}
  \item It is an error to invoke twice \cppkey{delete} on the same pointer value.
\begin{lstlisting}
int * p = new int{10};
delete p; // Release p
delete p; // Undefined behavior
\end{lstlisting}
\end{itemize}
\end{frame}

\begin{frame}[fragile]{Array deallocation}
\begin{itemize}
  \item There is a different version of \cppkey{delete} to deallocate \emph{arrays}.
\begin{lstlisting}
int * p = new int{10};
int * v = new int[10];
delete p;   // Release p
delete []v; // Release v
\end{lstlisting}

  \vfill\pause
  \item \textmark{Important}: 
        The right deallocation version must be used.
    \begin{itemize}
      \item If memory is allocated with \cppkey{new T} it must be released with \cppkey{delete}.
      \item If memory is allocated with \cppkey{new T[n]} it must be released with \cppkey{delete[]}.
    \end{itemize}
\begin{lstlisting}
int * p = new int{10};
int * v = new int[10];
delete []p; // Undefined behavior
delete v;   // Undefined behavior
\end{lstlisting}
\end{itemize}
\end{frame}

\begin{frame}[fragile]{Reasons to deallocate}
\begin{itemize}
  \item If memory is allocated and not released it remains allocated.
\begin{lstlisting}
void f() {
  int * v = new int[1024*1024];
  // ...
}
\end{lstlisting}
    \begin{itemize}
      \item Each time \cppid{f()} is invoked 8 MB are lost (assuming \cppkey{sizeof(int)}\cppid{==8}).
    \end{itemize}

  \mode<presentation>{\vfill}
  \item Problems with \textmark{memory leaks}:
    \begin{itemize}
      \item Each memory allocation might require more time.
      \item If program runs for a long time, memory could be exhausted.
    \end{itemize}

  \mode<presentation>{\vfill}
  \item If memory is exhausted \cppid{bad\_alloc} exception is thrown.
\end{itemize}
\end{frame}
