\subsection{Vectors}

\begin{frame}{Collection of values}
\begin{itemize}
  \item \cppid{vector} 
        allows storing and processing a set of values from the same type.
      \begin{itemize}
        \item C++ also has \emph{arrays} but they are too limited and simple.
      \end{itemize}
  \item A \cppid{vector}:
    \begin{itemize}
      \item Has a sequence of elements.
      \item Elements can be accessed by index.
      \item Includes size information
    \end{itemize}
\begin{tikzpicture}
\tikzset{
    bloque/.style={rectangle,draw=black, top color=white, bottom color=blue!50,
                   very thick, inner sep=0.5em, minimum size=0.6cm, text centered, font=\tiny},
    flecha/.style={->, >=latex', shorten >=1pt, thick},
    etiqueta/.style={text centered, font=\tiny} 
}  
\node[bloque] (bsize) {4};
\node[bloque,right=0cm of bsize] (bptr) { };
\node[bloque,below right=0.5cm and 0.75cm of bptr] (v0) {1};
\node[bloque,right=0cm of v0] (v1) {2};
\node[bloque,right=0cm of v1] (v2) {4};
\node[bloque,right=0cm of v2] (v3) {8};
\draw[flecha] (bptr) -- (v0);
\node[etiqueta, left=0.1cm of bsize] {v:};
\node[etiqueta, above=0cm of bsize] {size()};
\node[etiqueta, above=0cm of v0] {v[0]};
\node[etiqueta, above=0cm of v1] {v[1]};
\node[etiqueta, above=0cm of v2] {v[2]};
\node[etiqueta, above=0cm of v3] {v[3]};
\end{tikzpicture}
  \item Alternative to direct use of arrays.
\end{itemize}
\end{frame}

\begin{frame}{Basic use}
\vspace{-1.5em}
\begin{columns}[t]

\column{.5\textwidth}
\begin{block}{Using a vector}
\lstinputlisting{int/m6-01-cppintro/vector/vec1.cpp}
\end{block}

\column{.5\textwidth}
\begin{itemize}
  \item Header file:
      \cppid{<vector>}

  \mode<presentation>{\vspace{1em}}
  \item Must specify element type.
    \begin{itemize}
      \item All from the same type.
    \end{itemize}

  \mode<presentation>{\vspace{1em}}
  \item Constructor argument:
      \textmark{Initial size}.

  \mode<presentation>{\vspace{1em}}
  \item Indices beyond size (included) cannot be accessed.
    \begin{itemize}
      \item \textbad{Undefined behavior}.
    \end{itemize}
\end{itemize}
\end{columns}
\end{frame}

\begin{frame}
\begin{block}{Vectors and types}
\lstinputlisting{int/m6-01-cppintro/vector/vec2.cpp}
\end{block}
\end{frame}

\begin{frame}[t,fragile]{Vectors and initialization}
\begin{itemize}
  \item A vector with size initializes all its values to the default value for the element type.
    \begin{itemize}
      \item Numeric values: \cppid{0}
      \item String values: \cppid{``''}
    \end{itemize}

  \vfill
  \item If no initial value is provided, vector has a size of \cppid{0}.
\begin{lstlisting}
vector<double> v; // Vector with 0 elements
\end{lstlisting}

  \vfill
  \item A different initial value can be provided.
\begin{lstlisting}
vector<double> v(100, 0.5); // 100 elements initialized to 0.5
\end{lstlisting}
\end{itemize}
\end{frame}

\begin{frame}
\begin{block}{Initializing at definition}
\lstinputlisting{int/m6-01-cppintro/vector/vec3.cpp}
\end{block}
\end{frame}

\begin{frame}[fragile]{Growing vectors}
\begin{itemize}
  \item A \cppid{vector} may \emph{grow} when new elements are added.
    \begin{itemize}
      \item Operation \cppid{push\_back()}: Adds an element at the end of the vector.
    \end{itemize}
\end{itemize}
\pause
\begin{lstlisting}
vector<int> v;
\end{lstlisting}
\vspace{-.5em}
\begin{tikzpicture}
\tikzset{
    bloque/.style={rectangle,draw=black, top color=white, bottom color=blue!50,
                   very thick, inner sep=0.5em, minimum size=0.6cm, text centered, font=\tiny},
    flecha/.style={->, >=latex', shorten >=1pt, thick},
    etiqueta/.style={text centered, font=\tiny} 
}  
\node[bloque] (bsize) {0};
\node[bloque,right=0cm of bsize] (bptr) { };
\node[etiqueta, left=0.1cm of bsize] {v:};
\node[etiqueta, above=0cm of bsize] {size()};
\end{tikzpicture}

\pause
\begin{lstlisting}
v.push_back(1);
\end{lstlisting}
\vspace{-.5em}
\begin{tikzpicture}
\tikzset{
    bloque/.style={rectangle,draw=black, top color=white, bottom color=blue!50,
                   very thick, inner sep=0.5em, minimum size=0.6cm, text centered, font=\tiny},
    flecha/.style={->, >=latex', shorten >=1pt, thick},
    etiqueta/.style={text centered, font=\tiny} 
}  
\node[bloque] (bsize) {1};
\node[bloque,right=0cm of bsize] (bptr) { };
\node[bloque,right=0.75cm of bptr] (v0) {1};
\draw[flecha] (bptr) -- (v0);
\node[etiqueta, left=0.1cm of bsize] {v:};
\node[etiqueta, above=0cm of bsize] {size()};
\node[etiqueta, above=0cm of v0] {v[0]};
\end{tikzpicture}

\pause
\begin{lstlisting}
v.push_back(4);
\end{lstlisting}
\vspace{-.5em}
\begin{tikzpicture}
\tikzset{
    bloque/.style={rectangle,draw=black, top color=white, bottom color=blue!50,
                   very thick, inner sep=0.5em, minimum size=0.6cm, text centered, font=\tiny},
    flecha/.style={->, >=latex', shorten >=1pt, thick},
    etiqueta/.style={text centered, font=\tiny} 
}  
\node[bloque] (bsize) {2};
\node[bloque,right=0cm of bsize] (bptr) { };
\node[bloque,right=0.75cm of bptr] (v0) {1};
\node[bloque,right=0cm of v0] (v1) {4};
\draw[flecha] (bptr) -- (v0);
\node[etiqueta, left=0.1cm of bsize] {v:};
\node[etiqueta, above=0cm of bsize] {size()};
\node[etiqueta, above=0cm of v0] {v[0]};
\node[etiqueta, above=0cm of v1] {v[1]};
\end{tikzpicture}
\end{frame}

\begin{frame}[t,fragile]{Traversing a vector}
\begin{itemize}
  \item Size value can be obtained from a vector through \cppid{size} \emph{member function}.
\begin{lstlisting}
cout << v.size();
\end{lstlisting}

  \vfill
  \item \cppid{size()} allows defining a loop to traverse elements in a vector.
\begin{lstlisting}
for (int i=0; i<v.size(); ++i) {
  cout << "v[" << i << "] = " << v[i] << "\n";
}
\end{lstlisting}

\end{itemize}
\end{frame}

\begin{frame}[t,fragile]{Range based traversal}
\begin{itemize}
  \item A vector can be traversed with a range based loop.
\begin{lstlisting}
vector<int> v1 { 1, 2, 3, 4 };
for (auto x : v1) {
  cout << x << "\n";
}

vector<string> v2 { "Carlos", "Daniel", "José", "Manuel" };
for (auto x : v2) {
  cout << x << "\n";
}
\end{lstlisting}
\end{itemize}
\end{frame}

\begin{frame}[t]{Example: Statistics}
\begin{itemize}
  \item \alert{Goal}: Read from standard input a sequence of students marks
        and dump to standard output minimum, maximum and average mark.
    \begin{itemize}
      \item Finish reading when reaching to end-of-file.
      \item Finish reading if a value cannot be correctly read (e.g. letters instead of numbers).
      \item Number of values is unknown (and not asked).
    \end{itemize}
\end{itemize}
\end{frame}

\mode<presentation> {

\begin{frame}
\begin{columns}[T]
\column{0.5\textwidth}
\begin{block}{marks.cpp}
\lstinputlisting[lastline=16]{int/m6-01-cppintro/vector/marks.cpp}
\ldots
\end{block}

\column{0.5\textwidth}
\begin{block}{marks.cpp}
\ldots
\lstinputlisting[firstline=17]{int/m6-01-cppintro/vector/marks.cpp}
\end{block}
\end{columns}
\end{frame}

}

\begin{frame}[t]{Example: Unique words}
\begin{itemize}
  \item \alert{Goal}: Dump the sorted list of unique words from a text.
    \begin{itemize}
      \item Text is read from standard input until end-of-file.
      \item The list of words is printed to standard output.
    \end{itemize}
\end{itemize}
\end{frame}


\begin{frame}
\begin{columns}[T]

\column{0.5\textwidth}
\begin{block}{unique.cpp}
\lstinputlisting[lastline=15]{int/m6-01-cppintro/vector/unique.cpp}
\ldots
\end{block}

\column{0.5\textwidth}
\begin{block}{unique.cpp}
\ldots
\lstinputlisting[firstline=16]{int/m6-01-cppintro/vector/unique.cpp}
\end{block}
\end{columns}
\end{frame}

