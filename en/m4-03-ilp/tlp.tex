\section{Thread level parallelism}

\begin{frame}[t]{Why TLP?}
\begin{itemize}
  \item Some applications exhibit more \textgood{natural parallelism} than
        the achieved with \textmark{ILP}.
    \begin{itemize}
      \item Servers, scientific applications, \ldots
    \end{itemize}

  \mode<presentation>{\vfill\pause}
  \item \textgood{Two models} emerge:
    \begin{itemize}
      \item \textgood{Thread level parallelism (TLP)}:
        \begin{itemize}

          \item \textmark{Thread}: 
                Process with its own instructions and data.
          \item May be either part of a program or an independent program.
          \item Each thread has an associated \textmark{state} 
                (instructions, data, PC, registers, \ldots).
        \end{itemize}
      \item \textgood{Data level parallelism (DLP)}:
        \begin{itemize}
          \item Identical operation on different data items.
        \end{itemize}
    \end{itemize}
\end{itemize}
\end{frame}

\begin{frame}[t]{TLP}
\begin{itemize}
  \item \textgood{ILP} 
        exploits implicit parallelism within a basic block or a loop.

  \mode<presentation>{\vfill}
  \item \textgood{TLP} 
        uses multiple threads of execution inherently parallel.

  \mode<presentation>{\vfill\pause}
  \item \textgood{TLP Goal}:
    \begin{itemize}
      \item Use multiple instruction flows to improve:
        \begin{itemize}
          \item \textmark{Throughput} 
                in computers using many programs.
          \item \textmark{Execution time} 
                of multi-threaded programs.
        \end{itemize}
    \end{itemize}
\end{itemize}
\end{frame}

\begin{frame}[t]{Multi-threaded execution}
\begin{itemize}
  \item Multiple threads share processor functional units overlapping its use.
    \begin{itemize}
      \item Need to replicate state n-times.
        \begin{itemize}
          \item Register file, PC, page table (when threads do note belong to the same program).
          \item Shared memory through virtual memory mechanisms.
          \item Hardware for fast thread context switch.
        \end{itemize}

      \mode<presentation>{\vfill\pause}
      \item \textgood{Kinds}:
        \begin{itemize}
          \item \textmark{Fine grain}: Thread switch in every instruction.
          \item \textmark{Coarse grain}: Thread switch in stalls (e.g. Cache miss).
          \item \textmark{Simultaneous}: Fine grain with multiple-issue and dynamic scheduling.
        \end{itemize}
    \end{itemize}
\end{itemize}
\end{frame}

\begin{frame}[t]{Fine-grain multithreading}
\begin{itemize}
  \item Switches between threads in each instruction.
    \begin{itemize}
      \item Interleaves thread execution.
      \item Usually does \emph{round-robin}.
      \item Threads in a \emph{stall} are excluded from \emph{round-robin}.
      \item Processor must be able to switch every clock cycle.
    \end{itemize}

  \mode<presentation>{\vfill\pause}
  \item \textgood{Advantage}:
    \begin{itemize}
      \item Can hide short and long stalls.
    \end{itemize}

  \mode<presentation>{\vfill}
  \item \textbad{Drawback}:
    \begin{itemize}
      \item Delays individual thread execution due to sharing.
    \end{itemize}

  \mode<presentation>{\vfill}
  \item \textmark{Example}: Sun Niagara.
\end{itemize}
\end{frame}

\begin{frame}[t]{Coarse grain multithreading}
\begin{itemize}
  \item Switch thread only on long stalls.
    \begin{itemize}
      \item \textmark{Example}: L2 cache miss.
    \end{itemize}

  \mode<presentation>{\vfill\pause}
  \item \textgood{Advantages}:
    \begin{itemize}
      \item No need for a highly fast thread switch.
      \item Does not delay individual threads.
    \end{itemize}

  \mode<presentation>{\vfill}
  \item \textbad{Drawbacks}:
    \begin{itemize}
      \item Must flush or freeze the pipeline.
      \item Needs to fill pipeline with instructions from the new thread (latency).
    \end{itemize}

  \mode<presentation>{\vfill}
  \item Appropriate when filling the pipeline takes much shorter than a stall.
    \begin{itemize}
      \item \textmark{Example}: IBM AS/400.
    \end{itemize}
\end{itemize}
\end{frame}

\begin{frame}[t]{SMT: Simultaneous multithreading}
\begin{itemize}
  \item \textgood{Idea}: 
        Dynamically scheduled processors already have many mechanisms
        to support multithreading.
    \begin{itemize}
      \item Large sets of virtual registers.
        \begin{itemize}
          \item Registers for multiple threads.
        \end{itemize}
      \item Register renaming.
        \begin{itemize}
          \item Avoid conflicts in access to registers from threads.
        \end{itemize}
      \item Out-of-order finalization.
    \end{itemize}

  \mode<presentation>{\vfill\pause}
  \item \textgood{Modifications}:
    \begin{itemize}
      \item Per thread renaming table.
      \item Separate PC registers.
      \item Separate ROB.
    \end{itemize}

  \mode<presentation>{\vfill}
  \item \textmark{Examples}: Intel Core i7, IBM Power 7

\end{itemize}
\end{frame}

\begin{frame}[t]{TLP: Summary}
\makebox[\textwidth][c]{
\input{int/m4-03-ilp/tlp-alt.tkz}
}
\end{frame}

