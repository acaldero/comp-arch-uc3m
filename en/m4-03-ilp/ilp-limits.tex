\section{ILP limits}

\begin{frame}[t]{ILP limits}
\begin{itemize}
  \item To study maximum \textmark{ILP} we model an \textgood{ideal processor}.

  \mode<presentation>{\vfill}
  \item \textgood{Ideal processor}:
    \begin{itemize}
      \item \textmark{Infinite register renaming}: 
            All WAR and WAW hazards can be avoided.
      \item \textmark{Perfect branch prediction}: 
            All conditional branch predictions are a hit.
      \item \textmark{Perfect jump prediction}:
            All jumps (include returns) are correctly predicted.
      \item \textmark{Perfect memory address alias analysis}:
            A load can be safely moved before a store if address is not identical
      \item \textmark{Perfect caches}: 
            All cache accesses require one clock cycle (always hit).
    \end{itemize}
\end{itemize}
\end{frame}

\begin{frame}[t]{Available ILP}
\begin{tikzpicture}
  \begin{axis}[
    xbar,
    width=.8\textwidth, height=.9\textheight,
    xlabel={Instructions per cycle},
    symbolic y coords={gcc,expresso,li,fppp,doduc,tomcatv},
    ytick=data,
    nodes near coords, nodes near coords align={horizontal},
    ]
    \addplot coordinates {
  (54.8,gcc)
  (62.6,expresso)
  (17.9,li)
  (75.2,fppp)
  (118.7,doduc)
  (150.1,tomcatv)
};
  \end{axis}
\end{tikzpicture}
\end{frame}

\begin{frame}[t]{However \ldots}
\begin{itemize}
  \item More ILP implies more control logic:
    \begin{itemize}
      \item Smaller caches.
      \item Longer clock cycle.
      \item Higher energy consumption.
    \end{itemize}

  \mode<presentation>{\vfill}
  \item \textbad{Practical limitation}:
    \begin{itemize}
      \item Issue from 3 to 6 instructions per cycle.
    \end{itemize}
\end{itemize}
\end{frame}
