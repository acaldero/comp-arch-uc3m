\section{Advanced branch prediction techniques}

\begin{frame}[t]{Branch prediction}
\begin{itemize}
  \item High impact of \textgood{branches} on programs performance.

  \mode<presentation>{\vfill}
  \item To \textmark{reduce} impact:
    \begin{itemize}
      \item Loop unrolling.
      \item Branch prediction:
        \begin{itemize}
          \item Compile time.
          \item Each branch handled isolated.
        \end{itemize}
      \item Advanced branch prediction:
        \begin{itemize}
          \item Correlated predictors.
          \item Tournament predictors.
        \end{itemize}
    \end{itemize}
\end{itemize}
\end{frame}

\begin{frame}[t]{Dynamic scheduling}
\begin{itemize}
  \item Hardware \textgood{reorders} instructions execution to \emph{reduce} stalls
        while keeping data flow and exceptions.

  \mode<presentation>{\vfill\pause}
  \item Able to handle unknown cases at compile time:
    \begin{itemize}
      \item Cache misses/hits.
    \end{itemize}

  \mode<presentation>{\vfill}
  \item Code less dependent on a concrete pipeline.
    \begin{itemize}
      \item Simplifies compiler.
    \end{itemize}

  \mode<presentation>{\vfill}
  \item Permits the \textmark{hardware speculation}.

\end{itemize}
\end{frame}

\begin{frame}[t]{Correlated prediction}

\begin{columns}

\begin{column}{.7\textwidth}
\begin{itemize}
  \item If first and second branch are taken, third is NOT-taken.
\end{itemize}
\end{column}

\begin{column}{.3\textwidth}
\begin{block}{example}
\lstinputlisting{int/m4-03-ilp/branch-pred-ex1.cpp}
\end{block}
\end{column}
\end{columns}

\mode<presentation>{\vfill\pause}
\begin{itemize}
  \item Maintains last branches \textmark{history} to select among several
        predictors.
  \item A $(m,n)$ predictor:
    \begin{itemize}
      \item Uses the result of $m$ last branches to select among
            $2^m$ predictors.
      \item Each predictor has $n$ bits.
    \end{itemize}
  \item Predictor $(1,2)$:
    \begin{itemize}
      \item Result of last branch to select among 2 predictors.
    \end{itemize}
\end{itemize}
\end{frame}

\begin{frame}[t]{Size of predictor}
\begin{itemize}
  \item A predictor $(m,n)$ has several entries for each branch address.

  \item Total size:
\[
S = 2^m \times n \times entries_{\text{address}}
\]

  \item Examples:
    \begin{itemize}
      \item $(0,2)$ with 4K entries $\rightarrow$ 8 Kb
      \item $(2,2)$ with 4K entries $\rightarrow$ 32 Kb
      \item $(2,2)$ with 1K entries $\rightarrow$ 8 Kb
   \end{itemize}
\end{itemize}
\end{frame}

\begin{frame}[t]{Miss rate}
\begin{itemize}
  \item Correlated predictor has less misses that simple predictor with same size.
  \mode<presentation>{\vfill}
  \item Correlated predictor has less misses than simple predictor with unlimited number of entries.
\end{itemize}

\end{frame}


\begin{frame}[t]{Tournament prediction}
\begin{itemize}
  \item \textgood{Combines two predictors}:
    \begin{itemize}
      \item \textmark{Global information} based predictor.
      \item \textmark{Local information} based predictor.
    \end{itemize}
  \item Uses a selector to choose between predictors.
    \begin{itemize}
      \item Change among two selectors uses a saturation counter (2 bits).
    \end{itemize}
  \item \textgood{Advantage}:
    \begin{itemize}
      \item Allows different behavior for integer and FP.
    \end{itemize}
  \item \textgood{SPEC}:
    \begin{itemize}
      \item \textmark{Integer benchmarks} $\rightarrow$ global predictor 40\%
      \item \textmark{FP benchmarks} $\rightarrow$ global predictor 15\%.
    \end{itemize}
  \item \textgood{Uses}: Alpha and AMD Opteron.
\end{itemize}
\end{frame}


\begin{frame}[t]{Intel Core i7}
\begin{itemize}
  \item Predictor with \textgood{two levels}:
    \begin{itemize}
      \item Smaller first level predictor.
      \item Larger second level predictor as backup.
    \end{itemize}

  \mode<presentation>{\vfill}
  \item Each predictor combines \textgood{3 predictors}:
    \begin{itemize}
      \item Simple 2-bits predictor.
      \item Global history predictor.
      \item Exit-loop predictor (iterations counter).
    \end{itemize}

  \mode<presentation>{\vfill}
  \item Besides:
    \begin{itemize}
      \item \textmark{Indirect jumps} predictor.
      \item \textmark{Return address} predictor.
    \end{itemize}
\end{itemize}
\end{frame}
