\section{Memory model}

\begin{frame}[t]{Memory consistency}
\makebox[\textwidth][c]{\input{en/m5-02-consist/modelo.tkz}}
\mode<presentation>{\vfill}
\begin{itemize}
  \item \textgood{Memory consistency model}:
    \begin{itemize}
      \item Set of rules defining how the
            \textmark{memory system} processes memory operations from 
             \textmark{multiple processors}.
      \item \textmark{Contract} between programmer and system.
      \item Determines which \textmark{optimizations are valid} on correct programs.
    \end{itemize}
\end{itemize}
\end{frame}

\begin{frame}[t]{Memory model}
\begin{itemize}
  \item Interface between program and its transformers.
    \begin{itemize}
      \item Defines which values can be returned by a read operation.
    \end{itemize}
  \item The language's memory model has implications for hardware.
\end{itemize}
\mode<presentation>{\vfill}
\makebox[\textwidth][c]{\input{en/m5-02-consist/leng-mem.tkz}}
\end{frame}

\begin{frame}[t]{Single processor memory model}

\begin{columns}

\column{.2\textwidth}

\makebox[\textwidth][c]{\input{en/m5-02-consist/single-proc.tkz}}

\column{.8\textwidth}

\begin{itemize}
  \item \textgood{Memory behavior model}:
    \begin{itemize}
      \item Memory operations happen in \textgood{program order}.
        \begin{itemize}
          \item A read returns the value from the last write in program order.
        \end{itemize}
    \end{itemize}
  \mode<presentation>{\vfill\pause}
  \item Semantics defined by \textgood{sequential program order}:
    \begin{itemize}
      \item \textmark{Simple} but \textmark{constrained} reasoning.
        \begin{itemize}
          \item Solve \textmark{data and control dependencies}.
        \end{itemize}
      \item \textmark{Independent} operations may be executed in \textmark{parallel}.
      \item Optimizations \textmark{preserve semantics}.
    \end{itemize}
\end{itemize}

\end{columns}

\end{frame}
