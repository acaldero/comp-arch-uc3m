\section{Control hazards}

\begin{frame}[t]{Control hazard}
\begin{itemize}
  \item A \textgood{control hazard} happens in a \textmark{PC}
        modification instruction.

    \mode<presentation>{\vfill}
    \begin{itemize}
      \item \textgood{Terminology}:
        \begin{itemize}
          \item \textmark{Taken branch}: If PC is modified.
          \item \textmark{Not-taken branch}: if PC is not modified.
        \end{itemize}

      \mode<presentation>{\vfill}
      \item \textbad{Problem}:
        \begin{itemize}
          \item Pipelining assumes that branch will \textbad{not} be taken.
        \end{itemize}

      \mode<presentation>{\vfill}
      \item If branch is taken $\Rightarrow$ \textemph{PC} updated at the end of \textmark{ID}.
        \begin{itemize}
          \item \textgood{Effective address computation}: Adder at \textmark{ID}.
          \item \textgood{Condition evaluation}: Comparator at \textmark{ID}.
        \end{itemize}

      \mode<presentation>{\vfill}
      \item What if, after ID, we find out that branch \textgood{needs} to be taken?
    \end{itemize}
\end{itemize}
\end{frame}

\begin{frame}[t]{Alternatives in control hazards}
\begin{itemize}
  \item \textgood{Compile time}: Fixed assumption for the full program execution.
    \begin{itemize}
      \item Software may try to minimize impact if hardware behavior is known.
      \item Compiler can do this job.
    \end{itemize}

  \mode<presentation>{\vfill\pause}
  \item \textgood{Run-time}: Variable behavior during program execution.
    \begin{itemize}
      \item Hardware tries to predict what software will do.
    \end{itemize}
\end{itemize}
\end{frame}

\begin{frame}[t]{Control hazards: static solutions}
\begin{enumerate}
  \item Pipeline freezing.
  \item Fixed prediction.
    \begin{itemize}
      \item Always not taken.
      \item Always taken.
    \end{itemize}
  \item Delayed branching.
\end{enumerate}

\mode<presentation>{\vfill}
\begin{itemize}
  \item In many cases, compilers need to know what will be done to reduce impact.
\end{itemize}
\end{frame}

\begin{frame}[t]{Pipeline freezing}
\begin{itemize}
  \item \textgood{Idea}: If current instruction is a branch
        $\rightarrow$ 
        stop or flush subsequent instructions from pipeline
        until target is known.
    \begin{itemize}
      \item Run-time penalty is known.
      \item Software (compiler) cannot do anything.
    \end{itemize}

    \mode<presentation>{\vfill}
    \item Branch target is known at \textmark{ID} stage.
      \begin{itemize}
        \item Repeat next instruction \textmark{FETCH}.
      \end{itemize}
\end{itemize}
\end{frame}

\begin{frame}[t]{Repeating FETCH}
\makebox[\textwidth][c]{
\input{en/m4-02-ilp/rep-fetch.tkz}
}

\mode<presentation>{\vfill}
\begin{itemize}
  \item Repeating \textmark{IF} is equivalent to a stall.
  \item A branch stall may lead to a \textmark{performance loss} from 10\% to 30\%.
\end{itemize}
\end{frame}

\begin{frame}[t]{Fixed prediction: not taken}
\begin{itemize}
  \item \textgood{Idea}: Assume branch will be \textmark{not taken}.
    \begin{itemize}
      \item Avoid updating processor state until branch not taken
            is confirmed.
      \item If branch is finally taken, subsequent instructions are retired
            from pipeline and instruction from target address is fetched.
        \begin{itemize}
          \item Transform instructions into \textmark{NOP}s.
        \end{itemize}
    \end{itemize}

  \mode<presentation>{\vfill\pause}
  \item \textgood{Compiler task}:
    \begin{itemize}
      \item Organize code setting most frequent option as not-taken and inverting condition if needed.
    \end{itemize}
\end{itemize}
\end{frame}

\begin{frame}[t]{Fixed prediction: not taken}
\makebox[\textwidth][c]{
\input{en/m4-02-ilp/pred-not-taken.tkz}
}
\mode<presentation>{\vfill}
\begin{itemize}
  \item When hardware knows the branch will be taken  the new instruction is fetched.
\end{itemize}
\end{frame}

\begin{frame}[t]{Fixed prediction: taken}
\begin{itemize}
  \item \textgood{Idea}: Assume branch will be \textmark{taken}.
    \begin{itemize}
      \item As soon as branch instruction is decoded and target address
            is computed, target instruction starts to be fetched.
      \item In a 5-stages pipeline does not provide improvements.
        \begin{itemize}
          \item Target address is not known before branch outcome is known.
          \item Useful in processors with complex and slow conditions.
        \end{itemize}
    \end{itemize}

  \mode<presentation>{\vfill}
  \item \textgood{Compiler task}:
    \begin{itemize}
      \item Organize code setting the most frequent option as taken
            and inverting condition if needed.
    \end{itemize}
\end{itemize}
\end{frame}

\begin{frame}[t,fragile]{Delayed branching}
\begin{itemize}
  \item \textgood{Idea}: Branch happens after executing
        \textmark{n} subsequent instructions to branch instruction.
    \begin{itemize}
      \item \textmark{In a 5-stages pipeline} $\rightarrow$ 1 delay slot.
    \end{itemize}
\end{itemize}

\mode<presentation>{\vfill\pause}
\begin{columns}
\begin{column}{.4\textwidth}
\begin{block}{Example}
\lstinputlisting[language={generalasm}]{int/m4-02-ilp/delay-slot.asm}
\end{block}
\end{column}

\begin{column}{.6\textwidth}
\begin{itemize}
  \item Instructions \textmark{I1}, \textmark{I2}, \ldots, \textmark{IN} are
        executed independently of the branch condition outcome.
  \item Instruction \textmark{IN+1} is only executed if branch is not taken.
\end{itemize}
\end{column}
\end{columns}
\end{frame}

\begin{frame}[t]{Delayed branching}
\makebox[\textwidth][c]{
\input{en/m4-02-ilp/delayed-branch.tkz}
}
\mode<presentation>{\vfill}
\begin{itemize}
  \item Case with \textmark{delayed branch} and one \textmark{delay slot}.
  \item One instruction is always executed before taking the branch.
  \item Programmer are responsible for putting useful code in the slot.
\end{itemize}
\end{frame}

\begin{frame}[t]{Delayed branching}
\begin{itemize}
  \item Compiler effectiveness for the case of 1 slot.
    \begin{itemize}
      \item \textmark{Fills} around 60\% of slots.
      \item Around 80\% of executed instructions in slots \textmark{useful} for computations.
      \item Around 50\% of slots usefully \textmark{filled}.
    \end{itemize}

  \mode<presentation>{\vfill}
  \item used in first pipelines (short and simple):
    \begin{itemize}
      \item \textmark{Reason}: Hardware prediction was costly.
      \item Complicates implementation when combined with dynamic branch prediction.
      \item Not included in RISC-V.
    \end{itemize}

\end{itemize}
\end{frame}

\begin{frame}[t]{Performance and fixed prediction}
\begin{itemize}
  \item Number of stalls depends on:
    \begin{itemize}
      \item Branch frequency.
      \item Branch penalty.
    \end{itemize}
\end{itemize}

\mode<presentation>{\vfill\pause}
\begin{itemize}
  \item Penalty cycles per branch:
\end{itemize}
\begin{displaymath}
cycles_{branch} = frequency_{branch} \times penalty_{branch}
\end{displaymath}

\mode<presentation>{\vfill\pause}
\begin{itemize}
  \item Speedup:
\end{itemize}
\begin{displaymath}
S =
\frac{depth_{pipeline}}{1 + frequency_{branch} \times penalty_{branch}}
\end{displaymath}
\end{frame}

\begin{frame}[t]{Practical case}
\begin{itemize}
  \item \textgood{MIPS R4000} (deeper pipeline).
    \begin{itemize}
      \item 3 stages before knowing branch target.
      \item 1 additional stage to evaluate condition.
      \item Assuming no data stalls in comparisons.
      \item \textgood{Branch frequency}:
        \begin{itemize}
          \item \textmark{Unconditional branching}: 4\%.
          \item \textmark{Conditional branching, not-taken}: 6\%
          \item \textmark{Conditional branching, taken}: 10\%
        \end{itemize}
    \end{itemize}
\end{itemize}

\mode<presentation>{\vfill}
{\scriptsize
\begin{tabular}[c]{|p{.3\textwidth}|*{3}{p{.17\textwidth}|}}
\hline
branch &  \multicolumn{3}{c|}{Penalty}
\\
\cline{2-4}
scheme & unconditional & not-taken & taken
\\
\hline\hline

Flush pipeline &
\multicolumn{1}{r|}{2} & 
\multicolumn{1}{r|}{3} & 
\multicolumn{1}{r|}{3}
\\
\hline

Predict taken &
\multicolumn{1}{r|}{2} & 
\multicolumn{1}{r|}{3} &
\multicolumn{1}{r|}{2}
\\
\hline

Predict not-taken &
\multicolumn{1}{r|}{2} &
\multicolumn{1}{r|}{0} &
\multicolumn{1}{r|}{3}
\\

\hline
\end{tabular}
}
\end{frame}

\begin{frame}[t]{Solution}

{\scriptsize
\begin{tabular}[c]{|p{.2\textwidth}|*{3}{p{.18\textwidth}|}p{.06\textwidth}|}
\hline
branch &  \multicolumn{3}{c|}{Branch} & \multicolumn{1}{c|}{Total}
\\
\cline{2-4}
scheme & unconditional & not-taken & taken &
\\
\hline\hline

Frequency &
\multicolumn{1}{c|}{4\%} &
\multicolumn{1}{c|}{6\%} &
\multicolumn{1}{c|}{10\%} &
\multicolumn{1}{c|}{20\%} 
\\
\hline

Flush pipeline &
$0.04 \times 2 = 0.08$ &
$0.06 \times 3 = 0.18$ &
$0.10 \times 3 = 0.30$ &
$0.56$
\\
\hline

Predict taken &
$0.04 \times 2 = 0.08$ &
$0.06 \times 3 = 0.18$ &
$0.10 \times 2 = 0.20$ &
$0.46$
\\
\hline

Predict not-taken &
$0.04 \times 2 = 0.08$ &
$0.06 \times 0 = 0.00$ &
$0.10 \times 3 = 0.30$ &
$0.38$
\\
\hline

\hline
\end{tabular}
}
\begin{center}
\textmark{Contribution over ideal CPI}
\end{center}

\mode<presentation>{\vfill\pause}
\begin{columns}

\begin{column}{.49\textwidth}
{\footnotesize
\emph{Speedup} of predicting \textmark{taken} over flushing pipeline
}
\begin{displaymath}
S =
\frac{1 + 0.56}{1 + 0.46} =
1.068
\end{displaymath}
\end{column}

\begin{column}{.49\textwidth}
{\footnotesize
\emph{Speedup} of predicting \textmark{not taken} over flushing pipeline
}
\begin{displaymath}
S =
\frac{1 + 0.56}{1 + 0.38} =
1.130
\end{displaymath}
\end{column}

\end{columns}

\end{frame}
