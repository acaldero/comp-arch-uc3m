\section{Sequential consistency}

\begin{frame}[t]
\makebox[\textwidth][c]{\input{en/m5-03-consist/multi-proc.tkz}}
\mode<presentation>{\vfill\pause}
\begin{footnotesize}
A multiprocessor system is \textgood{sequentially consistent} if the result of any
execution is the same that would be obtained if operations from all processors
were executed in some sequential order, and operations from each individual
processor appear in that sequence in the order established by the program.
\begin{flushright}
Leslie Lamport, 1979
\end{flushright}
\end{footnotesize}
\end{frame}

\begin{frame}[t]{Sequential Consistency: Constraints}
\begin{itemize}
  \item \textgood{Program order}.
    \begin{itemize}
      \item Memory operations from a program must be made visible to
            \textmark{all processes} in
            \textgood{program order}.
    \end{itemize}

  \mode<presentation>{\vfill\pause}
  \item \textgood{Atomicity}.
    \begin{itemize}
      \item Total execution order between processes must be
            \textgood{consistent} requiring that all operations are
            \textmark{atomic}.
        \begin{itemize}
          \item All the operations that a processor does after it has seen the new value of a write
                are not visible to other processes until they have seen the value from that write.
        \end{itemize}
    \end{itemize}
\end{itemize}
\end{frame}

\begin{frame}[t]{Atomicity}
\makebox[\textwidth][c]{\input{en/m5-03-consist/atomicity.tkz}}
\mode<presentation>{\vfill\pause}
\begin{itemize}
  \item \textgood{Non atomic writes}:
    \begin{itemize}
      \item Write on \cppid{b} could bypass to
            \cppkey{while} loop and read from \cppid{a} would bypass the write.
        \begin{itemize}
          \item \cppid{X=0}.
        \end{itemize}
    \end{itemize}
  \item \textgood{Atomic writes}:
    \begin{itemize}
      \item \textmark{Sequential consistency is preserved}.
    \end{itemize}
\end{itemize}
\end{frame}

\begin{frame}[t]
\begin{itemize}
  \item \textgood{Sequential consistency} \textmark{constraints} all memory operations:
    \begin{itemize}
      \item Write $\rightarrow$ Read.
      \item Write $\rightarrow$ Write.
      \item Read $\rightarrow$ Read, Read $\rightarrow$ Write.
    \end{itemize}

  \mode<presentation>{\vfill\pause}
  \item \textmark{Simple model} to reason about parallel programs.

  \mode<presentation>{\vfill\pause}
  \item But, simple single processor reorderings may  \textbad{violate} 
        \textgood{sequential consistency} model:
    \begin{itemize}
      \item \textmark{Hardware reordering} to improve performance.
        \begin{itemize}
          \item Write buffers, overlapped writes, \ldots
        \end{itemize}
      \item \textmark{Compiler optimizations} apply transformations with memory operations reordering.
        \begin{itemize}
          \item Scalar replacement, register allocation, instruction scheduling, \ldots
        \end{itemize}
      \item \textmark{Transformations} by programmers, or
            \textmark{refactoring tools} also modify program semantics.
    \end{itemize}
\end{itemize}
\end{frame}

\begin{frame}[t]{Sequential consistency violation}
\begin{columns}[T]

\column{.5\textwidth}
\makebox[\textwidth][c]{\input{en/m5-03-consist/sc-viol.tkz}}

\pause 

\column{.5\textwidth}
\begin{itemize}
  \item If caches use a \textmark{write buffer}:
    \begin{itemize}
      \item Writes are \textbad{delayed} in buffer.
      \item Reads \textbad{obtain the old value}.
      \item \textmark{Dekker Algorithm} is no longer \textbad{valid}.
        \begin{itemize}
          \item \textmark{Dekker algorithm} is the first known solution
                to the mutual exclusion problem.
        \end{itemize}
    \end{itemize}
\end{itemize}

\end{columns}
\end{frame}

\begin{frame}[t]{Program order}
\begin{columns}[T]

\column{.5\textwidth}
\makebox[\textwidth][c]{\input{en/m5-03-consist/prog-order.tkz}}

\column{.5\textwidth}
\makebox[\textwidth][c]{\input{en/m5-03-consist/prog-order-res.tkz}}

\end{columns}
\end{frame}

\begin{frame}[t]{Program order}
\begin{columns}[T]

\column{.5\textwidth}
\makebox[\textwidth][c]{\input{en/m5-03-consist/ej-prog-order.tkz}}

\column{.5\textwidth}
\makebox[\textwidth][c]{\input{en/m5-03-consist/ej-prog-order-res.tkz}}

\end{columns}
\end{frame}

\begin{frame}[t]{Conditions for sequential consistency}
\begin{itemize}
  \item \textgood{Sufficient conditions}:
    \begin{itemize}
      \item Each process \textmark{issues memory operations} in program order.
      \item After \textmark{issuing a write}, 
            the process that performed the issue \textmark{waits for completions} of write
            \textmark{before issuing} another operation.
      \item After \textmark{issuing a read}, 
            the process that performed the issue \textmark{waits for completion} of read and
            for \textmark{completion} of the write of the value being read.
        \begin{itemize}
          \item Wait for write propagation to all processes.
        \end{itemize}
    \end{itemize}

  \mode<presentation>{\vfill\pause}
  \item \textbad{Very demanding conditions.}
    \begin{itemize}
      \item There might be necessary conditions that are less demanding.
    \end{itemize}
\end{itemize}
\end{frame}
