\section{Conclusion}

\begin{frame}[t]{Summary}
\begin{itemize}
  \item Consistency memory model determines which optimizations are valid.
  \item \textgood{Sequential consistency} establishes as constraints
        \textmark{atomicity} and \textmark{program order}.
  \item More relaxed models than sequential consistency can be used.
    \begin{itemize}
      \item \textmark{Weak consistency}.
      \item \textmark{Release/acquire consistency}
    \end{itemize}
  \item Intel memory model has evolved over last decade.
    \begin{itemize}
      \item Formalized and publicly available.
      \item Establishes what operations are atomic, when bus is blocked, and how barriers are defined.
      \item Defines the memory model within processor and between different processors.
    \end{itemize}
\end{itemize}
\end{frame}


\begin{frame}[t]{References}
\begin{itemize}
  \item \bibhennessy
    \begin{itemize}
      \item Section 5.6 -- \emph{Models of memory consistency: An introduction}.
    \end{itemize}

  \mode<presentation>{\vfill}
  \item \textmark{Shared memory consistency models: A tutorial.}\\
        Adve, S. V., and Gharachorloo, K.\\
        IEEE Computer 29, 12 (December 1996), 66-76.

  \mode<presentation>{\vfill}
  \item \textmark{Intel 64 and IA-32 Architectures Software Developer Manuals.}\\
        Volume 3: Systems Programming Guide.\\
        8.2: Memory Ordering

\end{itemize}
\end{frame}
