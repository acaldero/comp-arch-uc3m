\section{Introduction to pipelining}

\begin{frame}[t]{Pipeline}
\begin{itemize}
  \item Implementation technique to execute multiple instructions
        overlapped in time.
    \begin{itemize}
      \mode<presentation>{\vfill}
      \item A \textmark{costly} operation is divided into
            simple sub-operations.
      \mode<presentation>{\vfill}
      \item Sub-operations are executed into stages.
    \end{itemize}
  \mode<presentation>{\pause\vfill}
  \item \textgood{Effects}:
    \begin{itemize}
      \mode<presentation>{\vfill}
      \item \textgood{Increases} \textmark{\emph{throughput}}.
      \mode<presentation>{\vfill}
      \item \textmark{Latency} is \textbad{not decreased}.
    \end{itemize}
\end{itemize}
\end{frame}

\begin{frame}[t,fragile]{Segmentación}
\begin{columns}[T]
\begin{column}{0.6\textwidth}
\input{es/m4-01-ilp/intro-pipeline.tkz}
\end{column}

\pause
\begin{column}[t]{.4\textwidth}
\begin{itemize}
  \item \textgood{Latency}:
    \begin{itemize}
      \item One instruction requires 5 stages.
      \item 5 cycles.
    \end{itemize}
  \item \textgood{Throughput}:
    \begin{itemize}
      \item One instruction finalized per cycle (once pipeline is full).
      \item 1 instruction per cycle.
    \end{itemize}
\end{itemize}
\end{column}
\end{columns}
\end{frame}

\begin{frame}[t]{Pipeline stages}
\begin{itemize}
  \item Simplified model:
    \begin{itemize}
      \item 5 stages.
      \item \textbad{More realistic models require more stages}
    \end{itemize}
  \mode<presentation>{\pause\vfill}
  \item \textgood{Stages}:
    \begin{itemize}
      \item \textmark{Instruction Fetch}.
      \item \textmark{Instruction Decode}.
      \item \textmark{Execution}.
      \item \textmark{Memory}.
      \item \textmark{Write-back}.
    \end{itemize}
\end{itemize}
\end{frame}

\begin{frame}[t]{Instruction Fetch}
\begin{columns}
\begin{column}{.5\textwidth}
\input{int/m4-01-ilp/fetch-design.tkz}
\end{column}
\begin{column}{.5\textwidth}
\begin{itemize}
  \item Send PC to memory.
  \item Read instruction.
  \item Update PC.
\end{itemize}
\end{column}
\end{columns}
\end{frame}

\begin{frame}[t]{Instruction Decode}
\begin{columns}
\begin{column}{.5\textwidth}
\input{int/m4-01-ilp/decode-design.tkz}
\end{column}
\begin{column}{.5\textwidth}
\begin{itemize}
  \item Decode instruction.
  \item Read registers.
  \item Sign extend offsets.
  \item Compute possible branch address.
\end{itemize}
\end{column}
\end{columns}
\end{frame}

\begin{frame}[t]{Execution}
\begin{columns}
\begin{column}{.3\textwidth}
\input{int/m4-01-ilp/exec-design.tkz}
\end{column}
\begin{column}{.7\textwidth}
\begin{itemize}
  \item ALU operation on registers.
  \item Alternatives:
    \begin{itemize}
      \item \textmark{Memory reference}:
            Compute effective address (add base register and offset).
      \item \textmark{Register/register instruction}:
            ALU operation on registers obtained in decode.
      \item \textmark{Register/Immediate data instruction}:
            ALU operation betwee register and immediate data.
      \item \textmark{Branch}:
            Condition evaluation.
    \end{itemize}
\end{itemize}
\end{column}
\end{columns}
\end{frame}

\begin{frame}[t]{Memory}
\begin{columns}
\begin{column}{.3\textwidth}
\begin{center}
\input{int/m4-01-ilp/mem-design.tkz}
\end{center}
\end{column}
\begin{column}{.7\textwidth}
\begin{itemize}
  \item Read from or write into memory.
    \begin{itemize}
      \item \textmark{Read}:
            Memory access using the effective memory address computed in EX.
      \item \textmark{Write}:
            Memory write using the effective memory address computed in EX
            and value of second register obtained in ID.
    \end{itemize}
\end{itemize}
\end{column}
\end{columns}
\end{frame}

\begin{frame}[t]{Write back}
\begin{columns}
\begin{column}{.3\textwidth}
\input{int/m4-01-ilp/wb-design.tkz}
\end{column}
\begin{column}{.5\textwidth}
\begin{itemize}
  \item Write result into register file.
    \begin{itemize}
      \item \textmark{Load operation}:
            Value read from memory.
      \item \textmark{ALU Instructions}:
            Value computed in EX stage.
    \end{itemize}
\end{itemize}
\end{column}
\end{columns}
\end{frame}

\begin{frame}[t]{General architecture}
\input{int/m4-01-ilp/general-design.tkz}
\end{frame}

\begin{frame}[t]{Pipeline effects}
\begin{itemize}
  \item An \textbad{n} \textmark{depth} pipeline 
  multiplies by \textbad{n} the needed \textmark{bandwidth} 
  compared to a non-pipelined version with the same  clock rate.
    \begin{itemize}
      \item Caching, caching, \ldots
    \end{itemize}

  \mode<presentation>{\vfill}
  \item Separation among \textmark{data} and \textmark{instructions} 
        \textgood{caches} suppresses some memory \textbad{conflicts} 

  \mode<presentation>{\vfill}
  \item Instructions in the pipeline \textbad{should not} try to use
        the same resource at the same time.
    \begin{itemize}
      \item Pipelining registers between successive stages.
    \end{itemize}
\end{itemize}
\end{frame}

\begin{frame}[t]{Stages communication}
\input{int/m4-01-ilp/general-reg-design.tkz}
\end{frame}

\begin{frame}[t]{Pipeline over time}
\input{es/m4-01-ilp/pipeline-over-time.tkz}
\begin{itemize}
  \item Lectura registros segunda mitad de ciclo.
  \item Escritura registros primera mitad de ciclo.
\end{itemize}
\end{frame}

\begin{frame}[t]{Example}
\begin{itemize}
  \item Non-pipelined processor.
    \begin{itemize}
      \item Frecuencia de reloj: 4 GHz.
      \item 40\% ALU operations $\rightarrow$ 4 cycles.
      \item 20\% branch operations $\rightarrow$ 4 cycles.
      \item 40\% memory operations $\rightarrow$ 5 cycles.
      \item Pipeline overhead $\rightarrow$ 0.1 ns.
    \end{itemize}
  \mode<presentation>{\pause}
  \item Which is the pipeline speedup?
  \mode<presentation>{\pause}
\end{itemize}
\[
t_{orig} = cycle_{clock} \times CPI = 0.25 ns \times (0.6 \times 4 + 0.4 \times 5) = 1.1 ns
\]
\[
t_{new} = 0.25 ns + 0.1 ns = 0.35 ns
\]
\[
S = \frac{1.1 ns}{0.35 ns} = 3.1
\]
\end{frame}
