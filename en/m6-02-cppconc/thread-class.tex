\section{Class \texttt{thread}}

\begin{frame}[t]{Class \texttt{thread}}
\begin{itemize}
  \item Abstraction of a \emph{thread} represented through class \cppid{thread}.
  \item One-to-one correspondence with operating system thread.
  \item All threads in an application run in the same address space.
  \item Each thread has its own stack.
  \item \textbad{Dangers}:
    \begin{itemize}
      \item Pass a pointer or a non-const reference to another thread.
      \item Pass a reference through capture in lambda expressions.
    \end{itemize}
  \item \cppid{thread} represents a link to a system thread.
    \begin{itemize}
      \item Cannot be copied.
      \item They can be moved.
    \end{itemize}
\end{itemize}
\end{frame}

\begin{frame}[t,fragile]{Thread construction}
\begin{itemize}
  \item A thread is constructed from a function and arguments that mast be passed to that function.
    \begin{itemize}
      \item Template with variable number of arguments.
      \item Type safe.
    \end{itemize}
\end{itemize}
\begin{block}{Example}
\begin{lstlisting}
void f();
void g(int, double);

std::thread t1{f}; // OK
std::thread t2{f, 1}; // Error: Too many arguments

std::thread t3{g, 1, 0.5}; // OK
std::thread t4{g}; // Error: Missing arguments
std::thread t5{g, 1, "Hello"}; // Error: Wrong types
\end{lstlisting}
\end{block}
\end{frame}

\begin{frame}[t,fragile]{Construction and references}
\begin{itemize}
  \item Constructor of \cppid{thread} is a template with variable number of arguments.
\begin{lstlisting}
template <class F, class ...Args> 
explicit thread(F&& f, Args&&... args);
\end{lstlisting}
  \item Arguments passing to a thread is by value.
  \item To force passing by reference:
    \begin{itemize}
      \item Use a helper function for \cppid{reference\_wrapper}.
      \item Use lambdas and reference captures.
    \end{itemize}
\end{itemize}
\begin{block}{}
\begin{lstlisting}[basicstyle=\tiny]
void f(record & r);

void g(record & s) {
  std::thread t1{f,s}; // Error if f takes a reference
  std::thread t2{f, std::ref(s)}; // OK. Reference to s
  std::thread t3{[&] { f(s); }}; // Reference to s
}
\end{lstlisting}
\end{block}
\end{frame}

\begin{frame}[t,fragile]{Two-phase construction}
\begin{itemize}
  \item Construction includes thread launching.
    \begin{itemize}
      \item There is no separate operation to \emph{start} execution.
    \end{itemize}
\end{itemize}
\begin{columns}

\begin{column}{.5\textwidth}
\begin{block}{Producer/Consumer}
\begin{lstlisting}
struct producer {
  producer(std::queue<request> & q);
  void operator()();
  // ...
};

struct consumer {
  consumer(std::queue<request> & q);
  void operator()();
  // ...
};
\end{lstlisting}
\end{block}
\end{column}

\begin{column}{.5\textwidth}
\begin{block}{Stages}
\begin{lstlisting}
void f() {
  // Stage 1: Construction
  std::queue<request> q;
  producer prod{q};
  consumer cons{q};

  // Stage 2: Launching
  std::thread tp{prod};
  std::thread tc{cons};

  //...
\end{lstlisting}
\end{block}
\end{column}

\end{columns}
\end{frame}

\begin{frame}[t,fragile]{Empty thread}
\begin{itemize}
  \item Default constructor creates a thread without associated execution task.
\begin{lstlisting}
thread() noexcept;
\end{lstlisting}
  \item Useful in combination with move constructor.
\begin{lstlisting}
thread(thread &&) noexcept;
\end{lstlisting}
  \item An execution takes can be moved from a thread to another thread.
    \begin{itemize}
      \item Original thread remains without associated execution task.
    \end{itemize}
\begin{lstlisting}
std::thread create_task();
std::thread t1 = create_task();
std::thread t2 = move(t1); // t1 is empty now
\end{lstlisting}
\end{itemize}
\end{frame}

\begin{frame}[t]{Thread identity}
\begin{itemize}
  \item Each thread has a unique identifier.
    \begin{itemize}
      \item Type: \cppid{thread::id}.
      \item If the \cppid{thread} 
            is not associated with a thread
            \cppid{get\_id()} returns \cppid{id\{\}}.
      \item Current thread identifier is obtained with
            \cppid{this\_thread::get\_id()}.
    \end{itemize}

  \mode<presentation>{\vfill\pause}
  \item \cppid{t.get\_id()} returns \cppid{id\{\}} if:
    \begin{itemize}
      \item An execution task has not been assigned to it.
      \item It has finished.
      \item Task has been moved to another \cppid{thread}.
      \item It has been detached (\cppid{detach()}).
    \end{itemize}
\end{itemize}
\end{frame}

\begin{frame}[fragile]{Operations on \texttt{thread::id}}
\begin{itemize}
  \item Is an implementation dependent type, but it must allow:
    \begin{itemize}
      \item Copying.
      \item Comparison operators (\cppid{==}, \cppid{<}, ...).
      \item Output to streams through operator \cppid{<<}.
      \item \emph{hash} transformation through specialization \cppid{hash<thread::id>}.
    \end{itemize}
\end{itemize}
\begin{block}{Example}
\begin{lstlisting}
void print_id(std::thread & t) {
  if (t.get_id() == id{}) 
    std::cout << "Invalid thread" << "\n";
  else {
    std::cout << "Current thread: " << std::this_thread::get_id() << "\n";
    std::cout << "Received thread: " << t.get_id() << "\n";
  }
}
\end{lstlisting}
\end{block}
\end{frame}

\begin{frame}[t,fragile]{Joining}
\begin{itemize}
  \item When a thread wants to wait for other thread termination, it may use operation \cppid{join()}.
    \begin{itemize}
      \item \cppid{t.join()} $\rightarrow$ waits until \cppid{t} has finished.
    \end{itemize}
\end{itemize}
\begin{block}{Example}
\begin{lstlisting}
void f() {
  std::vector<thread> vt;
  for (int i=0; i< 8; ++i) {
    vt.push_back(std::thread(f,i));
  }

  for (auto & t : vt) { // Waits until all threads have finished
    t.join();
  }
}
\end{lstlisting}
\end{block}
\end{frame}

\begin{frame}[t,fragile]{Periodic tasks}
\begin{block}{Initial idea}
\begin{lstlisting}
void update_bar() {
  while (!task_has_finished()) {
    using namespace std::literals;
    std::this_thread::sleep_for(1s); // 1 second
    update_progress();
  }
}

void f() {
  std::thread t{update_bar};
  t.join();
}
\end{lstlisting}
\end{block}
\begin{itemize}
  \item Problems?
\end{itemize}
\end{frame}

\begin{frame}[fragile]{What if I forget \texttt{join}?}
\begin{itemize}
  \item When scope where thread was defined is exited, its destructor is invoked.
  \item \textbad{Problem}: 
    \begin{itemize}
      \item Link with operating system thread might be lost. 
      \item System thread goes on running but cannot be accessed.
    \end{itemize}
  \mode<presentation>{\pause}
  \item If \cppid{join()} was not called, destructor invokes \cppid{terminate()}.
\end{itemize}

\mode<presentation>{\vfill\pause}
\begin{columns}

\begin{column}{.5\textwidth}
\begin{block}{Example}
\begin{lstlisting}[basicstyle=\tiny]
void update() {
  for (;;) {
    using namespace std::chrono;
    show_clock(steady_clock::now());
    std::this_thread::sleep_for(1s);
  }
}

void f() {
  std::thread t{update};
}
\end{lstlisting}
\end{block}
\end{column}

\begin{column}{.5\textwidth}
\begin{itemize}
  \item \cppid{terminate()} is called when exiting \cppid{f()}.
\end{itemize}
\end{column}
\end{columns}
\end{frame}

\begin{frame}[t,fragile]{Destruction}
\begin{itemize}
  \item \textgood{Goal}: 
        Avoid a thread to survive its \cppid{thread} object.
  \item \textgood{Solution}: If a \cppid{thread} is \emph{joinable} its destructor invokes \cppid{terminate()}.
    \begin{itemize}
      \item A \cppid{thread} is joinable if it is linked to a system thread.
    \end{itemize}
\end{itemize}
\begin{block}{Example}
\begin{lstlisting}
void check() {
  for (;;) {
    check_state();
    using namespace std::literals;
    std::this_thread::sleep_for(100ms);
  }
}

void f() {
  std::thread t{check};
} // Destruction without join() -> Invokes terminate()
\end{lstlisting}
\end{block}
\end{frame}

\begin{frame}[t,fragile]{Problems with destruction}
\begin{columns}

\begin{column}{.65\textwidth}
\begin{block}{Example}
\begin{lstlisting}
void f();
void g();

void example() {
  std::thread t1{f}; // Thread running task f
  std::thread t2; // Empty thread

  if (mode == mode1) {
    std::thread tg {g}; 
    // ...
    t2 = std::move(tg); // tg empty, t2 running g()
  }

  std::vector<int> v{10000}; // Might throw exceptions
  t1.join();
  t2.join();
}
\end{lstlisting}
\end{block}
\end{column}

\begin{column}{.35\textwidth}
\begin{itemize}
  \item What if constructor of \cppid{v} throws an exception?
  \mode<presentation>{\vfill}
  \item What if end of example is reached with \cppid{mode==mode1}?
\end{itemize}
\end{column}

\end{columns}
\end{frame}

\begin{frame}[fragile]{Automatic thread}
\begin{itemize}
  \item RAII pattern can be used.
    \begin{itemize}
      \item Resource Acquisition Is Initialization.
    \end{itemize}
\end{itemize}
\begin{block}{A joining thread}
\begin{lstlisting}
struct auto_thread : std::thread {
  using thread::thread; // All thread constructors
  ~auto_thread() { 
    if (joinable()) join(); 
  }
};
\end{lstlisting}
\end{block}
\begin{itemize}
  \item Constructor acquires resource.
  \item Destructor releases resource.
  \item Avoids resource leakage.
\end{itemize}
\end{frame}

\begin{frame}[t,fragile]{Simplifying with RAII}
\begin{itemize}
  \item Simpler code and higher safety.
\end{itemize}
\begin{block}{Example}
\begin{lstlisting}
void example() {
  auto_thread t1{f}; // Thread running task f
  auto_thread t2; // Empty thread

  if (modo == mode1) {
    auto_thread tg {g}; 
    // ...
    t2 = move(tg); // tg empty, t2 running g()
  }

  vector<int> v{10000}; // Might throw exceptions
}
\end{lstlisting}
\end{block}
\end{frame}

\begin{frame}[t,fragile]{Detached threads}
\begin{itemize}
  \item A thread can be specified to go on running after destructor, with \cppid{detach()}.
  \item Useful for task running as daemons.
\end{itemize}
\begin{block}{Example}
\begin{lstlisting}
void update() {
  for (;;) {
    show_clock(steady_clock::now());
    using namespace::std::literals;
    std::this_thread::sleep_for(1s);
  }
}

void f() {
  std::thread t{update};
  t.detach();
}
\end{lstlisting}
\end{block}
\end{frame}

\begin{frame}[t,fragile]{Problems with detached threads}
\begin{itemize}
  \item \textbad{Drawbacks}:
    \begin{itemize}
      \item Control of active threads is lost.
      \item Uncertain whether the result generated by a thread can be used.
      \item Uncertain whether a thread has released its resources.
      \item Access to objects that might have already been destroyed.
    \end{itemize}
  \vfill
  \item Recommendations:
    \begin{itemize}
      \item Avoid using detached threads.
      \item Move threads to other scope (via return value).
      \item Move threads to a container in a larger scope.
    \end{itemize}
\end{itemize}
\end{frame}

\begin{frame}[t,fragile]{A hard to catch bug}
\begin{itemize}
  \item \alert{Problem}: 
        Access to local variables from a detached thread after destruction.
\end{itemize}
\begin{block}{Example}
\begin{lstlisting}
void g() {
  double x = 0;
  std::thread t{
    [&x]{ 
      f1(); 
      x = f2();
    }
  }; // If g has finished -> Problem
  t.detach();
}
\end{lstlisting}
\end{block}
\end{frame}

\begin{frame}{Operations on current thread}
\begin{itemize}
  \item Operations on current thread as global functions in name subspace \cppid{this\_thread}.
    \begin{itemize}
      \item \cppid{get\_id()}: Gets identifier from current thread.
      \item \cppid{yield()}: Allows potential selection of another thread for execution.
      \item \cppid{sleep\_until(t)}: Wait until a certain point in time.
      \item \cppid{sleep\_for(d)}: Wait for a given duration of time.
    \end{itemize}
  \item Timed waits:
    \begin{itemize}
      \item If clock can be modified, \cppid{wait\_until()} is affected.
      \item If clock can be modified, \cppid{wait\_for()} is \textbf{not} affected.
    \end{itemize}
\end{itemize}
\end{frame}

\begin{frame}[t]{Thread local variables}
\begin{itemize}
  \item Alternative to \cppkey{static} as storage specifier: \cppkey{thread\_local}.
    \begin{itemize}
      \item A variable \cppkey{static} has a single shared copy for all threads.
      \item A variable \cppkey{thread\_local} has a per thread copy.
    \end{itemize}
  \item Lifetime: \emph{thread storage duration}.
    \begin{itemize}
      \item Starts before its first usage in thread.
      \item Destroyed upon thread exit.
    \end{itemize}
  \item Reasons to used thread local storage:
    \begin{itemize}
      \item Transform data from static storage to thread local storage.
      \item Keep data caches to be thread local (exclusive access).
        \begin{itemize}
          \item Important in machines with separate caches and coherence protocols.
        \end{itemize}
    \end{itemize}
\end{itemize}
\end{frame}

\begin{frame}[t,fragile]
\begin{block}{A function with computation caching}
\begin{lstlisting}
thread_local std::map<int, int> cache;

int compute_key(int x) {
  auto i = cache.find(x);
  if (i != cache.end()) return i->second;
  return cache[x] = slow_and_complex_algorithm(x);
}

std::vector<int> generate_list(const std::vector<int> & v) {
  std::vector<int> r;
  for (auto x : v) {
    r.push_back(compute_key(x));
  }
  return r;
}
\end{lstlisting}
\end{block}
\begin{itemize}
  \item Avoids need for synchronization.
  \item Some computations might be repeated in multiple threads.
\end{itemize}
\end{frame}
