\subsection{Access to shared data}

\begin{frame}[t,fragile]{Mutual exclusion}
\begin{itemize}
  \item \cppid{mutex} allows to control access with \textmark{mutual exclusion to a resource}.
    \begin{itemize}
      \item \cppid{lock()}: \textmark{Acquires} associated lock.
      \item \cppid{unlock()}: \textmark{Releases} associated lock.
    \end{itemize}
\end{itemize}

\mode<presentation>{\vfill}
\begin{columns}[T]
\begin{column}{.5\textwidth}
\begin{block}{Use of mutex}
\begin{lstlisting}
std::mutex m;
int x = 0;

void f() {
  m.lock();
  ++x;
  m.unlock();
}

\end{lstlisting}
\end{block}
\end{column}

\begin{column}{.5\textwidth}
\begin{block}{Launching threads}
\begin{lstlisting}
void g() {
  std::thread t1(f); 
  std::thread t2(f);
  t1.join(); 
  t2.join();
  std::cout << x << "\n";
}
\end{lstlisting}
\end{block}
\end{column}
\end{columns}
\end{frame}

\begin{frame}[t,fragile]{Problems with \texttt{lock()}/\texttt{unlock()}}
\begin{itemize}
  \item \textgood{Possible problems}:
    \begin{itemize}
      \item Forgetting to \textbad{release a lock}.
      \item \textbad{Exceptions}.
    \end{itemize}
\end{itemize}

\mode<presentation>{\vfill}
\begin{columns}[T]
\begin{column}{.5\textwidth}
\begin{block}{Forgetting unlocking}
\begin{lstlisting}
std::mutex m;
int x = 0;

void f() {
  m.lock();
  ++x;
  // Ouch! Forgot to unlock
}
\end{lstlisting}
\end{block}
\end{column}

\begin{column}{.5\textwidth}
\begin{block}{Exceptions}
\begin{lstlisting}
std::mutex m;
int x = 0;

void f() {
  m.lock();
  g(x); // Exceptions?
  m.unlock();
}
\end{lstlisting}
\end{block}
\end{column}
\end{columns}

\end{frame}

\begin{frame}[t,fragile]{Avoiding problems with \texttt{unique\_lock()}}
\begin{itemize}
  \item Solution: \cppid{unique\_lock}.
    \begin{itemize}
      \item \textgood{Pattern}: \textmark{RAII} (Resource Acquisition Is Initialization).
    \end{itemize}
\end{itemize}

\mode<presentation>{\vfill}
\begin{columns}[T]
\begin{column}{.5\textwidth}
\begin{block}{Automatic lock}
\begin{lstlisting}
std::mutex m;
int x = 0;

void f() {
  // Acquires lock
  std::unique_lock l{m}; 
  ++x;
} // Releases lock
\end{lstlisting}
\end{block}
\end{column}

\begin{column}{.5\textwidth}
\begin{block}{Launching threads}
\begin{lstlisting}
void g() {
  std::thread t1(f); 
  std::thread t2(f);
  t1.join(); 
  t2.join();

  std::cout << x << "\n";
}
\end{lstlisting}
\end{block}
\end{column}
\end{columns}
\end{frame}

\begin{frame}[t,fragile]{Acquiring multiple \texttt{mutex}}
\begin{itemize}
  \item \cppid{lock()} allows for acquiring simultaneously several \cppid{mutex}.
    \begin{itemize}
      \item Acquires all or none.
      \item If some is blocked it waits releasing all of them.
    \end{itemize}
\end{itemize}

\mode<presentation>{\vfill}
\begin{block}{Multiple acquisition}
\begin{lstlisting}
std::mutex m1, m2, m3;

void f() {
  std::lock(m1, m2, m3);

  // Access to shared data

  // Beware: Locks are not released
  m1.unlock();
  m2.unlock();
  m3.unlock()
} 
\end{lstlisting}
\end{block}
\end{frame}

\begin{frame}[t,fragile]{Acquiring multiple \texttt{mutex}}
\begin{itemize}
  \item Specially useful in cooperation with \cppid{unique\_lock}
\end{itemize}

\mode<presentation>{\vfill}
\begin{block}{Multiple automatic acquisition}
\begin{lstlisting}
void f() {
  std::unique_lock l1{m1, defer_lock}; 
  std::unique_lock l2{m2, defer_lock}; 
  std::unique_lock l3{m3, defer_lock};
  
  lock(l1, l2, l3);
  // Access to shared data

} // Automatic release
\end{lstlisting}
\end{block}
\end{frame}

