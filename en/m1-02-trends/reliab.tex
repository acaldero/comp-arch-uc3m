\section{Reliability and availability}

\subsection{Reliability}

\begin{frame}[t]{Reliability}
\begin{itemize}
  \item The lifetime of a system represented as a random variable
        $X$.

  \item System reliability defined as function $R(t)$
\begin{displaymath}
R(t) = P(X > t) : R(0) = 1 \quad y \quad R(\infty) = 0
\end{displaymath}

  \mode<presentation>{\vfill\pause}
  \item Obtained from study of components failures.

  \mode<presentation>{\vfill\pause}
  \item \textmark{Reliability}:
        Probability that a device works properly during a given period of 
        time under specific operating conditions.
        
\end{itemize}
\end{frame}

\begin{frame}[t]{Reliability distributions}
\begin{itemize}
  \item Examples of distributions used for reliability:
    \begin{itemize}
      \mode<presentation>{\vfill}
      \item \textgood{Exponential}:
        \begin{itemize}
          \item If error rate is constant (generally true for electronic components),
                reliability follows an exponential distribution.
        \end{itemize}
      \mode<presentation>{\vfill}
      \item \textgood{Weibull}:
        \begin{itemize}
          \item Models failure distribution when failure rate is proportional to a power of time.
        \end{itemize}
    \end{itemize}
\end{itemize}
\end{frame}


\begin{frame}[t]{Serial systems}
\begin{itemize}
  \item Let $R_i(t)$ reliability for component \textmark{i}.
  \item System fails when some component fails.
\end{itemize}
\begin{center}
\begin{tikzpicture}
\tikzset{
  etiqueta/.style={text centered, font=\tiny} ,
  bloque/.style={rectangle,draw=black,fill=blue!70,text=white,thick, text centered, font=\tiny, minimum width=1cm, minimum height=0.5cm},
  flecha/.style={->,thick,draw=black,font=\tiny},
  sinflecha/.style={-,thick,draw=black,font=\tiny},
  intersec/.style={fill,shape=circle,minimum size=3pt, inner sep=0pt},
  connpoint/.style={minimum width=0cm,minimum height=0cm,inner sep=0pt},
}  
\node[bloque] (c1) {};
\node[connpoint, left=0.5cm of c1] (c0) {};
\node[etiqueta, below=0.1cm of c1] {$R_1(t)$};
\node[bloque, right=0.5cm of c1] (c2) {};
\node[etiqueta, below=0.1cm of c2] {$R_2(t)$};
\node[bloque, right=0.5cm of c2] (c3) {};
\node[etiqueta, below=0.1cm of c3] {$R_3(t)$};
\node[bloque, right=0.5cm of c3] (c4) {};
\node[etiqueta, below=0.1cm of c4] {$R_4(t)$};
\node[connpoint, right=0.5cm of c4] (c5) {};
\path[sinflecha] (c0) -- (c1);
\path[sinflecha] (c1) -- (c2);
\path[sinflecha] (c2) -- (c3);
\path[sinflecha] (c3) -- (c4);
\path[sinflecha] (c4) -- (c5);
\end{tikzpicture}
\end{center}
\begin{itemize}
  \item If failures are independent then:
\end{itemize}
\begin{equation*}
R(t) = \prod_{i=1}^{N} R_i(t)
\end{equation*}
\mode<presentation>{\pause}
\begin{itemize}
  \item System reliability is lower:
\end{itemize}
\begin{equation*}
R(t) < R_i(t) \forall i
\end{equation*}
\end{frame}

\begin{frame}[t]{Paralel system}
\begin{itemize}
  \item System fails when all components fail.
\end{itemize}
\begin{equation*}
R(t) = 1 - \prod_{i=1}^N Q_i(t) : Q_i(t) = 1 - R_i(t)
\end{equation*}
\begin{center}
\begin{tikzpicture}
\tikzset{
  etiqueta/.style={text centered, font=\tiny} ,
  bloque/.style={rectangle,draw=black,fill=blue!70,text=white,thick, text centered, font=\tiny, minimum width=1cm, minimum height=0.5cm},
  flecha/.style={->,thick,draw=black,font=\tiny},
  sinflecha/.style={-,thick,draw=black,font=\tiny},
  intersec/.style={fill,shape=circle,minimum size=3pt, inner sep=0pt},
  connpoint/.style={minimum width=0cm,minimum height=0cm,inner sep=0pt},
}  
\node[bloque] (c1) {};
\node[etiqueta,below=0.05cm of c1] {$R_1(t)$};
\node[connpoint, left=0.5cm of c1] (lc1) {};
\node[connpoint, right=0.5cm of c1] (rc1) {};
\path[sinflecha] (lc1) -- (c1);
\path[sinflecha] (rc1) -- (c1);
\node[bloque, below=0.5cm of c1] (c2) {};
\node[etiqueta,below=0.05cm of c2] {$R_2(t)$};
\node[connpoint, left=0.5cm of c2] (lc2) {};
\node[connpoint, right=0.5cm of c2] (rc2) {};
\path[sinflecha] (lc2) -- (c2);
\path[sinflecha] (rc2) -- (c2);
\node[bloque, below=0.5cm of c2] (c3) {};
\node[etiqueta,below=0.05cm of c3] {$R_3(t)$};
\node[connpoint, left=0.5cm of c3] (lc3) {};
\node[connpoint, right=0.5cm of c3] (rc3) {};
\path[sinflecha] (lc3) -- (c3);
\path[sinflecha] (rc3) -- (c3);
\path[sinflecha] (lc1) -- (lc2);
\path[sinflecha] (lc3) -- (lc2);
\path[sinflecha] (rc1) -- (rc2);
\path[sinflecha] (rc3) -- (rc2);
\node[connpoint,left=1cm of lc2] (inicio) {};
\path[sinflecha] (inicio) -- (lc2);
\node[connpoint,right=1cm of rc2] (fin) {};
\path[sinflecha] (fin) -- (rc2);
\end{tikzpicture}
\end{center}
\end{frame}

\begin{frame}[t]{Example}
\begin{center}
\begin{tikzpicture}
\tikzset{
  etiqueta/.style={text centered, font=\tiny} ,
  bloque/.style={rectangle,draw=black,fill=blue!70,text=white,thick, text centered, font=\tiny, minimum width=1cm, minimum height=0.5cm},
  flecha/.style={->,thick,draw=black,font=\tiny},
  sinflecha/.style={-,thick,draw=black,font=\tiny},
  intersec/.style={fill,shape=circle,minimum size=3pt, inner sep=0pt},
  connpoint/.style={minimum width=0cm,minimum height=0cm,inner sep=0pt},
}  
\node[etiqueta] (e1) {Para $t=100$};
\node[etiqueta,right=0.25cm of e1] (e2) {$R_i(t)=0.9$};
\node[bloque,above=0.1cm of e2, minimum width=0.5cm, minimum height=0.2cm] {};
\end{tikzpicture}
\end{center}

\begin{columns}[T]

\column{.5\textwidth}

\begin{tikzpicture}
\tikzset{
  etiqueta/.style={text centered, font=\tiny} ,
  bloque/.style={rectangle,draw=black,fill=blue!70,text=white,thick, text centered, font=\tiny, minimum width=1cm, minimum height=0.5cm},
  flecha/.style={->,thick,draw=black,font=\tiny},
  sinflecha/.style={-,thick,draw=black,font=\tiny},
  intersec/.style={fill,shape=circle,minimum size=3pt, inner sep=0pt},
  connpoint/.style={minimum width=0cm,minimum height=0cm,inner sep=0pt},
}  
\node[bloque] (c1) {};
\node[connpoint, left=0.5cm of c1] (c0) {};
\node[etiqueta, below=0.1cm of c1] {$R_1(t)$};
\node[bloque, right=0.5cm of c1] (c2) {};
\node[etiqueta, below=0.1cm of c2] {$R_2(t)$};
\node[bloque, right=0.5cm of c2] (c3) {};
\node[etiqueta, below=0.1cm of c3] {$R_3(t)$};
\node[connpoint, right=0.5cm of c3] (c4) {};
\path[sinflecha] (c0) -- (c1);
\path[sinflecha] (c1) -- (c2);
\path[sinflecha] (c2) -- (c3);
\path[sinflecha] (c3) -- (c4);
\end{tikzpicture}

\column{.5\textwidth}

\begin{tikzpicture}
\tikzset{
  etiqueta/.style={text centered, font=\tiny} ,
  bloque/.style={rectangle,draw=black,fill=blue!70,text=white,thick, text centered, font=\tiny, minimum width=1cm, minimum height=0.5cm},
  flecha/.style={->,thick,draw=black,font=\tiny},
  sinflecha/.style={-,thick,draw=black,font=\tiny},
  intersec/.style={fill,shape=circle,minimum size=3pt, inner sep=0pt},
  connpoint/.style={minimum width=0cm,minimum height=0cm,inner sep=0pt},
}  
\node[bloque] (c1) {};
\node[etiqueta,below=0.05cm of c1] {$R_1(t)$};
\node[connpoint, left=0.5cm of c1] (lc1) {};
\node[connpoint, right=0.5cm of c1] (rc1) {};
\path[sinflecha] (lc1) -- (c1);
\path[sinflecha] (rc1) -- (c1);
\node[bloque, below=0.5cm of c1] (c2) {};
\node[etiqueta,below=0.05cm of c2] {$R_2(t)$};
\node[connpoint, left=0.5cm of c2] (lc2) {};
\node[connpoint, right=0.5cm of c2] (rc2) {};
\path[sinflecha] (lc2) -- (c2);
\path[sinflecha] (rc2) -- (c2);
\node[bloque, below=0.5cm of c2] (c3) {};
\node[etiqueta,below=0.05cm of c3] {$R_3(t)$};
\node[connpoint, left=0.5cm of c3] (lc3) {};
\node[connpoint, right=0.5cm of c3] (rc3) {};
\path[sinflecha] (lc3) -- (c3);
\path[sinflecha] (rc3) -- (c3);
\path[sinflecha] (lc1) -- (lc2);
\path[sinflecha] (lc3) -- (lc2);
\path[sinflecha] (rc1) -- (rc2);
\path[sinflecha] (rc3) -- (rc2);
\node[connpoint,left=1cm of lc2] (inicio) {};
\path[sinflecha] (inicio) -- (lc2);
\node[connpoint,right=1cm of rc2] (fin) {};
\path[sinflecha] (fin) -- (rc2);
\end{tikzpicture}

\end{columns}

\begin{columns}[T]

\column{.5\textwidth}

\begin{equation*}
R(t) = 0.9 \cdot 0.9 \cdot 0.9 = 0.729
\end{equation*}

\column{.5\textwidth}

\begin{equation*}
R(t) = 1 - (1 - 0.9)^3 = 0.999
\end{equation*}

\end{columns}

\end{frame}

\subsection{Availability}

\begin{frame}[t]{Availability}
\begin{itemize}
  \item In many cases, it is more interesting to know availability.
  \item Availability of a system $A(t)$ defined as the
        probability that the system is working correctly at instant $t$.
    \begin{itemize}
      \item Reliability considers interval $[0,t]$.
      \item Availability considers a concrete instant in time.
    \end{itemize}
  \item A system modelled as following state diagram.
\end{itemize}
\begin{center}
\begin{tikzpicture}[->,>=stealth',shorten >=1pt,auto,node distance=3.5cm,
                    semithick]
  \tikzstyle{every state}=[fill=blue,draw=none,text=white,font=\tiny,minimum width=1.5cm]
  \tikzset{every edge/.append style={font=\tiny}}

  \node[state] (working) {Working};

  \node[state] (stopped) [right=2.5cm of working] {Not working};


  \path (working) edge[bend left=30]  
          node{Failure}                  
        (stopped);

  \path (stopped) edge[bend left=30]   
          node{Repair} 
        (working);
\end{tikzpicture}
\end{center}
\end{frame}

\begin{frame}[t]{Availability measurement}
\begin{itemize}
  \item Let MTTF the average time to failure.
  \item Let MTTR the average time to repair.
  \item System availability $A$ is defined as:
\end{itemize}

\begin{equation*}
A = \frac{MTTF}{MTTF + MTTR}
\end{equation*}

\begin{itemize}
  \item What does a reliability of 99\% mean?	
    \begin{itemize}
      \item In 365 days, it works correctly $\frac{99 \cdot 365}{100} = 361.35$ days.
      \item Out of service $3.65$ days.
    \end{itemize}
\end{itemize}
\end{frame}

\begin{frame}{Annual time without service}
\begin{center}
{\small
\begin{tabular}{|l|l|}
\hline
Availability (\%) & Days without service in a year\\
\hline
\hline
98\% & 7.3 days\\
\hline
99\% & 3.65 days\\
\hline
99.8\% & 17 hours y 30 minutes\\
\hline
99.9\% & 8 hours y 45 minutes\\
\hline
99.99\% & 52 minutes y 30 seconds\\
\hline
99.999\% & 5 minutes y 15 seconds\\
\hline
99.9999\% & 31.5 seconds\\
\hline
\end{tabular}
}
\end{center}
\end{frame}

\begin{frame}{Computing availability}
\begin{itemize}
  \item Elements availability
    \begin{itemize}
      \item HW: 99.99\%
      \item Disk: 99.9\%
      \item OS: 99.99\%
      \item Application: 99.9\%
      \item Communications: 99.9\%
    \end{itemize}

  \mode<presentation>{\vfill\pause}
  \item System availability:
    \begin{itemize}
      \item Product of elements availability.
    \end{itemize}
\end{itemize}
\begin{equation*}
A(t) = \prod_{i=1}^{N} A_i(t) = 99.6804 \Rightarrow 1.17 \text{days without service}
\end{equation*}
\end{frame}

