\section{Trends in Technology}

\begin{frame}[t]{Impact of technology}
\begin{itemize}
  \item An \textmark{Instruction Set Architecture} may last decades 
        through \textemph{evolution}.
    \begin{itemize}
      \item Need to plan for \textmark{long term} evolution.
      \item Learn from \textmark{critical technologies} evolution.
    \end{itemize}

  \mode<presentation>{\vfill}
  \item \textgood{Critical technologies}:
    \begin{itemize}
      \item Integrated circuit logic.
      \item Semiconductor DRAM.
      \item Semiconductor Flash.
      \item Magnetic disks.
      \item Networks.
    \end{itemize}
\end{itemize}
\end{frame}

\begin{frame}[t]{Integrated circuit logic}
\begin{itemize}
  \item Historical transistor technology evolution:
    \begin{itemize}
      \item Density: +35\% per year.
      \item Die size increase: 10\% - 20\% per year.
      \item Combined effect: 40\% - 55\% transistor count per year.
      \item \textemph{Moore's Law}.
    \end{itemize}

  \mode<presentation>{\vfill\pause}
  \item Evolution of \textemph{Moore's Law}
    \begin{itemize}
      \item 1965: Number of transistors per chip doubled every year.
      \item 1975: Number of transistors per chip doubled ever two years.
      \item \textbad{No longer true!}
        \begin{itemize}
          \item 2010: Intel processor: 1,170 millions transistors.
          \item 2016 (predicted): 18,720 millions transistors.
          \item 2016 (reality): 1,750 millions.
        \end{itemize}
    \end{itemize}
\end{itemize}
\end{frame}

\begin{frame}[t]{Semiconductor DRAM}
  \begin{itemize}
    \item \textbad{Activity}:
      \begin{enumerate}
        \item \emph{Read} \textmark{Section 2.2} -- 
              Memory Technology Optimizations (pages 84--90).
          \begin{itemize}
            \item Only SRAM, DRAM and Graphics Data RAMs
            \item \bibhennessy
          \end{itemize}

        \mode<presentation>{\vfill}
        \item \emph{Read} \textgood{\url{https://en.wikipedia.org/wiki/DDR_SDRAM}}
          \begin{itemize}
            \item Look at table of DDR SDRAM generations.
          \end{itemize}

        \mode<presentation>{\vfill}
        \item \textgood{Key aspects}:
          \begin{itemize}
            \item Evolution of clock rate.
            \item Evolution of bandwidth (MT/s and MB/s).
          \end{itemize}
      \end{enumerate}
  \end{itemize}
\end{frame}

\begin{frame}[t]{Flash memory}
  \begin{itemize}
    \item \textbad{Activity}
      \begin{enumerate}
        \item \emph{Continue reading} \textmark{Section 2.2} -- 
              Memory Technoloty Optimizations (pages 92--93).
          \begin{itemize}
            \item Flash Memory and Phase-Change Memory Technology
            \item \bibhennessy
          \end{itemize}

        \mode<presentation>{\vfill}
        \item Try to find information about NVMe.
          \begin{itemize}
            \item What is the relation with Flash Memory?
            \item Where are NVMe's used?
          \end{itemize}
      \end{enumerate}
  \end{itemize}
\end{frame}

\begin{frame}[t]{Bandwidth and latency}
  \begin{itemize}
    \item \textgood{Bandwidth} or \textgood{throughput}.
      \begin{itemize}
        \item \textmark{Amount of work} performed \textmark{per unit of time}.
        \item \textemph{Processors}: Increased between 32,000 and 40,000 times.
        \item \textemph{Memory/disks}: Increased between 400 and 2,400 times.
      \end{itemize}

    \mode<presentation>{\vfill\pause}
    \item \textgood{Latency}:
      \begin{itemize}
        \item Time betwee event \textgood{start} and \textgood{end}.
        \item \textemph{Processors}: Improved between 50 and 90 times.
        \item \textemph{Memory/disks}: Improved between 8 and 9 times.
      \end{itemize}

    \mode<presentation>{\vfill\pause}
    \item Bandwidth improves much faster than latency.

    \mode<presentation>{\vfill\pause}
    \item \textbad{Note}: See figures 1.9 (page 21) and 1.10 (page 22).
      \begin{itemize}
        \item \bibhennessy
      \end{itemize}
  \end{itemize}
\end{frame}
