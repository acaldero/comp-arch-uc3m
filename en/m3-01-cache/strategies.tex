\section{Policies and strategies}

\begin{frame}[t]{Four questions about memory hierarchy}
\begin{enumerate}
  \item Where is a block placed in the upper level?
    \begin{itemize}
      \item \textmark{Block placement}.
    \end{itemize}

  \mode<presentation>{\vfill\pause}
  \item How is a block found in the upper level?
    \begin{itemize}
      \item \textmark{Block identification}.
    \end{itemize}

  \mode<presentation>{\vfill\pause}
  \item Which block must be replaced on a miss?
    \begin{itemize}
      \item \textmark{Block replacement}.
    \end{itemize}


  \mode<presentation>{\vfill\pause}
  \item What happens on a write?
    \begin{itemize}
      \item \textmark{Write strategy}.
    \end{itemize}

\end{enumerate}
\end{frame}

\begin{frame}[t,shrink=10]{Q1: Block placement}
\begin{itemize}
  \item \textgood{Direct mapping}.
    \begin{itemize}
      \item Cache organized in cache lines.
      \item A memory block is placed in a line determined by its address.
      \item Placement $\rightarrow$ $block \mod n_{blocks}$
    \end{itemize}

  \mode<presentation>{\vfill\pause}
  \item \textgood{Fully associative mapping}.
    \begin{itemize}
      \item Each block has associated a unique tag.
      \item Each line stores the tag of the currently contained block.
      \item A block is placed in any line in the cache.
      \item Placement $\rightarrow$ Anywhere.
    \end{itemize}

  \mode<presentation>{\vfill\pause}
  \item \textgood{Set associative mapping}.
    \begin{itemize}
      \item Cache lines organized in sets.
      \item Each set has multiple lines.
      \item Set selection determined by block address 
            $\rightarrow$ $block \mod n_{sets}$
      \item Block placement within set determined by tag 
            $\rightarrow$ Anywhere in set.
    \end{itemize}
\end{itemize}
\end{frame}

\begin{frame}[t]{Reasons for cache misses}
\begin{itemize}
  \item A miss may be due to any of the following reasons:

    \begin{itemize}
    
      \mode<presentation>{\vfill\pause}
      \item \textmark{Compulsory}:
        \begin{itemize}
          \item First access to a memory address in a block.
          \item They would happen even with an infinite cache.
        \end{itemize}

      \mode<presentation>{\vfill\pause}
      \item \textmark{Capacity}:
        \begin{itemize}
          \item Accessing a block that was evicted from cache.
          \item They might not happen in a larger cache.
        \end{itemize}

      \mode<presentation>{\vfill\pause}
      \item \textmark{Conflict}:
        \begin{itemize}
          \item Accessing a block that was evicted because 
                it is mapped to the same set/line than another block.
          \item Only in caches that are not fully associative.
        \end{itemize}
    \end{itemize}
\end{itemize}
\end{frame}


\begin{frame}[t]{Q2: Block identification}
\begin{itemize}
  \item \textgood{Block address}:
    \begin{itemize}
      \item \textmark{Tag}: Identifies entry address.
        \begin{itemize}
          \item Validity bit in every entry to signal whether content is valid.
        \end{itemize}
      \item \textmark{Index}: Selects the set.
    \end{itemize}

  \mode<presentation>{\vfill\pause}
  \item \textgood{Block offset}:
    \begin{itemize}
      \item Selects data within block.
    \end{itemize}

  \mode<presentation>{\vfill\pause}
  \item \textmark{Higher associativity means}:
    \begin{itemize}
      \item Less \textmark{index} bits.
      \item More \textmark{tag} bits.
    \end{itemize}
\end{itemize}
\end{frame}

\begin{frame}[t]{Q3: Block replacement}
\begin{itemize}
  \item \textgood{Relevant} for \textmark{associative mapping} and 
        \textmark{set associative mapping}:
    \begin{itemize}
      \item \textmark{Random}.
        \begin{itemize}
          \item Easy to implement.
        \end{itemize}
      \item \textmark{LRU}: Least Recently Used.
        \begin{itemize}
          \item Increasing complexity as associative increases.
        \end{itemize}
      \item \textmark{FIFO}.
        \begin{itemize}
          \item Approximates LRU with a lower complexity.
        \end{itemize}
    \end{itemize}
\end{itemize}
\end{frame}

\begin{frame}[t]{Q4: Write strategy}
\begin{columns}[T]

\begin{column}{.5\textwidth}
\begin{block}{Write-through}
\begin{itemize}
  \item All writes sent to bus and memory.
  \item Easy to implement.
  \item Performance issues in SMPs.
\end{itemize}
\end{block}
\end{column}

\pause
\begin{column}{.5\textwidth}
\begin{block}{Write-back}
\begin{itemize}
  \item Many writes are a hit.
  \item Write hits do \alert{not} go to bus and memory.
  \item Propagation and serialization problems.
  \item More complex.
\end{itemize}
\end{block}
\end{column}

\end{columns}
\end{frame}

\begin{frame}[t,shrink=10]{Q4: Write strategy}
\begin{itemize}
  \item \textgood{Where is write done?}:
    \begin{itemize}
      \item \textmark{write-through}: In cache block and next level in memory.
      \item \textmark{write-back}: Only in cache block.
    \end{itemize}

  \mode<presentation>{\vfill\pause}
  \item \textgood{What happens when a block is evicted from cache?}:
    \begin{itemize}
      \item \textmark{write-through}: Nothing else.
      \item \textmark{write-back}: Next level in memory is updated.
    \end{itemize}

  \mode<presentation>{\vfill\pause}
  \item \textgood{Debugging}:
    \begin{itemize}
      \item \textmark{write-through}: Easy.
      \item \textmark{write-back}: Difficult.
    \end{itemize}

  \mode<presentation>{\vfill\pause}
  \item \textgood{Miss causes write?}:
    \begin{itemize}
      \item \textmark{write-through}: No.
      \item \textmark{write-back}: Yes.
    \end{itemize}

  \mode<presentation>{\vfill\pause}
  \item \textgood{Repeated write goes to next level?}:
    \begin{itemize}
      \item \textmark{write-through}: Yes.
      \item \textmark{write-back}: No.
    \end{itemize}
\end{itemize}
\end{frame}

\begin{frame}[t]{Write buffer}
\makebox[\textwidth][c]{
\input{en/m3-01-cache/write-buffer.tkz}
}

\begin{columns}[T]

\begin{column}{.5\textwidth}
\begin{itemize}
  \item \textmark{Why a buffer?}
    \begin{itemize}
      \item To avoid stalls in CPU.
    \end{itemize}
  \item \textmark{Why a buffer instead of a register?}
    \begin{itemize}
      \item Write bursts are frequent.
    \end{itemize}
\end{itemize}
\end{column}

\begin{column}{.5\textwidth}
\begin{itemize}
  \item \textmark{Are RAW hazards possible?}
    \item Yes.
    \item \textgood{Alternatives}:
      \begin{itemize}
        \item \textmark{Write allocate}: Block allocated in cache.
        \item \textmark{No-write allocate}: Block modified in next level.
      \end{itemize}
\end{itemize}
\end{column}

\end{columns}
\end{frame}

\begin{frame}[t]{Miss penalty}
\begin{itemize}
  \item \textgood{Miss penalty}:
    \begin{itemize}
      \item Total latency miss.
      \item Exposed latency (generating CPU stalls).
    \end{itemize}
\end{itemize}

\mode<presentation>{\vfill}
\begin{block}{Miss penalty}
\[
\frac{\text{stall\_cycles}_{memory}}{IC} =
\]
\[
\frac{misses}{IC} \times \left( latency_{total} - latency_{overlapped} \right)
\]
\end{block}
\end{frame}
