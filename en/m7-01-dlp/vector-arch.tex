\section{Vector architectures}

\begin{frame}[t]{Vector architecture approach}
\begin{itemize}
  \item Perform computations on vectors of data:
    \begin{enumerate}
      \item Read data from memory placing in large sequential register files.
      \item Operate on data in register file.
      \item Store data into memory reading from register file.
    \end{enumerate}

  \mode<presentation>{\vfill\pause}
  \item Register file behaves as compiler cotrolled buffer.
    \begin{itemize}
      \item Hides memory latency.
      \item Memory access deeply pipelined.
      \item Long memory latency only one per vector load/store.
    \end{itemize}
\end{itemize}
\end{frame}

\begin{frame}[t]{Example: RV64V}
\begin{itemize}
  \item \textmark{Vector registers}: 
    \begin{itemize}
      \item 32 vector registers.
      \item Each vector element 64 bits wide.
    \end{itemize}

  \mode<presentation>{\vfill\pause}
  \item \textmark{Vector functional units}: 
    \begin{itemize}
      \item Each unit is fully pipelined.
    \end{itemize}

  \mode<presentation>{\vfill\pause}
  \item \textmark{Vector load/store unit}: 
    \begin{itemize}
      \item Transfer vector to/from memory.
      \item Fully pipelined.
    \end{itemize}

  \mode<presentation>{\vfill\pause}
  \item \textmark{Scalar registers}:
    \begin{itemize}
      \item Scalar data input to vector functional units.
      \item Computing addresses for vector load/store.
    \end{itemize}
\end{itemize}
\end{frame}

\begin{frame}[t]{Dynamic register typing}
\begin{itemize}
  \item Vector instructions omit data type and size.
    \begin{itemize}
      \item Associate data type and size with each vector register.
      \item Instructions to configure vector registers.
      \item Simplifies the number of vector instructions.
    \end{itemize}

  \mode<presentation>{\vfill\pause}
  \item Vector register memory used by enabled vector registers.
    \begin{itemize}
      \item 4 vectors of doubles:
        \begin{itemize}
          \item \asminst{vsetdcfg} \asmreg{4*FP64} 4 vectors of doubles
        \end{itemize}
      \item 1 vector of floats and 3 vectors of doubles:
        \begin{itemize}
          \item \asminst{vsetdcfg} \asmreg{1*FP32,3*FP64}
        \end{itemize}
    \end{itemize}

  \mode<presentation>{\vfill\pause}
  \item Maximum vector length:
    \begin{itemize}
      \item Set by processor and derived from number of enabled vectors.
      \item 1024 bytes vector memory with 4 vectors of doubles $\Rightarrow$ 256 / 8 = 32 elements.
    \end{itemize}
\end{itemize}
\end{frame}

\begin{frame}[t,fragile]{Example DAXPY}
\begin{columns}[T]

\column{.5\textwidth}
\begin{itemize}
  \item Linear transformation of a vector:
\[
\vec{y} = a \cdot \vec{x} + \vec{y}
\]

  \item Assume vector length match array sizes.
    \begin{itemize}
      \item Arrays of 32 elements.
    \end{itemize}
\end{itemize}

\pause
\column{.5\textwidth}
\begin{block}{Scalar code}
\begin{lstlisting}[language=generalasm2]
      fld f0, a
      addi t2,t0,#256
Loop: fld f1,0(t0)
      fmul.d f1,f1,f0
      fld f2,0(t1)
      fadd.d f2,f2,f1
      fsd f2,0(t1)
      addi t0,t0,#8
      addi t1,t1,#8
      bne t2,t0,Loop
\end{lstlisting}
\end{block}

\end{columns}
\end{frame}

\begin{frame}[t,fragile]{DAXPY with vector instructions}
\begin{columns}[T]

\column{.65\textwidth}
\begin{block}{Vectorized DAXPY}
\begin{lstlisting}[language=generalasm2]
vsetdcfg 4*FP64  # 4 vectors double precision
fld f0, a        # Load scalar a in f0
vld v0, t0       # Load vector v0 from address t0
vmul v1, v0, f0  # Multiply vector by scalar
vld v2, t1       # Load vector v2 from address t1
vadd v3, v1, v2  # Add vectors
vst v3, t1       # Store vector v3 to address t1
vdisable         # Disable vector registers
\end{lstlisting}
\end{block}

\column{.35\textwidth}
\begin{itemize}
  \item Reduction in number of instructions.
    \begin{itemize}
      \item 258 $\rightarrow$ 8.
    \end{itemize}

  \item Vectorization requires independent iterations.

  \item Reduction of stalls thanks to instruction chaining.
\end{itemize}

\end{columns}
\end{frame}
