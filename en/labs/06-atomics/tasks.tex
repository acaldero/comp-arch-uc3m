\section{Tasks}

\subsection{Compilation and execution}

In this task you will have to compile and run the program.
You will also compare the \emph{Debug} and \emph{Release} versions.

Compile the project and generate the \emph{Debug} version.

Run the generated program 3 times.
Take note for each execution of the following information:
\begin{itemize}
\item The final value of \cppid{counter}.
\item The execution time measured by the program.
\end{itemize}

Now generate the \emph{Release} version.

Run the program again and take note of the new results.

\subsection{Detecting data races}

To detect data races you will use a compiler \textmark{sanitizer}: 
\textgood{Thread Sanitizer}.

We will generate two new execution profiles from the \emph{debug} and \emph{release} profiles.
They will be named \emph{debug-tsan} and \emph{release-tsan}.

When generating the corresponding projects with cmake you will need to add the flag
\cppid{"-fsanitize=thread"} with option
\cppkey{-DCMAKE\_CXX\_FLAGS="-fsanitize=thread"}.

Use these new profiles to identify possible data races.

\subsection{Protecting with locks}

Protect the \cppid{value\_} data member with a \cppid{std::mutex}. For this:

\begin{itemize}
  \item Add a data member to class \cppid{counter} of type \cppid{std::mutex}.
  \item Protect by using \cppid{std::mutex} the critical sections.
\end{itemize}

Place this modifications in the source file \cppid{counter\_mutex.cpp}.

Check that the program is free of race conditions with \textmark{Thread Sanitizer}.

Evaluate the performance for the cases of 2, 4, 8 and 16 threads.

\subsection{Protecting with an atomic}

In C++20 you may use atomic types for protecting floating point values.
We will use this to provide a lock free solution.

Change the type of data member \cppid{value\_} to \cppid{std::atomic<double>}.

Place this modifications in the source file \cppid{counter\_atomic.cpp}.

Check that the program is free of race conditions with \textmark{Thread Sanitizer}.

Evaluate the performance for the cases of 2, 4, 8 and 16 threads.

\subsection{Protecting with a spin lock and sequential consistency}

Now we will study the cases where an atomic variable is not enough.
Consider the case of updating two values in the critical section.

\begin{itemize}

  \item Modify class \cppid{counter} to have a second data member named \cppid{time\_} 
        of type \cppkey{float}.

  \item Modify the \cppid{update()} function so that each invocation increments
        \cppid{time\_} in \cppid{0.25} units.

\end{itemize}

To control access to the critical section, use an object of type \cppid{spinlock\_mutex} 
as the following, with sequential consistency:
\begin{lstlisting}
class spinlock_mutex {
public:
  spinlock_mutex() : flag_{} {}

  void lock() {
    while (flag_.test_and_set(std::memory_order_seq_cst)) {}
  }

  void unlock() {
    flag_.clear(std::memory_order_seq_cst);
  }

private:
  std::atomic_flag flag_;
};
\end{lstlisting}

Place this modifications in the source file \cppid{counter\_spin\_seq.cpp}.

Check that the program is free of race conditions with \textmark{Thread Sanitizer}.

Evaluate the performance for the cases of 2, 4, 8 and 16 threads.

\subsection{Protecting with a spin lock and release/acquire consistency}

Repeat the previous section with a \cppid{spinlock\_mutex} that uses
release/acquire consistency.

\begin{lstlisting}
class spinlock_mutex {
public:
  spinlock_mutex() : flag_{} {}

  void lock() {
    while (flag_.test_and_set(std::memory_order_acquire)) {}
  }

  void unlock() {
    flag_.clear(std::memory_order_release);
  }

private:
  std::atomic_flag flag_;
};
\end{lstlisting}

Place this modifications in the source file \cppid{counter\_spin\_ra.cpp}.

Check that the program is free of race conditions with \textmark{Thread Sanitizer}.

Evaluate the performance for the cases of 2, 4, 8 and 16 threads.

\subsection{Optimizing the spin lock}

For the two previous cases, consider optimizing the spin lock by inserting a relaxed read
inside the loop for operation \cppid{lock()}.

\begin{lstlisting}
  void lock() {
    while (flag_.test_and_set(std::memory_order_XXXX)) {
      while (flag_.test(std::memory_order_relaxed)) {}
    }
  }
\end{lstlisting}

Place this modifications in the source files 
\cppid{counter\_spin\_seq\_opt.cpp} and
\cppid{counter\_spin\_ra\_opt.cpp}.

Check that the program is free of race conditions with \textmark{Thread Sanitizer}.

Evaluate the performance for the cases of 2, 4, 8 and 16 threads.
