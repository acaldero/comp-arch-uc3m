\clearpage
\subsection{Impact of branch prediction}

In this task you will study the impact of branch prediction.

The program uses a function generating random values in the $[-100.0, 100.0]$ interval. 
This function is provided in the header file \cppid{generate.hpp} 
and implemented in the file \cppid{generate.cpp}.

\lstinputlisting[frame=single,title={File: generate.hpp}]{lab/05-ilp/generate.hpp}

\lstinputlisting[frame=single,title={File: generate.cpp}]{lab/05-ilp/generate.cpp}

You are also provided with a main program that computed the average of 
positive numbers from the generated sample.

\lstinputlisting[frame=single,title={File: bpred1.cpp}]{lab/05-ilp/bpred1.cpp}

The program computes the needed time to print the average.

\subsection{Task: Program execution}

Compile and run the original program. Determine the total execution time and 
the execution time for computing the average.

\subsection{Task: Branches and branch prediction}

Use the \cppkey{perf} tool to determine the behavior of branches.
Query the list of available events with the option \cppkey{perf list}.

In particular, you must identify events for:
\begin{itemize}
  \item Total number of executed branch instructions (\emph{branch}).
  \item Total number of prediction misses.
\end{itemize}

\subsection{Task: Impact on sorted sequences}

Copy the program \cppid{bpred1.cpp} in a new program \cppid{bpred2.cpp}.

Before entering the time measurement block, sort the vector.
To perform sorting you may select between the following functions from standards library.

\begin{itemize}
  \item \cppid{std::sort()}: \url{https://en.cppreference.com/w/cpp/algorithm/sort}.
  \item \cppid{std::ranges::sort()}: \url{https://en.cppreference.com/w/cpp/algorithm/ranges/sort}.
\end{itemize}

Repeat the measurements, paying attention to the time needed to compute the average.

Find an explanation for new experimental results.

\subsection{Task: Branch elimination}

Copy the program \cppid{bpred1.cpp} in a new program \cppid{bpred3.cpp}.

Try to eliminate branches inside the loop at function \cppid{average\_positive()}. 
To achieve this, you may use the ternary conditional operator to generate a bitmask:

\begin{lstlisting}
result = (condition)?expr1:expr2;
\end{lstlisting}

The evaluation of this statement results in \cppid{expr1} when the condition is true
or \cppid{expr2} when the condition is false.
You may use this statement to generate a bitmask that has all bits to 1 when a
condition is satisfied and all bits to 0 otherwise.

To apply the bitmask on a \cppkey{double} you may use function
\cppid{std::bit\_cast<>} from C++ 20. 
You may consult the details about this function at
\url{https://en.cppreference.com/w/cpp/numeric/bit_cast}.
Remind to perform a cast to an intger type with the smae size than the floating
point type.

Repeat measurements, paying attention to time needed to perform the average computation.

Find an explanation to the new expremiental results.
