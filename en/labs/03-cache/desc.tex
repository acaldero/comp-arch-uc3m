\section{Description of the laboratory}

In this laboratory the you will evaluate several C++ programs  with
\textmark{cachegrind} the tool, which is part of the tools available inside
\textmark{valgrind} (\textgood{\url{http://valgrind.org/}}). This tool is
capable of simulating the execution of a program and evaluating the impact of
the cache performance. 

If the evaluated program is compiled with the debugging information flag
activated(\textmark{-g} in \cppid{gcc}), \textmark{cachegrind} can also
provide the cache utilization for every line in the source code. For this
purpose, the executable must be generated using the commands:

\begin{lstlisting}[style=terminal,aboveskip=1em,belowskip=1em]
cmake -S . -B debug -DCMAKE_BUILD_TYPE=Debug
cmake --build debug
\end{lstlisting}

\fbox{ \parbox{.9\textwidth}{
\textbad{ATTENTION}: 
Keep in mind that for executing jobs in the \cppid{avignon} cluster
you must first create a \emph{script} to launch the compilation.
}}
\vspace{1em}

This way, it is possible to obtain the information of the execution of the
program using \textemph{valgrind} and \textemph{cachegrind}. If you want
to analyze an executable named \cppid{./test}, you can use the following
command:

\begin{lstlisting}[style=terminal,aboveskip=1em,belowskip=1em]
valgrind --tool=cachegrind ./test
\end{lstlisting}

\fbox{ \parbox{.9\textwidth}{
\textbad{ATTENTION}: 
Keep in mind that for executing jobs in the \cppid{avignon} cluster
you must first create a \emph{script} to launch \textemph{valgrind} executions.
}}
\vspace{1em}

If no additional parameter is provided, \textmark{cachegrind} simulates code
execution on the real hardware where it is being executed. Otherwise,
it can be used to evaluate a given cache configuration.
When the execution finishes it shows basic statistics of cache usage.
It also generates a file named \cppid{cachegrind.out.<pid>},
where \cppid{pid} is the process identifier. 
This file contains additional information that can be analyzed using the 
\cppid{cg\_annotate} utility. 
(The name of the file can be modified with the option
\cppid{--cachegrind-out-file=<file\_name>}).

The following option can be used to simulate a given cache configuration:

\begin{lstlisting}[style=terminal,aboveskip=1em,belowskip=1em]
--I1=<size in bytes>,<associativity>,<line size in bytes>
\end{lstlisting}

They specify the size (in bytes), the associativity (number of ways) and the
size of the cache line of the instruction cache of level 1.

\begin{lstlisting}[style=terminal,aboveskip=1em,belowskip=1em]
--D1=<size>,<associativity>,<line size>
\end{lstlisting}

They specify the size (in bytes), the associativity (number of ways) and the
size of the cache line for the data cache of level 1.

\begin{lstlisting}[style=terminal,aboveskip=1em,belowskip=1em]
--LL=<size>,<associativity>,<line size>
\end{lstlisting}

They specify the size (in bytes), the associativity (number of ways) and the
size of the cache line of the last level cache.

The tool \cppid{cg\_annotate} receives as parameter the file to be analyzed and
the absolute path of the files that will be annotated line by line.  If the path
or the files to be annotated are not known the option \cppid{--auto=yes} can be
used to annotate every file that can be of interest.

\begin{lstlisting}[style=terminal,aboveskip=1em,belowskip=1em]
cg_annotate cachegrind.out.XXXXX --auto=yes
\end{lstlisting}

\fbox{ \parbox{.9\textwidth}{
\textbad{ATTENTION}: 
Keep in mind that for executing jobs in the \cppid{avignon} cluster
you must first create a \emph{script} to launch \textemph{cg\_annotate} executions.
}}
\vspace{1em}

Additionally, if only one file is going to be annotated: 

\begin{lstlisting}[style=terminal,aboveskip=1em,belowskip=1em]
cg_annotate cachegrind.out.XXXXX <absolute path to file .cpp>
\end{lstlisting}

For more information about the usage of \textmark{cachegrind} and
\cppid{cg\_annotate},the documentation can be found in the web-page of Valgrind:
\textgood{\url{http://valgrind.org/docs/manual/cg-manual.html}}.
