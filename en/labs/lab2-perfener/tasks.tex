\section{Tasks}

\subsection{Lab account}

If you do not have an account in the Computer Science and Engineering Department Lab, 
you will need one.

You will find all the information in the following page:

\begin{itemize}
  \item \url{https://www.lab.inf.uc3m.es/en/services/account-opening/}
\end{itemize}

This is the account you will use to access the cluster from the Computer Science
and Engineering Department.

\subsection{Access to the cluster}

You have the following options:


\begin{itemize}
  \item Web connection. You will find all the information at:
    \begin{itemize}
      \item \url{https://www.lab.inf.uc3m.es/en/services/virtual-classroom/}
      \item Once connected, you must select the \textemph{Avignon} server.
    \end{itemize}

  \item Connect through \cppid{ssh} to \cppkey{avignon.lab.inf.uc3m.es}. 
        From a terminal, you must use the \cppid{ssh} command, with
        your user name and password from the Computer Science and Engineering Lab.
\begin{lstlisting}[style=terminal]
$ ssh -l username avignon.lab.inf.uc3m.es
username@avignon.lab.inf.uc3m.es's password: 
Linux avignon-frontend 5.10.0-15-amd64 #1 SMP Debian 5.10.120-1 (2022-06-09) x86_64
...
$
\end{lstlisting}

\end{itemize}

\subsection{Command execution}

Once you have entered your password you will be in the \emph{front-end} node.
This machine simply gives you access to launch jobs in other machines.
However you must not run any program directly in tghe \emph{front-end}.

To run a job, you must write a \emph{shell script} with the tasks you want
to run. Then you will launch the job with the command \cppkey{sbatch}.

You will create a script to know the hardware features form the computing node.
To know those features you may use the \cppkey{lscpu} command.

Write the following code in a file named \cppid{runlscpu.sh}

\begin{lstlisting}[language=bash,frame=single,title={File: runlscpu.sh}]
#!/bin/sh
lscpu
\end{lstlisting}

To launch the job, you shall use the \cppkey{sbatch} command.
The name of the \emph{shell script} is an argument.

\begin{lstlisting}[style=terminal]
$ sbatch ./runlscpu.sh
Submitted batch job 408
$
\end{lstlisting}

You will be able to see the list of active jobs with the \textmark{squeue} command.

When the job has finished, the standard output is saved in a file named
\cppid{slurm-XXX.out}, where XXX is the job number.

Compare the features from the \emph{front-end} node (running \cppkey{lscpu})
and the computing node (using \cppkey{sbatch ./lscpu.sh}).

Pay attention to the following:
\begin{itemize}
  \item Architecture.
  \item Number of CPUs.
  \item CPU model.
  \item Clock frequency rante.
  \item Sizes in all cache levels.
\end{itemize}

\subsection{Compiling C++ programs}

During this lab, three versions of a program will be used.
Teh program generates an image for the Mandelbrot set in PBM format.

\begin{itemize}
  \item \cppid{mandel.py}: Python version.
  \item \cppid{mandel.cpp}: Simplified sequential version in C++.
  \item \cppid{mandel-par.cpp}: Manually optimized parallel version in C++.
\end{itemize}

\subsubsection{File transfer}

In Aula Global you will find a file named \cppid{mandel.zip} with all the code you need.
Place this file in a directory in you local machine named \cppid{lab2}.
Unzip the file in a subdirectory named \cppid{mandel}:

\begin{lstlisting}[style=terminal]
username@machine:~/lab2$ unzip mandel.zip mandel
Archive:  mandel.zip
  inflating: mandel/CMakeLists.txt   
  inflating: mandel/mandel.cpp       
  inflating: mandel/mandel.py        
  inflating: mandel/mandel-par.cpp   
username@machine:~/lab2$ ls
mandel  mandel.zip
\end{lstlisting}

To transfer the fils you may use the \cppkey{scp} command.

Firstly, create in the remote directoyr a subdirectory named \cppid{mandel}.

\begin{lstlisting}[style=terminal]
$user@avignon-frontened:/~lab2$ mkdir
$user@avignon-frontened:/~lab2$ ls
mandel  runlscpu.sh  ...
\end{lstlisting}

Go to your local directory where you downloaded the \cppid{mandel.zip} fil and 
use the following command.

\begin{lstlisting}[style=terminal,basicstyle=\tiny\ttfamily,columns=fixed]
username@machine:~/lab2$ scp mandel/* user@avignon.lab.inf.uc3m.es:~/lab2/mandel
user@avignon.lab.inf.uc3m.es password: 
CMakeLists.txt                            100%  427   166.6KB/s   00:00    
mandel.cpp                                100% 2193   783.0KB/s   00:00    
mandel-par.cpp                            100%   11KB   3.0MB/s   00:00    
mandel.py                                 100% 1692   596.9KB/s   00:00    
username@machine:~/lab2$ 
\end{lstlisting}

You will have transfered all the needed files to the remote directory
\cppid{~/lab2/mandel}.

\subsubsection{Compiling from command line}

\cppid{CMake} files allow you to automate the compilation process.
In this task you will compile two versions for each program:
a \textmark{Debug} version and a \textmark{Release} version.

Note that the \emph{front-end} machine and the computin nodes are quite different
(as you noticed with \cppid{lscpu}).
It is very important that you compile in a computing node.

To achieve this, you will create a compilation script.
\begin{lstlisting}[language=bash,frame=single,title={File: runlscpu.sh}]
#!/bin/sh
cmake -S mandel -B mandel-debug -DCMAKE_BUILD_TYPE=Debug
cmake -S mandel -B mandel-release -DCMAKE_BUILD_TYPE=Release
cmake --build mandel-debug
cmake --build mandel-release 
\end{lstlisting}

\textbad{CLARIFICATION}: 
A small description of the effects of each command is given below.

\begin{itemize}
\begin{lstlisting}[style=terminal]
cmake -S mandel -B mandel-debug -DCMAKE_BUILD_TYPE=Debug
\end{lstlisting}
  \item Generates a project configuration in directoyr \cppid{madel-debug}
        from the source files in directory \cppid{mandel}.
        Use this option with the variable \cppkey{DCMAKE\_BUILD\_TYPE=Debug},
        to use de debug compiler options.

\begin{lstlisting}[style=terminal]
cmake -S mandel -B mandel-release -DCMAKE_BUILD_TYPE=Release
\end{lstlisting}
  \item Generates a project configuration in directoyr \cppid{madel-release}
        from the source files in directory \cppid{mandel}.
        Use this option with the variable \cppkey{DCMAKE\_BUILD\_TYPE=Release},
        to use de debug compiler options.

\begin{lstlisting}[style=terminal]
cmake --build mandel-debug
\end{lstlisting}
  \item Build the project generated in directoyr \cppid{mandel-debug}.

\begin{lstlisting}[style=terminal]
cmake --build mandel-release
\end{lstlisting}
  \item Build the project generated in directoyr \cppid{mandel-debug}.

\end{itemize}

\textbad{IMPORTANT}: 
Do not run thos commands in the \emph{front-end}.
In such case you would generate code that is not valid to be executed
in the computing nodes.
Run the script, using the command:

\begin{lstlisting}[style=terminal]
sbatch build.sh
\end{lstlisting}

Check the files generated in directories\cppid{mandel-debug}
and \cppid{mandel-release}.

\subsection{Running the programs}

In this task you will run the programs in a computing node.
For the C++ versions, you will only use the \textmark{Release} versions.

\begin{enumerate}
  \item \cppid{mandel.py}: 
        To run this program you will use the Python interpreter.
        You have availabe version 3.9.

    \begin{itemize}
      \item Do not forget to pass the option \cppid{-OO} to Python in order
            to remove debug information.
      \item The program takes as a parameter the image size, which is square.
      \item You must redirect the output to a file with extension \cppkey{.pbm}.
    \end{itemize}

\begin{lstlisting}[language=bash,title={File: runpy.sh},frame=single]
#!/bin/sh
python3.9 -OO mandel/mandel.py 8000 > py-mandel.pbm
echo "Finished running mandel.py"
\end{lstlisting}

    \begin{itemize}
      \item Run this script using the \cppkey{sbatch} command.
    \end{itemize}

  \item \cppid{mandel}: 
        To run this program, you shall invoke directly the executable.
    \begin{itemize}
      \item It takes as a first argument the size of the image, which is square.
      \item It takes as a second argument the name of the output file.
    \end{itemize}

\begin{lstlisting}[language=bash,title={File: run-seq-mandel.sh},frame=single]
#!/bin/sh
release/mandel 8000 seq-mandel.pbm
echo "Finished running release/mandel"
\end{lstlisting}
    \begin{itemize}
      \item Run this script using the \cppkey{sbatch} command.
    \end{itemize}

  \item \cppid{mandel-par}: 
        To run this program, you shall invoke directly the executable.
    \begin{itemize}
      \item It takes as a first argument the size of the image, which is square.
      \item You must redirect the output to a file with extension \cppkey{.pbm}.
    \end{itemize}

\begin{lstlisting}[language=bash,title={Archivo run-par-mandel.sh},frame=single]
#!/bin/sh
release/mandel 8000 > par-mandel.pbm
echo "Finished running release/mandel-par"
\end{lstlisting}
    \begin{itemize}
      \item Run this script using the \cppkey{sbatch} command.
    \end{itemize}
\end{enumerate}


\textbad{IMPORTANT}: 
Check that the three generate \cppid{.pbm} are identical.

Look at the \cppid{slurm-XXX} file to check that there is a finalization message.

\subsection{Basic performance measurement}

To carry out a basic peformance measurement, you shall use the \cppkey{perf} tool.
This is a Linux tool that gives you access to the operating system kernel
and to processor hardware counters.

The \cppkey{per} toool offers several subcommands.
In this lab we will use the \cppkey{stat} subcommand to gather
basic information on the execution of a program.

\begin{lstlisting}[style=terminal,basicstyle=\ttfamily]
$ perf stat program param1 param2

 Performance counter stats for 'program param1 param2':

         56.050,69 msec task-clock                #    0,992 CPUs utilized          
             6.851      context-switches          #    0,122 K/sec                  
                30      cpu-migrations            #    0,001 K/sec                  
             2.082      page-faults               #    0,037 K/sec                  
   192.416.969.761      cycles                    #    3,433 GHz                    
   437.336.887.712      instructions              #    2,27  insn per cycle         
    46.898.229.743      branches                  #  836,711 M/sec                  
        94.154.392      branch-misses             #    0,20% of all branches        

      56,521233276 seconds time elapsed

      55,930658000 seconds user
       0,120151000 seconds sys
$
\end{lstlisting}

If you want to gather informatio from all CPUs you must use the option \cppkey{-a}.

\begin{lstlisting}[style=terminal]
$ perf stat -a program params
$
\end{lstlisting}

You may find intersting some provided information.
We will focus, for now, in basic aspects.
Other aspects will be covered later in the course.

\begin{itemize}
  \item Number consumed clock cycles.
  \item Number of executed instructions.
  \item Elapsed time during program execution.
  \item Total user and system time.
  \item Equivalent number of used CPUs. This valu is higher than 1 when several cores are used
        at the same time.
  \item Average clock frequency.
  \item Number of instructions per cycle.
\end{itemize}

In this taks you will gather information on the execution parameters for the 
following versions of the program when a 8000x8000 image is generated.

\begin{enumerate}
  \item \cppkey{mandel.py}
  \item \cppkey{debug/mandel}
  \item \cppkey{release/mandel}
  \item \cppkey{debug/mandel-par}
  \item \cppkey{release/mandel-par}
\end{enumerate}

\textbad{IMPORTANT}: 
You will have to write the corresponding scritps and launch them with \cppkey{sbatch}.
For each \cppkey{sbatch} invokation a \cppid{slurm-XXX.out} file is generated
with the standard output form the execution.

Intenta sacar unas breves conclusiones de los resultados obtenidos.

\subsection{Energy measurement}

The \cppkey{stat} subcommand can also be used to measure certain events.

You can get the full event list that are availabe with the subcommand \cppkey{list}

\begin{lstlisting}[style=terminal]
$ perf list
 branch-instructions OR branches                    [Hardware event]
  branch-misses                                      [Hardware event]
  bus-cycles                                         [Hardware event]
  cache-misses                                       [Hardware event]
  cache-references                                   [Hardware event]
  cpu-cycles OR cycles                               [Hardware event]
  instructions                                       [Hardware event]
  ref-cycles                                         [Hardware event]
  alignment-faults                                   [Software event]
  bpf-output                                         [Software event]
  context-switches OR cs                             [Software event]
  cpu-clock                                          [Software event]
  cpu-migrations OR migrations                       [Software event]
  dummy                                              [Software event]
  emulation-faults                                   [Software event]
  major-faults                                       [Software event]
  minor-faults                                       [Software event]
  page-faults OR faults                              [Software event]
...
$
\end{lstlisting}

\textbad{ATTENTION}:
You must run this command in a computing node.

Use \cppkey{perf list} to get thos counters starting with \cppid{power}

Once you find thos event you may pass them to \cppkey{perf} with
option \cppkey{-e}.

\begin{lstlisting}[style=terminal]
$ perf stat -a -e 'event1,event2,event3' program params
...
$
\end{lstlisting}

In this task you shall obtain the energy information during the execution of each program.

The counter values that you obtain express the consumed energy in Joules.
As you also have the total execution time, you may get also the average power in Watts.

Try to draw some briev conclusions from your results.
