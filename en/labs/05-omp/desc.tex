\section{Lab description}

In this lab you will carry out the evaluation of running several
options in a program that performs some operation on a bitmap format image.
The program is able to convert an image to grayscale and to generate an
histogram of a given image.

The program executable is called \cppid{imgtool} and it takes three parameters:
\begin{itemize}
\item \textmark{operation}: Operation to be applied.
\item \textmark{input}: The name of the input file. This file will be bitmap image file.
\item \textmark{output}: The name of the output file
\end{itemize}

There are five available options:

\begin{itemize}
\item \textmark{copy}: Copies the input file to the output file.
\item \textmark{grayscale}: Generates an output bitmap file with the image in gray-scale.
\item \textmark{histogram}: Generates an output text file with a text representation of the image histogram.
\item \textmark{par\_grayscale}: Generates an output bitmap file with the image in gray-scale.
This option performs the computation in parallel and it is currently not correctly implemented.
\item \textmark{par\_histogram}: Generates an output text file with a text representation of the image histogram.
This option performs the computation in parallel and it is currently not correctly implemented.
\end{itemize}

\subsection{Building the program}

Please, remind that for compiling the optimized version of a program, 
you will have to generate a \emph{release} configuration.

\begin{lstlisting}[style=terminal,aboveskip=1em,belowskip=1em]
cmake -S . -B release -DCMAKE_BUILD_TYPE=Release
\end{lstlisting}

Additionally, to compile a given configuration, you must invoke \cppkey{cmake}
with the corresponding configuration name:

\begin{lstlisting}[style=terminal,aboveskip=1em,belowskip=1em]
cmake --build release
\end{lstlisting}

\fbox{ \parbox{.9\textwidth}{
\textbad{REMINDER}: 
To run jobs in the \cppid{avignon} cluster, you must first create a
script to launch the compilation.
}}

\subsection{Source code structure}

The source code is provided with the following structure:


\begin{itemize}

\item \textmark{util}: This folder contains the utility library. It contains
some basic utilities for binary input/output and program arguments parsing.

\item \textmark{img}: This folder contains all the source code for the image manipulation library.

\item \textmark{imgtool}: This folder contains the main program and some helper
functions.

\end{itemize}

In particular the image library contains a number of software components:

\begin{itemize}

\item \cppid{checking}: Contains error message management and checks.

\item \cppid{pixel}: Represents a pixel in the image.

\item \cppid{normalized\_pixel}: Alternate representation of a pixel used during
grayscale conversion.

\item \cppid{image\_header}: The header of a bitmap file. It is used for input
and output.

\item \cppid{image\_metadata}: Metadata on the image as read from the BMP file.

\item \cppid{image}: Implements an image as a 2-dimensional matrix of pixels.

\item \cppid{parallel\_image}: Another implementation of image that you will
have to modify to implement parallel version of algorithms.

\item \cppid{histogram}: A representation of histograms for the three components
of the image (red, green, blue).

\end{itemize}

Note that during this lab, you will have to modify component
\cppid{parallel\_image} to provide a parallel implementation of the
functionality to convert to grayscale and to compute the histogram.
