\section{Tasks}

During this lab you will perform the following tasks:

\begin{enumerate}

\item Build the initial version of the application and evaluate performance and
energy.

\item Modify the provided source code to correctly implement the parallel
options of the tool.

\item Evaluate the parallel version that you have produced studying:

\begin{enumerate}
  \item The impact of the number of threads from 2 to the double of the physical
number of cores.
  \item The impact of different scheduling policies (optional). 
\end{enumerate}

\end{enumerate}

\subsection{Building the initial version}

Your first task is to build the original version of the software. You shall
build only the release mode version. To do so use a script like the following
\cppid{build.sh}:

\lstinputlisting[language=bash]{lab/05-omp/build.sh}

You will run this script in one node of the regular cluster by doing:

\begin{lstlisting}[style=terminal]
sbatch build.sh
\end{lstlisting}

To run a grayscale conversion, you may use the following script
\cppid{run-gray.sh}:

\lstinputlisting[language=bash]{lab/05-omp/run-gray.sh}

The program will print the time in microseconds dedicated to the grayscale
conversion and \textmark{perf} will give you the total execution time.
With this two values you will be able to deduce which es the fraction of the
program that can be improved when parallelizing the grayscale conversion.

Now you can do the same thing by running the histogram computation with the
script \cppid{run-histo.sh}:

\lstinputlisting[language=bash]{lab/05-omp/run-histo.sh}

\subsection{Using a NUMA node}

Through the avignon front-end you may also use a high-end compute server (named
\textmark{stan}).

Prepare a script to see the hardware features of \textmark{stan} named
\cppid{lscpu.sh}

\lstinputlisting[language=bash]{lab/05-omp/lscpu.sh}

Modify the previous scripts to repeat the previous exercise on \textmark{stan}.

The new \cppid{build-stan.sh} should be:

\lstinputlisting[language=bash]{lab/05-omp/build-stan.sh}

The new script \cppid{run-gray-stan.sh} should be:

\lstinputlisting[language=bash]{lab/05-omp/run-gray-stan.sh}

The new script \cppid{run-histo-stan.sh} should be:

\lstinputlisting[language=bash]{lab/05-omp/run-histo-stan.sh}

\textbad{IMPORTANT}: To run a script on \textmark{stan} you need to specify the
execution queue name.

\begin{lstlisting}[style=terminal]
sbatch -p stan script.sh
\end{lstlisting}

Deduce again which es the fraction of the program that can be improved when
parallelizing the grayscale conversion and the histogram computation.

\subsection{Parallel implementation}

In this section you will parallelize the grayscale conversion and the histogram
computation. You will modify the implementation in software component
\cppid{parallel\_image}.

Pay attention to the following member functions:

\begin{itemize}

\item \cppid{parallel\_image::to\_grayscale()}: Performs the grayscale
conversion.
\item \cppid{parallel\_image::generate\_histogram()}: Generates the histogram
information.

\end{itemize}

Please do not forget the following:

\begin{itemize}

\item In the \cppid{CMakeLists.txt} at \textmark{img} folder you will need to
add the following:

\lstinputlisting{lab/05-omp/cmake-omp.txt}

\item Do not specify the number of threads in the code. We will change that by
using corresponding environment variable.

\end{itemize}

\subsection{Evaluation of the parallel version}

In this section you will evaluate the obtained speedup both in a regular
node and in \textmark{stan}.

As an example, you may use the following script to run the grayscale conversion
with different number of threads using the executable for \textmark{stan}:

\lstinputlisting[language=bash]{lab/05-omp/run-gray-omp-stan.sh}

You may run this on \textmark{stan} with the following command:

\begin{lstlisting}[style=terminal]
sbatch -p stan run-gray-omp-stan.sh
\end{lstlisting}

