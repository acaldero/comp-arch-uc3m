\section{Centralized shared memory architectures}

\begin{frame}[t]{SMP and memory hierarchy}
\begin{itemize}
  \item Why using centralized memory?
    \begin{itemize}
      \item Multi-level large caches decrease memory bandwidth demand on
            main memory accesses.
    \end{itemize}

  \mode<presentation>{\vfill\pause}
  \item \textgood{Evolution}:
    \begin{enumerate}[1.]
      \item Single-core with memory in \textmark{shared bus}.
      \item Memory connection in \textmark{separated bus} only for memory.
    \end{enumerate}
\end{itemize}
\end{frame}

\begin{frame}[t]{Cache memory}
\begin{itemize}
  \item \textgood{Kinds of data} in cache memory:
    \begin{itemize}
      \item \textmark{Private data}: Data used by a single processor.
      \item \textmark{Shared data}: Data used by multiple processors.
    \end{itemize}

  \mode<presentation>{\vfill\pause}
  \item \textbad{Problem} with shared data:
    \begin{itemize}
      \item \pause Datum may be replicated in multiple caches.
      \item \pause Contention is decreased.
        \begin{itemize}
          \item Each processors accesses its local copy.
        \end{itemize}
      \item \pause If two processors modify their copies \ldots
        \begin{itemize}
          \item Cache coherence?
        \end{itemize}
    \end{itemize}
\end{itemize}
\end{frame}

\begin{frame}[t,fragile]{Cache coherence}

\begin{columns}[T]

\column{.3\textwidth}
\begin{block}{Thread 1}
\begin{lstlisting}[language=generalasm]
lw $t0, dirx
addi $t0, $t0, 1
sw $t0, dirx
\end{lstlisting}
\end{block}

\column{.3\textwidth}
\begin{block}{Thread 2}
\begin{lstlisting}[language=generalasm]
lw $t0, dirx
\end{lstlisting}
%$
\end{block}

\column{.3\textwidth}
\begin{itemize}
\item \cppid{\$t0} initially 1.
\item Assuming write through.
\end{itemize}

\end{columns}

\mode<presentation>{\vfill\pause}
{\footnotesize
\begin{tabular}{l|l|l|l|l}

Process &
Instruction &
P1 Cache &
P2 Cache &
Main memory
\\

\hline
\hline

\pause
T1 & Initially & Not present & Not present & 1 \pause\\
\hline
T1 & \asminst{lw} \asmreg{\$t0}, \asmlabel{dirx} & 1 & Not present & 1 \pause\\
\hline
T1 & \asminst{addi} \asmreg{\$t0}, \asmreg{\$t0}, \asmlabel{1} & 1 & Not present & 1 \pause\\
\hline
T2 & \asminst{lw} \asmreg{\$t0}, \asmlabel{dirx} & 1 & 1 & 1 \pause\\
\hline
T1 & \asminst{sw} \asmreg{\$t0}, \asmlabel{dirx} & 2 & 1 & 1 \\
\hline

\end{tabular}
}

\end{frame}

\begin{frame}[t]{Cache incoherence}
\begin{itemize}
  \item Why does incoherence happen?
    \begin{itemize}
      \item \textgood{State duality}:
        \begin{itemize}
          \item \textgood{Global state} $\rightarrow$ \textmark{Main memory}.
          \item \textgood{Local state} $\rightarrow$ \textmark{Private cache}.
      \end{itemize}
    \end{itemize}

  \mode<presentation>{\vfill\pause}
  \item A memory system is \textgood{coherent} 
        if any read from a location returns the \textmark{most recent value}
        that has been written to that location.

  \mode<presentation>{\vfill\pause}
  \item \textgood{Two aspects}:
    \begin{itemize}
      \item \textmark{Coherence}: 
            Which value does a read return?
      \item \textmark{Consistency}: 
            When does a read get the written value?
    \end{itemize}
\end{itemize}
\end{frame}

\begin{frame}[t]{Conditions for coherence}
\begin{itemize}
  \item \textmark{Program order preservation}
    \begin{itemize}
      \item A read from processor P on location X
            after a write from processor P on location X,
            without intermediate writes on X by any other processor Q,
            always returns the value written by P.
    \end{itemize}

  \mode<presentation>{\vfill\pause}
  \item \textmark{Coherent view of memory}:
    \begin{itemize}
      \item A read from processor P on a memory location X,
            after a write form other processor Q on location X,
            returns the written value
            if both operations are separate enough in time and
            there are no intermediate writes on X.
    \end{itemize}

  \mode<presentation>{\vfill\pause}
  \item \textmark{Writes serialization}:
    \begin{itemize}
      \item Two writes on the same memory location by two different processors
            are seen in the same order by all the processors.
    \end{itemize}

\end{itemize}
\end{frame}

\begin{frame}[t]{Memory consistency}
\begin{itemize}
  \item Defines in which point in time a process reading values will see a written value.

  \mode<presentation>{\vfill\pause}
  \item \textgood{Coherence} y \textgood{consistency} are complementary:
    \begin{itemize}
      \item \textmark{Coherence}: 
            Behavior of reads and writes on a single memory location.
      \item \textmark{Consistency}: 
            Behavior of reads and writes with respect to accesses to other memory locations.
    \end{itemize}

  \mode<presentation>{\vfill\pause}
  \item There are different consistency memory models.
    \begin{itemize}
      \item \textbad{We will have a specific lecture on this problem}
    \end{itemize}

\end{itemize}
\end{frame}
