\begin{acexercise}\end{acexercise}

Consider the following global variables definition:

\begin{lstlisting}
const unsigned int max = 1024 * 1024;
double x[max];
double y[max];
double z[max];
double vx[max];
double vy[max];
double vz[max];
\end{lstlisting}

And the following function:

\begin{lstlisting}
void update_positions(double dt) {
  for (unsinged int i=0; i<max; ++i) {
    x[i] = vx[i] * dt + x[i];
    y[i] = vy[i] * dt + y[i];
    z[i] = vz[i] * dt + z[i];
  }
}
\end{lstlisting}

Consider a system with a 4 ways set associative L1 cache with a size of 32 KB and a line
size of 64 bytes. L2 cache is 8 ways set associative with a size of 1 MB and a line size
of 64 bytes. Replacement policy is LRU.

Arrays are consecutively stored in memory and first of them starts at 1024 multiple address.

\begin{enumerate}
  \item Determine the hit rate for L1 and L2 caches for running function
        \cppid{update\_positions()}. What is the global hit rate?

  \item Modify code applying the array merging optimization.

  \item Repeat computations from the first question for the code resulting from
        the second question.

  \item Assume that a hit at L1 cache requires 4 cycles and that L2 cache requires
        14 cycles. Also assume that penalty associated to bringing a block from
        main memory to L2 cache is 80 cycles. Which is the average access time in 
        each case?

\end{enumerate}


\begin{acsolution}\end{acsolution}

\paragraph{Point 1}

The loop access pattern is:

\begin{lstlisting}
vx[i], x[i], x[i], vy[i], y[i], y[i], vz[i], z[i], z[i], ...
\end{lstlisting}

Each of the arrays has $2^{20}$ positions of 8 bytes each. This results in a total of 
8 MB per array. Each line of the cache has 64 bytes, which allows hosting 8 elements of the array.

The L1 cache has 4 ways, so each way is 8KB ($2^{13}$ bytes), which results in 
$\frac{2^{13}}{2^6}$ = $2^7$ sets.

Due to each array has $2^{23}$ bytes (which is multiple of the way size), each \cppid{i} 
index of all arrays corresponds to the same cache set. The sequence of hits and misses in 
the cache will be:

\begin{itemize}
  \item \cppid{VX[0]} $\rightarrow$ Miss. Selecting way 0.
  \item \cppid{X[0]} $\rightarrow$ Miss. Selecting way 1.
  \item \cppid{X[0]} $\rightarrow$ Hit.
  \item \cppid{VY[0]} $\rightarrow$ Miss. Selecting way 2.
  \item \cppid{Y[0]} $\rightarrow$ Miss. Selecting way 3.
  \item \cppid{Y[0]} $\rightarrow$ Hit.
  \item \cppid{VZ[0]} $\rightarrow$ Miss. Selecting way 0. Evicted \cppid{vx}.
  \item \cppid{Z[0]} $\rightarrow$ Miss. Selecting way 1. Evicted \cppid{x}.
  \item \cppid{Z[0]} $\rightarrow$ Hit.
\end{itemize}

When you move to position 1, the same sequence of misses and hits is repeated. 
So in total you have:

\[
h_{L1} = \frac{3}{9} = \frac{1}{3}
\]

For the L2 cache, only accesses occur in cases where there is a miss in the L1 cache. 
For position 0, 6 failures occur, but for positions 1 through 7, 6 hits are produced 
for each element.

\[
h(L2) = 
\frac{42}{48} = 
\frac{7}{8}
\]

The hit rate will be:

\[
H = 1 - M
\]

Where:

\[
M = 
m_{h1} \cdot m_{h2} = 
\frac{2}{3} \cdot \frac{1}{8} = 
\frac{1}{12}
\]

\[
H = \frac{11}{12}
\]


\paragraph{Point 2}

Applying the required optimization, the code is:

\begin{lstlisting}
const unsigned int max = 1024 * 1024;
struct part {
  double x, y, z, vx, vy, vz;
};
part vec[max];

void update_postions(double dt) {
  for (unsinged int i=0; i<max; ++i) {
    vec[i].x = vec[i].vx * dt + vec[i].x;
    vec[i].y = vec[i].vy * dt + vec[i].y;
    vec[i].z = vec[i].vz * dt + vec[i].z;
  }
}
\end{lstlisting}

\paragraph{Point 3}

The access pattern in this case is:

\begin{lstlisting}
vec[i].vx, vec[i].x, vec[i].x, vec[i].vy, vec[i].y, vec[i].y, vec[i]. vz, vec[i].z, vec[i]. z, ...
\end{lstlisting}

Since each position of the array needs 6 positions and a cache entry occupies 8 
values, in 3 lines of cache will fit 4 complete positions of the array.

\begin{itemize}
  \item Index 0 produces 1 miss and 8 hits.
  \item Index 1 produces 2 hits, 1 miss and 6 hits.
  \item Index 2 produces 4 hits, 1 miss and 4 hits.
  \item Index 3 produces 9 hits.
\end{itemize}

This pattern is repeated throughout the array. Thus:

\[
h_{L1} = 
\frac{33}{36} = \frac{11}{12}
\]

For L2 cache, all misses in L1 cache are also misses in the L2 cache. Therefore:


\[
h_{L2} = 0
\]

And the global rate:

\[
H = 1 - M
\]

\[
M = m_{h1} \cdot m_{h2} = 
\frac{1}{12}
\]

\[
H = \frac{11}{12}
\]

\paragraph{Point 4}

For the original case:

\[
T = 
4 + (\frac{2}{3}) \cdot (14 + (\frac{1}{8}) \cdot 80) = 
4 + (\frac{2}{3}) \cdot 24 = 
4 +16 = 
20
\]

For the second case:

\[
T = 
4 + (\frac{1}{12}) \cdot (14 + 1 \cdot 80) = 
4 + (\frac{1}{12}) \cdot 94 = 
4 + 7.83 = 
11.83
\]

