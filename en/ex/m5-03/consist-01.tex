\begin{acexercise}\end{acexercise}

The following code is executed on a machine with 3 processors. The variables that used are located in shared memory and initialized to 
\cppid{0}.

Processor 1 executes:
\begin{lstlisting}
X=1;
\end{lstlisting}

Processor 2 executes:
\begin{lstlisting}
Print(X);
Print(Y);
\end{lstlisting}

Processor 3 executes:
\begin{lstlisting}
Y=1;
Print(X);
\end{lstlisting}

Assume that all the variables are atomics. Answer the following questions: 

\begin{itemize}
  \item Analyze the following outputs
    \begin{itemize}
      \item Is it possible to produce this output?
      \item If so, provide an execution order that produces this output. 
    \end{itemize}
\end{itemize}

\bigskip
\begin{tabular}{|c|c|c|}
\hline
\cppid{P2: Print(X)} & \cppid{P2: Print(Y)} & \cppid{P3: Print(X)}\\
\hline
0 & 0 & 0\\\hline
0 & 0 & 1\\\hline
0 & 1 & 0\\\hline
0 & 1 & 1\\\hline
1 & 0 & 0\\\hline
1 & 0 & 1\\\hline
1 & 1 & 0\\\hline
1 & 1 & 1\\\hline
\end{tabular}
\bigskip

\begin{itemize}
  \item Show a global execution order that is not possible under sequential consistency, 
        but that is possible under a relaxed consistency model. 
\end{itemize}

\newpage
\begin{acsolution}\end{acsolution}

The solution to this exercise consists of drawing a dependence graph with the program order
 and for each case analyse the ordering relationships.

Case 0,0,0:

\begin{lstlisting}
P2: Print(X)
P2: Print(Y)
P3: Y=1
P3: Print(X)
P1: X=1
\end{lstlisting}

Case 0,0,1

\begin{lstlisting}
P2: Print(X)
P2: Print(Y)
P3: Y=1
P1: X=1
P3: Print(X)
\end{lstlisting}

Case 0,1,0

\begin{lstlisting}
P2: Print(X)
P3: Y=1
P2: Print(Y)
P3: Print(X)
P1: X=1
\end{lstlisting}

Case 0,1,1

\begin{lstlisting}
P2: Print(X)
P3: Y=1
P2: Print(Y)
P1: X=1
P3: Print(X)
\end{lstlisting}

Case 1,0,0: {\color{red}not possible}

Case 1,0,1:

\begin{lstlisting}
P1: X=1
P2: Print(X)
P2: Print(Y)
P3: Y=1
P3: Print(X)
\end{lstlisting}

Case 1,1,0

\begin{lstlisting}
P3: Y=1
P3: Print(X)
P1: X=1
P2: Print(X)
P2: Print(Y)
\end{lstlisting}

Case 1,1,1

\begin{lstlisting}
P1: X=1
P3: Y=1
P2: Print(X)
P2: Print(Y)
P3: Print(X)
\end{lstlisting}

Therefore, the only option that does not occur with sequential consistency is the 1.0.0 combination.

However, with a relaxed consistency by eliminating the program order requirement, we could have:

\begin{lstlisting}
P2: Print(Y)
P3: Y=1
P2: Print(X)
P3: Print(X)
P1: X=1
\end{lstlisting}
