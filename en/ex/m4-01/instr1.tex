\begin{acexercise}\end{acexercise}
\label{ex:m4-01:instr-01}

Given the following code fragment:

\lstinputlisting[language=generalasm2]{int/ex/m4-01/instr1-code-question.asm}

Assume a 5 stages pipeline.
For branch instructions both branch target address and branch outcome are known
at the end of instruction decode stage. All branches are predicted to not taken.

Answer the following questions:
\begin{enumerate}
  \item Identify the RAW hazards in the given code.

  \item Provide a timing table (chronogram) when there is no forwarding hardware.
        Determine how many cycles per iteration are required assuming that the loop
        will be executed during 100 iterations.

  \item Provide a timing table (chronogram) when there is forwarding hardware.
        Determine how many cycles per iteration are required assuming that the loop
        will be executed during 100 iterations.

  \item Provide a timing table (chronogram) assuming now that all cache accesses
        require 2 cycles for reading and 3 cycles for writing, and that instruction
        cache read accesses always require two cycles. Evaluate the cases with and
        without forwarding.
\end{enumerate}


\begin{acsolution}\end{acsolution}

Numbering the instructions, we get:

\lstinputlisting[language=generalasm2]{int/ex/m4-01/instr1-code-num.asm}

\paragraph{Question 1}: 
We find the following RAW hazards:

\begin{enumerate}
  \item \asmreg{t1}: \asmlabel{I1} $\rightarrow$ \asmlabel{I3}
  \item \asmreg{t2}: \asmlabel{I2} $\rightarrow$ \asmlabel{I3}
  \item \asmreg{t2}: \asmlabel{I3} $\rightarrow$ \asmlabel{I4}
  \item \asmreg{s0}: \asmlabel{I5} $\rightarrow$ \asmlabel{I6}
\end{enumerate}

\paragraph{Question 2}: Chronogram is presented in table~\ref{ex:m4-01:instr-01:chrono:noforward}.

\begin{table}[htb]
\begin{tabular}{|l|l|*{14}{>{\footnotesize}c|}}
\hline
\# &
\textbf{Instr.} &
1 & 2 & 3 & 4 & 5 &
6 & 7 & 8 & 9 & 10 &
11 & 12 & 13 & 14 
\\
\hline
\hline

I1 &
\asminst{lw} \asmreg{t1}, \asmlabel{1000}(\asmreg{s0}) &
IF & ID & EX & M & WB 
\\
\hline

I2 &
\asminst{lw} \asmreg{t2}, \asmlabel{2000}(\asmreg{s0}) &
& 
IF & ID & EX & M & WB 
\\
\hline

I3 &
\asminst{add} \asmreg{t2}, \asmreg{t1}, \asmreg{t2} &
& &
IF & ID & -- & 
-- & EX & M & WB
\\
\hline

I4 &
\asminst{sw} \asmreg{t2}, 3000(\asmreg{s0}) &
& & & & &
IF & ID & -- & -- & EX & M & WB
\\
\hline

I5 &
\asminst{addi} \asmreg{s0}, \asmreg{s0}, 4 &
& & & & &
& IF & -- & -- & ID & EX & M & WB
\\
\hline

I6 &
\asminst{bne} \asmreg{s0}, \asmreg{s1}, \asmlabel{loop} &
& & & & &
& & & & IF & ID & -- & -- & EX
\\
\hline

I7 &
\asmlabel{instr} &
& & & & &
& & & &
& IF & -- & -- & ID/--
\\
\hline

I1 &
\asminst{lw} \asmreg{t1}, \asmlabel{1000}(\asmreg{s0}) &
& & & & &
& & & & &
& & & IF
\\
\hline
\end{tabular}

\caption{Time diagram for exercise~\ref{ex:m4-01:instr-01} without forwarding.}
\label{ex:m4-01:instr-01:chrono:noforward}
\end{table}

To determine the number of cycles per interation, we compute the number of cycles
from the first instruction in an iteration to the first instruction in next iteration.

For all iterations except the las one, this will be untile instruction \asmlabel{I1}
is fetched. For the last iteration, it will be until instruction \asmlabel{I7} 
is fetched.

\[
cycles_{iteration} = \frac{10 + 13 \cdot 99}{100} = \frac{1297}{100} = 12.97
\]

\paragraph{Question 3}: 
The chronogram is presented in table~\ref{ex:m4-01:instr-01:chrono:forward}.

\begin{table}[htb]
\begin{tabular}{|l|l|*{12}{>{\footnotesize}c|}}
\hline
\# &
\textbf{Instr.} &
1 & 2 & 3 & 4 & 5 &
6 & 7 & 8 & 9 & 10 &
11 & 12 
\\
\hline
\hline

I1 &
\asminst{lw} \asmreg{t1}, \asmlabel{1000}(\asmreg{s0}) &
IF & ID & EX & M & WB 
\\
\hline

I2 &
\asminst{lw} \asmreg{t2}, \asmlabel{2000}(\asmreg{s0}) &
& IF & ID & EX & M & WB 
\\
\hline

I3 &
\asminst{add} \asmreg{t2}, \asmreg{t1}, \asmreg{t2} &
& 
& IF & ID & -- & EX & M & WB
\\
\hline

I4 &
\asminst{sw} \asmreg{t2}, 3000(\asmreg{s0}) &
& & &
IF & -- & ID & EX & M & WB
\\
\hline

I5 &
\asminst{addi} \asmreg{s0}, \asmreg{s0}, 4 &
& & & &
& IF & ID & EX & M & WB
\\
\hline

I6 &
\asminst{bne} \asmreg{s0}, \asmreg{s1}, \asmlabel{loop} &
& & & & &
& IF & ID & -- & EX & M & WB 
\\
\hline

I7 &
\asmlabel{instr} &
& & & & & &
& IF & -- & ID/--
\\
\hline

I1 &
\asminst{lw} \asmreg{t1}, \asmlabel{1000}(\asmreg{s0}) &
& & & & & & &
& & IF
\\
\hline
\end{tabular}

\caption{Time diagram for exercise~\ref{ex:m4-01:instr-01} with forwarding.}
\label{ex:m4-01:instr-01:chrono:forward}
\end{table}

\[
cycles_{iteration} = \frac{7 + 9 \cdot 99}{100} = \frac{898}{100} = 8.98
\]

Consequently, the speedup due to forwarding is:

\[
S = \frac{12.97}{8.98} = 1.44
\]

\paragraph{Question 4}: 
In this case, time diagrams from previous sections must be repeated with the
following modifications:
\begin{itemize}
  \item Two cycles must always be used for \textemph{IF}.
  \item Stage \textemph{M} for instructions no performing memory accesses will take one cycle.
  \item Stage \textemph{M} for load instructions (like \asminst{lw}) will take always two cycles.
  \item State \textemph{M} for store instructions (like \asminst{sw}) will take always three cycles.
\end{itemize}
