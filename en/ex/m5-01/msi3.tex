\begin{acexercise}
\label{ex:m5-02:dir-01}
\end{acexercise}

Given the following code fragments that execute 3 threads using the
MSI consistency protocol.


\begin{lstlisting}
// Thread 0 code
for(i=0;i<16;i++){
  a[i]= 16;
}
 
// Thread 1 code
tmp=0;
for(i=0;i<16;i++){
  a[i]=tmp;
  tmp+=a[i];
}
 
// Thread 2 code
cnt=0;
for(i=0;i<4;i++){
  cnt+=b[i];
}
\end{lstlisting}

The computer has a CC-NUMA architecture with:

\begin{itemize}
  \item 2 32-bit processors with a single cache memory level. 
        Cache memory is private to each processor and has a direct-mapped correspondence. The cache line size is 16 bytes.

   \item All cache memories are initially empty.

   \item Threads 0 and 2 are executed on processor 0 and thread 1 runs on processor 1.

   \item Variables \cppid{i}, \cppid{tmp} and \cppid{cnt} 
         are allocated to registers (they are not stored in memory).     

   \item The block containing \cppid{a[0]} is associated with the same cache line 
         than the block containing \cppid {b[0]}.
\end{itemize}

Complete the following tables and justify the answer for the following sections:

\begin{enumerate}

\item Starting from the initial situation, indicate in the following table the
state transitions of the block that stores \cppid{a[0]} when
thread 0 is executed before thread 1. 
Include bus traffic associated with each thread.
 
  \begin{itemize}
    \item \textbf{Note}: If there are several transitions associated with a thread,
          indicate all of them.
  \end{itemize}

\begin{tabular}{|l|l|l|l|}
\hline
Code & Transition P0 & Transition P1 & Bus signals\\
\hline\hline
Thread 0 & & & \\
\hline
Thread 1 & & & \\
\hline
\end{tabular}

\item Starting from the initial situation, indicate in the following table the
state transitions of the block that stores \cppid{a[0]} when the
thread 1 is executed before the thread 0. Include the bus traffic
associated with each thread.
 

  \begin{itemize}
    \item \textbf{Note}: If there are several transitions associated with a thread,
          indicate all of them.
  \end{itemize}

\begin{tabular}{|l|l|l|l|}
\hline
Code & Transition P0 & Transition P1 & Bus signals\\
\hline\hline
Thread 0 & & & \\
\hline
Thread 1 & & & \\
\hline
\end{tabular}

\item Starting from the initial situation, indicate in the following table the
state transitions of the block that stores \cppid{a[0]} when 
thread 0 is executed, then  thread 2 and finally  thread 1. Include the bus traffic
associated with each thread.

  \begin{itemize}
    \item \textbf{Note}: If there are several transitions associated with a thread,
          indicate all of them.
  \end{itemize}

\begin{tabular}{|l|l|l|l|}
\hline
Code & Transition P0 & Transition P1 & Bus signals\\
\hline\hline
Thread 0 & & & \\
\hline
Thread 1 & & & \\
\hline
Thread 2 & & & \\
\hline
\end{tabular}

\item For each of the above scenarios, indicate the number of existing cache misses for each process.

\begin{tabular}{|l|l|l|}
\hline
Scenario & P0 cache misses & P1 cache misses\\
\hline
\hline
Thread 0 $\rightarrow$ Thread 1 & & \\		
\hline
Thread 1 $\rightarrow$ Thread 0 & & \\	
\hline
Thread 0 $\rightarrow$ Thread 2 $\rightarrow$ Thread 1 &&\\
\hline
\end{tabular}

\end{enumerate}


\begin{acsolution}\end{acsolution}

\paragraph{Section 1}

Table~\ref{tab:ex:m5-02:dir-01:a} shows the transitions. 

\begin{table}[htbp]

\begin{tabular}{|l|l|l|l|}

\hline
Code & Transition P0 & Transition P1 & Bus signals\\
\hline
\hline

Thread 0	&
I $\rightarrow$ E & I & Write miss
\\
\hline

Thread 1 &
E $\rightarrow$ I & I $\rightarrow$ E , E $\rightarrow$ E &
Write miss; Write-back block
\\
\hline

\end{tabular}

\caption{Transtions for section 1, exercise~\ref{ex:m5-02:dir-01}}
\label{tab:ex:m5-02:dir-01:a}
\end{table}

\paragraph{Section 2}

Table~\ref{tab:ex:m5-02:dir-01:b}  shows the transitions. 

\begin{table}[htbp]

\begin{tabular}{|l|l|l|l|}

\hline
Code & Transition P0 & Transition P1 & Bus signals\\
\hline
\hline

Thread 1 &
I &  I $\rightarrow$ E ; E $\rightarrow$ E & Write miss
\\
\hline

Thread 0 &
I $\rightarrow$ E  & E $\rightarrow$ I & Write miss; Write-back block
\\
\hline

\end{tabular}

\caption{Transtions for section 2, exercise~\ref{ex:m5-02:dir-01}}
\label{tab:ex:m5-02:dir-01:b}
\end{table}

\paragraph{Section 3}

Table~\ref{tab:ex:m5-02:dir-01:c}  shows the transitions. 

Each block has 4 words. So the Threads 0 and 1 access to 4 blocks and
while Thread 2 is accessing to the block in Thread 1 the first line causes
a  cache miss while the second, a cache hit.


\begin{table}[htbp]

\begin{tabular}{|l|l|l|l|}

\hline
Code & Transition P0 & Transition P1 & Bus signals\\
\hline
\hline

Thread 0 &
I $\rightarrow$ E (stores \cppid{a[0]}) & I &
Write miss
\\
\hline

Thread 2 & 
E $\rightarrow$ S (stores \cppid{b[0]}) & I &
Read miss; write back block
\\
\hline

Thread 1 &
S $\rightarrow$ S (\cppid{b[0]}) &
I $\rightarrow$ E ; E $\rightarrow$ E &
Write miss
\\
\hline

\end{tabular}

\caption{Transtions for section 3, exercise~\ref{ex:m5-02:dir-01}}
\label{tab:ex:m5-02:dir-01:c}
\end{table}

\paragraph{Section 4}

Table~\ref{tab:ex:m5-02:dir-01:d}  shows the transitions. 

\begin{table}[htbp]

\begin{tabular}{|l|l|l|}

\hline
Escenario & Cache misses P0 & Cache misses P1\\
\hline
\hline

Thread 0 $\rightarrow$ Thread 1 &
4 & 4
\\
\hline

Thread 1 $\rightarrow$ Thread 0 &
4 & 4
\\
\hline

Thread 0 $\rightarrow$ Thread 2 -> Thread 1 &
4 + 1 & 4
\\
\hline

\end{tabular}

\caption{Transtions for section 4, exercise~\ref{ex:m5-02:dir-01}}
\label{tab:ex:m5-02:dir-01:d}
\end{table}

