\begin{acexercise}\end{acexercise}

Consider a large data center with many servers. The data center operates most of 
the time at 50\% of its capacity. However, 10\% of the time it operates at its
full capacity. When the servers operate at 50\% of its capacity 90\% of the power
is consumed. Server nodes can be put into a suspend mode where they only use
20\% of the power.

Consider the following questions:

\begin{enumerate}

\item How much would be the power saving by suspending 50\% of the servers
compared to turning off the servers.

\item How much power saving can be obtained by decreasing voltage 10\% and 
frequency by 25\%.

\item How much would be the power saving by suspending 25\% of the servers
and turning off 25\% of the servers

\end{enumerate}

\begin{acsolution}\end{acsolution}

\begin{enumerate}

\item It is relevant to consider that turning off or suspending servers only
makes sense when the data center is operating at 50\% of its full capacity.

When 50\% of servers is turned off, we transition from a situation where 100\%
of servers are using 90\% of power to another situation in which 50\% of servers
are using 100\% of power.

\[P_{old} = 1 \cdot 0.9 = 0.9\]
\[P_{new} = 0.5 \]

If we consider the power ratio, we get:

\[\frac{P_{new}}{P_{old}} = \frac{0.5}{0.9} = 0.56\]

Thus, we need 56\% of the original power.

When 50\% of servers is suspended, we transition from a situation where 100\% of
servers are using 90\% of power to another situation in which 50\% of servers
are using 100\% of power and the other 50\% of servers are using 20\% of power.

\[P_{old} = 1 \cdot 0.9 = 0.9\]
\[P_{new} = 0.5 + 0.5 \cdot 0.2 = 0.6 \]

If we consider the power ratio, we get:

\[\frac{P_{new}}{P_{old}} = \frac{0.6}{0.9} = 0.67\]

\item We will consider the power ratio:

\[
\frac{P_{new}}{P_{old}} =
\frac{(0.9 \cdot V)^2 0.75 \cdot f}{V^2 \cdot f} =
0.9^2 \cdot 0.75 =
0.6075
\]

\item 25\% of the servers are turned off while other 25\% need 20\% of
its original energy.

\[
0.5 + 0.25 \cdot 0.2 = 0.55
\]

\end{enumerate}

