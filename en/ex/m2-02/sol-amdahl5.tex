\input{en/m2-ex-01/ex-jan-2014}

\begin{comparchsol}
\end{comparchsol}

Time for executing instructions in the original computer will be:

\begin{equation}
T_{orig} = 
0.75 \times 12 \times IC \times P + 0.25 \times 4 \times IC \times P = 
(9+1) \times IC \times P
\end{equation}

\paragraph{Alternative A}

Time for executing instructions in computer A will be:

\begin{equation}
T_{A} =
(0.75 \times (1,1 \times 12) + 0.25 \times (1.25 \times 4)) \times IC \times \frac{P}{1.5} = 
\frac{(9.9 + 1.25) \times IC  \times P}{1.5} = 
\frac{11.15}{1.5} \times IC * P 
\end{equation}

\begin{equation}
T_{A} = 7.433 \times IC \times P
\end{equation}

Speedup due to instructions will be:

\begin{equation}
S_{A}^{I} = 
\frac{T_{orig}}{T_{A}} = 
\frac{10}{7.433} = 
1.345
\end{equation}

Applying Amdahl's law, the global speedup will be:

\begin{equation}
S_{A} = \frac{1}{0.1 + \frac{0.9}{1.345}} = 1.3
\end{equation}

\paragraph{Alternative B}

In this case, assuming complete parallelization for the computing part,
we may consider the number of instructions to be executed in each core is
one fourth from the original.

\begin{equation}
T_{B} = 
(0.75 \times 0.8 \times 12 + 0.25 \times 4) \times \frac{IC}{4} \times \frac{P}{0.5} = 
( 7.2 + 1) \times \frac{2}{4} \times IC \times P = 
\end{equation}

\begin{equation}
T_{B} = 4.1 \times IC \times P
\end{equation}

The speedup due to instructions will be:

\begin{equation}
S^{I}_{B} = \frac{T_{orig}}{T_{B}} = \frac{10}{4.1} = 2.439
\end{equation}

Applying Amdahl's law, the global speedup will be:

\begin{equation}
S_{B} = \frac{1}{0.1 + \frac{0.9}{2.439}} = 2.132
\end{equation}

