\begin{acexercise}\end{acexercise}

An application for generating 3D video is run currently in a computer with 4
cores. The application has a portion that is not parallelized and takes 25\% of
total execution time. The application spends 50\% of the total time in memory
processing and is totally parallelized. The remaining 25\% of the total time is
dedicated to read/write operations on disks.

Two options are being considered.

\begin{itemize}
\item Replacing the current disks for a new storage technology that requires one fourth of the time for input/output.
\item Replace the processor by a processor with 32 cores.
\end{itemize}

You are asked:

\begin{enumerate}
\item Determine the speedup obtained when replacing the disks.
\item Determine the speedup when replacing the processor.
\item Determine the number of cores that would give the same speedup than replacing the disks.
\end{enumerate}


\begin{acsolution}\end{acsolution}

\paragraph{Section 1}
If the disk is replaced, the improvement fraction $F$ is $0.25$.
The improvement speedup $S(m)$ will be $4$.
Consequently, applying Amdahl's Law, we get:

\[
S = \frac{1}{0.75 + \frac{0.25}{4}} = \frac{1}{0.8125} =  1.2308
\]

\paragraph{Section 2}
If the processor is replaced, the improvement speedup is given by the ratio
between the ne number of cores $n_n$ and the old number of colres $n_o$.

\[
S_m = \frac{n_n}{n_o} = \frac{32}{4} = 8
\]

The improvement fraction $F$ is in this case $0.5$.
\[
S = \frac{1}{0.5 + \frac{0.5}{8}} = \frac{1}{0.5 + 0.0625} = \frac{1}{0.5625} = 1.7778
\]

\paragraph{Section 3}
The improvement speedup for a given number of cores $n$, is:

\[
S_i = \frac{n}{4}
\]

Consequently, the total speedup is:

\[
S = \frac{1}{0.5 + 
\frac{0.5}{\frac{n}{4}}} = 
\frac{1}{0.5 + \frac{2}{n}}
\]

If we make this value equal to $\frac{1}{0.8125}$, we get:
\[
\frac{1}{0.8125} = \frac{1}{0.5+\frac{2}{n}}
\]
\[ 0.8125 = 0.5 + \frac{2}{n} \]
\[ \frac{2}{n} = 0.8125 - 0.5 \]
\[ n = \frac{2}{0.3125} = 6.4 \]

Consequently a minimum of 7 cores are needed.
