\begin{acexercise}\end{acexercise}

A RISC-V like processor has an integer register file with 32 registers.
It also has another register file with 32 floating point registers.

Assume that there is enough fetch and decode bandwidth so that no stalls are generated
due to this. A new instruction execution can be initiated in each cycle,
except whe there are stalls due to data dependency.

The following table shows the additional latencies for some instructions categories,
when there is a data dependency. If there is no dependency this additional latency
is not applicable.

\begin{table}[htb]
\begin{tabular}{|l|r|}

\hline
\textbf{Instruction} &
\textbf{Additional latency}
\\
\hline

\asminst{fld} & +2\\
\hline

\asminst{fsd} & +2\\
\hline

\asminst{fadd.d} & +4\\
\hline

\asminst{fmul.d} & +6\\
\hline

\asminst{addi} & +0\\
\hline

\asminst{subi} & +0\\
\hline

\asminst{bnez} & +1\\
\hline

\end{tabular}

\end{table}

Instruction \asminst{bnez} uses delayed branch with one delay slot.

The following code fragment will be executed in this machine:

\lstinputlisting[language=generalasm2]{int/ex/m4-02/unroll-2-ex.asm}

Initially registers have the following values:

\begin{itemize}
  \item \asmreg{t1}: \asmlabel{0x00100000}.
  \item \asmreg{t2}: \asmlabel{0x00140000}.
  \item \asmreg{t3}: \asmlabel{0x00180000}.
  \item \asmreg{t4}: \asmlabel{0x00000100}.
\end{itemize}

Answer the following questions:

\begin{enumerate}

\item How many iterations does the loop execute?

\item Identify the RAW dependencies in the code.

\item Identify all stalls happening when the code is executed.
      Compute the number of cycles per iteration and the total number of cycles
      for all iterations.

\item Modify the loop using instruction scheduling.
      Compute the number of cycles per iteration and the total number of cycles
      for all iterations.

\item Apply loop unrolling processing two elements per iteration and using
      instruction scheduling.
      Compute the number of cycles per iteration and the total number of cycles
      for all iterations.
      Determine the speedup.

\item If we wanted to apply a loop unrolling factor of four elements per iteration,
      what complication migh we find?

\end{enumerate}

\begin{acsolution}\end{acsolution}

\paragraph{Number of iterations}.
The loop is repeated until register \asmreg{t4} reaches value \asmlabel{0}. 
That register is decremented one unit per iteration.
Initial value for that register is $2^8 = 256$. 
Consequently, \asmlabel{256} iterations are executed.

\paragraph{RAW Dependencies}.
The following dependencies are identified in the original fragment:

\begin{enumerate}

\item \asmreg{ft0}: \asmlabel{I1} $\rightarrow$ \asmlabel{I3}.
\item \asmreg{ft1}: \asmlabel{I2} $\rightarrow$ \asmlabel{I3}.
\item \asmreg{ft2}: \asmlabel{I3} $\rightarrow$ \asmlabel{I4}.
\item \asmreg{ft2}: \asmlabel{I4} $\rightarrow$ \asmlabel{I5}.
\item \asmreg{t4}: \asmlabel{I8} $\rightarrow$ \asmlabel{I9}.
\end{enumerate}

\paragraph{Original fragment execution}.

\lstinputlisting[language=generalasm2]{int/ex/m4-02/unroll-2-stalls.asm}

In total, 22 cycles per iteration are required.

To execute 256 iterations, we require:

\[
cycles = 256 \times 22 = 5632
\]

\paragraph{Loop scheduling}.
Reordering instructions we get:

\lstinputlisting[language=generalasm2]{int/ex/m4-02/unroll-2-sched.asm}

In total, 19 cycles are required.

To execute 256 iterations, we require:

\[
cycles = 256 \times 19 = 4864
\]
\[
S = \frac{5632}{4864} = 1.16
\]

\paragraph{Loop unrolling}

\lstinputlisting[language=generalasm2]{int/ex/m4-02/unroll-2-unrollx2.asm}

In total, 21 cycles are required for every two elements.

To execute the 128 needed iterations (2 elements per iteration) we require:

\[
cycles = 128 \times 21 = 2688
\]
\[
S = \frac{5632}{2688} = 2.1
\]

\paragraph{Loop unrolling with factor 4}

In this case a higher number of floating point registers are needed and
temporal registers (\asmreg{ft}) would not be enough.
Thus, some other register would have to be saved on the stack before using.
Later, that register would have to be restored.
