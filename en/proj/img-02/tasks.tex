\section{Tasks}

\subsection{Software development}

You will parallelize concrete functionalities from the given project.

In particular, you must parallelize the following functionalities:

\begin{itemize}
  \item Gray-scale transformation
    \begin{itemize}
      \item Function \cppid{bitmap\_aos::to\_gray()}.
      \item Function \cppid{bitmap\_soa::to\_gray()}.
    \end{itemize}

  \item Histogram generation:
    \begin{itemize}
      \item Function \cppid{bitmap\_aos::generate\_histogram()}.
      \item Function \cppid{bitmap\_soa::generate\_histogram()}
    \end{itemize}

  \item Gaussian blur:
    \begin{itemize}
      \item Function \cppid{bitmap\_aos::gauss()}.
      \item Function \cppid{bitmap\_soa::gauss()}
    \end{itemize}
\end{itemize}

Remind that you must perform all the evaluations with compiler optimizations
enabled with the CMake option
\cppid{-DCMAKE\_BUILD\_TYPE=Release}.

\subsubsection{Code quality rules}

Rules from previous project shall be followed.

\textbad{VERY IMPORTANT}: 
The source code shall not depend in any manner on the concrete number of threads or
the scheduling policy. Those parameters shall be controlled from the corresponding
environment variables that are external to the program.

\subsubsection{Unit tests}

A set of unit tests are provided with the source code.
Your time shall check that all unit tests are successful after modifying the code.

\subsection{Performance and energy evaluation}

In this task you will perform a comparative evaluation of performance and energy
for the two implementation strategies \cppid{image-aos} and \cppid{image-soa}.

For each case you must perform evaluations varying the following parameters:
\begin{itemize}
  \item \textmark{Number of threads}: 
        You shall evaluate the impact for number of threads considering values 1, 2, 4, 8, and 16.
        You can control the number of threads with the environment variable
        \cppid{OMP\_NUM\_THREADS}.

  \item \textmark{Scheduling policy}: 
        You shall consider all the available scheduling policies in OpenMP.
        You can control the scheduling policy with the environment variable
        \cppid{OMP\_SCHEDULE}.
      
\end{itemize}

To conduct the performance evaluation, the total execution time shall be measured.
In addition, energy use must be also measured.
The power use must be derived.

All performance evaluations shall be performed in a node from the
\cppid{avignon} cluster.

Represent in a graphic all total execution times, energy use and power
for the provided image \cppkey{sabatini.bmp}.

\textbf{The project report shall include conclusions drawn from results}.
Please, do not limit to simply describing data.
Try to find convincing explanations of those results.
