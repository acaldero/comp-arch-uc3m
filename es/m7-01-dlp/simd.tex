\section{Extensiones SIMD para multimedia}

\begin{frame}[t]{SIMD en procesadores escalares}
\begin{itemize}
  \item Operaciones estrechas en procesadores de 32 bits:
    \begin{itemize}
      \item Colores de 8-bits en imágenes.
      \item Muestras de audio de 8 o 16 bits.
    \end{itemize}

  \mode<presentation>{\vfill\pause}
  \item Uso de unidad aritmética de 256 bits y partición de acarreo.
    \begin{itemize}
      \item 32 operandos de 8-bit.
      \item 16 operandos de 16-bit.
      \item 8 operandos de 32-bit.
      \item 4 operandos de 64-bit.
    \end{itemize}

  \mode<presentation>{\vfill\pause}
  \item Instrucciones SIMD operan sobre pequeños vectores de datos.
\end{itemize}
\end{frame}

\begin{frame}[t]{Extensiones SIMD frente a procesadores vectoriales}
\begin{itemize}

  \mode<presentation>{\pause}
  \item Sin registro de longitud de vector.
    \begin{itemize}
      \item Diferentes versiones de intrucciones para cada longitud.
      \item Más códigos de operación.
    \end{itemize}

  \mode<presentation>{\vfill\pause}
  \item Sin transferencias de datos con huecos (\emph{strided}, \emph{gather/scatter}).
    \begin{itemize}
      \item Limita los programas que se pueden vectorizar.
      \item Recientemente añadidas.
    \end{itemize}

  \mode<presentation>{\vfill\pause}
  \item Sin registros de máscaras.
    \begin{itemize}
      \item Limita soporte de ejecución condicional.
      \item Recientemente añadidas.
    \end{itemize}

  \mode<presentation>{\vfill\pause}
  \item Código SIMD difícil de generar.
\end{itemize}
\end{frame}

\begin{frame}[t]{Extensiones SIMD en x86}
\begin{itemize}
  \item MMX (MultiMedia eXtensions) -- 1996.
    \begin{itemize}
      \item Coma flotante de 64-bit $\rightarrow$ 
            8 operciones de 8 bits o 4 operaciones de 16 bits.
    \end{itemize}

  \mode<presentation>{\vfill\pause}
  \item SSE (Streaming SIMD Extensions) -- 1999.
    \begin{itemize}
      \item 16 registros de 128-bit $\rightarrow$ registros XMM.
      \item 16 operaciones de 8-bit, 8 operaciones de 16-bit, o 4 operaciones de 32-bit.
      \item Coma flotante de simple precisión paralela.
    \end{itemize}

  \mode<presentation>{\vfill\pause}
  \item SSE2 (2001), SSE3 (2004), SSE4 (2007).
    \begin{itemize}
      \item Mas operaciones de coma flotante.
    \end{itemize}

\end{itemize}
\end{frame}

\begin{frame}[t]{AVX}
\begin{itemize}
  \item AVX (Advanced Vector eXtensions) -- 2010.
    \begin{itemize}
      \item Duplica longitud de vector.
      \item Registros de 256-bit $\rightarrow$ registros YMM.
      \item 4 operaciones de coma flotante de doble precisión.
    \end{itemize}

  \mode<presentation>{\vfill\pause}
  \item AVX2 -- 2013
    \begin{itemize}
      \item Añade instrucción de gather (\asminst{VGATHER}).
      \item Instrucciones de desplazamiento vectorial
            (\asminst{VPSLL}, \asminst{VPSRL}, \asminst{VPSRA}).
    \end{itemize}

  \mode<presentation>{\vfill\pause}
  \item AVX512 -- 2017.
    \begin{itemize}
      \item Duplica longitud de vector.
      \item Registros de 512-bit $\rightarrow$ registros ZMM.
      \item 8 operaciones de coma flotante de doble precisión.
      \item 250 nuevas instrucciones (\asminst{VPSCATTER}, \asminst{OPMASK}).
    \end{itemize}
\end{itemize}
\end{frame}

\begin{frame}[t]{Evolución de extensiones SIMD}
\begin{itemize}
  \item Enfoque incremental a lo largo del tiempo.
    \begin{itemize}
      \item Centrado en permitir la escritura de bibliotecas.
      \item Pero los compiladores están mejorando la autovectorización (mejora limitada).
      \item Los códigos de operación dependen de la longitud del vector
            $\Rightarrow$ 
            Duplciar códigos de operación con cada nueva longitud.
    \end{itemize}

  \mode<presentation>{\vfill\pause}
  \item ¿Por qué son más populares la extensiones SIMD que las arquitecturas vectoriales?
    \begin{enumerate}
      \item Bajo coste de añadir una unidad aritmética estándar.
      \item Pequeño estado de procesador extra $\rightarrow$ Cambios de contexto más simples.
      \item Se necesita ancho de banda más bajo.
      \item Longitud fija pequeña $\rightarrow$ Interacciones más simples con memoria virtual.
    \end{enumerate}
\end{itemize}
\end{frame}

\begin{frame}[t]{Actividad}
\begin{enumerate}
  \item \textemph{Lee} la \textmark{sección 4.3} --
        \emph{SIMD Instruction Set Extensions for Multimedia}.
    \begin{itemize}
      \item \textbad{Solamente} \textmark{The Roofline Visual Performance Model}
            (páginas 307--310).
      \item \credithennessy
    \end{itemize}

  \mode<presentation>{\vfill\pause}
  \item \textgood{Aspectos clave}:
    \begin{itemize}
      \item ¿Cómo se mide la intensidad aritmética?
      \item ¿Como se mide el rendimiento máximo de coma flotante?
      \item ¿Cuál es el límite superior de rendimiento para intensidad aritmética baja?
      \item ¿Cuál es el límite superior de rendimiento para intensidad aritmética alta?
    \end{itemize}
\end{enumerate}
\end{frame}
