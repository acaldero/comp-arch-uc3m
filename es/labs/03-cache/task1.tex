\clearpage
\subsection{Tarea 1: Fusión de bucles}

En esta tarea se pretende analizar dos programas: \cppid{loop\_merge.cpp} y \cppid{loop\_merge\_opt.cpp}.

\lstinputlisting[caption={loop\_merge.cpp},frame=single,numbers=left,basicstyle=\small]{lab/03-cache/loop_merge.cpp}
\lstinputlisting[caption={loop\_merge\_opt.cpp},frame=single,numbers=left,basicstyle=\small]{lab/03-cache/loop_merge_opt.cpp}

Ambos programas implementan la misma funcionalidad:
dados dos vectores $\vec{z}$ y $\vec{t}$, calculan otros dos vectores $\vec{u}$ y $\vec{v}$:

\[
\vec{u} = \vec{z} + \vec{t}
\]
\[
\vec{v} = \vec{u} + \vec{t}
\]

Para ello, los programas hacen uso de 4 arrays de tamaño fijo.
El programa no imprime ningún resultado.

Se pide: 

\begin{enumerate}

\item Ejecute \cppid{loop\_merge} y \cppid{loop\_merge\_opt} con el programa
\textmark{valgrind} y la herramienta \textmark{cachegrind} para las
siguientes configuraciones:

\begin{itemize}
\item Caché de último nivel fijada a 128 KiB.
\item Evalúe con tamaños de caché L1D de 16 KiB, 32 KiB y 64 KiB.
\end{itemize}

\item Observe los resultados obtenidos e inspeccione el código con la
herramienta \textmark{cg\_annotate}. Anote los resultados globales y observe los
resultados prestando especial atención al cuerpo de los bucles.

\item Compare ambos resultados.
Estudie los resultados para Dr, D1mr, DLmr, Dw,
D1mw y DLmw.
 
\end{enumerate}

