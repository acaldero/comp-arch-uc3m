\section{Descripción de la práctica}

En esta práctica se realizará la evaluación de la ejecución de diversos
programas en C++ con la herramienta \textmark{cachegrind}, que forma parte de
las herramientas disponibles dentro de \textmark{valgrind}
(\textgood{\url{http://valgrind.org/}}).  La herramienta es capaz de simular
la ejecución de un programa y el impacto que tiene la memoria caché sobre él. 

Si el programa evaluado está compilado con las opciones de depuración
\textmark{cachegrind} puede además
indicar cuál es la utilización de la memoria caché con nivel de detalle de
línea de código fuente.  Para ello, se debe generar primero el código
ejecutable mediante el comando:

\begin{lstlisting}[style=terminal,aboveskip=1em,belowskip=1em]
cmake -S . -B debug -DCMAKE_BUILD_TYPE=Debug
cmake --build debug
\end{lstlisting}

\fbox{ \parbox{.9\textwidth}{
\textbad{ATENCIÓN}: Recuerde que para ejecutar trabajos en el clúster
\cppid{avignon} debe primero crear un \emph{script} para lanzar
la compilación.
}}
\vspace{1em}

De este modo, se puede obtener la información de ejecución del programa
utilizando \textemph{valgrind} y \textemph{cachegrind}. Si quiere analizar
un ejecutable llamado \cppid{./test}, puede utilizar el siguiente mandato.

\begin{lstlisting}[style=terminal,aboveskip=1em,belowskip=1em]
valgrind --tool=cachegrind ./test
\end{lstlisting}

\fbox{ \parbox{.9\textwidth}{
\textbad{ATENCIÓN}: Recuerde que para ejecutar trabajos en el clúster
\cppid{avignon} debe primero crear un \emph{script} para lanzar
la ejecución de \textemph{valgrind}.
}}
\vspace{1em}

Si no se le indica ningún parámetro adicional, \textmark{cachegrind} simula
la ejecución del código sobre la configuración real de memoria caché del
procesador de la máquina sobre la que se está ejecutando. Al finalizar la
ejecución, muestra estadísticas básicas sobre el uso de la memoria caché y
genera un fichero de nombre \cppid{cachegrind.out.<pid>} donde \cppid{pid} es
el identificador del proceso. Este fichero contiene información adicional que
puede ser analizada mediante la utilidad \cppid{cg\_annotate}. (El nombre del
fichero puede ser modificado mediante la opción
\cppid{--cachegrind-out-file=<file\_name>}).

Si se desea modificar la configuración de caché se deben utilizar las opciones:

\begin{lstlisting}[style=terminal,aboveskip=1em,belowskip=1em]
--I1=<tamaño>,<asociatividad>,<tamaño línea>
\end{lstlisting}

Especifica el tamaño (en bytes), la asociatividad (número de vías) y el tamaño
de línea (en bytes) de la memoria caché de instrucciones de nivel 1.

\begin{lstlisting}[style=terminal,aboveskip=1em,belowskip=1em]
--D1=<tamaño>,<asociatividad>,<tamaño línea>
\end{lstlisting}

Especifica el tamaño (en bytes), la asociatividad (número de vías) y el tamaño de
línea (en bytes) de la caché de datos de nivel 1.

\begin{lstlisting}[style=terminal,aboveskip=1em,belowskip=1em]
--LL=<tamaño>,<asociatividad>,<tamaño línea>
\end{lstlisting}

Especifica el tamaño (en bytes), la asociatividad (número de vías) y el tamaño de
línea (en bytes) de la caché de último nivel.

La herramienta \cppid{cg\_annotate} recibe por parámetro el fichero que se
desee analizar y la ruta de los ficheros que se quieren anotar línea por línea.
Si no se  sabe la ruta o cuáles son los ficheros de interés, se puede utilizar
la opción \cppid{--auto=yes} para que anote todos los ficheros que puedan ser de
interés.

\begin{lstlisting}[style=terminal,aboveskip=1em,belowskip=1em]
cg_annotate cachegrind.out.XXXXX --auto=yes
\end{lstlisting}

\fbox{ \parbox{.9\textwidth}{
\textbad{ATENCIÓN}: Recuerde que para ejecutar trabajos en el clúster
\cppid{avignon} debe primero crear un \emph{script} para lanzar
la ejecución de \textemph{cg\_annotate}.
}}
\vspace{1em}

Adicionalmente, si sólo se quiere anotar un fichero: 

\begin{lstlisting}[style=terminal,aboveskip=1em,belowskip=1em]
cg_annotate cachegrind.out.XXXXX <ruta absoluta al fichero .cpp>
\end{lstlisting}

Si se desea más información sobre el funcionamiento o la utilización de
\textmark{cachegrind} y \cppid{cg\_annotate}, se puede visitar la
documentación en la página web de Valgrind:
\textgood{\url{http://valgrind.org/docs/manual/cg-manual.html}}.
