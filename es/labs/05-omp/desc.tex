\section{Descripción del laboratorio}

En este laboratorio llevarás a cabo la evaluación al ejecutar diversas opciones
de un programa que realiza algunas operaciones sobre una imagen en formato bitmap.
El programa es capaz de convertir una imagen a escala de grises y también puede 
generar el histograma de la imagen.

El programa ejecutable se llama \cppid{imgtool} y toma tres parámetros:
\begin{itemize}
\item \textmark{operación}: Operación a aplicar sobre la imagen.
\item \textmark{entrada}: El nombre del archivo de entrada. Este será el archivo con la imagen en formato bitmap.
\item \textmark{salida}: El nombre de archivo de salida.
\end{itemize}

Hay cinco opciones disponibles

\begin{itemize}
\item \textmark{copy}: Copia el archivo de entrada al archivo de salida.
\item \textmark{grayscale}: Genera un archivo de salida en formato bitmap con la imagen en escala de grises.
\item \textmark{histogram}: Genera un archivo de texto con una representación textual del histograma de la imagen.
\item \textmark{par\_grayscale}: Genera un archivo de salida en formato bitmap con la imagen en escala de grises.
Esta opción realiza el cómputo en paralelo y actualmente no está implementada de forma correcta.
\item \textmark{par\_histogram}: Genera un archivo de texto con una representación textual del histograma de la imagen.
Esta opción realiza el cómputo en paralelo y actualmente no está implementada de forma correcta.
\end{itemize}

\subsection{Construcción del programa}

Por favor, recuerda que para compilar la versión optimizada del programa 
deberás generar una configuración de \emph{release}.

\begin{lstlisting}[style=terminal,aboveskip=1em,belowskip=1em]
cmake -S . -B release -DCMAKE_BUILD_TYPE=Release
\end{lstlisting}

Además, para compilar una determinada configuración, deberás invocar a 
\cppkey{cmake} con el correspondiente nombre de configuración:

\begin{lstlisting}[style=terminal,aboveskip=1em,belowskip=1em]
cmake --build release
\end{lstlisting}

\fbox{ \parbox{.9\textwidth}{
\textbad{REMINDER}: 
Para ejecutar trabajos en el clúster \cppid{avignon}, deberás crear un
\emph{script} para lanzar la compilación.
}}

\subsection{Estructura del código fuente}

El código fuente que se suministra tiene la siguiente estructura:


\begin{itemize}

\item \textmark{util}: 
Esta carpeta contiene la biblioteca de utilidades. Incluye algunas utilidades
básicas para entrada/salida binaria y procesamiento de argumentos de programa.

\item \textmark{img}: 
Esta carpeta contiene todo el código fuente para la biblioteca de manipulación
de imágenes.

\item \textmark{imgtool}: 
Esta carpeta contiene el programa principal y algunas funciones de ayuda.

\end{itemize}

En particular, la biblioteca de imágenes contiene diversos componentes software:

\begin{itemize}

\item \cppid{checking}: 
Contiene la gestión de mensajes de error y comprobaciones.

\item \cppid{pixel}: 
Representa un pixel de la imagen.

\item \cppid{normalized\_pixel}: 
Representación alternativa de un pixel utilizada durante la conversión a escala
de grises.

\item \cppid{image\_header}: 
La cabecera de un archivo de bitmap. Se utiliza para la entrada/salida.

\item \cppid{image\_metadata}: 
Metadatos de una image tal y como se leen de un archivo BMP.

\item \cppid{image}: 
Impelementa una imagen como una matriz bi-dimensional de pixels.

\item \cppid{parallel\_image}: 
Otra implementación de la imagen que tendrás que modificar para implementar la
versión paralela de los algoritmos.

\item \cppid{histogram}: 
Una representación de histogramas para los tres componentes de la imagen (rojo,
verde y azul).

\end{itemize}

Ten en cuenta que durante este laboratorio, tendrás que modificar el componente
\cppid{parallel\_image} para suministrar una implementación paralela de la
funcionalidad de conversión a escala de grises y cálculo del histograma.
