\section{Tareas}

Durante este laboratorio realizarás las siguientes tareas:

\begin{enumerate}

\item Construir la versión inicial de la aplicación y evaluar su rendimiento y energía.

\item Modificar el código fuente suministrado para implementar correctamente las
opciones paralelas de la herramienta.

\item Evaluar la versión paralela que hayas generado estudiando:

\begin{enumerate}
  \item El impacto del número de hilos desde 2 hasta el doble del número de
núcleos físicos disponibles.

  \item El impacto de diferentes políticas de planificación (opcional).
\end{enumerate}

\end{enumerate}

\subsection{Construcción de la versión inicial}

Tu primera tarea es construir la versión original del software. Deberás
construir solamente la versión de \emph{release}. Para hacerlo, utiliza un
\emph{script} como el siguiente \cppid{build.sh}:

\lstinputlisting[language=bash]{lab/05-omp/build.sh}

Podrás ejecutar este script en un nodo normal del clúster haciendo:

\begin{lstlisting}[style=terminal]
sbatch build.sh
\end{lstlisting}

Para ejecutar la conversión a escala de grises, puedes usar el siguiente
\emph{script} \cppid{run-gray.sh}:

\lstinputlisting[language=bash]{lab/05-omp/run-gray.sh}

El programa imprimirá el tiempo en microsegundos dedicado a la conversión a
escala de grises y \textmark{perf} te dará el tiempo total de ejecución.
Con estos dos valores podrás deducir cuál es la fracción de tiempo del programa
que se puede mejorar al paralelizar la conversión a escala de grises.

Ahora puedes hacer los mismo ejecutando el cálculo del histograma con el
\emph{script} \cppid{run-histo.sh}:

\lstinputlisting[language=bash]{lab/05-omp/run-histo.sh}

\subsection{Uso de un nodo NUMA}

A través del \emph{front-end} \textmark{avignon} puedes tener acceso a un
servidor de cómputo (llamado \textmark{stan}).

Prepara un \emph{script} para ver las características hardware de
\textmark{stan} llamado \cppid{lscpu.sh}

\lstinputlisting[language=bash]{lab/05-omp/lscpu.sh}

Modifica los \emph{scripts} previos para repetir el ejercicio anterior en
\textmark{stan}:

El nuevo \cppid{build-stan.sh} debería ser:

\lstinputlisting[language=bash]{lab/05-omp/build-stan.sh}

El nuevo \emph{script} \cppid{run-gray-stan.sh} sería:

\lstinputlisting[language=bash]{lab/05-omp/run-gray-stan.sh}

El nuevo script \cppid{run-histo-stan.sh} sería:

\lstinputlisting[language=bash]{lab/05-omp/run-histo-stan.sh}

\textbad{IMPORTANTE}: 
Para ejecutar un \emph{script} en \textmark{stan} debes especifica el nombre de
la cola de ejecución.

\begin{lstlisting}[style=terminal]
sbatch -p stan script.sh
\end{lstlisting}

Deduce otra vez cuál es la fracción del programa que puede mejorarse al
paralelizar la conversión a escala de grises y el cálculo del histograma.

\subsection{Implementación paralela}

En esta sección paralelizarás la conversión a escala de grises y el cálculo del
histograma. Modificarás la implementación del componente software
\cppid{parallel\_image}.

Presta atención a las siguientes funciones:

\begin{itemize}

\item \cppid{parallel\_image::to\_grayscale()}: 
Realiza la conversión a escala de grises.
\item \cppid{parallel\_image::generate\_histogram()}: 
Genera la información del histograma.

\end{itemize}

No olvides lo siguiente:

\begin{itemize}

\item En el \cppid{CMakeLists.txt} de la carpeta \textmark{img} 
necesitarás añadir lo siguiente:

\lstinputlisting{lab/05-omp/cmake-omp.txt}

\item No especifiques el número de hilos en el código. Modificaremos esto
mediante la correspondiente variable de entorno.

\end{itemize}

\subsection{Evaluación de la versión paralela}

En esta sección evaluarás la aceleración obtenido tanto en un nodo normal como
en \textmark{stan}.

Como ejemplo, puedes usar el siguiente \emph{script} para ejecutar la conversión
a escala de grises con diferentes números de hilos usando el ejecutable generado
para \textmark{stan}:

\lstinputlisting[language=bash]{lab/05-omp/run-gray-omp-stan.sh}

Puedes ejecutar este \emph{script} en \textmark{stan} con el siguiente mandato:

\begin{lstlisting}[style=terminal]
sbatch -p stan run-gray-omp-stan.sh
\end{lstlisting}
