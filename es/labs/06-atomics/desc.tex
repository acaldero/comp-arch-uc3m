\section{Descripción del laboratorio}

En este laboratorio partirá de un proyecto suministrado que contiene un programa principal,
que ejecuta una aplicación con varios hilos.

Observa que para compilar un programa con varios hilos con CMake debes tener en cuenta:

\begin{itemize}

\item Debes incluir una línea para encontrar el paquete de compilación de hilos
\begin{lstlisting}
find_package(Threads REQUIRED)
\end{lstlisting}

\item Debes enlazar tu programa con la biblioteca de hilos.
      Para ello puedes poner la siguiente línea antes de la definición del ejecutable
      (afectará a todos los ejecutables).
\begin{lstlisting}
link_libraries(Threads::Threads)
\end{lstlisting}

\end{itemize}

Se suministra un programa \cppid{counter.cpp} con la funcionalidad básica.

\begin{itemize}

\item Un clase \cppid{counter} que encapsula un valor en doble precisión, que se inicia a cero.
      Tiene las siguientes operaciones:
\begin{itemize}
  \item Una función \cppid{update()} que incrementa el valor encapsulado.
  \item Una función \cppid{print()} que imprime el valor del contador.
\end{itemize}

\item Un programa principal que lanza varios hilos concurrentes.
\begin{itemize}
  \item El número de hilos se define en la constante \cppid{num\_threads}.
  \item El contador compartido es la variable local \cppid{count}.
  \item Cada hilo realiza $100,000$ actualizaciones del contador.
  \item Cuando todos los hilos han finalizado se imprime el resultado final del contador.
  \item También se imprime el tiempo transcurrido en la ejecución de todos los hilos.
\end{itemize}

\end{itemize}


Para llevar a cabo este laboratorio debes realizar todas las pruebas en
el clúster \textemph{avignon}.
