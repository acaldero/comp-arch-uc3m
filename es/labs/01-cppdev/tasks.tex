\section{Tareas}

\subsection{Comprueba tu versión del compilador}

La interfaz para el compilador gcc es el mandato \textmark{g++}.

Comprueba la versión del compilador que tienes:

\begin{lstlisting}[style=terminal]
user$machine:~$g++ --version
\end{lstlisting}

Deberías obtener el número de versión, si se encuentra el mandato \textmark{g++}.

\subsubsection{En tu máquina}

\begin{enumerate}

\item Instala el paquete \textmark{build-essential}:

\begin{lstlisting}[style=terminal]
user$machine:~$sudo apt install build-essential
\end{lstlisting}

\item Comprueba la versión del compilador:

\begin{lstlisting}[style=terminal]
user$machine:~$g++ --version
\end{lstlisting}

\item Si quieres instalar varias versiones del compilador,
      puede que quieras leer la siguientes páginas:
  \begin{itemize}
    \item \url{https://itslinuxfoss.com/install-gcc-ubuntu-22-04}
    \item \url{https://linuxize.com/post/how-to-install-gcc-on-ubuntu-20-04/}
    \item \url{https://lindevs.com/install-gcc-on-ubuntu}
  \end{itemize}
\end{enumerate}

\subsection{Compilar un programa sencillo}

Para este ejercicio de compilación utilizaremos un archivo \cppid{hello.cpp} file 
con el siguiente contenido:

\begin{lstlisting}
#include <iostream>

int main() {
  using namespace std;

  cout << "Hello C++\n";
}
\end{lstlisting}

Utilizaremos diversas opciones para observar diferentes comportamientos de la interfaz de línea de comando
\textmark{g++}:

\begin{itemize}
  \item \cppid{g++ hello.cpp -E -o hello.ii}: 
  Ejecuta solamente el preprocesador y obtiene la salida en el archivo \cppid{hello.ii}.
    \begin{itemize}
      \item Intenta determinar cuantas líneas de código C++ se generan despues de procesar \cppid{hello.cpp}.
    \end{itemize}

  \item \cppid{g++ hello.cpp -S -o hello.s}:
  Genera código ensamblador para la unidad de traducción actual.
    \begin{itemize}
      \item Intenta determinar cuantas líneas de código ensamblador se generan para \cppid{hello.cpp}.
    \end{itemize}

  \item \cppid{g++ hello.cpp -c -o hello.o}:
  Genera un archivo objeto \cppid{hello.o} que puede enlazarse con otros archivos objeto.
    \begin{itemize}
      \item Intenta determinar cuantos bytes se requieren para el archivo \cppid{hello.o}.
    \end{itemize}

  \item \cppid{g++ hello.cpp -o hello}:
  Genera un archivo ejecutable \cppid{hello}.
    \begin{itemize}
      \item Intenta determinar cuantos bytes se requieren para el ejecutable binario \cppid{hello}.
    \end{itemize}
\end{itemize}

Hay otras herramientas relacionadas que te podrían ser útiles como desarrollador de sistemas,
pero que no se cubren en este laboratorio.
Intenta encontrar información sobre las siguientes:

\begin{itemize}
  \item \cppid{nm}
  \item \cppid{readelf}
  \item \cppid{objdump}
  \item \cppid{c++filt}
\end{itemize}

\subsection{Uso del entorno de desarrollo \textbf{CLion}}

\begin{enumerate}

\item
Incia el IDE CLion.
Cuando te pida la licencia, utiliza el usuario y la clave que obtuviste cuando
registraste tu cuenta de usuario en la página de JetBrains.

\item
Crea un nuevo proyecto llamado \cppid{hello}:

\begin{itemize}
  \item Project type: C++ Executable.
  \item Language standard: C++20
\end{itemize}

\item
Comprueba que se han generado los siguientes archivos:

\begin{itemize}
  \item \cppid{main.cpp}: 
        Modifícalo para escribir los contenidos de \cppid{hello.cpp} en este archivo.
  \item \cppid{CMakeLists.txt}
\end{itemize}

\item
Construye el programa mediante \textmark{Build | Build Project} 
y comprueba que el programa se ha compilado correctamente.

\item
Ejecuta el programa con \textmark{Run | Run 'hello'}. 
Comprueba que puedes ver la salida del programa en el IDE.

\end{enumerate}

\subsection{Uso de varios archivos fuente}

En esta sección crearemos un proyecto con varios archivos fuente:

\begin{enumerate}

\item
Crea un nuevo proyecto llamado \cppid{areas}.

\item
Cambia la extensión para los archivos fuente siguiendo estos pasos:
\begin{itemize}
  \item Menú \textmark{File | Settings}.
  \item En el cuadro de diálogo, selecciona \textmark{Editor | Code Style | C/C++}.
  \item Selecciona la pestaña \textmark{New File Extensions}.
  \item Cambia \textmark{Header extensions} a \textgood{hpp}.
  \item Cambia \textmark{File Naming convention} a \textgood{snake\_case}.
\end{itemize}

\item
Añade un archivo llamado \cppid{geom.hpp}

\begin{lstlisting}
#ifndef AREAS_GEOM_HPP
#define AREAS_GEOM_HPP

namespace geom {

  double area(double w, double h);

}

#endif
\end{lstlisting}

\item 
Añade un archivo llamado \cppid{geom.cpp}:

\begin{lstlisting}
#include "geom.hpp"

namespace geom {
  double area(double w, double h) {
    return w * h;
  }
}
\end{lstlisting}

\item
Modifica tu programa \cppid{main.cpp}:

\begin{lstlisting}
#include <iostream>
#include "geom.hpp"

int main() {
  double x = 2.0;
  double y = 3.0;
  std::cout << "area(" << x << "," << y << ")= " << geom::area(x,y) << "\n";
}
\end{lstlisting}

\item
Modifica tu archivo \cppid{CMakeLists.txt} para añadir algunas opciones de compilación y 
habilitar las advertencias:

\begin{lstlisting}
...
set(CMAKE_CXX_STANDARD 20)
set(CMAKE_CXX_STANDARD_REQUIRED ON)
set(CMAKE_CXX_EXTENSIONS_OFF)
add_compile_options(-Wall -Wextra -Werror -pedantic -pedantic-errors)

add_executable(area main.cpp geom.cpp)
\end{lstlisting}

\item
Ejecuta el programa.

\end{enumerate}

\subsection{Versión de \emph{Debug} y versión de \emph{Release}}

Cuando utilizas CLion la compilación por defecto genera la versión de depuración.
No obstante, cuando no estas depurando prefieres tener una versión optimizada o versión de \textgood{Release}.

\begin{enumerate}

\item
Abre el \textgood{cuadro de diálogo de opciones} con \textmark{File | Settings}.

\item
Selecciona \textmark{Build, Execution, Deployment | CMake}.

\item 
Verás la configuración de \textgood{Debug}.

\item 
Haz clic en el símbolo \textmark{+} 
para añadir otra configuración.
La configuración de \textgood{Release} se añadirá.

\item
Aplica el cambio de configuración.

\item
En la lista de objetivos de construcción, ahora puedes seleccionar 
\cppid{areas | Debug} o bien \cppid{areas | Release}.

\end{enumerate}

\subsection{Detección de goteos de memoria con \emph{valgrind}}

\begin{enumerate}

\item
Crea una aplicación llamada \cppid{numbers}.

\item Añade un archivo llamado \cppid{primitive\_vecnum.hpp}

\begin{lstlisting}
#ifndef NUMBERS_PRIMITIVE_VECNUM_HPP
#define NUMBERS_PRIMITIVE_VECNUM_HPP

class vecnum {
public:
  explicit vecnum(int n) : size_{n}, buffer_{new double[n]{}} {}
  [[nodiscard]] int size() const { return size_; }

  double operator[](int i) const { return buffer_[i]; }
  double &operator[](int i) { return buffer_[i]; }
private:
  int size_;
  double * buffer_;
};

#endif//NUMBERS_PRIMITIVE_VECNUM_HPP
\end{lstlisting}

\item
Escribe el siguiente \cppid{main.cpp}:

\begin{lstlisting}
#include <iostream>
#include "primitive_vecnum.hpp"

int main() {
  vecnum v(5);
  v[2] = 3.0;
  for (int i=0; i<v.size(); ++i) {
    std::cout << "v[" << i << "] = " << v[i] << '\n';
  }
}
\end{lstlisting}

\item
Ejecuta el programa y comprueba los resultados.

\item
Ejecuta el programa con \textmark{Run | Run with valgrind memcheck} y analiza la salida.

\end{enumerate}

\subsection{Análisis dinámico de código con \emph{sanitizers}}

Hay varios \textmark{sanitizers} disponibles en algunos compiladores:

\begin{itemize}
  \item \textmark{Address Sanitizer}: 
        Detecta errores de memoria
  \item \textmark{Leak Sanitizer}: 
        Subconjunto de \emph{AddressSanitizer} solamente para goteos de memoria.
  \item \textmark{Thread Sanitizer}: 
        Detecta errores de concurrencia.
  \item \textmark{Undefined Behavior Sanitizer}: 
        Detecta algunos comportamientos no definidos.
  \item \textmark{Memory Sanitizer} (solo con \cppid{clang++}): 
        Detecta el uso de memoria sin iniciar.
\end{itemize}

Puedes utilizarlos añadiendo perfiles extra en CLion:

\begin{enumerate}

\item
Abre el diálogo de opciones con \textmark{File | Settings}.

\item
Selecciona \textmark{Build, Execution, Deployment | CMake}.

\item
Copia el perfil de \textmark{Debug} en un nuevo perfil 
y llámalo \textmark{Debug-address-sanitizer}.

\item
En el campo de opciones añade
\cppid{-DCMAKE\_CXX\_FLAGS="-fsanitize=address -fno-omit-frame-pointer"}.

\end{enumerate}

\subsection{Corrección de goteos de memoria}

Intenta corregir el goteo de memoria con al menos una de las siguientes alternativas:

\begin{enumerate}

\item Añade un destructor a la clase existente.
\begin{itemize}
  \item Asegúrate de que en el destructor utilizas 
        un \cppkey{delete []} \cppid{var} 
        y no un \cppkey{delete} \cppid{var}.
\end{itemize}

\item Modifica el dato miembro \cppkey{double*} 
para utilizar en su lugar un dato miembro \cppid{std::unique\_ptr<double[]>}.
Ten en cuenta lo siguiente:

\begin{itemize}
  \item Incluye la cabecera \cppid{<memory>}.
  \item Modifica el dato miembro \cppid{buffer\_} al tipo \cppid{std::unique\_ptr<double[]>}.
  \item Remplaza el \cppkey{new} del constructor por una llamada a \cppid{std::make\_unique<double[]>(n)}.
  \item Incluye la cabecera \cppid{<algorithm>}.
  \item Añade un constructor de copia.
\begin{lstlisting}
vecnum(const vecnum & v) : size_{v.size_}, buffer_{std::make_unique<double[]>(size_) {
  std::copy_n(v.buffer_.get(), v.size_, buffer_.get());
}
\end{lstlisting}
\end{itemize}

\end{enumerate}

Después, analiza los efectos sobre los goteos de memoria cuando usas el siguiente programa \cppid{main.cpp}:

\begin{lstlisting}
#include <iostream>
#include "primitive_vecnum.hpp"

int main() {
  vecnum v(5);
  v[2] = 3.0;

  vecnum w{v};

  for (int i=0; i<v.size(); ++i) {
    std::cout << "v[" << i << "] = " << v[i] << '\n';
  }

  for (int i=0; i<w.size(); ++i) {
    std::cout << "w[" << i << "] = " << w[i] << '\n';
  }
}
\end{lstlisting}

