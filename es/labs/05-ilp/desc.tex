\section{Descripción de la práctica}

En esta práctica se realizará la evaluación de la ejecución de diversos
programas en C++ analizando su rendimiento y estudiando los valores
obtenidos con contadores hardware.

Recuerda que para compilar un programa con opciones de depuración
(y sin optimizaciones) debes generar una configuración de depuración.

\begin{lstlisting}[style=terminal,aboveskip=1em,belowskip=1em]
cmake -S . -B debug -DCMAKE_BUILD_TYPE=Debug
\end{lstlisting}

Si quieres generar una configuración con todas las optimizaciones
activadas deberás generar una configuración de \emph{release}.

\begin{lstlisting}[style=terminal,aboveskip=1em,belowskip=1em]
cmake -S . -B release -DCMAKE_BUILD_TYPE=Release
\end{lstlisting}

Además, para compilar una configuración, debes invocar a \cppkey{cmake}
con el nombre de dicha configuración:

\begin{lstlisting}[style=terminal,aboveskip=1em,belowskip=1em]
cmake --build release
\end{lstlisting}

\fbox{ \parbox{.9\textwidth}{
\textbad{RECORDATORIO}: 
Para ejecutar trabajos en el clúster
\cppid{avignon} primero debes crear un \emph{script} para lanzar
la compilación.
}}
