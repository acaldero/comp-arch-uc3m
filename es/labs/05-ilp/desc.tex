\section{Descripción de la práctica}

En esta práctica se realizará la evaluación de la ejecución de diversos
programas en C++ analizando su rendimiento y estudiando los valores
obtenidos con contadores hardware.

Recuerda que para compilar un programa con opciones de depuración
(y sin optimizaciones) debes generar una configuración de depuración.

\begin{lstlisting}[style=terminal,aboveskip=1em,belowskip=1em]
cmake -S . -B debug -DCMAKE_BUILD_TYPE=Debug
\end{lstlisting}

Si quieres generar una configuración que active de forma selectiva alguna optimización,
puedes utilizar la opción \cppid{CMAKE\_CXX\_FLAGS}. Por ejemplo,
si quieres compilar en modo depuración, pero activar selectivamente
la optimización \cppkey{-funroll-loops}, deberás generar una configuración
como la siguiente:

\begin{lstlisting}[style=terminal,aboveskip=1em,belowskip=1em]
cmake -S . -B debug-unroll -DCMAKE_BUILD_TYPE=Debug CMAKE_CXX_FLAGS=-funroll-loops
\end{lstlisting}

Si quieres generar una una configuración con todas las optimizaciones
activadas deberás genrar una configuración de \emph{release}.

\begin{lstlisting}[style=terminal,aboveskip=1em,belowskip=1em]
cmake -S . -B release -DCMAKE_BUILD_TYPE=Release
\end{lstlisting}

Además, para compilar una configuración debes invocar a \cppkey{cmake}
con el nombre de la configuración:

\begin{lstlisting}[style=terminal,aboveskip=1em,belowskip=1em]
cmake --build debug-unroll
\end{lstlisting}

\fbox{ \parbox{.9\textwidth}{
\textbad{ATENCIÓN}: Recuerda que para ejecutar trabajos en el clúster
\cppid{avignon} primero debes crear un \emph{script} para lanzar
la compilación.
}}
