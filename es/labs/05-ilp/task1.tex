\clearpage
\subsection{Impacto de predicción de saltos}

En esta tarea se va a estudiar el impacto de la predicción de saltos.

El programa utiliza una función que genera valores aleatorios en el intervalo
$[-100.0, 100.0]$. Esta función se suministra en el archivo de cabecera
\cppid{generate.hpp} y se implementa en el archivo \cppid{generate.cpp}.

\lstinputlisting[frame=single,title={Archivo: generate.hpp}]{lab/05-ilp/generate.hpp}

\lstinputlisting[frame=single,title={Archivo: generate.cpp}]{lab/05-ilp/generate.cpp}

También se suministra un programa principal, calcula el promedio de los números
positivos de la muestra generada.

\lstinputlisting[frame=single,title={Archivo: bpred1.cpp}]{lab/05-ilp/bpred1.cpp}

El programa calcula el tiempo necesario para imprimir el promedio.

\subsection{Tarea: Ejecución del programa}

Compila y ejecuta el programa original. Determine el tiempo total de ejecución y
el tiempo de ejecución del cálculo del promedio.

\subsection{Tarea: Saltos y predicción de saltos}

Utiliza la herramienta \cppkey{perf} para determinar el comportamiento de los saltos.
Consulte la lista de eventos disponibles con la opción \cppkey{perf list}. 

En particular, debe identificar los eventos para:
\begin{itemize}
  \item Número total de instrucciones de bifurcación (\emph{branch}) ejecutadas.
  \item Número de fallos de predicción.
\end{itemize}

\subsection{Tarea: Impacto en secuencias ordenadas}

Copia el programa \cppid{bpred1.cpp} en un nuevo programa \cppid{bpred2.cpp}.

Antes de entrar en el bloque de medición de tiempos, ordena el vector. Para ello
puedes optar por una de las siguientes funciones de la biblioteca estándar.

\begin{itemize}
  \item \cppid{std::sort()}: \url{https://en.cppreference.com/w/cpp/algorithm/sort}.
  \item \cppid{std::ranges::sort()}: \url{https://en.cppreference.com/w/cpp/algorithm/ranges/sort}.
\end{itemize}

Repite las mediciones, prestando especial atención al tiempo necesario para
realizar el cálculo del promedio. 

Busca una explicación a las diferencias de tiempo.

\subsection{Tarea: Eliminación de saltos}

Copia el programa \cppid{bpred1.cpp} en un nuevo programa \cppid{bpred3.cpp}.

Trata de eliminar las bifurcación que hay dentro del bucle de la función \cppid{average\_positive()}. 
Para ello, puedes hacer uso del operador condicional ternario:

\begin{lstlisting}
result = (condicion)?expr1:expr2;
\end{lstlisting}

La evaluación de esta sentencia da lugar a la expresión \cppid{expr1} cuando la condición
es cierta y la expresión \cppid{expr2} cuando la condición es falsa.

Repite las mediciones, prestando especial atención al tiempo necesario para
realizar el cálculo del promedio. 

Busca una explicación a las diferencias de tiempo.
