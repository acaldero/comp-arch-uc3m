\section{Tareas}

\subsection{Apertura de cuenta}

Si no tienes cuenta en el Laboratorio de Informática, debes abrir una cuenta.

Encontrarás toda la información en la siguiente página:

\begin{itemize}
  \item \url{https://www.lab.inf.uc3m.es/servicios/apertura-de-cuenta/}
\end{itemize}

Esta es la cuenta que usarás para acceder al clúster docente del Departamento de Informática.

\subsection{Acceso al Clúster}

Tienes las siguientes opciones:


\begin{itemize}
  \item Conectarte por Web. Tienes toda la información de acceso en:
    \begin{itemize}
      \item \url{https://www.lab.inf.uc3m.es/servicios/aulas-virtuales-del-laboratorio/}.
      \item Una vez conectado, debes seleccionar el equipo \textemph{Avignon}.
    \end{itemize}

  \item Conectarte mediante \cppid{ssh} a \cppkey{avignon.lab.inf.uc3m.es}. Para ello,
        desde un terminal deberás usar el mandato \cppid{ssh}, utilizando tu
        usuario y clave del Laboratorio de Informática.
\begin{lstlisting}[style=terminal]
$ ssh -l usuario avignon.lab.inf.uc3m.es
usuario@avignon.lab.inf.uc3m.es's password: 
Linux avignon-frontend 5.10.0-15-amd64 #1 SMP Debian 5.10.120-1 (2022-06-09) x86_64
...
$
\end{lstlisting}

\end{itemize}

\subsection{Ejecución de mandatos}

Una vez introducida la clave estarás en el nodo del \emph{front-end}.
Ten en cuenta que esta máquina simplemente te da acceso a lanzar trabajos
en otros equipos, pero que no debes ejecutar nada directamente.

Para ejecutar un trabajo debes escribir un \emph{shell script} con las tareas
a ejecutar y lanzarlo utilizando el mandato \textmark{sbatch}.

Vamos a crear un script para conocer las características del hardware del nodo
en el que vas a ejecutar los trabajos. Para conocer estas características
se puede usar el mandato \textmark{lscpu}. 

Escribe el siguiente código en un archivo que llamarás \cppid{runlscpu.sh}

\begin{lstlisting}[language=bash,frame=single,title={Archivo: runlscpu.sh}]
#!/bin/sh
lscpu
\end{lstlisting}

Para lanzar el trabajo debes usar el mandato \cppid{sbatch}.
Debe pasarse como argumento siempre el nombre de un \emph{shell script}:

\begin{lstlisting}[style=terminal]
$ sbatch ./runlscpu.sh
Submitted batch job 408
$
\end{lstlisting}

Podrás consultar la lista de trabajos activos con el mandato \textmark{squeue}.

Cuando el trabajo haya terminado, la salida estándar del mismo se habrá guardado
en un archivo llamado \cppid{slurm-XXX.out}, donde XXX es el número de trabajo.

Compara las características del nodo \emph{front-end} (ejecutando \cppid{lscpu})
y el nodo de ejecución (ejecutando \cppid{sbatch ./lscpu.sh}).

Presta atención a las siguientes características:
\begin{itemize}
  \item Arquitectura.
  \item Número de CPUs.
  \item Modelo de CPU.
  \item Rango de frecuencias de reloj.
  \item Tamaño de los niveles de caché.
\end{itemize}

\subsection{Compilación de los programas C++}

Durante este laboratorio, vamos a trabajar con tres versiones de un programa
que generan una imagen en formato PBM del conjunto de Mandelbrot.

\begin{itemize}
  \item \cppid{mandel.py}: Versión en Python.
  \item \cppid{mandel.cpp}: Versión secuencial simplificada en C++.
  \item \cppid{mandel-par.cpp}: Versión paralela optimizada manualmente en C++.
\end{itemize}

\subsubsection{Transferencia de los archivos}

En Aula Global encontrarás un archivo \cppid{mandel.zip} con toda el código que
necesitas. Coloca este archivo en un directorio en tu máquina local llamado
\cppkey{lab2}. A continuación descomprime el archivo en un subdirectorio que
se llamará \cppkey{mandel}:

\begin{lstlisting}[style=terminal]
usuario@maquina:~/lab2$ unzip mandel.zip mandel
Archive:  mandel.zip
  inflating: mandel/CMakeLists.txt   
  inflating: mandel/mandel.cpp       
  inflating: mandel/mandel.py        
  inflating: mandel/mandel-par.cpp   
usuario@maquina:~/lab2$ ls
mandel  mandel.zip
\end{lstlisting}

Para transferir los archivos puedes usar el mandato \cppid{scp}.

Primero crea en el directorio remoto un subdirectorio llamado \cppid{mandel}.

\begin{lstlisting}[style=terminal]
$usuario@avignon-frontened:/~lab2$ mkdir
$usuario@avignon-frontened:/~lab2$ ls
mandel  runlscpu.sh  ...
\end{lstlisting}

Colócate en el directorio donde hayas descargado el archivo \cppid{mandel.zip}
y usa el siguiente mandato:

\begin{lstlisting}[style=terminal,basicstyle=\tiny\ttfamily,columns=fixed]
usuario@maquina:~/lab2$ scp mandel/* usuario@avignon.lab.inf.uc3m.es:~/lab2/mandel
usuario@avignon.lab.inf.uc3m.es password: 
CMakeLists.txt                            100%  427   166.6KB/s   00:00    
mandel.cpp                                100% 2193   783.0KB/s   00:00    
mandel-par.cpp                            100%   11KB   3.0MB/s   00:00    
mandel.py                                 100% 1692   596.9KB/s   00:00    
usuario@maquina:~/lab2$ 
\end{lstlisting}

Con esta orden habrás transferido todos los archivos necesarios al directorio
remoto \cppid{~/lab2/mandel}.

\subsubsection{Compilación desde línea de mandatos}

Los archivos \cppid{CMake} permiten automatizar la compilación. En esta tarea
se compilarán dos versiones de cada programa: una versión de \textmark{Debug}
y una versión de \textmark{Release}.

Un detalle importante, es que las máquinas del \emph{front-end} y los nodos de
ejecución son muy diferentes (como habrás observado con \cppid{lscpu}). Por
tanto es importante que compiles en un nodo de ejecución.

Para ello vamos a crear un script de compilación:
\begin{lstlisting}[language=bash,frame=single,title={Archivo: build.sh}]
#!/bin/sh
cmake -S mandel -B mandel-debug -DCMAKE_BUILD_TYPE=Debug
cmake -S mandel -B mandel-release -DCMAKE_BUILD_TYPE=Release
cmake --build mandel-debug
cmake --build mandel-release 
\end{lstlisting}

\textbad{ACLARACIÓN}: A continuación se presenta una pequeña descripción
de lo que hace cada uno de los mandatos

\begin{itemize}
\begin{lstlisting}[style=terminal]
cmake -S mandel -B mandel-debug -DCMAKE_BUILD_TYPE=Debug
\end{lstlisting}
  \item Genera una configuración de proyecto en el directorio \cppid{mandel-debug}
        a partir de los archivos fuente del directorio \cppid{mandel}.
        Utiliza como opción la variable \cppkey{DCMAKE\_BUILD\_TYPE=Debug},
        que hace que se utilicen las opciones de compilación de depuración.

\begin{lstlisting}[style=terminal]
cmake -S mandel -B mandel-release -DCMAKE_BUILD_TYPE=Release
\end{lstlisting}
  \item Genera una configuración de proyecto en el directorio \cppid{mandel-release}
        a partir de los archivos fuente del directorio \cppid{mandel}.
        Utiliza como opción la variable \cppkey{DCMAKE\_BUILD\_TYPE=Release},
        que hace que se utilicen las opciones de compilación de depuración.

\begin{lstlisting}[style=terminal]
cmake --build mandel-debug
\end{lstlisting}
  \item Construye el proyecto generado en el directorio \cppid{mandel-debug}.

\begin{lstlisting}[style=terminal]
cmake --build mandel-release
\end{lstlisting}
  \item Construye el proyecto generado en el directorio \cppid{mandel-release}.

\end{itemize}

\textbad{IMPORTANTE}: No ejecutes estos mandatos en el \emph{front-end}.
En ese caso generarías código que no es válido para ejecutar en los 
nodos de cómputo. En su lugar, ejecuta el \emph{script} utilizando el mandato.

\begin{lstlisting}[style=terminal]
sbatch build.sh
\end{lstlisting}

Recuerda comprobar los archivos generados en los directorios \cppid{mandel-debug}
y \cppid{mandel-release}.

\subsection{Ejecución de los programas}

En esta tarea vamos a ejecutar los tres programas en un nodo de cómputo.
Para las versiones de C++ ejecutaremos solamente las versiones de \textmark{Release}.

\begin{enumerate}
  \item \cppid{mandel.py}: Para ejecutar este programa usaremos el intérprete de Python.
        Tienes disponible la versión 3.9.

    \begin{itemize}
      \item Recuerda pasar a Python la opción \cppid{-OO} para eliminar información
            de depuración.
      \item El programa toma como parámetro el tamaño de la imagen, que es cuadrada.
      \item Debes redirigir la salida a un archivo con extensión \cppkey{.pbm}.
    \end{itemize}

\begin{lstlisting}[language=bash,title={Archivo: runpy.sh},frame=single]
#!/bin/sh
python3 -OO mandel/mandel.py 8000 > py-mandel.pbm
echo "Ejecución de mandel.py finalizada"
\end{lstlisting}

    \begin{itemize}
      \item Ejecuta este \emph{scritp} usando el mandato \cppid{sbatch}.
    \end{itemize}

  \item \cppid{mandel}: Para ejecutar este programa invocaremos directamente al
        ejecutable.
    \begin{itemize}
      \item El programa toma como primer parámetro el tamaño de la imagen, que es cuadrada.
      \item El programa toma como segundo parámetro el nombre del archivo de salida.
    \end{itemize}

\begin{lstlisting}[language=bash,title={Archivo: run-seq-mandel.sh},frame=single]
#!/bin/sh
mandel-release/mandel 8000 seq-mandel.pbm
echo "Ejecución de release/mandel finalizada"
\end{lstlisting}
    \begin{itemize}
      \item Ejecuta este \emph{scritp} usando el mandato \cppid{sbatch}.
    \end{itemize}

  \item \cppid{mandel-par}: Para ejecutar este programa invocaremos directamente al
        ejecutable.
    \begin{itemize}
      \item El programa toma como primer parámetro el tamaño de la imagen, que es cuadrada.
      \item Debes redirigir la salida a un archivo con extensión \cppkey{.pbm}.
    \end{itemize}

\begin{lstlisting}[language=bash,title={Archivo: run-par-mandel.sh},frame=single]
#!/bin/sh
mandel-release/mandel-par 8000 > par-mandel.pbm
echo "Ejecución de release/mandel-par finalizada"
\end{lstlisting}
    \begin{itemize}
      \item Ejecuta este \emph{scritp} usando el mandato \cppid{sbatch}.
    \end{itemize}
\end{enumerate}


\textbad{IMPORTANTE}: Recuerda comprobar que los tres archivos \cppkey{.pbm}
generados son idénticos.

Fíjate que el fichero \cppid{slurm-XXX} debe tener un mensaje de finalización.

\subsection{Medición básica del rendimiento}

Para realizar una medición básica del rendimiento, vamos a usar la
herramienta \cppkey{perf}. Esta herramienta desarrollada para el sistema
operativo Linux y que ofrece acceso a a información del kernel
del sistema operativo así como a contadores hardware del procesador.

La herramienta \cppkey{perf} ofrece varios subcomandos. En este
laboratorio usaremos el subcomando \cppkey{stat} que recoge información
básica sobre la ejecución de un programa:

\begin{lstlisting}[style=terminal,basicstyle=\ttfamily]
$ perf stat program param1 param2

 Performance counter stats for 'program param1 param2':

         56.050,69 msec task-clock                #    0,992 CPUs utilized          
             6.851      context-switches          #    0,122 K/sec                  
                30      cpu-migrations            #    0,001 K/sec                  
             2.082      page-faults               #    0,037 K/sec                  
   192.416.969.761      cycles                    #    3,433 GHz                    
   437.336.887.712      instructions              #    2,27  insn per cycle         
    46.898.229.743      branches                  #  836,711 M/sec                  
        94.154.392      branch-misses             #    0,20% of all branches        

      56,521233276 seconds time elapsed

      55,930658000 seconds user
       0,120151000 seconds sys
$
\end{lstlisting}

Ten en cuenta que si quieres obtener la información de todas las CPUs
debes usar la opción \cppkey{-a}.

\begin{lstlisting}[style=terminal]
$ perf stat -a program params
$
\end{lstlisting}

Puedes observar alguna información interesante que se puede determinar.
Nos centraremos por ahora en algunos aspectos básicos. Otros se
cubrirán más adelante en la asignatura.

\begin{itemize}
  \item El número de ciclos de reloj consumidos.
  \item El número de instrucciones ejecutadas.
  \item El tiempo transcurrido en la ejecución del programa
  \item El tiempo total de usuario y del sistema.
  \item El número equivalente de CPUs usada. Este valor será mayor de 1 si
        se usan varios cores simultáneamente.
  \item La frecuencia de reloj promedio a la que se ha ejecutado.
  \item El número de instrucciones por ciclo.
\end{itemize}

En esta tarea deberás obtener información sobre los parámetros de ejecución de
las siguientes versiones del programa cuando se genera una imagen de 8000x8000.

\begin{enumerate}
  \item \cppkey{mandel.py}
  \item \cppkey{debug/mandel}
  \item \cppkey{release/mandel}
  \item \cppkey{debug/mandel-par}
  \item \cppkey{release/mandel-par}
\end{enumerate}

\textbad{IMPORTANTE}: Recuerda que tendrás que escribir los correspondientes
scripts y lanzarlos con \cppkey{sbatch}. Por cada invocación a
\cppkey{sbatch} se genera un archivo \cppid{slurm-XXX.out} con la salida
estándar de la ejecución.

Intenta sacar unas breves conclusiones de los resultados obtenidos.

\subsection{Medición de la energía}

El subcomando \cppkey{stat} también puede usarse per medir determinados eventos
en la ejecución de un programa.

Puedes obtener la lita completa de eventos disponibles con el subcomando \cppkey{list}.

\begin{lstlisting}[style=terminal]
$ perf list
 branch-instructions OR branches                    [Hardware event]
  branch-misses                                      [Hardware event]
  bus-cycles                                         [Hardware event]
  cache-misses                                       [Hardware event]
  cache-references                                   [Hardware event]
  cpu-cycles OR cycles                               [Hardware event]
  instructions                                       [Hardware event]
  ref-cycles                                         [Hardware event]
  alignment-faults                                   [Software event]
  bpf-output                                         [Software event]
  context-switches OR cs                             [Software event]
  cpu-clock                                          [Software event]
  cpu-migrations OR migrations                       [Software event]
  dummy                                              [Software event]
  emulation-faults                                   [Software event]
  major-faults                                       [Software event]
  minor-faults                                       [Software event]
  page-faults OR faults                              [Software event]
...
$
\end{lstlisting}

\textbad{Cuidado}: Debes ejecutar el mandato en un nodo de cómputo.

Utiliza \cppkey{perf list} para obtener aquellos contadores que comienzan 
por \cppid{power}.

Una vez hayas encontrado estos eventos puedes pasarlo a \cppkey{perf}
con la opción \cppkey{-e}.

\begin{lstlisting}[style=terminal]
$ perf stat -a -e 'evento1,evento2,evento3' program params
...
$
\end{lstlisting}

En esta tarea debes obtener información energética de la ejecución de cada uno de los
cinco programas del apartado anterior.

Los contadores obtenidos expresan la energía consumida en Julios. Como además
dispondrás del tiempo total de ejecución, puedes obtener también la potencia
media expresada en Vatios.

Intenta sacar unas breves conclusiones de los resultados obtenidos.
