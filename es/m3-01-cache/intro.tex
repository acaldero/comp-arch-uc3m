\section{Introducción}

\begin{frame}[t]{Evolución de la latencia}
\begin{itemize}
  \item Hay múltiples visiones del rendimiento.
    \begin{itemize}
      \item $rendimiento = \frac{1}{latencia}$
      \item Útil para comparar evolución de procesadores y memoria.
    \end{itemize}

  \mode<presentation>{\vfill\pause}
  \item \textgood{Procesadores}
    \begin{itemize}
      \item Incrementos de rendimiento anuales de de entre 25\% y 52%.
      \item Efecto combinado de 1980 a 2005 $\rightarrow$ superior a $10,000$ veces.
      \item Crecimiento marginal desde entonces.
    \end{itemize}

  \mode<presentation>{\vfill\pause}
  \item \textgood{Memoria}
    \begin{itemize}
      \item Incrementos de rendimiento anuales alrededor del 7\%
      \item Efecto combinado de 1980 a 2015 $\rightarrow$ alrededor de $10$ veces.
    \end{itemize}
\end{itemize}
\end{frame}

\begin{frame}[t]{Efecto multi-core}
\begin{itemize}
  \item \textmark{Intel Core i7 6700}.
    \begin{itemize}
      \item 2 accesos a datos de 64 bits por ciclo.
      \item 4 cores, 4.2 GHz $\rightarrow$ $32.8 \times 10^9$ referencias/seg
      \item Demanda instrucciones: $12.8 \times 10^9$ de 128 bits.
      \item Pico de ancho de banda: $524.8$ GB/s + $204.8$ GB/s
    \end{itemize}

  \mode<presentation>{\vfill}
  \item \textmark{Memoria SDRAM}.
    \begin{itemize}
      \item DDR2 (2003): 3.2 GB/seg – 8.5 GB/seg
      \item DDR3 (2007):  8.53 GB/seg – 18 GB/seg
      \item DDR4 (2014): 17.06 GB/seg – 25.6 GB/seg
    \end{itemize}

  \mode<presentation>{\vfill\pause}
  \item \textgood{Soluciones}:
    \begin{itemize}
      \item Memoria multipuerta, cachés segmentadas, cachés multinivel, 
            cachés por core, separación instrucciones y datos.
    \end{itemize}

\end{itemize}
\end{frame}

\begin{frame}[t]{Principio de localidad}
\begin{itemize}
  \item \textgood{Principio de localidad}.
    \begin{itemize}
      \item Es una \textmark{propiedad de los programas} que se explota en el diseño del hardware.
      \item Los programas acceden una porción relativamente pequeña del espacio de direcciones.
    \end{itemize}

  \mode<presentation>{\vfill\pause}
  \item \textgood{Tipos de localidad}:
    \begin{itemize}
      \item \textmark{Localidad temporal}: 
            Los elementos accedidos recientemente tienden a volver a ser accedidos.
        \begin{itemize}
          \item Ejemplos: Bucles, reutilización de variables, \ldots
        \end{itemize}
      \item \textmark{Localidad espacial}: 
            Los elementos próximos a los accedidos recientemente tienden a ser accedidos.
        \begin{itemize}
          \item Ejemplos: Ejecución secuencial de instrucciones, arrays, \ldots
        \end{itemize}
    \end{itemize}
\end{itemize}
\end{frame}

\begin{frame}[t]{Jerarquía de memoria}
\makebox[\textwidth][c]{
\input{es/m3-01-cache/mem-hier.tkz}
}
\end{frame}

\begin{frame}[t]{Situación (2017)}
\begin{itemize}
  \item \textgood{Registros} $\rightarrow$ \emph{CMOS}.
    \begin{itemize}
      \item \textmark{Tiempo de acceso}: 0.1 ns -- 0.2 ns.
      \item \textmark{Tamaño típico}: < 4KiB.
      \item \textmark{Ancho de banda}: $10^6$ -- $10^7$ MiB/s
    \end{itemize}
  \item \textgood{Caché} $\rightarrow$ \emph{SRAM CMOS en chip}.
    \begin{itemize}
      \item \textmark{Tiempo de acceso}: 0.5 ns -- 10 ns.
      \item \textmark{Tamaño típico}: 32KiB -- 8 MiB.
      \item \textmark{Ancho de banda}: $2 \cdot 10^4$ -- $5 \cdot 10^4$ MiB/s. 
    \end{itemize}

  \mode<presentation>{\vfill}
  \item \textgood{Memoria principal} $\rightarrow$ \emph{DRAM CMOS}.
    \begin{itemize}
      \item \textmark{Tiempo de acceso}: $30$ns -- $150$ ns
      \item \textmark{Typical size}: < 1 TB
      \item \textmark{Bandwidth}: $10^4$ --$3 \cdot 10^4$ MiB/s 
    \end{itemize}

\end{itemize}
\end{frame}

\begin{frame}[t]{Escala de las latencias}
\begin{itemize}
  \item \textgood{Tiempos típicos}:
    \begin{itemize}
      \item \textmark{Ciclo de CPU}: 0.3 ns $\rightarrow$ 1 s
      \item \textmark{Caché L1}: 0.9 ns $\rightarrow$ 3 s
      \item \textmark{Caché L2}: 2.8 ns $\rightarrow$ 9 s
      \item \textmark{Memoria principal}: 120 ns $\rightarrow$ 6 min.
      \item \textmark{SSD}: 50--150 $\mu$s 
      \item \textmark{Disco}: 1--10 ms
    \end{itemize}

  \pause
  \item \textgood{Si un ciclo fuese un segundo}:
    \begin{itemize}
      \item \textmark{Ciclo de CPU}: 1 s
      \item \textmark{Caché L1}: 3 s
      \item \textmark{Caché L2}: 9 s
      \item \textmark{Memoria principal}: 6 min.
      \item \textmark{SSD}: 2--6 días.
      \item \textmark{Disco}: 1-12 meses.
    \end{itemize}
\end{itemize}
\end{frame}

\begin{frame}[t]{Jerarquía de memoria}
\begin{itemize}
  \item \textgood{Bloque o línea}: Unidad de copia.
    \begin{itemize}
      \item Suele estar formada por múltiples palabras.
    \end{itemize}

  \mode<presentation>{\vfill\pause}
  \item Si dato accedido está \textmark{presente} en nivel superior:
    \begin{itemize}
      \item \textgood{Acierto}: Satisfecho por nivel superior.
    \end{itemize}
\[
h = \frac{aciertos}{accesos}
\]

  \mode<presentation>{\vfill\pause}
  \item Si dato accedido está \textmark{ausente}.
    \begin{itemize}
      \item \textgood{Fallo}: Bloque se copia de nivel inferior.
      \item Acceso a dato en nivel superior.
      \item Tiempo necesario $\rightarrow$ \textmark{Penalización de fallo}.
    \end{itemize}
\[
m = \frac{fallos}{accesos} = 1 - h
\]

\end{itemize}
\end{frame}

\begin{frame}[t]{Medidas}
\begin{itemize}
  \item Tiempo medio de acceso a memoria:
\[
t_M = t_A + (1-h) t_F
\]
  \mode<presentation>{\vfill\pause}
  \item \textgood{Penalización de fallo}:
    \begin{itemize}
      \item Tiempo en remplazar y bloque y entregarlo a CPU.
      \item \textmark{Tiempo de acceso}.
        \begin{itemize}
          \item Tiempo de obtenerlo de nivel inferior.
          \item Depende de latencia a nivel inferior.
        \end{itemize}
      \item \textmark{Tiempo de transferencia}.
        \begin{itemize}
          \item Tiempo de transferir un bloque.
          \item Depende de ancho de banda entre niveles.
        \end{itemize}
    \end{itemize}
\end{itemize}
\end{frame}

\begin{frame}[t]{Medidas}
\begin{itemize}
  \item \textgood{Tiempo de ejecución de CPU}:
\[
t_{CPU} = 
\left( ciclos_{CPU} + ciclos_{\text{detencion memoria}} \right) \times t_{ciclo}
\]

  \mode<presentation>{\vfill\pause}
  \item \textgood{Ciclos de reloj CPU}:
\[
ciclos_{CPU} =
IC \times CPI
\]

  \mode<presentation>{\vfill\pause}
  \item \textgood{Ciclos de detención de memoria}:
\[
ciclos_{\text{detención memoria}} =
n_{fallos} \times \text{penalización}_{fallo}=
\]
\[
IC \times fallo_{instr} \times \text{penalización}_{fallo} =
\]
\[
IC \times accesos\_memoria_{instr} \times (1 - h ) \times \text{penalización}_{fallo}
\]
\end{itemize}
\end{frame}

\begin{frame}[t]{Distinguiendo lecturas de escrituras}
\begin{itemize}
  \item Patrones de lectura y escritura diferenciados:
    \begin{itemize}
      \item Diferentes tasas de acierto: $h_{lect}$, $h_{escr}$.
      \item Diferentes pnealizaciones: $penalización_{lect}$, $penalización_{escr}$
    \end{itemize}

\[
ciclos_{\text{detención memoria}} =
IC \times lecturas_{instr} \times (1 - h_{lect}) \times \text{penalización}_{lect} +
\]
\[
IC \times escrituras_{instr} \times (1 - h_{escr}) \times \text{penalización}_{escr}
\]
\end{itemize}
\end{frame}
