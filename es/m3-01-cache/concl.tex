\section{Conclusión}

\begin{frame}[t,shrink=10]{Resumen}
\begin{itemize}

  \item Cachés como solución para mitigar el \textbad{muro de memoria}.

  \mode<presentation>{\vfill\pause}
  \item La \textmark{evolución} tecnológica y el principio de \textmark{localidad}
        generan más presión sobre las cachés.
  
  \mode<presentation>{\vfill\pause}
  \item \textgood{Penalización de fallos} dependiente de \textmark{tiempo de acceso}
        y \textmark{tiempo de transferencia}.

  \mode<presentation>{\vfill\pause}
  \item Cuatro dimensiones clave en el \textgood{diseño de cachés}:
    \begin{itemize}
      \item 
        \textmark{Ubicación de bloque}, 
        \textmark{identificación de bloque},
        \textmark{remplazo de bloque}, y
        \textmark{estrategia de escritura}.
    \end{itemize}

  \mode<presentation>{\vfill\pause}
  \item Seis \textgood{optimizaciones básicas} de caché:
    \begin{itemize}
      \item \textmark{Reducir tasa de fallos}: 
        Aumentar tamaño de bloque,
        aumentar tamaño de caché,
        incrementar asociatividad.
      \item \textmark{Reducir penalización de fallos}:
        Cachés multi-nivel,
        dar prioridad a lecturas sobre escrituras.
      \item \textmark{Reducir tiempo de acierto}:
        Evitar traducción de direcciones.
    \end{itemize}
\end{itemize}
\end{frame}


\begin{frame}[t]{Referencias}
\begin{itemize}
  \item \bibhennessy
    \begin{itemize}
       \item Sección B.1 -- Introduction.
       \item Sección B.2 -- Cache Performance.
       \item Sección B.3 -- Size Cache Optimizations.
    \end{itemize}

  \item Otras lecturas:
    \begin{itemize}
      \item Exploring How Cache Memory Really Works:
            \textgood{\url{https://pikuma.com/blog/understanding-computer-cache}}.
    \end{itemize}
\end{itemize}
\end{frame}
