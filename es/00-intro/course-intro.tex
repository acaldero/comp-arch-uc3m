\section{Visión general}

\subsection{Objetivos y competencias}

\begin{frame}[t]{Asignatura}
\begin{itemize}
  \item \textmark{Objetivo}: que el estudiante conozca los conceptos básicos 
sobre la \textemph{arquitectura de un computador} y el impacto que estos tienen sobre 
el \textgood{rendimiento de las aplicaciones} y \textgood{los sistemas informáticos}.

\end{itemize}
\end{frame}

\begin{frame}[t]{Competencias}
\begin{itemize}
  \item Para alcanzar este \textgood{objetivo}, el estudiante profundizará en aspectos de las siguientes \textmark{competencias}:
    \begin{itemize}
      \mode<presentation>{\vfill\pause}
      \item Capacidad de conocer, comprender y evaluar la 
            \textgood{arquitectura de los computadores}, 
            así como los \textmark{componentes básicos} que los conforman.

      \mode<presentation>{\vfill\pause}
      \item Conocimiento y aplicación de los \textmark{principios fundamentales} y 
            \textmark{técnicas básicas} de la \textgood{programación paralela y concurrente}.

      \mode<presentation>{\vfill\pause}
      \item Capacidad de \textmark{analizar y evaluar} 
            \textgood{arquitecturas de computadores}, 
            incluyendo \textgood{plataformas paralelas}, así como desarrollar 
            y \textmark{optimizar software} para las mismas.
    \end{itemize}
\end{itemize}
\end{frame}

\begin{frame}[t]{Conocimientos previos}
\begin{itemize}
  \item Son necesarios conocimientos de:
    \begin{itemize}

      \mode<presentation>{\vfill}
      \item \textgood{Estructura de Computadores}.
        \begin{itemize}
          \item Programación en ensamblador.
          \item Procesador.
          \item Jerarquía de memoria.
        \end{itemize}

      \mode<presentation>{\vfill}
      \item \textgood{Sistemas Operativos}.
        \begin{itemize}
          \item Lenguaje de programación C.
          \item Procesos e hilos.
          \item Programación multi-hilo.
        \end{itemize}

      \mode<presentation>{\vfill}
      \item \textgood{Programación}.
        \begin{itemize}
          \item Técnicas de programación.
        \end{itemize}
    \end{itemize}
\end{itemize}
\end{frame}

\subsection{Programa de contenidos}

\begin{frame}[t]{Programa}
\begin{enumerate}
  \mode<presentation>{\vfill}
  \item Fundamentos del diseño de computadores.

  \mode<presentation>{\vfill}
  \item Evaluación del rendimiento de sistemas informáticos.

  \mode<presentation>{\vfill}
  \item Jerarquía de memoria.

  \mode<presentation>{\vfill}
  \item Paralelismo a nivel de instrucción. 

  \mode<presentation>{\vfill}
  \item Introducción a los multiprocesadores.

  \mode<presentation>{\vfill}
  \item Modelos de programación paralela y concurrente.
\end{enumerate}
\end{frame}
