\section{Visión general}

\subsection{Objetivos y competencias}

\begin{frame}[t]{Asignatura}
\begin{itemize}
  \item \bulletmark{Objetivo}: que el estudiante conozca los conceptos básicos 
sobre la \bulletgood{arquitectura de un computador} y el impacto que estos tienen sobre 
el \bulletgood{rendimiento de las aplicaciones} y los sistemas informáticos.

\end{itemize}
\end{frame}

\begin{frame}[t]{Competencias}
\begin{itemize}
  \item Para alcanzar este objetivo, el estudiante profundizará en aspectos de las siguientes competencias:
    \begin{itemize}
      \vspace{1.5em}
      \pause
      \item Capacidad de conocer, comprender y evaluar la \textbf{arquitectura de los computadores}, 
            así como los \textbf{componentes básicos} que los conforman.
      \vspace{1.5em}
      \pause
      \item Conocimiento y aplicación de los principios fundamentales y técnicas básicas de la \textbf{programación paralela y concurrente}.
      \vspace{1.5em}
      \pause
      \item Capacidad de analizar y evaluar arquitecturas de computadores, 
            incluyendo \textbf{plataformas paralelas}, así como desarrollar 
            y \textbf{optimizar software} para las mismas.
    \end{itemize}
\end{itemize}
\end{frame}

\begin{frame}[t]{Conocimientos previos}
\begin{itemize}
  \item Son necesarios conocimientos de:
    \begin{itemize}
      \item Estructura de Computadores.
      \item Sistemas Operativos.
      \item Programación.
    \end{itemize}
\end{itemize}
\end{frame}

\subsection{Programa de contenidos}

\begin{frame}[t]{Programa}
\begin{enumerate}
  \item Fundamentos del diseño de computadores.
  \item Evaluación del rendimiento de sistemas informáticos.
  \item Jerarquía de memoria.
  \item Paralelismo a nivel de instrucción. 
  \item Introducción a los multiprocesadores.
  \item Modelos de programación paralela y concurrente.
\end{enumerate}
\end{frame}
