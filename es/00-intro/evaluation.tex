\section{Evaluación}

\begin{frame}[t]{Sistema de evaluación}
\begin{itemize}
  \item Resumen:
  \vspace{1em}
    \begin{itemize}
      \item Examen final: 40\% de la calificación final.
        \begin{itemize}
          \item Incluye todos los contenidos (teoría, prácticas, y proyectos).
        \end{itemize}

      \item Evaluación continua: 60\% de la calificación final.
        \begin{itemize}
          \item Evaluaciones durante el curso: 25\% de la calificación final.
          \item Prácticas: 35\% de la calificación final.
        \end{itemize}
    \end{itemize}
  \vspace{1em}
  \begin{itemize}
    \item Convocatorias:
      \begin{itemize}
        \item Convocatoria ordinaria: Enero.
        \item Convocatoria extraordinaria: Junio.
      \end{itemize}
  \end{itemize}
\end{itemize}
\end{frame}

\begin{frame}[t]{Evaluación continua}
\begin{itemize}
  \item Obtener buen resultado en la evaluación continua es \alert{clave} para superar 
        la asignatura.
  \vspace{1em}
  \item Elementos:
    \begin{itemize}
      \item Evaluaciones durante el curso: 25\% de la calificación final.
      \item Prácticas: 35\% de la calificación final.
    \end{itemize}
  \vspace{1em}
  \item No has seguido la evaluación continua si:
    \begin{itemize}
      \item obtienes menos de 2.0 en una práctica/laboratorio.
    \end{itemize}
\end{itemize}
\end{frame}

\begin{frame}[t]{Convocatoria ordinaria: Evaluación continua}
\begin{itemize}
  \item Si sigues el proceso de evaluación continua:
    \begin{itemize}
    \item Examen final: 40\%.
      \begin{itemize}
        \item Mínimo necesario: 3.5.
      \end{itemize}
    \item Evaluaciones durante el curso: 25\%
      \begin{itemize}
        \item Mínimo necesario: \alert{No hay mínimo}.
      \end{itemize}
    \item Prácticas: 35\%.
      \begin{itemize}
        \item Mínimo en cada práctica: 2.0.
      \end{itemize}
    \item Si no logras algún mínimo, la media no se calcula y serás calificado como suspenso.
  \end{itemize}
  \item \alert{Bonus}:
    \begin{itemize}
      \item Se añadirá un punto adicional a la calificación final si:
        \begin{itemize}
          \item Obtienes al menos 7.0 puntos en la evaluación continua, y además
          \item obtienes al menos 6.0 puntos en el examen final.
        \end{itemize}
    \end{itemize}
\end{itemize}
\end{frame}

\begin{frame}[t]{Evaluaciones durante el curso}
\begin{itemize}
  \item Cuestionarios realizados en clase
  \pause\vfill
  \item Cada cuestionario puede incluir:
    \begin{itemize}
      \item Preguntas teóricas.
      \item Preguntas prácticas.
      \item Resolución de problemas.
    \end{itemize}
  \pause\vfill
  \item Las fechas de cada prueba están ya publicadas.
  \pause\vfill
  \item La calificación de esta parte será el promedio de las cuatro mejores calificaciones
        de los cuestionarios. 
    \begin{itemize}
      \item Una prueba no realizada se califica con 0 puntos.
    \end{itemize}
\end{itemize}
\end{frame}

\begin{frame}[t]{Calendario de pruebas de evaluación}
\begin{itemize}
  \item Todas las pruebas de evaluación se realizan en el grupo reducido
        en el que el estudiante está oficialmente matriculado.

  \vfill
  \item \bulletmark{Calendario}:
  \begin{itemize}
    \item Semana 4 (26/09 -- 30/09): Introducción, fundamentos y rendimiento.
    \item Semana 6 (10/10 -- 14/10): Jerarquía de memoria.
    \item Semana 10 (7/11 -- 11/11): Paralelismo a nivel de instrucción.
    \item Semana 13 (28/11 -- 2/12): Multiprocesadores.
    \item Semana 14 (28/11 -- 13/12): Programación paralela y concurrente.
  \end{itemize}
\end{itemize}
\end{frame}


\begin{frame}[t]{Prácticas/Laboratorios}
\begin{itemize}
  \item Cuatro laboratorios.
    \begin{itemize}
{\scriptsize
      \item Realización durante la clase práctica en grupos de 2 personas.
      \item Entrega de cuestionario o memoria durante un plazo
            establecido.
      \item Peso de cada laboratorio: 3\% de la nota final.
      \item Calificación mínima: 2 puntos sobre 10.
}
    \end{itemize}
  \vfill\pause
  \item Una práctica de programación paralela.
    \begin{itemize}
{\scriptsize
      \item Realizada usando OpenMP en lenguaje C++.
      \item Dos entregas.
        \begin{itemize}
          {\scriptsize
          \item Semana 7 (24/10): Optimizaciones secuenciales.
          \item Semana 11 (14/11): Optimizaciones paralelas.
          }
        \end{itemize}
      \item Grupos de 4 personas.
      \item El rendimiento será un criterio clave.
      \item Se valorará la calidad del código y las pruebas y evaluaciones realizadas.
      \item Muy importante la calidad de la memoria elaborada.
      \item Peso de la práctica: 23\% de la nota final.
      \item Calificación mínima: 2 puntos sobre 10.
}
    \end{itemize}
\end{itemize}
\end{frame}

\begin{frame}[t]{Sesiones de laboratorio}
\begin{itemize}
  \item 5 sesiones de laboratorio.

  \vfill
  \item \bulletmark{Calendario}:
    \begin{enumerate}
      \item Semana 2 (12/09 -- 16/09): Laboratorio C++ básico.
        \begin{itemize}
          \item Sin entrega asociada.
          \item Preparación de proyecto.
        \end{itemize}
      \item Semana 3 (19/09 -- 23/09): Laboratorio de memoria caché.
        \begin{itemize}
          \item Entrega en semana 4 (26/09).
        \end{itemize}
      \item Semana 5 (3/10 -- 7/10): Laboratorio de memoria caché.
        \begin{itemize}
          \item Entrega en semana 6 (10/10).
        \end{itemize}
      \item Semana 8 (24/10 -- 28/10): Laboratorio de OpenMP.
        \begin{itemize}
          \item Sin entrega asociada.
          \item Preparación de proyecto.
        \end{itemize}
      \item Semana 9 (31/10 -- 4/11): Laboratorio de ILP.
        \begin{itemize}
          \item Entrega en semana 10 (7/11).
        \end{itemize}
      \item Semana 12 (21/11 -- 25/11): Lab. consistencia memoria.
        \begin{itemize}
          \item Entrega en semana 13 (28/11).
        \end{itemize}
    \end{enumerate}
\end{itemize}
\end{frame}

\begin{frame}[t]{Convocatoria ordinaria: Evaluación NO-continua}
\begin{itemize}
  \item Si no has seguido el proceso de evaluación continua:
    \begin{itemize}
      \item El examen final tiene un valor del 60\% de la calificación final.
      \item Necesitarás 8.33 en ele examen final para superar la asignatura.
    \end{itemize}

  \mode<presentation>{\vfill}
  \item \alert{CONSEJO}:
    \begin{itemize}
      \item Pon esfuerzo en seguir el proceso de evaluación continua.
    \end{itemize}
\end{itemize}
\end{frame}


\begin{frame}[t]{Convocatoria extraordinaria}
\begin{itemize}
  \item Examen extraordinario en el mes de junio.
  \vspace{1em}
  \item Normas:
    \begin{enumerate}
      \item Estudiantes que han completado el proceso de evaluación continua:
        \begin{itemize}
          \item El examen extraordinario vale el 40\% y la evaluación continua el otro 60\%.
          \item Solamente se aplica si la calificación en el examen es de al menos 3.5.
        \end{itemize}
      \item Estudiantes que no han completado el proceso de evaluación continua:
        \begin{itemize}
          \item El examen vale el 100\%.
        \end{itemize}
    \end{enumerate}
    \begin{itemize}
      \item A los estudiantes que hayan completado el proceso de evaluación continua se tomará la opción más favorable.
    \end{itemize}
\end{itemize}
\end{frame}

\begin{frame}[t]{Pruebas de evaluación}
\begin{itemize}
  \item Todas las pruebas de evaluación deberá realizarse en el grupo en que el estudiante se encuentra oficialmente matriculado.
    \begin{itemize}
      \item No se admitirán cambios de grupo que no se realicen oficialmente.
    \end{itemize}
  \item \alert{MUY IMPORTANTE}:
    
\begin{itemize}
      \item La no asistencia al examen final implica la calificación como NO-PRESENTADO, independientemente de cualquier otra calificación.
    \end{itemize}
\end{itemize}
\end{frame}
