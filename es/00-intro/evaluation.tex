\section{Evaluación}

\begin{frame}[t]{Sistema de evaluación}
\begin{itemize}
  \item \textmark{Resumen}:

  \mode<presentation>{\vfill}
  \begin{itemize}
    \item \textgood{Examen final}: 40\% de la calificación final.
      \begin{itemize}
        \item Incluye todos los contenidos (teoría, prácticas, y proyectos).
      \end{itemize}

    \mode<presentation>{\vfill}
    \item \textgood{Evaluación continua}: 60\% de la calificación final.
      \begin{itemize}
        \item \textmark{Exámenes parciales}: 24\% de la calificación final.
        \item \textmark{Exámenes de laboratorio}: 12\% de la calificación final.
        \item \textmark{Proyecto}: 24\% de la calificación final.
      \end{itemize}
  \end{itemize}

  \mode<presentation>{\vfill}
  \begin{itemize}
    \item \textemph{Convocatorias}:
      \begin{itemize}
        \item \textmark{Ordinaria}: Enero.
        \item \textmark{Extraordinaria}: Junio.
      \end{itemize}
  \end{itemize}
\end{itemize}
\end{frame}

\begin{frame}[t]{Evaluación continua}
\begin{itemize}
  \item Obtener buen resultado en la evaluación continua es \textemph{clave} para superar 
        la asignatura.
  
  \mode<presentation>{\vfill}
  \item \textgood{Elementos}:
    \begin{itemize}
      \item \textmark{Exámenes parciales}: 24\% de la calificación final.
        \begin{itemize}
          \item 2 exámenes $\Rightarrow$ 12\% cada uno.
        \end{itemize}
      \item \textmark{Exámenes de laboratorio}: 12\% de la calificación final.
        \begin{itemize}
          \item Lab exam 1: 4\%.
          \item Lab exam 2: 8\%.
        \end{itemize}
      \item \textmark{Proyecto}: 24\% de la calificación final.
    \end{itemize}
  
  \mode<presentation>{\vfill}
  \item \textbad{No has seguido} la evaluación continua si:
    \begin{itemize}
      \item obtienes menos de 2.0 en algún examen de laboratorio.
      \item obtienes menos de 2.0 en el proyecto.
    \end{itemize}
\end{itemize}
\end{frame}

\begin{frame}[t]{Convocatoria ordinaria: Evaluación continua}
\begin{itemize}
  \item Si sigues el proceso de \textmark{evaluación continua}:
    \begin{itemize}
    \item \textgood{Examen final}: 40\%.
      \begin{itemize}
        \item Mínimo necesario: \textbad{3.5}.
      \end{itemize}
    \item \textgood{Exámenes parciales}: 24\%
      \begin{itemize}
        \item Mínimo necesario: \alert{No hay mínimo}.
      \end{itemize}
    \item \textgood{Laboratorios}: 12\%.
      \begin{itemize}
        \item Mínimo necesario: \textbad{2.0} en cada examen.
      \end{itemize}
    \item \textgood{Proyecto}: 24\%.
      \begin{itemize}
        \item Mínimo necesario: \textbad{2.0}.
      \end{itemize}
  \end{itemize}

  \mode<presentation>{\vfill}
  \item \textbad{IMPORTANTE}: Si no logras algún mínimo, 
        la media no se calcula y serás calificado como \textbad{suspenso}.
\end{itemize}
\end{frame}

\begin{frame}[t]{Mejora tu calificación}
\begin{itemize}
  \item \textemph{Bonus}:
    \begin{itemize}
      \item Se añadirá un punto adicional a la calificación final si:
        \begin{itemize}
          \item Obtienes al menos 7.0 puntos en la evaluación continua, y además
          \item obtienes al menos 6.0 puntos en el examen final.
        \end{itemize}
    \end{itemize}
\end{itemize}
\end{frame}

\begin{frame}[t]{Evaluaciones durante el curso}
\begin{itemize}
  \item Exámenes realizados \textmark{durante clase}.

  \mode<presentation>{\pause\vfill}
  \item Cada examene \textmark{puede incluir}:
    \begin{itemize}
      \item Preguntas teóricas.
      \item Preguntas prácticas.
      \item Resolución de problemas.
    \end{itemize}

  \mode<presentation>{\pause\vfill}
  \item Las fechas de cada prueba están \textemph{ya publicadas}.

  \mode<presentation>{\pause\vfill}
  \item Cada examen tiene un valor de un 12\% de la calificación final.
    \begin{itemize}
      \item No presentado $\Rightarrow$ 0.
    \end{itemize}
\end{itemize}
\end{frame}

\begin{frame}[t]{Calendario de pruebas de evaluación}
\begin{itemize}
  \item Todas las pruebas de evaluación se realizan en el grupo reducido
        en el que el estudiante está \textbad{oficialmente matriculado}.

  \vfill
  \item \textmark{Calendario}:
  \begin{itemize}
    \item Semana 7 (14/10 -- 18/10): 
      \begin{itemize}
        \item Introducción y fundamentos del diseño computadores.
        \item Rendimiento energía y fiabilidad.
        \item Jerarquía de memoria.
      \end{itemize}

    \item Semana 11 (18/11 -- 22/11):
      \begin{itemize}
        \item Paralelismo a nivel de instrucción.
        \item Arquitecturas de memoria compartida.
        \item Memoria compartida distribuida.
      \end{itemize}
  \end{itemize}
\end{itemize}
\end{frame}


\begin{frame}[t]{Laboratorios}
\begin{itemize}
  \item \textmark{SEIS sesiones de laboratorio}:
    \begin{itemize}
      \item Realizadas durante clases de laboratoiro en grupos de 2 estudiantes.
      \item Sin entrega de resultados.
      \item \textbad{IMPORTANTE}: Exámenes de laboratorio sobre trabajo realizado.
    \end{itemize}

  \mode<presentation>{\vfill}
  \item \textemph{Sesiones de laboratorio}:
    \begin{itemize}
      \item Semana 2 (16/09 -- 20/09): \textgood{C++ básico}.
      \item Semana 3 (23/09 -- 27/09): \textgood{Rendimiento y energía}.
      \item Semana 5 (07/10 -- 11/10): \textgood{Memoria caché}.
      \item Semana 8 (28/10 -- 1/11): \textgood{Introducción a OpenMP}.
      \item Semana 10 (11/11 -- 15/11): \textgood{OpenMP}.
      \item Semana 13 (2/12 -- 06/12): \textgood{Consistencia de memoria}.
    \end{itemize}
\end{itemize}
\end{frame}

\begin{frame}[t]{Calendario de exámenes de laboratorio}
\begin{itemize}
  \item Todos los exámenes realizados en el grupo reducido
        en el que el estudiante está \textbad{oficialmente matriculado}.

  \vfill
  \item \textmark{Calendario}:
  \begin{itemize}
    \item Semana 7 (14/10 -- 18/10): 4\% de la calificación final
      \begin{itemize}
        \item C++ básico.
        \item Rendimiento y energía.
      \end{itemize}
    \item Semana 11 (18/11 -- 22/11): 8\% de la calificación final.
      \begin{itemize}
        \item Memoria caché.
        \item Programación paralela en OpenMP.
      \end{itemize}
  \end{itemize}

  \mode<presentation>{\vfill}
  \item \textmark{Nota}: Estos exámenes son en la misma sesión que 
        los exámenes parciales.
\end{itemize}
\end{frame}

\begin{frame}[t]{Proyectos}
\begin{itemize}
  \item Un proyecto de programación orientada al rendimiento y la energía.
    \begin{itemize}
      \item Realizado usando C++.
      \item Entrega única.
        \begin{itemize}
          \item Semana 9 (8/11).
        \end{itemize}
      \item Grupos de 4 personas.
    \end{itemize}

  \item \textgood{Evaluación}:
    \begin{itemize}
      \item El \textmark{rendimiento} y el \textmark{consumo energético} serán criterios clave.
      \item Se valorará la \textemph{calidad del código}, las 
            \textemph{pruebas} y las \textemph{evaluaciones de rendimiento/energía}.
      \item Muy importante la \textmark{calidad de la memoria} elaborada.
      \item Evaluación de la \textbad{contribución de cada participante}.
      \item \textgood{Peso}: 24\% de la nota final.
      \item \textbad{Calificación mínima}: \textbad{2.0}.
    \end{itemize}
\end{itemize}
\end{frame}

\begin{frame}[t]{Reglas y resolución de conflictos}
\begin{itemize}
  \item \textemph{Reglas}:
    \begin{itemize}
      \item El proyecto está diseñado para equipos de 4 integrantes.
      \item Se pueden permitir grupos más pequeños sin reducción de carga de trabajo.
      \item Las contribuciones individuales se deben declarar en el informe.
        \begin{itemize}
          \item Una contribución no puede tener más de un autor.
        \end{itemize}
    \end{itemize}

  \mode<presentation>{\vfill\pause}
  \item \textbad{Conflictos}:
    \begin{itemize}
      \item Si hay un conflicto se comunicará al coordinador lo antes posible.
      \item Si no puede resolverse, el coordinador podrá:
        \begin{itemize}
          \item Expulsar a una persona del equipo.
          \item Dividir el equipo.
          \item En ningún caso se reducirá la carga de trabajo.
        \end{itemize}
    \end{itemize}
\end{itemize}
\end{frame}


\begin{frame}[t]{Convocatoria ordinaria: Evaluación NO-continua}
\begin{itemize}
  \item Si \textbad{no has seguido} el proceso de \textgood{evaluación continua}:
    \begin{itemize}
      \item El \textgood{examen final} tiene un valor del 
            \textbad{60\%} de la calificación final.
      \item Necesitarás \textbad{8.33} en el \textgood{examen final} 
            para superar la asignatura.
    \end{itemize}

  \mode<presentation>{\vfill}
  \item \textbad{CONSEJO}:
    \begin{itemize}
      \item Pon esfuerzo en seguir el proceso de evaluación continua.
    \end{itemize}
\end{itemize}
\end{frame}


\begin{frame}[t]{Convocatoria extraordinaria}
\begin{itemize}
  \item \textgood{Examen extraordinario} en el mes de junio.

  \mode<presentation>{\vfill}
  \item \textmark{Normas}:
    \begin{enumerate}
      \item Estudiantes que \textgood{han completado} el proceso de evaluación continua:
        \begin{itemize}
          \item El examen extraordinario vale el \textbad{40\%} 
                y la evaluación continua el otro \textbad{60\%}.
          \item \textbad{Solamente se aplica} si la calificación en el examen es de 
                al menos \textbad{3.5}.
        \end{itemize}
      \item Estudiantes que \textbad{no han completado} el proceso de evaluación continua:
        \begin{itemize}
          \item El examen vale el \textbad{100\%}.
        \end{itemize}
    \end{enumerate}

    \mode<presentation>{\vfill}
    \begin{itemize}
      \item Para los estudiantes que hayan completado el proceso de evaluación continua 
            se tomará \textgood{la opción más favorable}.
    \end{itemize}
\end{itemize}
\end{frame}

\begin{frame}[t]{Pruebas de evaluación}
\begin{itemize}
  \item Todas las pruebas de evaluación \textbad{deberán realizarse} 
        en el grupo en que el estudiante se encuentra \textmark{oficialmente matriculado}.
    \begin{itemize}
      \item No se admitirán cambios de grupo que no se realicen oficialmente.
    \end{itemize}

  \mode<presentation>{\vfill}
  \item \textbad{MUY IMPORTANTE}:   
\begin{itemize}
      \item La no asistencia al examen final 
            implica la calificación como \textbad{NO-PRESENTADO}, 
            independientemente de cualquier otra calificación.
    \end{itemize}
\end{itemize}
\end{frame}
