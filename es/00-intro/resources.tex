\section{Recursos}

\begin{frame}[t]{Bibliografía}
\begin{itemize}
  \item \textgood{Bibliografía básica}:
    \begin{itemize}
      \item \textmark{Computer Architecture: A quantiative approach, 6th Edition}. 
            Hennessy, JL and Patterson, DA. 
            Morgan Kaufmann, 2017.
    \end{itemize}
  \vspace{1em}
  \pause
  \item \textgood{Bibliografía complementaria}:
    \begin{itemize}
      \item \textmark{Computer Organization and Design, MIPS Edition: The Hardware Software Interface}. 
            Patterson, DA; Hennessy, JL. 
            Morgan Kaufmann, 2020.
      \item \textmark{The OpenMP Common Core: Making OpenMP Simple Again}.
            Mattson, TG; He, Y.; Koniges, A.E.
            MIT Press, 2019. 
      \item \textmark{C++ Concurrency in Action. Practical Multithreading, 2nd Edition} 
            Williams, A.  
            Manning. 2018.
    \end{itemize}
\end{itemize}
\end{frame}

\begin{frame}[t]{Otro material}
\begin{itemize}
  \item Los materiales usados en clase se publicarán a través de \textgood{Aula Global}.

  \mode<presentation>{\vfill}
  \item \textbad{AVISO MUY IMPORTANTE}:
    \begin{itemize}
      \item Las transparencias y otros materiales publicados mediante \textgood{Aula Global} solamente son un \textemph{guión de clase}.
        \begin{itemize}
          \item \textmark{No son los materiales del curso}.
        \end{itemize}
      \item El conocimiento de los contenidos de dichos guiones 
            \textbad{puede ser insuficiente} para alcanzar los objetivos de la asignatura.
        \begin{itemize}
          \item \textmark{Es muy probable que suspendas si no haces más}.
        \end{itemize}
      \item Es \textemph{altamente recomendable} usar, estudiar y trabajar con la bibliografía básica y complementaria.
        \begin{itemize}
          \item \textmark{p. ej.: resolución individual de ejercicios y libros}.
        \end{itemize}
    \end{itemize}
\end{itemize}
\end{frame}
