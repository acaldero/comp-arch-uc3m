\subsection{Vectores}

\begin{frame}{Colecciones de valores}
\begin{itemize}
  \item \cppid{vector} permite almacenar y procesar un conjunto de valores
  de un mismo tipo.
    \mode<article>{
      \begin{itemize}
        \item C++ también tiene \emph{arrays} pero son demasiado limitados
          y simples. Se revisará su uso más adelante.
      \end{itemize}
    }
  \item Un \cppid{vector}:
    \begin{itemize}
      \item Tiene una secuencia de elementos.
      \item Se puede acceder a los elementos por su índice.
      \item Incluye información de su tamaño.
    \end{itemize}
\begin{tikzpicture}
\tikzset{
    bloque/.style={rectangle,draw=black, top color=white, bottom color=blue!50,
                   very thick, inner sep=0.5em, minimum size=0.6cm, text centered, font=\tiny},
    flecha/.style={->, >=latex', shorten >=1pt, thick},
    etiqueta/.style={text centered, font=\tiny} 
}  
\node[bloque] (bsize) {4};
\node[bloque,right=0cm of bsize] (bptr) { };
\node[bloque,below right=0.5cm and 0.75cm of bptr] (v0) {1};
\node[bloque,right=0cm of v0] (v1) {2};
\node[bloque,right=0cm of v1] (v2) {4};
\node[bloque,right=0cm of v2] (v3) {8};
\draw[flecha] (bptr) -- (v0);
\node[etiqueta, left=0.1cm of bsize] {v:};
\node[etiqueta, above=0cm of bsize] {size()};
\node[etiqueta, above=0cm of v0] {v[0]};
\node[etiqueta, above=0cm of v1] {v[1]};
\node[etiqueta, above=0cm of v2] {v[2]};
\node[etiqueta, above=0cm of v3] {v[3]};
\end{tikzpicture}
  \item Alternativa a usar arrays directamente.
\end{itemize}
\end{frame}

\begin{frame}{Uso básico}
\vspace{-1.5em}
\begin{columns}[t]

\column{.5\textwidth}
\begin{block}{Uso de vector}
\lstinputlisting{int/m6-01-cppintro/vector/vec1.cpp}
\end{block}

\column{.5\textwidth}
\begin{itemize}
  \item Archivo de cabecera: \cppid{<vector>}

  \mode<presentation>{\vspace{1em}}
  \item Se debe indicar el tipo del elemento.
    \begin{itemize}
      \item Todos del mismo tipo.
    \end{itemize}

  \mode<presentation>{\vspace{1em}}
  \item Parámetro del constructor: \textmark{Tamaño inicial}.

  \mode<presentation>{\vspace{1em}}
  \item No se puede acceder a indices más allá del tamaño (inclusive).
    \begin{itemize}
      \item \textbad{Comportamiento no definido}.
    \end{itemize}
\end{itemize}
\end{columns}
\end{frame}

\begin{frame}
\begin{block}{Vectores y tipos}
\lstinputlisting{int/m6-01-cppintro/vector/vec2.cpp}
\end{block}
\end{frame}

\mode<article>{
\begin{itemize}
  \item Cada vector debe indicar su tipo de elemento.
  \item Se realiza comprobación de tipos.
    \begin{itemize}
      \item No se pueden mezclar tipos.
    \end{itemize}
\end{itemize}
}

\begin{frame}[t,fragile]{Vectores e iniciación}
\begin{itemize}
  \item Un vector con tamaño inicia todos sus valores al valor por defecto del tipo.
    \begin{itemize}
      \item Valores numéricos: \mode<presentation>{\cppid{0}}
      \item Valores de cadena: \mode<presentation>{\cppid{``''}}
    \end{itemize}

  \vfill
  \item Si no se indica tamaño inicial, el vector tiene tamaño \cppid{0}.
\begin{lstlisting}
vector<double> v; // Vector con 0 elementos
\end{lstlisting}

  \vfill
  \item Se puede suministrar un valor inicial distinto.
\begin{lstlisting}
vector<double> v(100, 0.5); // 100 posiciones iniciadas a 0.5
\end{lstlisting}
\end{itemize}
\end{frame}

\begin{frame}
\begin{block}{Iniciación en la declaración}
\lstinputlisting{int/m6-01-cppintro/vector/vec3.cpp}
\end{block}
\end{frame}

\begin{frame}[fragile]{Vectores que crecen}
\begin{itemize}
  \item Un \cppid{vector} puede \emph{crecer} cuando se añaden elementos.
    \begin{itemize}
      \item Operación \cppid{push\_back()}: \mode<presentation>{Añade un elemento al final del vector.}
    \end{itemize}
\end{itemize}
\pause
\begin{lstlisting}
vector<int> v;
\end{lstlisting}
\mode<presentation>{\vspace{-.5em}}
\begin{tikzpicture}
\tikzset{
    bloque/.style={rectangle,draw=black, top color=white, bottom color=blue!50,
                   very thick, inner sep=0.5em, minimum size=0.6cm, text centered, font=\tiny},
    flecha/.style={->, >=latex', shorten >=1pt, thick},
    etiqueta/.style={text centered, font=\tiny} 
}  
\node[bloque] (bsize) {0};
\node[bloque,right=0cm of bsize] (bptr) { };
\node[etiqueta, left=0.1cm of bsize] {v:};
\node[etiqueta, above=0cm of bsize] {size()};
\end{tikzpicture}
\pause
\begin{lstlisting}
v.push_back(1);
\end{lstlisting}
\mode<presentation>{\vspace{-.5em}}
\begin{tikzpicture}
\tikzset{
    bloque/.style={rectangle,draw=black, top color=white, bottom color=blue!50,
                   very thick, inner sep=0.5em, minimum size=0.6cm, text centered, font=\tiny},
    flecha/.style={->, >=latex', shorten >=1pt, thick},
    etiqueta/.style={text centered, font=\tiny} 
}  
\node[bloque] (bsize) {1};
\node[bloque,right=0cm of bsize] (bptr) { };
\node[bloque,right=0.75cm of bptr] (v0) {1};
\draw[flecha] (bptr) -- (v0);
\node[etiqueta, left=0.1cm of bsize] {v:};
\node[etiqueta, above=0cm of bsize] {size()};
\node[etiqueta, above=0cm of v0] {v[0]};
\end{tikzpicture}
\pause
\begin{lstlisting}
v.push_back(4);
\end{lstlisting}
\mode<presentation>{\vspace{-.5em}}
\begin{tikzpicture}
\tikzset{
    bloque/.style={rectangle,draw=black, top color=white, bottom color=blue!50,
                   very thick, inner sep=0.5em, minimum size=0.6cm, text centered, font=\tiny},
    flecha/.style={->, >=latex', shorten >=1pt, thick},
    etiqueta/.style={text centered, font=\tiny} 
}  
\node[bloque] (bsize) {2};
\node[bloque,right=0cm of bsize] (bptr) { };
\node[bloque,right=0.75cm of bptr] (v0) {1};
\node[bloque,right=0cm of v0] (v1) {4};
\draw[flecha] (bptr) -- (v0);
\node[etiqueta, left=0.1cm of bsize] {v:};
\node[etiqueta, above=0cm of bsize] {size()};
\node[etiqueta, above=0cm of v0] {v[0]};
\node[etiqueta, above=0cm of v1] {v[1]};
\end{tikzpicture}
\end{frame}

\begin{frame}[t,fragile]{Recorrido de un vector}
\begin{itemize}
  \item Se puede consultar el tamaño de un vector mediante la \emph{función miembro} \cppid{size}.
\begin{lstlisting}
cout << v.size();
\end{lstlisting}

  \vfill
  \item \cppid{size()} permite definir un bucle para recorrer los elementos de un vector.
\begin{lstlisting}
for (int i=0; i<v.size(); ++i) {
  cout << "v[" << i << "] = " << v[i] << "\n";
}
\end{lstlisting}

\end{itemize}
\end{frame}

\begin{frame}[t,fragile]{Recorrido basado en rango}
\begin{itemize}
  \item Se puede usar un recorrido basado en rango para un vector.
\begin{lstlisting}
vector<int> v1 { 1, 2, 3, 4 };
for (auto x : v1) {
  cout << x << "\n";
}

vector<string> v2 { "Carlos", "Daniel", "José", "Manuel" };
for (auto x : v2) {
  cout << x << "\n";
}
\end{lstlisting}
\end{itemize}
\end{frame}

\begin{frame}[t]{Ejemplo: Estadísticas}
\begin{itemize}
  \item \alert{Objetivo}: Leer de la entrada estándar una secuencia de calificaciones
    y volcar en la salida estándar la calificación mínima, la máxima y la calificación media.
    \begin{itemize}
      \item Finalizar la lectura si se llega a fin de fichero.
      \item Finalizar la lectura si no se lee un valor correctamente (p. ej. letras en lugar de números).
      \item Se desconoce (y  no se pregunta) el número de valores.
    \end{itemize}
\end{itemize}
\end{frame}

\mode<presentation> {

\begin{frame}
\begin{columns}[T]
\column{0.5\textwidth}
\begin{block}{notas.cpp}
\lstinputlisting[lastline=16]{int/m6-01-cppintro/vector/marks.cpp}
\ldots
\end{block}

\column{0.5\textwidth}
\begin{block}{notas.cpp}
\ldots
\lstinputlisting[firstline=17]{int/m6-01-cppintro/vector/marks.cpp}
\end{block}
\end{columns}
\end{frame}

}

\mode<article> {
\begin{frame}
\begin{block}{notas.cpp}
\lstinputlisting{int/m6-01-cppintro/vector/notas.cpp}
\end{block}
\end{frame}
}

\begin{frame}[t]{Ejemplo: Palabras únicas}
\begin{itemize}
  \item \alert{Objetivo}: Volcar la lista ordenada de palabras únicas de un texto.
    \begin{itemize}
      \item El texto se lee de la entrada estándar hasta fin de fichero.
      \item La lista de palabras se imprime en la salida estándar.
    \end{itemize}
\end{itemize}
\end{frame}


\mode<presentation> {

\begin{frame}
\begin{columns}[T]

\column{0.5\textwidth}
\begin{block}{unique.cpp}
\lstinputlisting[lastline=15]{int/m6-01-cppintro/vector/unique.cpp}
\ldots
\end{block}

\column{0.5\textwidth}
\begin{block}{unique.cpp}
\ldots
\lstinputlisting[firstline=16]{int/m6-01-cppintro/vector/unique.cpp}
\end{block}
\end{columns}
\end{frame}

}

\mode<article> {

\begin{frame}
\begin{block}{unique.cpp}
\lstinputlisting{int/m6-01-cppintro/vector/unique.cpp}
\end{block}
\end{frame}

}
