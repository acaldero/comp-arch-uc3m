\section{Técnicas de compilación e ILP}

\begin{frame}[t]{Aprovechamiento de ILP}
\begin{itemize}
  \item ILP directamente aplicables a bloques básicos.
    \begin{itemize}
      \item \textgood{Bloque básico}: secuencia de instrucciones sin saltos.
      \item Programa \textmark{típico} en procesadores RISC:
        \begin{itemize}
          \item Tamaño medio de bloque básico de 3 a 6 instrucciones.
          \item Poco aprovechamiento de ILP dentro del bloque.
        \end{itemize}
      \item Es necesario explotar ILP entre bloques básicos.
    \end{itemize}
\end{itemize}

\mode<presentation>{\vfill\pause}
\begin{columns}

  \begin{column}{.4\textwidth}
    \begin{block}{Ejemplo}
      \lstinputlisting{int/m4-02-ilp/comp-ilp-ex1.cpp}
    \end{block}
  \end{column}

  \begin{column}{.6\textwidth}
    \begin{itemize}
      \item \textgood{Paralelismo a nivel de bucle}.
        \begin{itemize}
          \item Transformable a ILP.
          \item Por compilador o hardware.
        \end{itemize}
      \item \textgood{Alternativa}:
        \begin{itemize}
          \item Instrucciones vectoriales.
          \item Instrucciones SIMD en procesador.
        \end{itemize}
    \end{itemize}
  \end{column}
\end{columns}
\end{frame}

\begin{frame}[t]{Planificación y desenrollamiento de bucles}
\begin{itemize}
  \item \textgood{Aprovechamiento de paralelismo}:
    \begin{itemize}
      \item Entrelazar ejecución de instrucciones no relacionadas.
      \item Rellenar detenciones con instrucciones.
      \item No alterar los efectos del programa original.
    \end{itemize}

\mode<presentation>{\vfill\pause}
  \item El compilador puede usar el conocimiento detallado de la arquitectura.
\end{itemize}
\end{frame}

\begin{frame}[t]{Aprovechamiento de ILP}
\begin{columns}

\begin{column}{.4\textwidth}
\begin{block}{Ejemplo}
\lstinputlisting{int/m4-02-ilp/comp-ilp-ex2.cpp}
\end{block}
\end{column}

\begin{column}{.6\textwidth}
\begin{itemize}
  \item Cuerpo de cada iteración es independiente.
\end{itemize}
\end{column}

\end{columns}

\mode<presentation>{\vfill}
\begin{block}{Latencias entre instrucciones}
{\footnotesize
\begin{tabular}{|*{2}{p{.3\textwidth}|}p{0.25\textwidth}|}

\hline
Instrucción que &
Instrucción que &
Latencia (ciclos)
\\
produce resultado &
usa resultado &
\\
\hline
\hline

Operación ALU FP & Operación ALU FP & 3\\
\hline

Operación ALU FP & Almacenar doble & 2\\
\hline

Cargar doble & Operación ALU FP & 1\\
\hline

Cargar doble & Almacenar doble & 0\\
\hline

\end{tabular}
}
\end{block}

\end{frame}

\begin{frame}[t]{Código compilado}
\begin{itemize}
  \item \asmreg{x1} $\rightarrow$ Último elemento de array.
  \item \asmreg{f2} $\rightarrow$ Escalar \cppid{s}.
  \item \asmreg{x2} $\rightarrow$ Precalculado para que \asmlabel{8(x2)} sea primer elemento de array.
\end{itemize}

\mode<presentation>{\vfill}
\begin{block}{Código ensamblador}
\lstinputlisting[language=generalasm2]{int/m4-02-ilp/comp-ilp-ex3.asm}
\end{block}

\end{frame}

\begin{frame}[t]{Detenciones en la ejecución}

\begin{columns}[T]

\begin{column}{.45\textwidth}
\begin{block}{Original}
\lstinputlisting[language=generalasm2]{int/m4-02-ilp/comp-ilp-ex4.asm}
\end{block}
\end{column}

\pause
\begin{column}{.45\textwidth}
\begin{block}{Detenciones}
\lstinputlisting[language=generalasm2]{int/m4-02-ilp/comp-ilp-ex4b.asm}
\end{block}
\end{column}

\end{columns}
\end{frame}

\begin{frame}[t]{Planificación de bucle}

\begin{columns}[T]

\begin{column}{.45\textwidth}
\begin{block}{Original}
\lstinputlisting[language=generalasm2]{int/m4-02-ilp/comp-ilp-ex4b.asm}
\end{block}

\begin{itemize}
  \item 8 ciclos por elemento.
\end{itemize}
\end{column}

\pause
\begin{column}{.45\textwidth}
\begin{block}{Planificado}
\lstinputlisting[language=generalasm2]{int/m4-02-ilp/comp-ilp-ex4c.asm}
\end{block}
\begin{itemize}
  \item 7 ciclos por elemento.
\end{itemize}
\end{column}

\end{columns}
\end{frame}

\begin{frame}[t]{Desenrollamiento de bucles}
\begin{itemize}
  \item \textgood{Idea}:
    \begin{itemize}
      \item Replicar varias veces el cuerpo del bucle.
      \item Ajustar el código de terminación.
      \item Usa registros distintos para cada réplica para reducir dependencias.
    \end{itemize}

  \mode<presentation>{\vfill}
  \item \textmark{Efecto}:
    \begin{itemize}
      \item Aumenta la longitud del bloque básico.
      \item Incrementa el aprovechamiento ILP disponible.
    \end{itemize}
\end{itemize}
\end{frame}

\begin{frame}[t]{Desenrollamiento}

\begin{columns}

\begin{column}{.45\textwidth}
\begin{block}{Desenrollado (x4)}
\lstinputlisting[language=generalasm2,lastline=7]{int/m4-02-ilp/loop-unroll0.asm}
\end{block}
\end{column}

\begin{column}{.45\textwidth}
\begin{block}{Desenrollado (x4)}
\lstinputlisting[language=generalasm2,firstline=8]{int/m4-02-ilp/loop-unroll0.asm}
\end{block}
\end{column}

\end{columns}
\mode<presentation>{\vfill}
\begin{itemize}
  \item 4 iteraciones requieren más registros.
  \item Este ejemplo asume tamaño de array múltiplo de 4.
\end{itemize}
\end{frame}

\begin{frame}[t,shrink=10]{Detenciones y desenrollamiento}

\mode<presentation>{\vspace{-1em}}
\begin{columns}[T]

\begin{column}{.45\textwidth}
\begin{block}{Desenrollado (x4)}
\lstinputlisting[language=generalasm2,firstline=1,lastline=14]{int/m4-02-ilp/loop-unroll1.asm}
\end{block}
\end{column}

\begin{column}{.45\textwidth}
\begin{block}{Desenrollado (x4)}
\lstinputlisting[language=generalasm2,firstline=15]{int/m4-02-ilp/loop-unroll1.asm}
\end{block}
\end{column}

\end{columns}
\mode<presentation>{\vfill}
\begin{itemize}
  \item 26 ciclos por 4 iteraciones $\rightarrow$ $6.5$ ciclos por elemento.
\end{itemize}
\end{frame}

\begin{frame}[t]{Planificación y desenrollamiento}

\begin{columns}

\begin{column}{.45\textwidth}
\begin{block}{Desenrollado (x4)}
\lstinputlisting[language=generalasm2]{int/m4-02-ilp/loop-unroll2.asm}
\end{block}
\end{column}

\begin{column}{.45\textwidth}
\begin{itemize}
  \item Reorganización de código.
    \begin{itemize}
      \item Respetando dependencias.
      \item Semánticamente equivalente.
      \item \textgood{Objetivo}: Aprovechar \emph{stalls}.
    \end{itemize}
  \item Actualización de \asmreg{x1} suficientemente separada de
        \asminst{BNE}.
  \item 14 ciclos por 4 iteraciones $\rightarrow$ $3.5$ ciclos por elemento.
\end{itemize}
\end{column}

\end{columns}

\end{frame}

\begin{frame}[t]{Límites del desenrollamiento de bucles}
\begin{itemize}
  \item \textgood{Decremento} de la \textmark{ganancia} con cada desenrollamiento.
    \begin{itemize}
      \item Ganancia limitada a eliminación de detenciones.
      \item Sobrecoste se reparte entre iteraciones.
    \end{itemize}

  \mode<presentation>{\vfill\pause}
  \item \textgood{Crecimiento} en \textmark{tamaño} de código.
    \begin{itemize}
      \item Puede afectar a la tasa de fallos de la caché de instrucciones.
    \end{itemize}

  \mode<presentation>{\vfill\pause}
  \item \textgood{Presión} sobre \textmark{banco registros}.
    \begin{itemize}
      \item Puede generar falta de registros.
      \item Si no hay registros suficientes se pierden las ventajas.
    \end{itemize}

\end{itemize}
\end{frame}

