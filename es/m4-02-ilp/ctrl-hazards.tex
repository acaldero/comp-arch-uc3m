\section{Riesgos de control}

\begin{frame}[t]{Riesgos de control}
\begin{itemize}
  \item Un \textgood{riesgo de control} se produce en una 
        instrucción de alteración del PC.
    \mode<presentation>{\vfill}
    \begin{itemize}
      \item \textgood{Terminología}:
        \begin{itemize}
          \item \textmark{Bifurcación tomada}: Si se modifica el PC.
          \item \textmark{Bifurcación no tomada}: Si no se modifica el PC.
        \end{itemize}
      \mode<presentation>{\vfill}
      \item \alert{Problema}:
        \begin{itemize}
          \item La segmentación asume que lo bifurcación \textbf{no} se tomará.
          \item ¿Qué hacer si después de la etapa ID se determina que hay que tomar la bifurcación?
        \end{itemize}
    \end{itemize}
\end{itemize}
\end{frame}

\begin{frame}[t]{Alternativas ante riesgos de control}
\begin{itemize}
  \item \textgood{Tiempo de compilación}: Prefijadas durante toda la ejecución del programa.
    \begin{itemize}
      \item El software puede intentar minimizar su impacto si conoce el comportamiento del hardware.
      \item El compilador puede hacer este trabajo.
    \end{itemize}
  \mode<presentation>{\vfill\pause}
  \item \textgood{Alternativas en tiempo de ejecución}: Comportamiento variable durante la ejecución del programa.
    \begin{itemize}
      \item Intentan predecir qué hará el software.
    \end{itemize}
\end{itemize}
\end{frame}

\begin{frame}[t]{Riesgos de control: soluciones estáticas}
\begin{enumerate}
  \item Congelación del \emph{pipeline}.
  \item Predicción prefijada.
    \begin{itemize}
      \item Siempre no tomado.
      \item Siempre tomado.
    \end{itemize}
  \item Bifurcación con retraso.
\end{enumerate}
\mode<presentation>{\vfill}
\begin{itemize}
  \item En muchos casos el compilador necesita saber qué se va a hacer para reducir el impacto.
\end{itemize}
\end{frame}

\begin{frame}[t]{Congelación del pipeline}
\begin{itemize}
  \item \textgood{Idea}: Si la instrucción actual es una bifurcación 
        $\rightarrow$ parar o eliminar del pipeline instrucciones posteriores 
        hasta que se conozca el destino.
    \begin{itemize}
      \item Penalización en tiempo de ejecución conocida.
      \item El software (compilador) no puede hacer nada.
    \end{itemize}

    \mode<presentation>{\vfill}
    \item El destino de la bifurcación se conoce en la etapa \textmark{ID}.
      \begin{itemize}
        \item Repetir el \textmark{FETCH} de la siguiente instrucción.
      \end{itemize}
\end{itemize}
\end{frame}

\begin{frame}[t]{Repetición de captación}
\makebox[\textwidth][c]{
\input{es/m4-02-ilp/rep-fetch.tkz}
}

\mode<presentation>{\vfill}
\begin{itemize}
  \item La repetición de IF equivale a una detención.
  \item Una detención por bifurcación puede dar lugar a una \textmark{pérdida de rendimiento} de 10\% a 30\%.
\end{itemize}
\end{frame}

\begin{frame}[t]{Predicción prefijada: no tomada}
\begin{itemize}
  \item \textgood{Idea}: Asumir que la bifurcación será \textmark{no tomada}.
    \begin{itemize}
      \item Se evita modificar el estado del procesador hasta que 
            se tiene la confirmación de que la bifurcación no se toma.
      \item Si la bifurcación se toma, las instrucciones siguientes se 
            retiran del pipeline y se capta la instrucción en el destino del salto.
        \begin{itemize}
          \item Transformar instrucciones en \textmark{NOP}.
        \end{itemize}
    \end{itemize}

  \mode<presentation>{\vfill\pause}
  \item \textgood{Tarea del compilador}:
    \begin{itemize}
      \item Organizar código poniendo la opción más frecuente como no tomada e invirtiendo condición si es necesario.
    \end{itemize}
\end{itemize}
\end{frame}

\begin{frame}[t]{Predicción prefijada: no tomada}
\makebox[\textwidth][c]{
\input{es/m4-02-ilp/pred-not-taken.tkz}
}
\mode<presentation>{\vfill}
\begin{itemize}
  \item Cuando se sabe que el salto se tomará se capta la nueva instrucción (\emph{j+1}).
\end{itemize}
\end{frame}

\begin{frame}[t]{Predicción prefijada: tomada}
\begin{itemize}
  \item \textgood{Idea}: Asumir que la bifurcación será \textmark{tomada}.
    \begin{itemize}
      \item Tan pronto como se decodifica la bifurcación y se calcula el destino se comienza a captar instrucciones del destino.
      \item En pipeline de 5 etapas no aporta ventajas.
        \begin{itemize}
          \item No se conoce dirección destino antes que decisión de bifurcación.
          \item Útil en procesadores con condiciones complejas y lentas.
        \end{itemize}
    \end{itemize}

  \mode<presentation>{\vfill}
  \item \textgood{Tarea del compilador}:
    \begin{itemize}
      \item Organizar código poniendo la opción más frecuente como tomada e invirtiendo condición si es necesario.
    \end{itemize}
\end{itemize}
\end{frame}

\begin{frame}[t,fragile]{Bifurcación retrasada}
\begin{itemize}
  \item \textgood{Idea}: La bifurcación se produce después de ejecutar 
        las \textmark{n} instrucciones posteriores a la propia instrucción 
        de bifurcación.
    \begin{itemize}
      \item \textmark{En pipeline de 5 etapas} $\rightarrow$ 
            1 ranura de retraso (\emph{delay slot}).
    \end{itemize}
\end{itemize}

\mode<presentation>{\vfill\pause}
\begin{columns}
\begin{column}{.4\textwidth}
\begin{block}{Ejemplo}
\begin{lstlisting}[language={generalasm}]
I0:    bnez $1, etiq
I1:    addi $2, $2, 1
I2:    mul  $3, $2, $4
...
IN:    sub  $1, $1, 1
IN+1:  mul  $3, $3, $4

\end{lstlisting}
\end{block}
\end{column}

\begin{column}{.6\textwidth}
\begin{itemize}
  \item Las instrucciones \textmark{I1}, \textmark{I2}, \ldots, \textmark{IN} se
        ejecutan independientemente del sentido de la condición de bifurcación.
  \item La instrucción \textmark{IN+1} solamente se ejecuta si no se produce la
        bifurcación.
\end{itemize}
\end{column}
\end{columns}
\end{frame}

\begin{frame}[t]{Bifrucación retrasada}
\makebox[\textwidth][c]{
\input{es/m4-02-ilp/delayed-branch.tkz}
}
\mode<presentation>{\vfill}
\begin{itemize}
  \item Caso de bifurcación retrasada con una ranura de retraso.
  \item Se espara siempre una instrucción antes de tomar la bifurcación.
  \item Es responsabilidad del programador poner código útil en la ranura.
\end{itemize}
\end{frame}

\begin{frame}[t,fragile]{Compilador y ranuras de retraso}
\mode<presentation>{\vspace{-1.5em}}
\begin{columns}[T]

\begin{column}{.33\textwidth}
\begin{block}{Original}
\begin{lstlisting}[language={generalasm},basicstyle=\scriptsize]
      add $1, $2, $3
      beqz $2, etiq
      nop
      xor $5, $6, $7
etiq: and $8, $9, $10
\end{lstlisting}
\end{block}
\mode<presentation>{\vfill\pause}
\begin{block}{Mejorado}
\begin{lstlisting}[language={generalasm},basicstyle=\scriptsize]
      beqz $2, etiq
      add $1, $2, $3
      xor $5, $6, $7
etiq: and $8, $9, $10
\end{lstlisting}
\end{block}
\end{column}

\mode<article>{
\begin{itemize}
  \item Esta es la reordenación preferible.
  \item Se pasa la instrucción anterior a la bifurcación a la ranura
        porque siempre se ejecuta.
\end{itemize}
}

\begin{column}{.33\textwidth}
\mode<presentation>{\vfill\pause}
\begin{block}{Original}
\begin{lstlisting}[language={generalasm},basicstyle=\scriptsize]
etiq: add $1, $2, $3
      xor $5, $6, $7
      beqz $5, etiq
      nop
      and $8, $9, $10
\end{lstlisting}
\end{block}
\mode<presentation>{\vfill\pause}
\begin{block}{Mejorado}
\begin{lstlisting}[language={generalasm},basicstyle=\scriptsize]
      add $1, $2, $3
etiq: xor $5, $6, $7
      beqz $5, etiq
      add $1, $2, $3
      and $8, $9, $10
\end{lstlisting}
\end{block}
\end{column}

\mode<article>{
\begin{itemize}
  \item Esta reordenación se aplica cuando otra instrucción no se puede
        mover a la ranura por dependencia de datos con la bifurcación.
  \item Se hace necesario repetir la instrucción de la ranura dos veces
        (una para la primera iteración y otra para el resto).
\end{itemize}
}

\begin{column}{.33\textwidth}
\mode<presentation>{\vfill\pause}
\begin{block}{Original}
\begin{lstlisting}[language={generalasm},basicstyle=\scriptsize]
      add $1, $2, $3
      beqz $1, etiq
      nop
      xor $5, $6, $7
etiq: and $8, $9, $10
\end{lstlisting}
\end{block}
\mode<presentation>{\vfill\pause}
\begin{block}{Mejorado}
\begin{lstlisting}[language={generalasm},basicstyle=\scriptsize]
      add $1, $2, $3
      beqz $1, etiq
      xor $5, $6, $7
etiq: and $8, $9, $10
\end{lstlisting}
\end{block}
\end{column}

\mode<article>{
\begin{itemize}
  \item Esta reordenación se aplica si la instrucción introducida
        en la ranura no se vuelve a usar después de la dirección
        de salto.
\end{itemize}
}
\end{columns}
\end{frame}

\begin{frame}[t]{Bifurcaciones retrasadas}
\begin{itemize}
  \item Efectividad del compilador para el caso de 1 ranura:
    \begin{itemize}
      \item \textmark{Rellena} alrededor del 60\% de ranuras.
      \item En torno al 80\% de instrucciones ejecutadas en ranuras \textmark{útiles} para computación.
      \item En torno al 50\% de ranuras \textmark{rellenados} de forma útil.
    \end{itemize}

  \mode<presentation>{\vfill}
  \item Al usarse pipelines más profundos y emisión múltiple hacen falta más ranuras.
    \begin{itemize}
      \item En desuso a favor de enfoque dinámicos.
    \end{itemize}
\end{itemize}
\end{frame}

\begin{frame}[t]{Rendimiento y predicción prefijada}
\begin{itemize}
  \item El número de detenciones de bifurcaciones depende de:
    \begin{itemize}
      \item Frecuencia de las bifurcaciones.
      \item Penalización por bifurcación.
    \end{itemize}
\end{itemize}
\mode<presentation>{\vfill\pause}
\begin{itemize}
  \item Ciclos de penalización por bifurcación:
\end{itemize}
\begin{displaymath}
ciclos_{bifurc} = frecuencia_{bifurc} \times penaliz_{bifurc}
\end{displaymath}
\mode<presentation>{\vfill\pause}
\begin{itemize}
  \item Aceleración
\end{itemize}
\begin{displaymath}
S =
\frac{profundidad_{pipeline}}{1 + frecuencia_{bifurc} \times penaliz_{bifurc}}
\end{displaymath}
\end{frame}

\begin{frame}[t]{Caso práctico}
\begin{itemize}
  \item \textgood{MIPS R4000} (pipeline más profundo).
    \begin{itemize}
      \item 3 etapas antes de conocer destino de bifurcación.
      \item 1 etapa adicional para evaluar condición.
      \item Asumiendo que no hay detenciones de datos en comparaciones.
      \item \textgood{Frecuencia de bifurcaciones}:
        \begin{itemize}
          \item \textmark{Bifurcación incondicional}: 4\%.
          \item \textmark{Bifurcación condicional, no-tomada}: 6\%
          \item \textmark{Bifurcación condicional, tomada}: 10\%
        \end{itemize}
    \end{itemize}
\end{itemize}

\mode<presentation>{\vfill}
{\scriptsize
\begin{tabular}[c]{|p{.3\textwidth}|*{3}{p{.17\textwidth}|}}
\hline
Esquema &  \multicolumn{3}{c|}{Penalización}
\\
\cline{2-4}
bifurcación & incondicional & no-tomada & tomada
\\
\hline\hline

Vaciar pipeline &
\multicolumn{1}{r|}{2} & 
\multicolumn{1}{r|}{3} & 
\multicolumn{1}{r|}{3}
\\
\hline

Predecir tomada &
\multicolumn{1}{r|}{2} & 
\multicolumn{1}{r|}{3} &
\multicolumn{1}{r|}{2}
\\
\hline

Predecir no-tomada &
\multicolumn{1}{r|}{2} &
\multicolumn{1}{r|}{0} &
\multicolumn{1}{r|}{3}
\\

\hline
\end{tabular}
}
\end{frame}

\begin{frame}[t]{Solución}

{\scriptsize
\begin{tabular}[c]{|p{.2\textwidth}|*{3}{p{.18\textwidth}|}p{.06\textwidth}|}
\hline
Esquema &  \multicolumn{3}{c|}{Bifurcación} & \multicolumn{1}{c|}{Total}
\\
\cline{2-4}
bifurcación & incondicional & no-tomada & tomada &
\\
\hline\hline

Frecuencia &
\multicolumn{1}{c|}{4\%} &
\multicolumn{1}{c|}{6\%} &
\multicolumn{1}{c|}{10\%} &
\multicolumn{1}{c|}{20\%} 
\\
\hline

Vaciar pipeline &
$0.04 \times 2 = 0.08$ &
$0.06 \times 3 = 0.18$ &
$0.10 \times 3 = 0.30$ &
$0.56$
\\
\hline

Predecir tomada &
$0.04 \times 2 = 0.08$ &
$0.06 \times 3 = 0.18$ &
$0.10 \times 2 = 0.20$ &
$0.46$
\\
\hline

Predecir no-tomada &
$0.04 \times 2 = 0.08$ &
$0.06 \times 0 = 0.00$ &
$0.10 \times 3 = 0.30$ &
$0.38$
\\
\hline

\hline
\end{tabular}
}
\begin{center}
\textmark{Contribución sobre CPI ideal}
\end{center}

\mode<presentation>{\vfill\pause}
\begin{columns}

\begin{column}{.49\textwidth}
{\footnotesize
\emph{Speedup} de predecir \textmark{tomada} sobre vaciar pipeline
}
\begin{displaymath}
S =
\frac{1 + 0.56}{1 + 0.46} =
1.068
\end{displaymath}
\end{column}

\begin{column}{.49\textwidth}
{\footnotesize
\emph{Speedup} de predecir \textmark{no tomada} sobre vaciar pipeline
}
\begin{displaymath}
S =
\frac{1 + 0.56}{1 + 0.38} =
1.130
\end{displaymath}
\end{column}

\end{columns}

\end{frame}
