\section{Otros modelos de consistencia}

\begin{frame}[t]{Optimizaciones}
\begin{itemize}
  \item Modelos que relajan el orden de ejecución de programa.
    \begin{itemize}
      \item Relajar W $\rightarrow$ R $\Rightarrow$
           \textmark{Orden de almacenamiento total}.
      \item Relajar W $\rightarrow$ W, W $\rightarrow$ W $\Rightarrow$
            \textmark{Orden de almacenamiento parcial}.
      \item Relajar todo $\Rightarrow$
        \begin{itemize}
          \item \textmark{Ordenamiento débil} (PowerPC).
          \item \textmark{Consistencia de liberación} (RISC V).
        \end{itemize}
    \end{itemize}

  \mode<presentation>{\vfill}
  \item \textgood{Notación}:
    \begin{itemize}
      \item X $\rightarrow$ Y
        \begin{itemize}
          \item X debe completarse antes de que Y se realice.
        \end{itemize}
    \end{itemize}
\end{itemize}
\end{frame}

\begin{frame}[t]{Reordenaciones}
\mode<presentation>{\vfill}
\makebox[\textwidth][c]
{
\begin{tabular}{|l|c|l|}
\hline

Sequential consistency &
Opcionalmente &
R $\rightarrow$ R
R $\rightarrow$ W\\
& &
W $\rightarrow$ R
W $\rightarrow$ W
\\
\hline

Total store order &
IBM/370 &
R $\rightarrow$ R
R $\rightarrow$ W\\
&
DEC VAX &
W $\rightarrow$ W\\

& SPARC & \\
\hline

Partial store order &
SPARC &
R $\rightarrow$ R
R $\rightarrow$ W\\
\hline

Weak ordering &
Power PC &
\\
\hline

Release consistency &
MIPS, RISC V, ARM v8 &
\\
\hline


\end{tabular}
}
\end{frame}

\begin{frame}[t]{Lecturas adelantas a escrituras (W$\rightarrow$R)}
\begin{itemize}
  \item Una \textgood{lectura} \textmark{puede ejecutarse antes} 
        que una \textgood{escritura} anterior.

  \mode<presentation>{\vfill}
  \item Típico en sistemas con \textgood{búfer de escritura}.
    \begin{itemize}
      \item Comprobación de consistencia con búfer.
      \item Permiten lectura de búfer.
    \end{itemize}
\end{itemize}
\end{frame}

\begin{frame}[t]{Otros modelos}
\begin{itemize}
  \item R $\rightarrow$ W, W $\rightarrow$ R.
    \begin{itemize}
      \item Permiten que las \textgood{escrituras} 
            \textmark{puedan llegar a memoria} fuera de orden de programa.
    \end{itemize}

  \mode<presentation>{\vfill\pause}
  \item R $\rightarrow$ W, 
        W $\rightarrow$ R, 
        R $\rightarrow$ R, 
        W $\rightarrow$ W.
    \begin{itemize}
      \item Solamente se evitan dependencias de datos y control dentro del procesador.
      \item \textgood{Alternativas}:
        \begin{itemize}
          \item Consistencia débil.
          \item Consistencia de liberación.
        \end{itemize}
    \end{itemize}
\end{itemize}
\end{frame}

\begin{frame}[t]{Ordenamiento débil}
\begin{itemize}
  \item Divide las operaciones a memoria en \textgood{operaciones de datos} 
        y \textgood{operaciones de sincronización}.

  \mode<presentation>{\vfill\pause}
  \item Las \textgood{operaciones de sincronización} actúan como una \textmark{barrera}.
    \begin{enumerate}
      \item Todas las operaciones de datos previas en orden de programa a una sincronización 
            deben completarse antes de ejecutar la sincronización.
      \item Todas las operaciones de datos posteriores en orden de programa a una sincronización 
            deben esperar a que se complete la sincronización.
      \item Las sincronizaciones se realizan en orden de programa.
    \end{enumerate}

  \mode<presentation>{\vfill\pause}
  \item Implementación hardware de la \textmark{barrera}.
    \begin{itemize}
      \item \textgood{Procesador mantiene un contador}:
        \begin{itemize}
          \item \textgood{Emisión} de operación de datos $\Rightarrow$ \textmark{incremento}.
          \item Operación de datos \textgood{completada} $\Rightarrow$ \textmark{decremento}.
        \end{itemize}
    \end{itemize}
\end{itemize}
\end{frame}

\begin{frame}[t]{Consistencia de adquisición/liberación}
\begin{itemize}
  \item \textmark{Más relajada} que la consistencia débil.
  \item  \textmark{Accesos de sincronización} divididos en:
    \begin{itemize}
      \item \textmark{Acquire} $\rightarrow$ Adquisición.
      \item \textmark{Relase} $\rightarrow$ Liberación.
    \end{itemize}

  \mode<presentation>{\vfill\pause}
  \item \textgood{Semántica}:
    \begin{itemize}
      \item \textmark{Acquire}
        \begin{itemize}
          \item Debe completarse antes que todos los accesos a memoria subsiguientes.
        \end{itemize}
      \item \textmark{Release}
        \begin{itemize}
          \item Deben completarse todos los accesos a memoria previos.
          \item Accesos memoria posteriores \textgood{SI} pueden iniciarse.
          \item Operaciones que siguen a \textmark{release} y que deben esperar se deben proteger con un \textmark{acquire}.
        \end{itemize}
    \end{itemize}
\end{itemize}
\end{frame}
