\section{Consistencia secuencial}

\begin{frame}[t]
\makebox[\textwidth][c]{\input{es/m5-02-consist/multi-proc.tkz}}
\mode<presentation>{\vfill\pause}
\begin{footnotesize}
Un sistema multiprocesador es \textgood{secuencialmente consistente} si el resultado de
cualquier ejecución es el mismo que se obtendría si las operaciones de todos
los procesadores se ejecutase en algún orden secuencial, y las operaciones de
cada procesador individual aparecen en esa secuencia en el orden indicado por
el programa.

\begin{flushright}
Leslie Lamport, 1979
\end{flushright}
\end{footnotesize}
\end{frame}

\begin{frame}[t]{Restricciones de la consistencia secuencial}
\begin{itemize}
  \item \textgood{Orden de programa}.
    \begin{itemize}
      \item Las operaciones de memoria de un programa deben 
            hacerse visibles a \textmark{todos los procesos} en el 
            \textgood{orden de programa}.
    \end{itemize}

  \mode<presentation>{\vfill\pause}
  \item \textgood{Atomicidad}.
    \begin{itemize}
      \item El orden total de ejecución entre procesos debe ser 
            \textgood{consistente} requiriendo que todas las 
             operaciones sean \textmark{atómicas}.
        \begin{itemize}
          \item Nada que un procesador haga 
                después de que haya visto el nuevo valor de una escritura 
                se hace visible a otros procesos antes de que 
                hayan visto el valor de esa escritura.
        \end{itemize}
    \end{itemize}
\end{itemize}
\end{frame}

\begin{frame}[t]{Atomicidad}
\makebox[\textwidth][c]{\input{es/m5-02-consist/atomicity.tkz}}
\mode<presentation>{\vfill\pause}
\begin{itemize}
  \item \textgood{Escrituras no atómicas}:
    \begin{itemize}
      \item La escritura en \cppid{b} podría adelantar al 
            bucle \cppkey{while} y la lectura de \cppid{a} 
            adelantaría la escritura.
        \begin{itemize}
          \item \cppid{X=0}.
        \end{itemize}
    \end{itemize}
  \item \textgood{Escrituras atómicas}:
    \begin{itemize}
      \item Se preserva la \textmark{consistencia secuencial}.
    \end{itemize}
\end{itemize}
\end{frame}

\begin{frame}[t]
\begin{itemize}
  \item La \textgood{consistencia secuencial} \textmark{restringe} todas las operaciones de memoria:
    \begin{itemize}
      \item Write $\rightarrow$ Read.
      \item Write $\rightarrow$ Write.
      \item Read $\rightarrow$ Read, Read $\rightarrow$ Write.
    \end{itemize}

  \mode<presentation>{\vfill\pause}
  \item \textmark{Modelo simple} para razonar sobre programas paralelos.

  \mode<presentation>{\vfill\pause}
  \item Pero, reordenaciones simples para monoprocesador pueden \textbad{violar} 
        el modelo de \textgood{consistencia secuencial}:
    \begin{itemize}
      \item \textmark{Reordenación de hardware} para mejora del rendimiento.
        \begin{itemize}
          \item Write buffers, escrituras solapadas, \ldots
        \end{itemize}
      \item \textmark{Optimizaciones de compilador} aplican transformaciones que reordenan operaciones de memoria.
        \begin{itemize}
          \item Remplazo de escalares, asignación de registros, planificación de instrucciones, \ldots
        \end{itemize}
      \item \textmark{Transformaciones} por programadores o 
            \textmark{herramientas de refactoring} también modifican 
            la semántica del programa.
    \end{itemize}
\end{itemize}
\end{frame}

\begin{frame}[t]{Violación de consistencia secuencial}
\begin{columns}[T]

\column{.5\textwidth}
\makebox[\textwidth][c]{\input{es/m5-02-consist/sc-viol.tkz}}

\pause 

\column{.5\textwidth}
\begin{itemize}
  \item Si las cachés usan búfer de escritura:
    \begin{itemize}
      \item Escrituras se \textbad{retrasan} en búfer.
      \item Lecturas \textbad{obtienen el valor antiguo}.
      \item Se \textbad{invalida} el \textmark{algoritmo de Dekker}.
        \begin{itemize}
          \item El \textmark{algoritmo de Dekker} es la primera solución 
                conocida al problema de la exclusión mutua.
        \end{itemize}
    \end{itemize}
\end{itemize}

\end{columns}
\end{frame}

\begin{frame}[t]{Orden de programa}
\begin{columns}[T]

\column{.5\textwidth}
\makebox[\textwidth][c]{\input{es/m5-02-consist/prog-order.tkz}}

\column{.5\textwidth}
\makebox[\textwidth][c]{\input{es/m5-02-consist/prog-order-res.tkz}}

\end{columns}
\end{frame}

\begin{frame}[t]{Orden de programa}
\begin{columns}[T]

\column{.5\textwidth}
\makebox[\textwidth][c]{\input{es/m5-02-consist/ej-prog-order.tkz}}

\column{.5\textwidth}
\makebox[\textwidth][c]{\input{es/m5-02-consist/ej-prog-order-res.tkz}}

\end{columns}
\end{frame}

\begin{frame}[t]{Condiciones para la consistencia secuencial}
\begin{itemize}
  \item \textgood{Condiciones suficientes}:
    \begin{itemize}
      \item Cada proceso \textmark{emite las operaciones} de memoria en orden de programa.
      \item Después de la \textmark{emisión de una escritura}, 
            el proceso de emisión \textmark{espera a que se complete} la escritura 
            \textmark{antes de emitir} otra operación.
      \item Después de \textmark{emitir una lectura}, el proceso que la emitió 
            \textmark{espera a que se complete} la lectura y a que 
            la escritura del valor que se está leyendo se \textmark{complete}.
        \begin{itemize}
          \item Esperar la propagación de escrituras a todos los procesos.
        \end{itemize}
    \end{itemize}
  \mode<presentation>{\vfill\pause}
  \item \textbad{Son condiciones muy exigentes.}
    \begin{itemize}
      \item Puede haber condiciones necesarias menos exigentes.
    \end{itemize}

\end{itemize}
\end{frame}
