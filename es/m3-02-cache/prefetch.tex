\subsection{Lectura adelantada hardware}

\begin{frame}[t]{Lectura adelantada de instrucciones}
\begin{itemize}
  \item \textmark{Observación}: Las instrucciones presentan 
        alta localidad espacial.

  \mode<presentation>{\vfill\pause}
  \item \textgood{Instruction prefetching}: Lectura adelantada
        de instrucciones.
    \begin{itemize}
      \item Lectura de dos bloques en caso de fallo.
        \begin{itemize}
          \item Bloque que provoca el fallo.
          \item Bloque siguiente.
        \end{itemize}
    \end{itemize}

  \mode<presentation>{\vfill\pause}
  \item \textgood{Ubicación}:
    \begin{itemize}
      \item Bloque que provoca el fallo $\rightarrow$ \textmark{caché de instrucciones}.
      \item Bloque siguiente $\rightarrow$ \textmark{búfer de instrucciones}.
    \end{itemize}
\end{itemize}
\end{frame}

\begin{frame}[t]{Lectura adelantada de datos}
\begin{itemize}
  \item \textmark{Ejemplo}: Pentium 4.

  \mode<presentation>{\vfill\pause}
  \item \textgood{Data prefetching}: Permite lectura adelantada de páginas de 4KB a caché L2.

  \mode<presentation>{\vfill\pause}
  \item Se invoca lectura adelantada si:
    \begin{itemize}
      \item 2 fallos en L2 debidos a una misma página.
      \item Distancia entre fallos menor que 256 bytes.
    \end{itemize}

\end{itemize}
\end{frame}
