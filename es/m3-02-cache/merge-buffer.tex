\section{Mezclas en búfer de escritura}

\begin{frame}[t]{Búfer de escritura}
\begin{itemize}
  \item Un \textmark{búfer de escritura} permite \textgood{reducir} la penalización de fallo.
    \begin{itemize}
      \item Cuando se ha escrito en el búfer el procesador da la escritura por efectuada.
      \item Escrituras simultáneas en memoria son más eficientes que una única escritura.
    \end{itemize}

  \mode<presentation>{\vfill\pause}
  \item \textgood{Usos}:
    \begin{itemize}
      \item \textmark{Escritura inmediata} (\emph{write-through}): En todas las escrituras.
      \item \textmark{Post escritura} (\emph{write-back}): Cuando se remplaza un bloque.
    \end{itemize}
\end{itemize}
\end{frame}

\begin{frame}[t]{Mezclas en búfer de escritura}
\begin{itemize}
  \item Si el búfer contiene bloques modificados, 
        se comprueban las direcciones para ver si se puede sobrescribir.

  \mode<presentation>{\vfill\pause}
  \item \textmark{Efectos}:
    \begin{itemize}
      \item Reduce el número de \textgood{escrituras en memoria}.
      \item Reduce \textgood{paradas} debidas a que el búfer esté lleno.
    \end{itemize}
\end{itemize}
\end{frame}

\begin{frame}[t]{Mezclas en buffer de escritura}
\input{es/m3-02-cache/merge-buffer-1.tkz}
\mode<presentation>{\vfill\pause}
\input{es/m3-02-cache/merge-buffer-2.tkz}
\end{frame}
