\section{Optimizaciones del compilador}

\begin{frame}[t]{Optimizaciones del compilador}
\begin{itemize}
  \item \textmark{Objetivo}:
        Generar código que provoque menos fallos de 
        instrucciones y datos.

  \mode<presentation>{\vfill\pause}
  \item \textgood{Instrucciones}: 
    \begin{enumerate}[1.]
      \item Reordenación de procedimientos.
      \item Alinear bloques de código al inicio de bloques de caché.
      \item Linearización de saltos.
    \end{enumerate}

  \mode<presentation>{\vfill}
  \item \textgood{Datos}: 
    \begin{enumerate}[1.]
      \item Fusión de arrays.
      \item Intercambio de bucles.
      \item Fusión de bucles.
      \item Acceso por bloques.
    \end{enumerate}
\end{itemize}
\end{frame}

\begin{frame}[t]{Reordenación de procedimientos}
\begin{columns}

\column{.7\textwidth}

\begin{itemize}
  \item \textgood{Objetivo}: Reducir los \textbad{fallos por conflicto} debidos a que 
        dos procedimientos coincidentes en el tiempo se corresponden 
        con la \textmark{misma línea de caché}.

  \mode<presentation>{\vfill}
  \item \textmark{Técnica}: Reordenar los procedimientos en memoria.

\end{itemize}

\column{.3\textwidth}
\pause
\input{es/m3-02-cache/reorder-proc.tkz}

\end{columns}
\end{frame}

\begin{frame}[t]{Alineación de bloques básicos}
\begin{itemize}
  \item \textmark{Definición}: 
        Un \textgood{bloque básico} es un conjunto de instrucciones 
        que se ejecuta secuencialmente (no contiene saltos).

  \mode<presentation>{\vfill\pause}
  \item \textmark{Objetivo}: Reducir la posibilidad de 
        \textbad{fallos de caché} para código secuencial.

  \mode<presentation>{\vfill\pause}
  \item \textmark{Técnica}: 
        Hacer coincidir la \textgood{primera instrucción} de un bloque básico 
        con la \textgood{primera palabra} de una línea de caché.
\end{itemize}
\end{frame}

\begin{frame}[t]{Linearización de saltos}
\begin{itemize}
  \item \textmark{Objetivo}: Reducir los fallos de caché debidos a saltos condicionales.

  \mode<presentation>{\vfill\pause}
  \item \textgood{Técnica}: Si el compilador sabe que es probable que se tome un salto, 
        puede cambiar el sentido de la condición en intercambiar los bloques 
        básicos de las dos alternativas.
    \begin{itemize}
      \item Algunos compiladores definen extensiones para dar pistas al compilador.
      \item \textmark{Ejemplo}: \textgood{ISO C++} (\cppkey{[[likely]]}, \cppkey{[[unlikely]]}).
    \end{itemize}

\end{itemize}
\end{frame}

\begin{frame}[t,fragile]{Fusión de arrays}
\begin{columns}[T]

\column{.5\textwidth}
\begin{block}{Arrays paralelos}
\begin{lstlisting}
vector<int> key;
vector<int> val;

for (int i=0;i<max;++i) {
  cout << key[i] << "," 
       << val[i] << "\n";
}
\end{lstlisting}
\end{block}

\pause
\column{.5\textwidth}
\begin{block}{Array fusionado}
\begin{lstlisting}
struct entry {
  int key;
  int val;
};
vector<entry> v;

for (int i=0;i<max;++i) {
  cout << v[i].key << "," 
       << v[i].val << "\n";
}
\end{lstlisting}
\end{block}

\end{columns}

\begin{itemize}
  \item Reducción de conflictos.
  \item Mejora de localidad espacial.
\end{itemize}
\end{frame}

\begin{frame}[t,fragile]{Intercambio de bucles}
\begin{columns}[T]

\column{.5\textwidth}
\begin{block}{Acceso con saltos}
\begin{lstlisting}
for (int j=0; j<100; ++j) {
  for (int i=0; i<5000; ++i) {
    v[i][j] = k * v[i][j];
  }
}
\end{lstlisting}
\end{block}

\pause
\column{.5\textwidth}
\begin{block}{Accesos secuenciales}
\begin{lstlisting}
for (int i=0; i<5000; ++i) {
  for (int j=0; j<100; ++j) {
    v[i][j] = k * v[i][j];
  }
}
\end{lstlisting}
\end{block}
\end{columns}

\mode<presentation>{\vfill}
\begin{itemize}
  \item \textmark{Objetivo}: Mejorar localidad espacial.
  \item Dependiente del modelo de almacenamiento vinculado al lenguaje de programación.
    \begin{itemize}
      \item FOTRAN versus C.
    \end{itemize}
\end{itemize}
\end{frame}

\begin{frame}[t,fragile]{Fusión de bucles}
\begin{columns}[T]

\column{.5\textwidth}
\begin{block}{Bucles separados}
\begin{lstlisting}
for (int i=0; i<rows; ++i) {
  for (int j=0; j<cols; ++j) {
    a[i][j] = b[i][j] * c[i][j];
  }
}
for (int i=0; i<rows; ++i) {
  for (int j=0; j<cols; ++j) {
    d[i][j] = a[i][j] + c[i][j];
  }
}
\end{lstlisting}
\end{block}

\pause
\column{.5\textwidth}
\begin{block}{Bucles fusionados}
\begin{lstlisting}
for (int i=0; i<rows; ++i) {
  for (int j=0; j<cols; ++j) {
    a[i][j] = b[i][j] * c[i][j];
    d[i][j] = a[i][j] + c[i][j];
  }
}
\end{lstlisting}
\end{block}
\end{columns}

\mode<presentation>{\vfill}
\begin{itemize}
  \item \textmark{Objetivo}: Mejorar localidad temporal.
  \item \textbad{Cuidado}: Puede reducir localidad espacial.
\end{itemize}

\end{frame}

\begin{frame}[t,fragile]{Acceso por bloques}
\begin{columns}[T]

\column{.4\textwidth}
\begin{block}{Producto original}
\begin{lstlisting}
for (int i=0; i<size; ++i) {
  for (int j=0; j<size; ++j) {
    double r = 0;
    for (int k=0; k<size; ++k) {
      r+= b[i][k] * c[k][j];
    }
    a[i][j] = r;
  }
}
\end{lstlisting}
\end{block}

\pause
\column{.6\textwidth}
\begin{block}{Producto con acceso por bloques}
\begin{lstlisting}
for (int bj=0; bj<size; bj+=bsize) {
  for (int bk=0; bk<size; bk +=bsize) {
    for (int i=0; i<size; ++i) {
      for (int j=bj; j<std::min(bj+bsize,size); ++j) {
        double r = 0;
        for (int k=bk; k<std::min(bk+bsize,size); ++k) {
          r +=b[i][k] * c[k][j];
        }
        a[i][j] += r;
      }
    }
  }
}
\end{lstlisting}
\end{block}
\end{columns}

\mode<presentation>{\vfill}
\begin{itemize}
  \item \cppid{bsize}: Factor de bloque.
\end{itemize}

\end{frame}
