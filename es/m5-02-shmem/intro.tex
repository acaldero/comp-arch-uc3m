\section{Introducción a memoria compartida distribuida}

\begin{frame}[t]{Protocolos de espionaje y escalabilidad}
\begin{itemize}
  \item \textbad{Problemas} de protocolos de espionaje.
    \begin{itemize}
      \item Requiere comunicación con todas las cachés:
        \begin{itemize}
          \item En cada fallo de caché.
          \item En cada escritura de dato compartido.
        \end{itemize}
    \end{itemize}

  \mode<presentation>{\vfill\pause}
  \item ¿Qué \textgood{ventaja} tienen los protocolos de espionaje?
    \begin{itemize}
      \item Ausencia de estructura de datos centralizada.
        \begin{itemize}
          \item Bajo coste de implementación.
        \end{itemize}
    \end{itemize}

  \mode<presentation>{\vfill\pause}
  \item ¿Qué \textbad{inconveniente} tiene los protocolos de espionaje?
    \begin{itemize}
      \item Ausencia de estructura de datos centralizadas.
        \begin{itemize}
          \item Comunicaciones limitan la escalabilidad.
        \end{itemize}
    \end{itemize}
\end{itemize}
\end{frame}

\begin{frame}{Modelo básico de DSM}
\makebox[\textwidth][c]{
\input{es/m5-02-shmem/dsm.tkz}
}
\mode<presentation>{\vfill\pause}
\begin{itemize}
  \item \textbad{Necesidad de eliminar tráfico de coherencia.}
\end{itemize}
\end{frame}

\begin{frame}[t]{Clases de protocolos de coherencia}
\begin{itemize}
  \item \textmark{Espionaje} (\emph{snooping}):
    \begin{itemize}
      \item Cada caché mantiene el estado de compartición de cada bloque que tiene.
      \item Las cachés accesibles mediante medio de multidifusión (bus).
      \item Todas las cachés monitorizan si tienen una copia del bloque.
    \end{itemize}

  \mode<presentation>{\vfill\pause}
  \item \textmark{Basados en directorio}:
    \begin{itemize}
      \item El estado de compartición se mantiene en un directorio.
        \item \textgood{SMP}: 
              Directorio centralizado en memoria o en caché de más alto nivel.
        \item \textgood{DSM}: 
              Para evitar cuello de botella se usa un directorio distribuido (más complejo).
    \end{itemize}
\end{itemize}
\end{frame}

\begin{frame}[t]{Protocolo basado en directorio}
\begin{itemize}
  \item \textmark{Idea}: Mantener el \textgood{estado} de cada bloque de caché.
    \begin{itemize}
      \item ¿Qué cachés tiene copia del bloque?
      \item Bits de estado del bloque.
    \end{itemize}

  \mode<presentation>{\vfill\pause}
  \item Multicores con caché externa compartida.
    \begin{itemize}
      \item \textmark{Vector de bits} de longitud igual a número de cores.
        \begin{itemize}
          \item Indica que cachés privadas pueden tener copia del bloque.
          \item Solamente se envía invalidación a cachés marcadas en mapa de bits.
        \end{itemize}
      \item Esquema funciona bien dentro de un único multicore.
      \item \textmark{Ejemplo}: \textgood{Intel Core i7}.
    \end{itemize}
\end{itemize}
\end{frame}

\begin{frame}[t]{Directorio centralizado y escalabilidad}
\begin{itemize}
  \item Un directorio centralizado evita broadcast pero
    \begin{itemize}
      \item Se convierte en cuello de botella.
      \item Problema de escalabilidad con número de procesadores.
    \end{itemize}

  \mode<presentation>{\vfill\pause}
  \item \textgood{Solución}: \textmark{Directorio distribuido}.
    \begin{itemize}
      \item Distribuir el directorio con la memoria.
      \item Cada directorio tiene información de la memoria local asociada.
        \begin{itemize}
          \item Siempre se sabe a qué directorio ir.
        \end{itemize}
      \item Distintas peticiones de coherencia van a distintos directorios.
    \end{itemize}


\end{itemize}
\end{frame}

\begin{frame}{Direcotrio distribuido}
\makebox[\textwidth][c]{
\input{es/m5-02-shmem/dsm-dist.tkz}
}
\mode<presentation>{\vfill\pause}
\end{frame}
