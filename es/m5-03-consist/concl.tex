\section{Conclusión}

\begin{frame}[t]{Resumen}
\begin{itemize}
  \item Modelo de consistencia de memoria determina qué optimizaciones son válidas.
  \item La \textgood{consistencia secuencial} establece como restricciones la
        \textmark{atomicidad} y el \textmark{orden de programa}.
  \item Se pueden usar modelos más relajados que la consistencia secuencial.
    \begin{itemize}
      \item \textmark{Consistencia débil}.
      \item \textmark{Consistencia adquisición liberación}
    \end{itemize}
  \item El modelo de memoria de Intel ha ido evolucionando en la última década.
    \begin{itemize}
      \item Formalizado y públicamente disponible.
      \item Establece qué operaciones son atómicas, cuándo se bloquea el bus y cómo se definen barreras.
      \item Define el modelo de memoria dentro del procesador y entre distintos procesadores.
    \end{itemize}
\end{itemize}
\end{frame}


\begin{frame}[t]{Referencias}
\begin{itemize}
  \item \bibhennessy
  \textgood{Sections}: 5.6

  \mode<presentation>{\vfill}
  \item \textmark{Shared memory consistency models: A tutorial.}\\
        Adve, S. V., and Gharachorloo, K.\\
        IEEE Computer 29, 12 (December 1996), 66-76.

  \mode<presentation>{\vfill}
  \item \textmark{Intel 64 and IA-32 Architectures Software Developer Manuals.}\\
        Volume 3: Systems Programming Guide.\\
        8.2: Memory Ordering

\end{itemize}
\end{frame}
