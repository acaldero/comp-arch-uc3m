\subsection{Modelo de consistencia}

\begin{frame}[t]{Consistencia de memoria en Intel}
\begin{itemize}
  \item Hasta el año 2005 Intel no había clarificado completamente su 
        \textmark{modelo de consistencia de memoria}.
    \begin{itemize}
      \item Complejidad para formalización del modelo.
      \item Problemas para implementaciones de lenguajes (Java, C++, \ldots).
    \end{itemize}

  \mode<presentation>{\vfill\pause}
  \item Actualmente el modelo está completamente clarificado y es público.
\end{itemize}
\end{frame}

\begin{frame}[t]{Modelo inicial de Intel}
\begin{itemize}
  \item \textmark{i486} y \textmark{Pentium}:
    \begin{itemize}
      \item Operaciones en orden de programa.
        \begin{itemize}
          \item \textbad{Excepción}: Fallos de lectura adelantan escrituras en 
                \emph{write buffer} solamente si todas las escrituras son aciertos de caché.
          \item Es imposible que el fallo de lectura coincida con una escritura.
        \end{itemize}
    \end{itemize}
\end{itemize}
\end{frame}

\begin{frame}[t]{Operaciones atómicas}
\begin{itemize}
  \item Desde \textmark{i486}:
    \begin{itemize}
      \item Leer o escribir 1 byte.
      \item Leer o escribir una palabra alineada a 16 bits.
      \item Leer o escribir una doble palabra alineada a 32 bits.
    \end{itemize}

  \mode<presentation>{\vfill\pause}
  \item Desde \textmark{Pentium}:
    \begin{itemize}
      \item Leer o escribir quadword alineado a 64 bits.
      \item Acceso a memoria no cacheada que cabe en bus de datos de 32 bits.
    \end{itemize}

  \mode<presentation>{\vfill\pause}
  \item Desde \textmark{P6}:
    \begin{itemize}
      \item Acceso no alineado a datos de 16, 32 o 64 bits que caben en una línea de caché.
    \end{itemize}
\end{itemize}
\end{frame}

\begin{frame}[t]{Bloqueo del bus (I)}
\begin{itemize}
  \item Un procesador puede emitir una \textmark{señal de bloqueo} del bus.
    \begin{itemize}
      \item Otros elementos \textmark{no pueden acceder} al bus.
    \end{itemize}

  \mode<presentation>{\vfill\pause}
  \item \textgood{Bloqueo automático del bus}:
    \begin{itemize}
      \item Instrucción \asminst{XCHG}.
      \item Actualización de \textmark{descriptores de segmento}, 
            \textmark{directorio de páginas} y 
            \textmark{tabla de páginas}.
      \item Aceptación de interrupciones.
    \end{itemize}
\end{itemize}
\end{frame}

\begin{frame}[t]{Bloqueo del bus (II)}
\begin{itemize}
  \item \textgood{Bloqueo software del bus}:
    \begin{itemize}
      \item Uso del prefijo \asminst{LOCK} en:
      \item Instrucciones de comprobación y modificación de bit (\asminst{BTS}, \asminst{BTR}, \asminst{BTC}).
      \item Instrucciones de intercambio (\asminst{XADD}, \asminst{CMPXCHG}, \asminst{CMPXCHG8B}).
      \item Instrucciones aritméticas de 1 operando (\asminst{INC}, \asminst{DEC}, \asminst{NOT}, \asminst{NEG}).
      \item Instrucciones aritmético-lógicas de 2 operandos (\asminst{ADD}, \asminst{ADC}, \asminst{SUB}, 
            \asminst{SBB}, \asminst{AND}, \asminst{OR}, \asminst{XOR}).
    \end{itemize}
\end{itemize}
\end{frame}

\begin{frame}[t]{Instrucciones de barrera}
\begin{itemize}
  \item \asminst{LFENCE}:
    \begin{itemize}
      \item Barrera para \textmark{operaciones de load}.
      \item Cada \textmark{load previo} a \asminst{LFENCE} 
            se hace \textgood{globalmente visible} antes que cualquier \textmark{load posterior}.
    \end{itemize}

  \mode<presentation>{\vfill}
  \item \asminst{SFENCE}:
    \begin{itemize}
      \item Barrera para \textmark{operaciones de store}.
      \item Cada \textmark{store previo} a \asminst{SFENCE} se hace \textgood{globalmente visible} 
            antes que cualquier \textmark{store posterior}.
    \end{itemize}

  \mode<presentation>{\vfill}
  \item \asminst{MFENCE}:
    \begin{itemize}
      \item Barrera para \textmark{operaciones de load/store}.
      \item Todos los \textmark{load y store previos} a \asminst{MFENCE} son \textgood{globalmente visibles} 
            antes que cualquier \textmark{load o store posterior}.
    \end{itemize}
\end{itemize}
\end{frame}

\begin{frame}[t]{Modelo de memoria actual dentro del procesador (I)}
\begin{itemize}
  \item Lecturas no adelantan lecturas (R $\rightarrow$ R).
  \item Escrituras no adelantan lecturas (R $\rightarrow$ W).
  \item Escrituras no adelantan escrituras (W $\rightarrow$ W).
  \item Hay excepciones para strings y movimientos no temporales.
  \item Lecturas si adelantan escrituras anteriores (W $\rightarrow$ R) a direcciones diferentes.
  \item Lecturas/escrituras no adelantan a operaciones de (E/S), instrucciones con cerrojo o instrucciones de serialización.
\end{itemize}
\end{frame}

\begin{frame}[t]{Modelo de memoria actual dentro del procesador (II)}
\begin{itemize}
  \item Lecturas no pueden sobrepasar \asminst{LFENCE} o \asminst{MFENCE} anteriores.
  \item Escrituras no pueden sobrepasar \asminst{LFENCE}, \asminst{SFENCE} o \asminst{MFENCE} anteriores.
  \item \asminst{LFENCE} no puede sobrepasar lectura anterior.
  \item \asminst{SFENCE} no puede sobrepasar escritura anterior.
  \item \asminst{MFENCE} no puede sobrepasar lectura o escritura anterior.
\end{itemize}
\end{frame}

\begin{frame}[t]{Modelo de memoria multiprocesador}
\begin{itemize}
  \item Cada procesador cumple con reglas anteriores individualmente.
  \item Las escrituras de un procesador se observan en el mismo orden por todos los demás.
  \item Las escrituras de un procesador NO se ordenan con respecto a las escrituras de otros procesadores.
  \item La ordenación de memoria es transitiva.
  \item Dos escrituras son vistas en un orden consistente por cualquier procesador distinto de esos dos.
  \item Las instrucciones de cerrojo tienen un orden total.
\end{itemize}
\end{frame}
