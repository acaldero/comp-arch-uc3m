\section{Tendencia en potencia y energía}

\begin{frame}[t]{Potencia y energía: Principales aspectos}
\begin{enumerate}
  \item \textgood{Potencia máxima}
    \begin{itemize}
      \item Procesador consume más potencia de la suministrada $\Rightarrow$ \textmark{caída de voltaje}
            $\Rightarrow$ \textbad{mal funcionamiento}.
      \item Indexado de voltaje para permitir reducir velocidad para regular voltaje.
    \end{itemize}

  \mode<presentation>{\vfill\pause}
  \item \textgood{Potencia Térmica de Diseño (TDP -- \emph{Thermal Design Power})}
    \begin{itemize}
      \item Consumo de potenica baj máxima carga teórica.
      \item Determina los requisitos de enfriamiento.
      \item Se puede superar durante periodos cortos de tiempo.
      \item Qué pasa si el TDP se supera por más tiempo?
        \begin{itemize}
          \item Bajar tasa de reloj.
          \item Apagar el chip.
        \end{itemize}
    \end{itemize}   

  \mode<presentation>{\vfill\pause}
  \item \textgood{Energía}
    \begin{itemize}
      \item La potencia es energía por unidad de tiempo.
      \item 1 Watio = 1 Julio / 1 segundo.
    \end{itemize}
\end{enumerate}
\end{frame}

\begin{frame}[t]{Potencia y Energía: Ejemplo}
\begin{itemize}
  \item Dos sistemas diferentes (\textmark{A} y \textmark{B}).
    \begin{itemize}
      \item \textmark{A} consume un 20\% más de potencia que \textmark{B}.
      \item \textmark{A} ejecuta una tarea en el 70\% del tiempo de \textmark{B}.
      \item Qué tiene un coste más bajo?
    \end{itemize}

  \mode<presentation>{\pause\vfill}
  \item La métrica adecuada para comparar es la \textgood{Energía}.
    \begin{itemize}
      \item $E(B) = P(B) \cdot t(B)$
      \item $E(A) = 1.2 \cdot P(B) \cdot 0.7 \cdot t(B) = 0.84 \cdot E(B)$
      \item El sistema \textmark{A} usa el 84\% de la energía de \textmark{B}.
    \end{itemize}
\end{itemize}
\end{frame}

\begin{frame}[t]{Potencia y energía en microprocesadores}
\begin{itemize}
  \item En tecnología CMOS, el \textmark{consumo de energía} se deriva de la
        \textmark{conmutación de transistores}.

  \mode<presentation>{\pause\vfill}
  \item \textgood{Energía dinámica}:
    \begin{itemize}
      \item Cantidad de energía necesaria para conmutar.
        \begin{itemize}
          \item $0 \rightarrow 1$ o $1 \rightarrow 0$.
          \item $E_d \approx \frac{1}{2} \cdot X_c \cdot V^2$
        \end{itemize}
    \end{itemize}

  \mode<presentation>{\pause\vfill}
  \item \textgood{Potencia dinámica}:
    \begin{itemize}
      \item Depende de la frecuencia de conmutación.
        \begin{itemize}
          \item $P_d \approx \frac{1}{2} \cdot X_c \cdot V^2 \cdot f$
        \end{itemize}
    \end{itemize}
\end{itemize}
\begin{block}{Nota}
\begin{small}
  $X_c$: Carga capacitiva\qquad
  $V$: Voltaje \qquad
  $f$: Frequencia
\end{small}
\end{block}
\end{frame}

\begin{frame}[t]{Ejemplo}
\begin{itemize}
  \item Si una reducción del voltaje del 15\% implica una reducción de frecuencia del 15\%:
    \begin{itemize}
      \item ¿Cuál es el efecto sobre la energía dinámica y la potencia dinámica?
    \end{itemize}
\end{itemize}

\mode<presentation>{\pause\vfill}
\begin{block}{Solución}
\begin{displaymath}
\frac{E_{nueva}}{E_{ant}} =
\frac
{(V \cdot 0.85)^2}
{V^2} =
0.85^2 =
0.72
\end{displaymath}
\pause
\begin{displaymath}
\frac{P_{nueva}}{P_{ant}} =
\frac
{(V \cdot 0.85)^2 \cdot (f \cdot 0.85)}
{V^2 \cdot f} =
0.85^3 =
0.61
\end{displaymath}
\end{block}
\end{frame}

\begin{frame}[t]{Evolución}
\begin{itemize}
  \item Tendencias a lo largo del tiempo:
    \begin{itemize}
      \item Incremento del \textgood{número de transistores} que conmutan.
      \item Incremento de la \textgood{frecuencia de conmutación}.
      \item \textmark{Más relevante} que el decremento en carga capacitiva y voltaje.
      \item \textemph{Efecto neto}: Crecimiento en consumo de potencia y energía.
    \end{itemize}

  \mode<presentation>{\vfill\pause}
  \item \textgood{Consumo de potencia}
    \begin{itemize}
      \item Intel 80386 (1986): 2W.
      \item Intel Core i7-6700K (2015): 91W.
      \item Intel Core i7-12700K (2021): 125W.
      \item \textbad{Aproximación a límites de enfriamiento}.
    \end{itemize}

  \mode<presentation>{\vfill\pause}
  \item \textemph{Impacto sobre frecuencia de reloj}
    \begin{itemize}
      \item Intel 80386 (1986): 33 MHz.
      \item Intel Core i7-6700K (2015): 4.00 -- 4.20 GHz.
      \item Intel Core i7-12700K (2021): 3.60 -- 4.90 GHz.
    \end{itemize}
\end{itemize}
\end{frame}

\begin{frame}[t]{Eficiencia energética}
\begin{itemize}
 \item \textgood{Técnicas}:
   \mode<presentation>{\vfill}
   \begin{enumerate}
     \item Desactivar reloj de módulos inactivos.
       \begin{itemize}
         \item Sin ejecución de coma flotante $\Rightarrow$ Desactivar unidad FP.
         \item Desactivar cores no usados.
       \end{itemize}

     \mode<presentation>{\vfill}
     \item Escalado dinámico de Voltaje-Frecuencia (DVFS -- \emph{Dynamic Voltage-Frequency Scaling}).
       \begin{itemize}
         \item Varias fecuencias de operación $\Rightarrow$ Ahorro de potencia.
       \end{itemize}

     \mode<presentation>{\vfill}
     \item Modos de baja potencia.
       \begin{itemize}
         \item Requiere mecanismo de reactivación.
       \end{itemize}
     \mode<presentation>{\vfill}

     \item \emph{Overclocking} automático.
       \begin{itemize}
         \item Activado cuando es seguro.
         \item Ejemplo: Core i7 3.3 GHz con ráfagas a 3.6 GHz.
       \end{itemize}
    \end{enumerate}
\end{itemize}
\end{frame}

\begin{frame}[t]{Efecto de potencia estática}
\begin{itemize}
  \item Potencia estática cada vez más importante.
    \begin{itemize}
      \item La corriente fluye incluso si el transistor está apagado.

\pause
\begin{block}{Potencia estática}
\begin{displaymath}
P_s \approx current\_{static} \times Voltage
\end{displaymath}
\end{block}

      \pause
      \item La potencia estática es proporcional al número de dispositivos.
    \end{itemize}

  \mode<presentation>{\vfill\pause}
  \item \textmark{Impacto}:
    \begin{itemize}
      \item Creciente número de transistores $\Rightarrow$ incremento de potencia.
        \begin{itemize}
          \item \textbad{Incluso si inactivo!}
        \end{itemize}

      \pause
      \item \textmark{Limitación de potencia}: 
            Apagar suministro de potencia a módulos inactivos.            

      \pause
      \item Chips de alto rendimiento pueden perder hasta el 50\%.
        \begin{itemize}
          \item Grandes memorias caché que usan SRAM (RAM estática).
        \end{itemize}
    \end{itemize}
\end{itemize}
\end{frame}
