\section{Barreras}

\begin{frame}[t]{Barrera}
\begin{itemize}
  \item Una barrera permite \textmark{sincronizar} varios procesos en algún punto.
  
    \mode<presentation>{\vfill}
    \begin{itemize}
      \item Garantiza que ningún proceso supera la barrera hasta que todos han llegado.
      \mode<presentation>{\vfill}
      \item Usadas para sincronizar fases de un programa.
    \end{itemize}
\end{itemize}
\end{frame}

\begin{frame}[t]{Barreras centralizadas}
\begin{itemize}
  \item \textgood{Contador} centralizado asociado a la \textmark{barrera}.
    \begin{itemize}
      \item Cuenta el número de procesos que han llegado a la barrera.
    \end{itemize}

  \mode<presentation>{\vfill\pause}
  \item \textgood{Función barrera}:
    \begin{itemize}
      \item Incrementa el contador.
      \item Espera a que el contador llegue al número de procesos a sincronizar.
    \end{itemize}
\end{itemize}
\end{frame}

\begin{frame}[t,fragile]{Barrera simple}
\begin{columns}

\column{.6\textwidth}
\begin{block}{Implementación simple}
\begin{lstlisting}
do_barrier(barrera, n) {
  lock(barrera.cerrojo);
  if (barrera.contador == 0) {
    barrera.flag=0;
  }
  contador_local = barrera.contador++;
  unlock(barrera.cerrojo);
  if (contador_local == np) {
    barrera.contador=0;
    barrera.flag=1;
  }
  else {
    while (barrera.flag==0) {}
  }
}
\end{lstlisting}
\end{block}

\column{.4\textwidth}
\begin{itemize}
  \item Problema si se reusa la barrera en un bucle.
\end{itemize}

\end{columns}
\end{frame}

\begin{frame}[t,fragile]{Barrera con inversión de sentido}
\begin{columns}

\column{.6\textwidth}
\begin{block}{Implementación simple}
\begin{lstlisting}
do_barrera(barrera, n) {
  flag_local = !flag_local;
  lock(barrera.cerrojo);
  contador_local = barrera.contador++;
  unlock(barrera.cerrojo);
  if (contador_local == np) {
    barrera.contador=0;
    barrera.flag=flag_local;
  }
  else {
    while (barrera.flag==flag_local) {}
  }
}
\end{lstlisting}
\end{block}

\column{.4\textwidth}

\end{columns}
\end{frame}

\begin{frame}[t]{Barreras en árbol}
\begin{itemize}
  \item Una implementación simple de barreras no es \textbad{escalable}.
    \begin{itemize}
      \item Contención en el acceso a variables compartidas.
    \end{itemize}
  \mode<presentation>{\vfill\pause}
  \item Estructuración de los procesos de llegada y liberación en forma de árbol.
    \begin{itemize}
      \item Especialmente útil en redes distribuidas.
    \end{itemize}

\end{itemize}
\end{frame}
