\section{Introducción}

\begin{frame}[t]{Sincronización en memoria compartida}
\begin{itemize}
  \item La comunicación se realiza a través de memoria compartida.
    \begin{itemize}
      \item Es necesario sincronizar accesos a variable compartida.
    \end{itemize}

  \mode<presentation>{\vfill\pause}
  \item \textgood{Alternativas}:
    \begin{itemize}
      \item Comunicación 1-1.
      \item Comunicación colectiva (1-N).
    \end{itemize}
\end{itemize}
\end{frame}

\begin{frame}[t]{Comunicación 1 a 1}
\begin{itemize}
  \item Asegurar que la lectura (\textmark{recepción}) se produce después de la
        escritura (\textmark{emisión}).
  
  \mode<presentation>{\vfill\pause}
  \item En caso de \textmark{reutilización} (bucles):
    \begin{itemize}
      \item Asegurar que escritura (\textmark{emisión}) después de lectura (\textmark{recepción}) anterior.
    \end{itemize}

  \mode<presentation>{\vfill\pause}
  \item Necesidad de accesos con \textgood{exclusión mutua}.
    \begin{itemize}
      \item Solamente uno de los procesos accede a una variable a la vez.
    \end{itemize}

  \mode<presentation>{\vfill\pause}
  \item \textbad{Sección crítica}:
    \begin{itemize}
      \item Secuencia de instrucciones que acceden a una o varias variables con exclusión mutua.
    \end{itemize}

\end{itemize}
\end{frame}

\begin{frame}[t]{Comunicación colectiva}
\begin{itemize}
  \item Necesidad de \textgood{coordinación} de \textmark{múltiples accesos} a variables.
    \begin{itemize}
      \item Escrituras sin interferencias.
      \item Lecturas \textmark{deben esperar} a que datos estén disponibles.
    \end{itemize}

  \mode<presentation>{\vfill\pause}
  \item Deben \textmark{garantizarse accesos} a variable en \textgood{exclusión mutua}.

  \mode<presentation>{\vfill\pause}
  \item Debe \textmark{garantizarse} que \textbad{no se lee resultado} 
        hasta que todos han ejecutado su sección crítica.
  
\end{itemize}
\end{frame}

\begin{frame}[fragile]{Suma de un vector}
\mode<presentation>{\vspace{-1.25em}}
\begin{columns}[T]

\column{.5\textwidth}
\begin{block}{Sección crítica en bucle}
\begin{lstlisting}
void f(const std::vector<double> & v) {
  double sum = 0;
  
  auto do_sum = [&](int start, int n) {
    for (int i=start; i<n; ++i) {
      sum += v[i];
    }
  }

  thread t1{do_sum, 0, std::ssize(v)/2};
  thread t2{do_sum, std::ssize(v)/2+1, std::ssize(v)};
  t1.join();
  t2.join();
}
\end{lstlisting}
\end{block}

\mode<presentation>{\pause}

\column{.5\textwidth}
\begin{block}{Sección crítica fuera bucle}
\begin{lstlisting}
void f(const std::vector<double> & v) {
  double sum = 0;
  
  auto do_sum = [&](int start, int n) {
    double local_sum = 0;
    for (int i=start; i<n; ++i) {
      local_sum += v[i];
    }
    sum += local_sum
  }

  thread t1{do_sum, 0, std::ssize(v)/2};
  thread t2{do_sum, std::ssize(v)/2+1, std::ssize(v)};
  t1.join();
  t2.join();
}
\end{lstlisting}
\end{block}

\end{columns}
\end{frame}
