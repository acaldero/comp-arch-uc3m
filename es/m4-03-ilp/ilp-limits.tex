\section{Límites del ILP}

\begin{frame}[t]{Límites de ILP}
\begin{itemize}
  \item Para estudiar el \textmark{ILP} máximo se modela el \textgood{procesador ideal}.

  \mode<presentation>{\vfill}
  \item \textgood{Procesador ideal}:
    \begin{itemize}
      \item \textmark{Renombrado de registros infinito}: 
            Se pueden evitar todos los riesgos WAR y WAW.
      \item \textmark{Predicción de bifurcación perfecta}: 
            Todos las predicciones de bifurcaciones condicionales dan acierto.
      \item \textmark{Predicción perfecta de saltos}: 
            Todos los saltos (incluyendo retornos) se predicen correctamente.
      \item \textmark{Análisis perfecto de alias de direcciones de memoria}: 
            Se puede mover un load antes de un store si la dirección no es idéntica.
      \item \textmark{Cachés perfectas}: 
            Todos los acceso a caché requieren un ciclo de reloj (siempre hay acierto).
    \end{itemize}
\end{itemize}
\end{frame}

\begin{frame}[t]{ILP disponible}
\begin{tikzpicture}
  \begin{axis}[
    xbar,
    width=.8\textwidth, height=.9\textheight,
    xlabel={Instrucciones por ciclo},
    symbolic y coords={gcc,expresso,li,fppp,doduc,tomcatv},
    ytick=data,
    nodes near coords, nodes near coords align={horizontal},
    ]
    \addplot coordinates {
  (54.8,gcc)
  (62.6,expresso)
  (17.9,li)
  (75.2,fppp)
  (118.7,doduc)
  (150.1,tomcatv)
};
  \end{axis}
\end{tikzpicture}
\end{frame}

\begin{frame}[t]{Sin embargo \ldots}
\begin{itemize}
  \item Más ILP implica más lógica de control:
    \begin{itemize}
      \item Cachés de menor tamaño.
      \item Ciclos de reloj más largos.
      \item Mayor consumo de energía.
    \end{itemize}

  \mode<presentation>{\vfill}
  \item \textbad{Limitación práctica}:
    \begin{itemize}
      \item Emisión de 3 a 6 instrucciones por ciclo.
    \end{itemize}
\end{itemize}
\end{frame}
