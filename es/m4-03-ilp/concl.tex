\section{Conclusión}

\begin{frame}[t]{Resumen}
\begin{itemize}[<+->]
  \mode<presentation>{\vfill}
  \item La planificación dinámica gestiona detenciones desconocidas en tiempo de compilación.
  
  \mode<presentation>{\vfill}
  \item Las técnicas especulativas se apoyan de la predicción de saltos y la planificación dinámica.
  
  \mode<presentation>{\vfill}
  \item La emisión múltiple en ILP queda limitada de forma práctica de 3 a 6.
  
  \mode<presentation>{\vfill}
  \item SMT como aproximación a TLP dentro un núcleo.
\end{itemize}
\end{frame}


\begin{frame}[t]{Referencias}
\begin{itemize}
  \item \bibhennessy
    \begin{itemize}
      \item 3.3 -- Reducing Branch Costs with advanced Branch Prediction.
      \item 3.4 -- Overcoming Data Hazards with Dynamic Scheduling.
      \item 3.6 -- Hardware-Based Speculation.
      \item 3.7 -- Exploiting ILP using Multiple Issue and Statich Scheduling.
      \item 3.11 -- Multithreading: Exploiting Thread-Level Parallelism to 
            Improve Uniprocessor Throughput.
    \end{itemize}
\end{itemize}
\end{frame}
