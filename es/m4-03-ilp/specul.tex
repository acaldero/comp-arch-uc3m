\section{Especulación}

\begin{frame}[t]{Bifurcaciones y límites de paralelismo}
\begin{itemize}
  \item Al aumentar el paralelismo conseguido, las \textbad{dependencias de control} se convierten en problema.
    \begin{itemize}
      \item La predicción de bifurcaciones no es suficiente.
    \end{itemize}

  \mode<presentation>{\vfill}
  \item El siguiente paso es la \textgood{especulación} sobre el 
        \textmark{resultado de las bifurcaciones} y la \textgood{ejecución} 
        asumiendo que la \textmark{especulación fue correcta}.
    \begin{itemize}
      \item Se capta, emite y ejecuta instrucciones.
      \item Se necesita un mecanismo de tratamiento si la especulación no era correcta.
    \end{itemize}
\end{itemize}
\end{frame}

\begin{frame}[t]{Componentes}
\begin{itemize}
  \item \textgood{Ideas}:
    \begin{itemize}
      \item \textmark{Predicción dinámica de saltos}: Selecciona las instrucciones a ejecutar.
      \item \textmark{Especulación}: Ejecución antes de que se resuelvan dependencias de control y capacidad para deshacer.
      \item \textmark{Planificación dinámica}.
    \end{itemize}

  \mode<presentation>{\vfill\pause}
  \item Para conseguirlo se debe \textgood{separar}:
    \begin{itemize}
      \item El \textmark{paso del resultado} de una instrucción a otra que lo usa.
      \item La \textmark{finalización} de la instrucción.
    \end{itemize}

  \mode<presentation>{\vfill}
  \item \textmark{IMPORTANTE}:No se actualiza el estado del procesador 
        (registros/memoria) hasta que no se tiene confirmación.
\end{itemize}
\end{frame}

\begin{frame}[t]{Solución}
\begin{itemize}
  \item \textgood{Reorder Buffer} (\textmark{ROB}):
    \begin{itemize}
      \item Cuando se finaliza una instrucción se escribe en el \textmark{ROB}.
      \item Cuando se confirma su ejecución se escribe en destino real.
      \item Las instrucciones leen datos modificados del \textmark{ROB}.
    \end{itemize}

  \mode<presentation>{\vfill\pause}
  \item Entradas del \textmark{ROB}:
    \begin{itemize}
      \item \textmark{Tipo de instrucción}: branch, store, operación de registro.
      \item \textmark{Destino}: Id de registro o dirección de memoria.
      \item \textmark{Valor}: Valor del resultado de la instrucción.
      \item \textmark{Ready}: Indica si la instrucción se ha completado.
    \end{itemize}
\end{itemize}
\end{frame}
