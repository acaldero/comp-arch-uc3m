\section{Técnicas avanzadas de predicción de bifurcación}

\begin{frame}[t]{Predicción de bifuración}
\begin{itemize}
  \item Las \textgood{bifurcaciones} tienen un alto impacto sobre 
        el rendimiento de los programas.

  \mode<presentation>{\vfill}
  \item Para \textmark{reducir} el impacto:
    \begin{itemize}
      \item Desenrollamiento de bucles.
      \item Predicción de saltos.
        \begin{itemize}
          \item Tiempo de compilación.
          \item Comportamiento de cada bifurcación de forma aislada.
        \end{itemize}
      \item Predicción avanzada de bifurcación:
        \begin{itemize}
          \item Predictores con correlación.
          \item Predictores por turnos.
        \end{itemize}
    \end{itemize}
\end{itemize}
\end{frame}

\begin{frame}[t]{Dependencia entre bifurcaciones}

\begin{columns}[T]

\begin{column}{.4\textwidth}
\begin{itemize}
  \item Si la primera y la segunda bifurcación se toman, la tercera 
        \textbad{NO} se toma.
\end{itemize}
\end{column}

\begin{column}{.3\textwidth}
\begin{block}{Ejemplo}
\lstinputlisting{int/m4-03-ilp/branch-pred-ex1.cpp}
\end{block}
\end{column}

\pause
\begin{column}{.3\textwidth}
\begin{block}{Código generado}
\lstinputlisting[language=generalasm2,basicstyle=\tiny]{int/m4-03-ilp/branch-pred-ex1.asm}\end{block}
\end{column}
\end{columns}

\mode<presentation>{\vfill\pause}
\begin{itemize}
  \item La predicción aislada para cada salto no aporta ventajas.
    \begin{itemize}
      \item Es necesaria predecir dependiendo de lo ocurrido en otras bifurcaciones.
    \end{itemize}
\end{itemize}

\end{frame}


\begin{frame}[t]{Predicción con correlación}
\begin{itemize}
  \item Se mantiene la \textmark{historia} de las últimas bifurcaciones 
        para seleccionar entre varios predictores.

  \mode<presentation>{\vfill\pause}
  \item Un predictor $(m,n)$:
    \begin{itemize}
      \item Usa el resultado de las $m$ últimas bifurcaciones para seleccionar 
            entre $2^m$ predictores.
      \item Cada predictor de $n$ bits.
    \end{itemize}

  \mode<presentation>{\vfill\pause}
  \item Predictor $(1,2)$:
    \begin{itemize}
      \item Resultado de última bifurcación para elegir entre 2 predictores.
    \end{itemize}
\end{itemize}
\end{frame}

\begin{frame}[t]{Tamaño del predictor}
\begin{itemize}
  \item Un predictor (m,n) tiene varias entradas por cada dirección de
        bifurcación.

  \item Tamaño total:
\[
T = 2^m \times n \times entradas_{\text{dirección}}
\]

  \item Ejemplos:
    \begin{itemize}
      \item $(0,2)$ con 4K entradas $\rightarrow$ 8 Kb
      \item $(2,2)$ con 4K entradas $\rightarrow$ 32 Kb
      \item $(2,2)$ con 1K entradas $\rightarrow$ 8 Kb
   \end{itemize}
\end{itemize}
\end{frame}

\begin{frame}[t]{Tasa de fallos}
\begin{itemize}
  \item Predictor con correlación tiene menos fallos que predictor simple de mismo tamaño.
  \item Predictor con correlación tiene menos fallos que predictor simple con número ilimitado de entradas.
\end{itemize}

\end{frame}


\begin{frame}[t]{Predicción por turnos}
\begin{itemize}
  \item \textgood{Combina dos predictores}:
    \begin{itemize}
      \item Predictor basado en \textmark{información global}.
      \item Predictor basado en \textmark{información local}.
    \end{itemize}
  \item Utiliza un selector para elegir entre predictores.
    \begin{itemize}
      \item El cambio entre dos selecciones usa un contador con saturación (2 bits).
    \end{itemize}
  \item \textgood{Ventaja}:
    \begin{itemize}
      \item Permite comportamiento distinto para enteros y FP.
    \end{itemize}
  \item \textgood{SPEC}:
    \begin{itemize}
      \item \textmark{Benchmarks enteros} $\rightarrow$ predictor global 40\%
      \item \textmark{Benchmarks FP} $\rightarrow$ predictor global 15\%.
    \end{itemize}
  \item \textgood{Usos}: Alpha y AMD Opteron.
\end{itemize}
\end{frame}


\begin{frame}[t]{Intel Core i7}
\begin{itemize}
  \item Predictor de \textgood{dos niveles}:
    \begin{itemize}
      \item Predictor de primer nivel más pequeño.
      \item Segundo predictor más grande como \emph{backup}.
    \end{itemize}

  \mode<presentation>{\vfill}
  \item Cada predictor combina \textgood{3 predictores}:
    \begin{itemize}
      \item Predictor simple de 2 bits.
      \item Predictor de historia global.
      \item Predictor de salida de bucle (contador de iteraciones).
    \end{itemize}

  \mode<presentation>{\vfill}
  \item Además:
    \begin{itemize}
      \item Predicción de \textmark{saltos indirectos}.
      \item Predicción de \textmark{direcciones de retorno}.
    \end{itemize}
\end{itemize}
\end{frame}
