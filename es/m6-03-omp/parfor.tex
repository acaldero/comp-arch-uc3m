\section{Bucles paralelos}

\begin{frame}[t,fragile]{Parallel for}
\begin{itemize}
  \item \textgood{División de bucle}: Realiza una partición de las iteraciones de
        un bucle entre los hilos disponible.
\end{itemize}

\mode<presentation>{\vfill\pause}
\begin{columns}[T]

\column{.35\textwidth}
\begin{block}{Sintaxis}
\begin{lstlisting}
#pragma omp parallel
{
  #pragma omp for
  for (i=0; i<n; ++i) {
    f(i);
  }
}
\end{lstlisting}
\end{block}

\column{.65\textwidth}
\begin{itemize}
  \item \cppkey{omp for} $\rightarrow$ División de bucle for.
  \item Se generar una copia privada de \cppid{i} para cada hilo.
    \begin{itemize}
      \item Se puede hacer también con \cppkey{private(}\cppid{i}\cppkey{)}
    \end{itemize}
\end{itemize}

\end{columns}

\end{frame}

\begin{frame}[t,fragile]{Ejemplo}
\mode<presentation>{\vspace{-1em}}
\begin{block}{Código secuencial}
\begin{lstlisting}
for (i=0;i<max;++i) { u[i] = v[i] + w[i]; }
\end{lstlisting}
\end{block}

\mode<presentation>{\vspace{-0.5em}}
\pause
\begin{columns}[T]

\column{.5\textwidth}

\begin{block}{Región paralela}
\begin{lstlisting}[basicstyle=\tiny]
#pragma omp parallel
{
  int id = omp_get_thread_num();  
  int nthreads = omp_get_num_threads();
  int istart = id * max / nthreads;
  int iend = (id==nthreads-1) ? ((id + 1) * max / nthreads):max;
  for (int i=istart;i<iend;++i) {
    u[i] = v[i] + w[i]; 
  }
}
\end{lstlisting}
\end{block}

\column{.5\textwidth}

\mode<presentation>{\vfill\pause}
\begin{block}{Región paralela + Bucle paralelo}
\begin{lstlisting}
#pragma omp parallel
#pragma omp for
for (i=0;i<max;++i) { 
  u[i] = v[i] + w[i]; 
}
\end{lstlisting}
\end{block}

\end{columns}
\end{frame}

\begin{frame}[t,fragile]{Construcción combinada}
\begin{itemize}
  \item Se puede usar una \textmark{forma abreviada} combinando las dos directivas.
\end{itemize}

\begin{columns}[T]

\column{.5\textwidth}

\mode<presentation>{\vfill\pause}
\begin{block}{Dos directivas}
\begin{lstlisting}
vector<double> vec(max);
#pragma omp parallel
{
  #pragma omp for
  for (i=0; i<max; ++i) {
    vec[i] = generate(i);
  }
}
\end{lstlisting}
\end{block}

\column{.5\textwidth}

\mode<presentation>{\vfill\pause}
\begin{block}{Directiva combinada}
\begin{lstlisting}
vector<double> vec(max);
#pragma omp parallel for
for (i=0; i<max; ++i) {
  vec[i] = generate(i);
}
\end{lstlisting}
\end{block}

\end{columns}
\end{frame}

\begin{frame}[t,fragile,shrink=10]{Reducciones}
\begin{columns}

\column{.4\textwidth}

\begin{block}{Ejemplo}
\begin{lstlisting}
double sum = 0.0; 
vector<double> v(max);
for (int i=0; i<max; ++i) {
  suma += v[i];
}
\end{lstlisting}
\end{block}

\column{.6\textwidth}
\begin{itemize}
  \item Una \textgood{reducción} es una operación de acumulación
        en un bucle.
  \pause
  \item Clausula de reducción: \cppkey{reduction (op:var)}
\end{itemize}

\end{columns}

\mode<presentation>{\vfill\pause}

\begin{columns}

\column{.4\textwidth}

\begin{itemize}
  \item \textgood{Efectos}:
    \begin{itemize}
      \item Copia privada de cada variable.
      \item Actualización de copia local en cada iteración.
      \item Copias locales combinadas al final.
    \end{itemize}
\end{itemize}

\column{.6\textwidth}

\pause
\begin{block}{Ejemplo}
\begin{lstlisting}
double sum = 0.0; 
vector<double> v(max);
#pragma omp parallel for reduction(+:suma)
for (int i=0; i<max; ++i) {
  suma += v[i];
}
\end{lstlisting}
\end{block}

\end{columns}

\end{frame}

\begin{frame}[t]{Operaciones de reducción}
\begin{itemize}
  \item Operaciones que son asociativas.
\[
(a \oplus b) \oplus c = a \oplus (b \oplus c)
\]

  \item Valor inicial definido por la operación.

  \mode<presentation>{\vfill\pause}
  \item \textgood{Operadores básicos}:
    \begin{itemize}
      \item \cppkey{+} (valor inicial: \cppid{0}).
      \item \cppkey{*} (valor inicial: \cppid{1}).
      \item \cppkey{-} (valor inicial: \cppid{0}).
    \end{itemize}

  \mode<presentation>{\vfill\pause}
  \item \textgood{Operadores avanzados}:
    \begin{itemize}
      \item \cppkey{\&} (valor inicial: \cppid{~0}).
      \item \cppkey{|} (valor inicial: \cppid{0}).
      \item \cppkey{\^} (valor inicial: \cppid{0}).
      \item \cppkey{\&\&} (valor inicial: \cppid{1}).
      \item \cppkey{||} (valor inicial: \cppid{0}).
    \end{itemize}

\end{itemize}
\end{frame}

\begin{frame}[t]{Ejercicio 4}
\begin{itemize}
  \item Modifique el programa de cálculo de $\pi$.
  \item Intente que el programa sea lo más parecido posible al programa secuencial
        con una reducción.
\end{itemize}
\end{frame}
