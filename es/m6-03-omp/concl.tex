\section{Conclusión}

\begin{frame}[t]{Resumen}
\begin{itemize}
  \item \textgood{OpenMP} permite anotar código secuencial para
        hacer uso de \textmark{paralelismo fork-join}.
    \begin{itemize}
      \item Basado en el concepto de región paralela.
    \end{itemize}

  \mode<presentation>{\vfill\pause}
  \item Los mecanismos de sincronización pueden ser de 
        \textmark{alto nivel} o \textmark{bajo nivel}.

  \mode<presentation>{\vfill\pause}
  \item Los bucles paralelos combinados con las reducciones permiten
        preservar el código original de muchos algoritmos.

  \mode<presentation>{\vfill\pause}
  \item Los \textmark{atributos de almacenamiento} permiten controlar
        las copias y compartición de datos con las regiones paralelas.

  \mode<presentation>{\vfill\pause}
  \item OpenMP ofrece varios tipos de planificación.
\end{itemize}
\end{frame}


\begin{frame}[t]{Referencias}
\begin{itemize}
  \item Libros:
    \begin{itemize}
      \item \textmark{The OpenMP Common Core}
            T. Mattson, Y. Hen, A.E. Koniges.
            MIT Press, 2019.
    \end{itemize}
  \mode<presentation>{\vfill}
  \item Páginas Web:
    \begin{itemize}
      \item OpenMP: \url{http://www.openmp.org}.
      \item Lawrence Livermore National Laboratory Tutorial: \url{https://computing.llnl.gov/tutorials/openMP/}.
    \end{itemize}
\end{itemize}
\end{frame}
