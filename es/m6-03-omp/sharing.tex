\section{Compartición de datos}

\begin{frame}[t]{Atributos de almacenamiento}
\begin{itemize}
  \item Modelo de programación en \textgood{memoria compartida}:
    \begin{itemize}
      \item \textgood{Variables compartidas}.
      \item \textgood{Variables privadas}.
    \end{itemize}

  \mode<presentation>{\vfill\pause}
  \item \textgood{Compartidas}:
    \begin{itemize}
      \item Variables globales (alcance de fichero y espacio de nombres).
      \item Variables \cppkey{static}.
      \item Objetos en memoria dinámica (\cppid{malloc()} y \cppkey{new}).
    \end{itemize}

  \mode<presentation>{\vfill\pause}
  \item \textgood{Privadas}:
    \begin{itemize}
      \item Variables locales de funciones invocadas desde una región paralela.
      \item Variables locales definidas dentro de un bloque.
    \end{itemize}
\end{itemize}
\end{frame}

\begin{frame}[t,fragile]{Modificación de atributos de almacenamiento}
\mode<presentation>{\vspace{-1em}}
\begin{itemize}
  \item Atributos en clausulas paralelas:
    \begin{itemize}
      \item \cppkey{shared}.
      \item \cppkey{private}.
      \item \cppkey{firstprivate}.
    \end{itemize}

  \item \cppkey{private} crea una nueva copia local por hilo.
    \begin{itemize}
      \item El valor de las copias no se inicializa.
    \end{itemize}
\end{itemize}

\mode<presentation>{\vfill\pause}
\begin{block}{Ejemplo}
\begin{lstlisting}
void f() {
  int x = 17;
  #pragma omp parallel for private(x) 
  for (int i=0;i<max;++i) {
    x += i;  // x inicialmente vale 17
  }
  cout << x << '\n'; // x==17
}
\end{lstlisting}
\end{block}
\end{frame}

\begin{frame}[t,fragile]{firstprivate}
\begin{itemize}
  \item Caso particular de \cppkey{private}.
    \begin{itemize}
      \item Cada copia privada se inicia con el valor de la variable en el
            hilo \textmark{maestro}.
    \end{itemize}
\end{itemize}

\mode<presentation>{\vfill\pause}
\begin{block}{Ejemplo}
\lstinputlisting[firstline=6,lastline=13]{int/m6-03-omp/examples/firstprivate/main.cpp}
\end{block}
\end{frame}

\begin{frame}[t,fragile]{lastprivate}
\begin{itemize}
  \item Pasa el valor de una variable privada de la última iteración \textmark{secuencial} a la
        variable global.
\end{itemize}

\mode<presentation>{\vfill\pause}
\begin{block}{Ejemplo}
\lstinputlisting[firstline=6,lastline=13]{int/m6-03-omp/examples/lastprivate/main.cpp}
\end{block}
\end{frame}
