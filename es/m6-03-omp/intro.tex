\section{Introducción}

\begin{frame}[t]{¿Qué es OpenMP?}
\begin{itemize}
  \item Es un \textmark{API} (y un conjunto de directivas) para expresar aplicaciones
        paralelas en sistemas de memoria compartida.

  \mode<presentation>{\vfill\pause}
  \item \textgood{Componentes}:
    \begin{itemize}
      \item Directivas de compilador.
      \item Funciones de biblioteca.
      \item Variables de entorno.
    \end{itemize}

  \mode<presentation>{\vfill\pause}
  \item Simplifica la forma de escribir programas paralelos.
    \begin{itemize}
      \item \emph{Mappings} para FORTRAN, C y C++.
    \end{itemize}
\end{itemize}
\end{frame}

\begin{frame}[t,fragile]{Construcciones}
\begin{itemize}
  \item Directivas:
\begin{lstlisting}
#pragma omp directiva [clausulas]
\end{lstlisting}

  \begin{itemize}
    \item \textmark{Ejemplo}: Fijar el número de hilos.
\begin{lstlisting}
#pragma omp parallel num_threads(4)
\end{lstlisting}
  \end{itemize}

  \mode<presentation>{\vfill\pause}
  \item Funciones de biblioteca
\begin{lstlisting}
#include <omp.h> // Incluir para llamar funciones OpenMP
\end{lstlisting}
    \begin{itemize}
      \item \textmark{ejemplo}: Obtener el número de hilos usado.
\begin{lstlisting}
int n = omp_get_num_threads();
\end{lstlisting}
    \end{itemize}

\end{itemize}
\end{frame}

\begin{frame}[t,fragile]{Ejercicio 1: Secuencial}
\begin{block}{exseq.cpp}
\lstinputlisting{lab/04-omp/hello/exseq.cpp}
\end{block}
\begin{itemize}
  \item Imprime en salida estándar.
\end{itemize}
\begin{lstlisting}[style=terminal]
Hello(0) world(0)
\end{lstlisting}
\end{frame}

\begin{frame}[t,fragile]{Ejercicio 1: Paralelo}
\begin{columns}[T]

\column{.5\textwidth}
\begin{block}{exseq.cpp}
\lstinputlisting{lab/04-omp/hello/expar.cpp}
\end{block}

\column{.5\textwidth}
\begin{itemize}
  \item Imprime en salida estándar.
\end{itemize}
\begin{lstlisting}[style=terminal]
Hello(Hello(10) world(0Hello() world(1)
3) world(3)
Hello(2) world(2)
)
\end{lstlisting}

\end{columns}
\end{frame}

\begin{frame}[t,fragile]{Compilación y OpenMP}

\begin{itemize}
  \item Flags de compilación: 
    \begin{itemize}
      \item \textgood{gcc}: \textmark{-fopenmp}
      \item \textgood{Intel Linux}: \textmark{-openmp}
      \item \textgood{Intel Windows}: \textmark{/Qopenmp}
      \item \textgood{Microsft Visual Studio}: \textmark{/openmp}
    \end{itemize}

  \mode<presentation>{\vfill\pause}
  \item Usando CMake:
  \begin{itemize}
    \item Activar el soporte para \textmark{OpenMP} en CMake.
\begin{lstlisting}
find_package(OpenMP REQUIRED)
\end{lstlisting}

    \item Activar las opciones de compilación y enlace para un ejecutable.
\begin{lstlisting}
add_executable(expar expar.cpp)
target_link_libraries(expar PUBLIC OpenMP::OpenMP_CXX)
\end{lstlisting}
  \end{itemize}
\end{itemize}
\end{frame}

\begin{frame}[t]{Ejercicio 1}
\begin{itemize}
  \item \textgood{Objetivo}: Verificar que el entorno funciona.
  
  \mode<presentation>{\vfill\pause}
  \item \textgood{Actividades}:
  \begin{enumerate}
    \item Compilar la versión secuencial y ejecutar.
    \item Compilar la versión paralela y ejecutar.
    \item Introduzca la función \cppid{omp\_get\_num\_threads()} para
          imprimir el número de hilos:
      \begin{enumerate}[a)]
        \item Antes del \cppkey{pragma}.
        \item Justo después del \cppkey{pragma}.
        \item Dentro del bloque.
        \item Antes de salir del programa, pero fuera del bloque.
      \end{enumerate}
  \end{enumerate}
\end{itemize}
\end{frame}

\begin{frame}[t]{Observaciones}
\begin{itemize}
  \item Modelo de memoria compartida multi-hilo.
    \begin{itemize}
      \item Comunicación mediante variables compartidas.
    \end{itemize}

  \mode<presentation>{\vfill\pause}
  \item Compartición accidental $\rightarrow$ \textbad{condiciones de carrera}.
    \begin{itemize}
      \item Resultado dependiente de la planificación de los hilos.
    \end{itemize}

  \mode<presentation>{\vfill\pause}
  \item Evitar condiciones de carrera.
    \begin{itemize}
      \item Sincronización para evitar conflictos.
        \begin{itemize}
          \item Coste de sincronización.
        \end{itemize}
      \item Modificación en patrón de accesos.
        \begin{itemize}
          \item Minimizar sincronizaciones necesarias.
        \end{itemize}
    \end{itemize}

\end{itemize}
\end{frame}
