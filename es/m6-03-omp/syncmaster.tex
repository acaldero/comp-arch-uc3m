\section{Sincronización con master}

\begin{frame}[t,fragile]{Barreras}
\mode<presentation>{\vspace{-1em}}
\begin{itemize}
  \item Permite sincronizar todos los hilos en un punto.
    \begin{itemize}
      \item Se espera hasta que todos los puntos llegan a la \textmark{barrera}.
    \end{itemize}
\end{itemize}
\begin{block}{Ejemplo}
\begin{lstlisting}[basicstyle=\tiny]
#pragma omp parallel
{
  int id = omp_get_thread_num();
  v[id] = f(id);
  #pragma omp barrier

  #pragma omp for
  for (int i=0;i<max;++i) {
    w[i] = g(i);
  } // Barrera implícita

  #pragma omp for nowait
  for (int i=0;i<max;++i) {
    w[i] = g(i);
  } // nowait -> No hay barrera implícita

  v[i] = h(i);
} // Barrera implícita
\end{lstlisting}
\end{block}
\end{frame}

\begin{frame}[t,fragile]{Ejecución única: master}
\begin{itemize}
  \item La clausula \cppkey{master} marca un bloque que solamente se ejecuta
        en el hilo \emph{maestro}.
\end{itemize}
\begin{block}{Ejemplo}
\begin{lstlisting}
#pragma omp parallel
{
  f(); // En todos los hilos
  #pragma omp master
  {
    g(); // Solamente en maestro
    h(); // Solamente en maestro
  }
  i(); // En todos los hilos
}
\end{lstlisting}
\end{block}
\end{frame}

\begin{frame}[t,fragile]{Ejecución única: single}
\begin{itemize}
  \item La clausula \cppkey{single} marca un bloque que solamente se ejecuta
        en un hilo.
    \begin{itemize}
      \item No tiene por qué ser el hilo maestro.
    \end{itemize}
\end{itemize}
\begin{block}{Ejemplo}
\begin{lstlisting}
#pragma omp parallel
{
  f(); // En todos los hilos
  #pragma omp single
  {
    g(); // Solamente en algún hilo 
    h(); // Solamente en algún hilo
  }
  i(); // En todos los hilos
}
\end{lstlisting}
\end{block}
\end{frame}

\begin{frame}[t,fragile]{Ordenación}
\begin{itemize}
  \item Una región \cppkey{ordered} se ejecuta en orden secuencial.
\end{itemize}
\begin{block}{Ejemplo}
\begin{lstlisting}
#pragma omp parallel
{
  #pragma omp for ordered reduction(+:res)
  for (int i=0;i<max;++i) {
    double tmp = f(i);
    #pragma ordered
    res += g(tmp);
  }
}
\end{lstlisting}
\end{block}

\end{frame}

\begin{frame}[t,fragile]{Cerrojos simples}
\begin{itemize}
  \item Cerrojos de la biblioteca OpenMP.
    \begin{itemize}
      \item También hay cerrojos anidados.
    \end{itemize}
\end{itemize}
\begin{block}{Ejemplo}
\begin{lstlisting}
omp_lock_t l;
omp_init_lock(&l);

#pragma omp parallel
{
  int id = omp_get_thread_num();
  double x = f(i);
  omp_set_lock(&l);
  cout << "ID=" << id << " tmp= " << tmp << endl;
  omp_unset_lock(&l);
}
omp_destroy_lock(&l);
\end{lstlisting}
\end{block}
\end{frame}

\begin{frame}[t]{Otras funciones de biblioteca}
\begin{itemize}
  \item \textgood{Cerrojos anidados}:
    \begin{itemize}
      \item
        \cppid{omp\_init\_nest\_lock()},
        \cppid{omp\_set\_nest\_lock()},
        \cppid{omp\_unset\_nest\_lock()},
        \cppid{omp\_test\_next\_lock()},
        \cppid{omp\_destroy\_nest\_lock()}.
    \end{itemize}
  \item \textgood{Consulta de procesadores}:
    \begin{itemize}
      \item
        \cppid{omp\_num\_procs()}.
    \end{itemize}
  \item \textgood{Número de hilos}:
    \begin{itemize}
      \item
        \cppid{omp\_set\_num\_threads()},
        \cppid{omp\_get\_num\_threads()},
        \cppid{omp\_get\_thread\_num()},
        \cppid{omp\_get\_max\_threads()}.
    \end{itemize}
  \item \textgood{Comprobación de región paralela}:
    \begin{itemize}
      \item
        \cppid{omp\_in\_parallel()}.
    \end{itemize}
  \item \textgood{Selección dinámica de número de hilos}:
    \begin{itemize}
      \item
        \cppid{omp\_set\_dynamic()},
        \cppid{omp\_get\_dynamic()}.
    \end{itemize}
\end{itemize}
\end{frame}
