\section{Hilos en OpenMP}

\begin{frame}[t,fragile]{Paralelismo \emph{fork-join}}
\begin{itemize}
  \item Aplicación secuencial con secciones paralelas:
    \begin{itemize}
      \item \textgood{Hilo maestro}: Iniciado con el programa principal.
      \item En una sección paralela se arranca un conjunto de hilos.
      \item Se puede \textmark{anidar} el paralelismo.
    \end{itemize}

  \mode<presentation>{\vfill\pause}
  \item Una región paralela es un bloque marcado con la directiva \cppkey{parallel}.
\begin{lstlisting}
#pragama omp parallel
\end{lstlisting}
\end{itemize}
\end{frame}

\begin{frame}[t,fragile]{Selección del número de hilos}
\begin{columns}[T]

\column{.5\textwidth}
\begin{block}{Función de biblioteca}
\begin{lstlisting}[basicstyle=\tiny]
omp_set_num_threads(4);
#pragma omp parallel
{
  // Región paralela
}
\end{lstlisting}
\end{block}

\mode<presentation>{\vfill\pause}
\column{.5\textwidth}
\begin{block}{Directiva OpenMP}
\begin{lstlisting}[basicstyle=\tiny]
#pragma omp parallel num_threads(4)
{
  // Región paralela
}
\end{lstlisting}
\end{block}
\end{columns}

\mode<presentation>{\vfill\pause}
\begin{columns}

\column{.25\textwidth}

\column{.5\textwidth}

\begin{block}{Variable de entorno}
\begin{lstlisting}[style=terminal,basicstyle=\tiny]
OMP_NUM_THREADS=8 program
\end{lstlisting}
\end{block}

\column{.25\textwidth}
\end{columns}

\end{frame}

\begin{frame}[t,shrink=20]{Ejercicio 2: Cálculo de $\pi$}
\begin{itemize}

  \item Cálculo de $\pi$.
\[
\int_{0}^{1} \frac{1}{1+x^2} dx =
\arctan{1} - \arctan{0} =
\frac{\pi}{4}
\]

  \mode<presentation>{\vfill\pause}
  \item Aproximación:
\[
\pi \approx 4 \cdot \sum_{i=0}^{N} f(x_{i} + \frac{\Delta x}{2}) \Delta x =
4 \cdot \Delta x \sum_{i=0}^{N} \frac{1}{1 + (x_i + \frac{\Delta x}{2} )^2}
\]

  \mode<presentation>{\vfill\pause}
  \item Suma del área de N rectángulos.
    \begin{itemize}
      \item \textmark{Base}: $\Delta x = \frac{1}{N}$.
      \item \textmark{Puntos}: $x_i = \Delta x \cdot i$
      \item \textmark{Altura}: $f(x_i + \frac{\Delta x}{2})$ = $f(\Delta x \cdot i + \frac{\Delta x}{2})$
            $f(\Delta x (i + \frac{1}{2})$.
    \end{itemize}

\[
\pi \approx 
4 \cdot \Delta x \sum_{i=0}^{N} \frac{1}{1 + (\Delta x \cdot (i + \frac{1}{2} ))^2}
\]
\end{itemize}
\end{frame}

\begin{frame}[t]{Ejercicio 2: Versión secuencial}
\mode<presentation>{\vspace{-1em}}
\begin{block}{Cálculo de $\pi$}
\lstinputlisting{lab/04-omp/pi/piseq.cpp}
\end{block}
\end{frame}

\begin{frame}[t,fragile]{Medición del tiempo en C++11}
\begin{itemize}

\item Ficheros de \textmark{include}:
\begin{lstlisting}
#include <chrono>
\end{lstlisting}

\item Tipo para el reloj:
\begin{lstlisting}
using clk = chrono::high_resolution_clock;
\end{lstlisting}

\item Obtener un punto del tiempo:
\begin{lstlisting}
auto t1 = clk::now();
\end{lstlisting}

\item Diferencias de tiempo (puede especificar unidad).
\begin{lstlisting}
auto diff = duration_cast<microseconds>(t2-t1);
\end{lstlisting}

\item Obtención del valor de la diferencia.
\begin{lstlisting}
cout << diff.count();
\end{lstlisting}

\end{itemize}
\end{frame}

\begin{frame}[t,fragile]{Ejemplo de medida de tiempo}
\mode<presentation>{\vspace{-1em}}
\begin{block}{Ejemplo}
\begin{lstlisting}
#include <chrono>

void f() {
  using namespace std::chrono;
  using clk = high_resolution_clock;

  auto t1 = clk::now();

  g();

  auto t2 = clk::now();
  auto diff = duration_cast<microseconds>(t2-t1);

  std::cout << "Time= " << diff.count << "microseconds\n";
}
\end{lstlisting}
\end{block}
\end{frame}

\begin{frame}[t,fragile]{Medición de tiempo en OpenMP}
\begin{itemize}

\item Punto del tiempo.
\begin{lstlisting}
double t1 = omp_get_wtime();
\end{lstlisting}

\item Diferencia de tiempo.
\begin{lstlisting}
double t1 = omp_get_wtime();
double t2 = omp_get_wtime();
double diff = t2-t1;
\end{lstlisting}

\item Diferencia de tiempo entre 2 ticks sucesivos:
\begin{lstlisting}
double tick = omp_get_wtick();
\end{lstlisting}

\end{itemize}
\end{frame}

\begin{frame}[t]{Midiendo el tiempo en cálculo de $\pi$}
\begin{block}{piseqtime.cpp}
\lstinputlisting[lastline=12]{lab/04-omp/pi/piseqtime.cpp}
\end{block}
\end{frame}

\begin{frame}[t]{Midiendo el tiempo en cálculo de $\pi$}
\begin{block}{piseqtime.cpp}
\lstinputlisting[firstline=14]{lab/04-omp/pi/piseqtime.cpp}
\end{block}
\end{frame}

\begin{frame}[t]{Ejercicio 2}
\begin{itemize}
\item Crear una versión paralela del programa de cálculo de $\pi$ usando
      una clausula \cppkey{parallel}.

\item \textgood{Observaciones}:
  \begin{itemize}
    \item Incluya mediciones de tiempo.
    \item Imprima el número de hilos que se están usando.
    \item Tenga cuidado con las variables compartidas.
    \item \textmark{Idea}: Use un array y acumule una suma parcial en cada hilo en la región paralela.
  \end{itemize}
\end{itemize}
\end{frame}
