\section{Paralelismo}

\begin{frame}[t]{Velocidad de máquina secuencial}
\begin{itemize}
  \item Una \textgood{máquina secuencial} de 1 TFLOP ($10^{12}$ FLOPS):
    \begin{itemize}
      \item Los datos deben viajar una cierta \textmark{distancia} ($r$) desde memoria a CPU.
      \item 1 \textmark{dato elemental} por ciclo:
        \begin{itemize}
          \item $\Rightarrow$ $10^{12}$ veces por segundo $\Rightarrow$ $10^{-12}$ s. por ciclo.
        \end{itemize}
      \item Datos viajando a la \textmark{velocidad de la luz}: $c=3 \cdot 10^8$ m/s.
      \item $r = 3 \cdot 10^8 \cdot 10^{-12} = 0.3 mm$
    \end{itemize}

  \mode<presentation>{\pause\vfill}
  \item 1 TB de datos en \textmark{superficie} de 0.3 mm$^2$:
    \begin{itemize}
      \item Cada dato debería almacenarse en 3 Ángstrom (aprox.).
        \begin{itemize}
          \item ¡El tamaño de un átomo pequeño!
        \end{itemize}
    \end{itemize}

  \mode<presentation>{\vfill\pause}
  \item \textemph{CONCLUSIÓN}: Aproximación secuencial no factible.
\end{itemize}
\end{frame}

\begin{frame}[t]{Tipos de paralelismo}
\begin{itemize}
  \item \textgood{Paralelismo}: La única alternativa para mejorar el rendimiento.

  \mode<presentation>{\vfill}
  \item \textmark{Restricciones}:
    \begin{itemize}
      \item Consumo energético.
      \item Costes.
    \end{itemize}

  \mode<presentation>{\vfill\pause}
  \item \textgood{Tipos de paralelismo en las aplicaciones}:
    \begin{itemize}
      \item \textmark{Paralelismo de datos}: Una operación aplicada a muchos datos.
        \begin{itemize}
          \item \textemph{Ejemplo}: Aplicar la misma operación a todos los píxeles de una imagen.
        \end{itemize}

      \item \textmark{Paralelismo de tareas}: Tareas operan independientemente y en paralelo.
        \begin{itemize}
          \item \textemph{Ejemplo}: Una cadena de filtros aplicada a una secuencia de imágenes.
        \end{itemize}
    \end{itemize}
\end{itemize}
\end{frame}

\begin{frame}[t]{Paralelismo hardware}
\begin{itemize}
  \item \textmark{ILP}: \emph{Instruction-Level Parallelism}.
    \begin{itemize}
      \item Explota paralelismo de datos con ayuda del compilador
      (segmentación, ejecución especulativa, \ldots).
    \end{itemize}

  \mode<presentation>{\pause\vfill}
  \item Arquitecturas \textmark{Vectoriales} y \textmark{GPUs}.
    \begin{itemize}
      \item Explota paralelismo de datos aplicando la misma operación
            a varios datos en paralelo.
    \end{itemize}

  \mode<presentation>{\pause\vfill}
  \item \textmark{TLP}: \emph{Thread-Level Parallelism}.
    \begin{itemize}
      \item Explota paralelismo de datos o tareas en hardware altamente acoplado.
      \item Permite interacciones entre hilos.
    \end{itemize}

  \mode<presentation>{\pause\vfill}
  \item \textmark{RLP}: Request-Level Parallelism.
    \begin{itemize}
      \item Explota paralelismo entre tareas altamente desacopladas.
    \end{itemize}
\end{itemize}
\end{frame}

\begin{frame}[t]{Taxonomía de Flynn (1966)}
\begin{itemize}
  \item Una clasificación de arquitecturas paralelas posibles.

  \mode<presentation>{\pause\vfill}
  \item \textgood{SISD}: \emph{Single Instruction Stream / Single Data Stream}.
    \begin{itemize}
      \item Mono-procesador.
      \item Puede usar técnicas de ILP.
    \end{itemize}

  \mode<presentation>{\pause\vfill}
  \item \textgood{SIMD}: \emph{Single Instruction Stream / Multiple Data Streams}.
    \begin{itemize}
      \item Las mismas instrucciones ejecutadas por procesadores diferentes sobre datos distintos.
      \item \textgood{Alternativas}: Procesadores vectoriales, extensiones multimedia y \emph{GPU}s.
    \end{itemize}

  \mode<presentation>{\pause\vfill}
  \item \textgood{MISD}: \emph{Multiple Instructions Streams/ Single Data Stream}.
    \begin{itemize}
      \item No se conocen implementaciones comerciales.
    \end{itemize}

  \mode<presentation>{\pause\vfill}
  \item \textgood{MIMD}: \emph{Multiple Instructions Streams / Multiple Data Streams}.
    \begin{itemize}
      \item Cada procesador opera sobre sus propios datos $\Rightarrow$ Paralelismo de tareas.
    \end{itemize}
\end{itemize}
\end{frame}

\begin{frame}[t]{Más sobre MIMD}
\begin{itemize}
  \item Variedad de arquitecturas \textmark{MIMD}:
    \begin{itemize}
      \item Arquitecturas altamente acopladas.
        \begin{itemize}
          \item \textgood{TLP} (\emph{Thread-Level Parallelism}): Arquitecturas \emph{Multi/Many-core}.
        \end{itemize}
      \item Arquitecturas débilmente acopladas:
        \begin{itemize}
          \item \textgood{RLP} (\emph{Request-Level Parallelism}): \emph{Clusters} y \emph{WSC}s.
        \end{itemize}
    \end{itemize}

  \mode<presentation>{\vfill\pause}
  \item \textmark{MIMD} es:
    \begin{itemize}
      \item Más flexible y general que SIMD.
      \item Más caro que SIMD.
      \item Requiere suficiente granularidad de tareas.
    \end{itemize}
\end{itemize}
\end{frame}

