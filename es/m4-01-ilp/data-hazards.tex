\section{Riesgos de datos}

\begin{frame}[t,fragile]{Riesgo de datos}
\begin{itemize}
  \item Se produce un \textgood{riesgo de datos} cuando la segmentación 
        modifica el orden de accesos de lectura/escritura a los operandos.
\end{itemize}
\begin{columns}
% Left column
\begin{column}{.45\textwidth}
\begin{block}{Ejemplo}
\lstinputlisting[language={generalasm},basicstyle=\small]{int/m4-01-ilp/ex-data-hazard.asm}
\end{block}
\end{column}
% Right column
\begin{column}{.55\textwidth}
\begin{itemize}
  \item \asmlabel{I2} lee \asmreg{x1} antes que \asmlabel{I1} la modifique.
  \item \asmlabel{I3} lee \asmreg{x1} antes de que \asmlabel{I1} la modifique.
  \item \asmlabel{I4} obtiene valor correcto.
    \begin{itemize}
      \item Banco de registros leído en segunda mitad de ciclo.
    \end{itemize}
  \item \asmlabel{I5} obtiene valor correcto.
\end{itemize}
\end{column}
\end{columns}
\end{frame}
      
\begin{frame}[t]{Riesgos de datos}
\makebox[\textwidth][c]{
\input{es/m4-01-ilp/data-hazard-ex1.tkz}
}
\end{frame}

\begin{frame}[t]{Detenciones en riesgos de datos}
\makebox[\textwidth][c]{
\input{es/m4-01-ilp/data-hazard-ex2.tkz}
}
\end{frame}

\begin{frame}[t,fragile]{Riesgos de datos: RAW}
\begin{itemize}
  \item Lectura después de escritura (\textmark{Read After Write}).
    \begin{itemize}
      \item La instrucción \textmark{i+1} trata de leer un dato antes de que la
            instrucción \textmark{i} lo escriba.
    \end{itemize}
\end{itemize}
\begin{columns}
\begin{column}{.25\textwidth}
\begin{block}{Ejemplo}
\lstinputlisting[language={generalasm}]{int/m4-01-ilp/raw.asm}
\end{block}
\end{column}
\begin{column}{.75\textwidth}
\begin{itemize}
  \item Si hay dependencia de datos, las instrucciones:
    \begin{itemize}
      \item No pueden ejecutarse en paralelo ni solaparse completamente.
      \item La instrucción \cppkey{sub} necesita el valor de \cppid{x1}
            producido por la instrucción \cppkey{add}.
    \end{itemize}
\end{itemize}
\end{column}
\end{columns}
\mode<presentation>{\pause\vfill}
\begin{itemize}
  \item \textgood{Soluciones}:
  \begin{itemize}
    \item \textmark{Detección hardware}.
    \item \textmark{Control por el compilador}.
  \end{itemize}
\end{itemize}
\end{frame}

\begin{frame}[t,fragile]{Riesgos de datos: WAR}
\begin{itemize}
  \item Escritura después de lectura (\textmark{Write After Read}).
    \begin{itemize}
      \item La instrucción \textmark{i+1} modifica operando antes de que la
            instrucción \textmark{i} lo lea.
    \end{itemize}
\end{itemize}
\begin{columns}
\begin{column}{.25\textwidth}
\begin{block}{Ejemplo}
\lstinputlisting[language={generalasm}]{int/m4-01-ilp/war.asm}
\end{block}
\end{column}
\begin{column}{.75\textwidth}
\begin{itemize}
  \item Conocido como \textmark{antidependencia} en compiladores.
    \begin{itemize}
      \item Reutilización de nombre
      \item La instrucción \cppkey{add} modifica \cppid{x1}
            antes de \cppkey{sub} la lea.
    \end{itemize}
\end{itemize}
\end{column}
\end{columns}
\mode<presentation>{\pause\vfill}
\begin{itemize}
  \item \textbad{No puede} ocurrir en un MIPS con \textmark{pipeline de 5 etapas}.
  \begin{itemize}
    \item Todas las instrucciones de 5 etapas.
    \item Lecturas siempre en etapa 2.
    \item Escrituras siempre en etapa 5.
  \end{itemize}
\end{itemize}
\end{frame}

\begin{frame}[t,fragile]{Riesgos de datos: WAW}
\begin{itemize}
  \item Escritura después de escritura (\textmark{Write After Write}).
    \begin{itemize}
      \item La instrucción \textmark{i+1} modifica el operando antes de que la
            instrucción \textmark{i} lo modifica.
    \end{itemize}
\end{itemize}
\begin{columns}
\begin{column}{.25\textwidth}
\begin{block}{Ejemplo}
\lstinputlisting[language={generalasm}]{int/m4-01-ilp/waw.asm}
\end{block}
\end{column}
\begin{column}{.75\textwidth}
\begin{itemize}
  \item Conocido como \textmark{dependencia de salida} en compiladores.
    \begin{itemize}
      \item Reutilización de nombre
      \item La instrucción \cppkey{add} modifica \cppid{x1}
            antes de \cppkey{sub} la modifique.
    \end{itemize}
\end{itemize}
\end{column}
\end{columns}
\mode<presentation>{\pause\vfill}
\begin{itemize}
  \item \textbad{No puede} ocurrir en un MIPS con \textmark{pipeline de 5 etapas}.
  \begin{itemize}
    \item Todas las instrucciones de 5 etapas.
    \item Escrituras siempre en etapa 5.
  \end{itemize}
\end{itemize}
\end{frame}

\begin{frame}[t]{Soluciones a los riesgos de datos}
\begin{itemize}
  \item \textgood{Dependencias RAW}:
    \begin{itemize}
      \item Envío adelantado (\textmark{forwarding}).
    \end{itemize}

  \mode<presentation>{\vfill}
  \item \textgood{Dependencias WAR y WAW}:
    \begin{itemize}
      \item \textmark{Renombrado de registros}.
        \begin{itemize}
          \item Estáticamente por el compilador.
          \item Dinámicamente por el hardware.
        \end{itemize}
    \end{itemize}  
\end{itemize}
 \end{frame}

\begin{frame}[t]{Envío adelantado (\emph{forwarding})}
\begin{itemize}
  \item Técnica para evitar algunas detenciones de datos.
  \mode<presentation>{\vfill}
  \item \textgood{Idea básica}:
    \begin{itemize}
      \item No hace falta esperar a que el resultado se escriba en el banco de registros. 
      \item Ya está en los registros de segmentación.
      \item Se puede usar ese valor en vez del que hay en el banco de registros.
    \end{itemize}
  \mode<presentation>{\vfill}
  \item \textgood{Implantación}:
    \begin{itemize}
      \item Los resultados de las fases EX y MEM se escriben en registros de entrada a ALU.
      \item La lógica de \emph{forwarding} selecciona entre entradas reales y registros de \emph{forwarding}.
    \end{itemize}
\end{itemize}
\end{frame}

\begin{frame}[t]{Forwarding}
\makebox[\textwidth][c]{
\input{int/m4-01-ilp/forward.tkz}
}
\end{frame}

\begin{frame}[t,fragile]{Limitaciones del \emph{forwarding}}
\begin{itemize}
  \item No todos los riesgos se pueden evitar con forwarding.
    \begin{itemize}
      \item ¡No se puede viajar hacia atrás en el tiempo!
    \end{itemize}
\end{itemize}

\begin{columns}

\begin{column}{.4\textwidth}
\begin{block}{Ejemplo}
\lstinputlisting[language={generalasm}]{int/m4-01-ilp/load-stall.asm}
\end{block}
\end{column}

\begin{column}{.6\textwidth}
\makebox[\textwidth][c]{
\input{int/m4-01-ilp/forward-limit.tkz}
}\end{column}

\end{columns}
\mode<presentation>{\vfill}
\begin{itemize}
  \item Si el riesgo no se puede evitar se debe introducir una detención.
\end{itemize}

\end{frame}

\begin{frame}[t]{Detenciones en acceso a memoria}
\makebox[\textwidth][c]{
\input{int/m4-01-ilp/memory-stall.tkz}
}
\end{frame}
