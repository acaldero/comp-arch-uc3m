\section{Riesgos estructurales}

\begin{frame}[t]{Riesgo estructural}
\begin{itemize}
  \item  Se produce cuando el hardware \alert{no puede} soportar
         todas las posibles secuencias de instrucciones.
    \begin{itemize} 
      \item En un mismo ciclo dos etapas de la segmentación
            necesitan hacer uso del \textmark{mismo recurso}.
    \end{itemize}

  \mode<presentation>{\vfill\pause}
  \item \textgood{Razones}:
    \begin{itemize}
      \item Unidades funcionales no totalmente segmentadas.
      \item Unidades funcionales no duplicadas.
    \end{itemize}

  \mode<presentation>{\vfill}
  \item Este tipo de riesgos es evitable pero encarece el
        hardware.
\end{itemize}
\end{frame}

\begin{frame}[t]{Speedup de la segmentación}
\begin{itemize}
  \item \textmark{Speedup de la segmentación}:
    \begin{itemize}
      \item $t_{noseg}$: Tiempo medio de instrucción en arquitectura no segmentada.
      \item $t_{seg}$: Tiempo medio de instrucción en arquitectura segmentada.
    \end{itemize}
\end{itemize}
\[
S = 
\frac{t_{noseg}}{t_{seg}} =
\frac{CPI_{noseg} \times ciclo_{noseg}}{CPI_{seg} \times ciclo_{seg}}
\]
\mode<presentation>{\vfill}
\begin{itemize}
  \item En el caso ideal el \textgood{CPI} segmentado es 1.
    \begin{itemize}
      \item Hay que añadir ciclos de detención por instrucción.
    \end{itemize}
\end{itemize}
\end{frame}

\begin{frame}[t]{Speedup de la segmentación}
\begin{itemize}
  \item En el caso del procesador no segmentado.
    \begin{itemize}
      \item $CPI=1$, con $ciclo_{noseg} > ciclo_{seg}$.
      \item $ciclo_{noseg} = N \times ciclo_{seg}$.
      \item $N$ $\rightarrow$ \textmark{Profundidad del pipeline}.
    \end{itemize}
\end{itemize}
\mode<presentation>{\vfill}
\begin{block}{Speedup}
\[
S =
\frac{N}{1 + \text{detenciones por instrucción}}
\]
\end{block}
\end{frame}

\begin{frame}[t]{Riesgos estructurales: ejemplo}
\makebox[\textwidth][c]{
\input{es/m4-01-ilp/str-hazard-ex1.tkz}
}
\begin{itemize}
  \item Asumiendo memoria de un único puerto.
\end{itemize}
\end{frame}

\begin{frame}[t]{Riesgos estructurales: ejemplo}
\makebox[\textwidth][c]{
\input{es/m4-01-ilp/str-hazard-ex2.tkz}
}
\begin{itemize}
  \item Asumiendo memoria de un único puerto.
\end{itemize}
\end{frame}

\begin{frame}[t]{Ejemplo}
\begin{itemize}
  \item Se consideran dos diseño alternativos:
    \begin{itemize}
      \item \textmark{A}: Sin riesgos estructurales.
        \begin{itemize}
          \item Ciclo de reloj $\rightarrow 1 ns$
        \end{itemize}
      \item \textmark{B}: Con riesgos estructurales.
        \begin{itemize}
          \item Ciclo de reloj $\rightarrow 0.9 ns$
        \end{itemize}
      \item Instrucciones de acceso a datos con riesgo $\rightarrow 30\%$.
    \end{itemize}
  
  \mode<presentation>{\vfill}
  \item ¿Alternativa más rápida?
\end{itemize}
\mode<presentation>{\pause}
\[
t_{inst}(A) =
CPI \times ciclo = 1 \times 1 ns = 1 ns
\]
\[
t_{inst}(B) =
CPI \times ciclo = (0.7 \times 1 + 0.3 \times (1 + 1)) \times 0.9 ns =
1.17 ns
\]
\end{frame}
