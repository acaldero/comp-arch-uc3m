\subsection{Acceso a datos compartidos}

\begin{frame}[t,fragile]{Exclusión mutua}
\begin{itemize}
  \item \cppid{mutex} permite controlar el acceso con \textmark{exclusión mutua} a un recurso.
    \begin{itemize}
      \item \cppid{lock()}: \textmark{Adquiere} el cerrojo asociado.
      \item \cppid{unlock()}: \textmark{Libera} el cerrojo asociado.
    \end{itemize}
\end{itemize}

\mode<presentation>{\vfill}
\begin{columns}[T]
\begin{column}{.5\textwidth}
\begin{block}{Uso de mutex}
\begin{lstlisting}
std::mutex m;
int x = 0;

void f() {
  m.lock();
  ++x;
  m.unlock();
}
\end{lstlisting}
\end{block}
\end{column}

\begin{column}{.5\textwidth}
\begin{block}{Lanzamiento de hilos}
\begin{lstlisting}
void g() {
  std::thread t1(f); 
  std::thread t2(f);
  t1.join(); 
  t2.join();
  cout << x << "\n";
}
\end{lstlisting}
\end{block}
\end{column}
\end{columns}
\end{frame}

\begin{frame}[t,fragile]{Problemas con \texttt{lock()}/\texttt{unlock()}}
\begin{itemize}
  \item \textgood{Posibles problemas}:
    \begin{itemize}
      \item Olvido de \textbad{liberar cerrojo}.
      \item \textbad{Excepciones}.
    \end{itemize}
\end{itemize}

\mode<presentation>{\vfill}
\begin{columns}[T]
\begin{column}{.5\textwidth}
\begin{block}{Olvidos}
\begin{lstlisting}
std::mutex m;
int x = 0;

void f() {
  m.lock();
  ++x;
  // Ouch! olvido liberar cerrojo
}
\end{lstlisting}
\end{block}
\end{column}

\begin{column}{.5\textwidth}
\begin{block}{Excepciones}
\begin{lstlisting}
std::mutex m;
int x = 0;

void f() {
  m.lock();
  g(x); // Excepciones?
  m.unlock();
}
\end{lstlisting}
\end{block}
\end{column}
\end{columns}
\end{frame}

\begin{frame}[t,fragile]{Evitando problemas con \texttt{unique\_lock}}
\begin{itemize}
  \item Solución: \cppid{unique\_lock}.
    \begin{itemize}
      \item \textgood{Patrón}: \textmark{RAII} (Resource Acquisition Is Initialization).
    \end{itemize}
\end{itemize}

\mode<presentation>{\vfill}
\begin{columns}[T]
\begin{column}{.5\textwidth}
\begin{block}{Cerrojo automático}
\begin{lstlisting}
mutex m;
int x = 0;

void f() {
  // Adquiere el cerrojo
  unique_lock l{m}; 
  ++x;
} // Libera el cerrojo
\end{lstlisting}
\end{block}
\end{column}

\begin{column}{.5\textwidth}
\begin{block}{Lanzamiento de hilos}
\begin{lstlisting}
void g() {
  thread t1(f); 
  thread t2(f);
  t1.join(); 
  t2.join();

  cout << x << "\n";
}
\end{lstlisting}
\end{block}
\end{column}
\end{columns}
\end{frame}

\begin{frame}[t,fragile]{Adquisición de múltiples \texttt{mutex}}
\begin{itemize}
  \item La función \cppid{lock()} permite adquirir a la vez varios \cppid{mutex}.
    \begin{itemize}
      \item Adquiere todos o ninguno.
      \item Si alguno está bloqueado espera dejando libres todos.
    \end{itemize}
\end{itemize}

\mode<presentation>{\vfill}
\begin{block}{Adquisición múltiple}
\begin{lstlisting}
std::mutex m1, m2, m3;

void f() {
  std::lock(m1, m2, m3); // Adquiere m1, m2 y m3

  // Acceso a datos compartidos

  // Cuidado: Hay que liberar los cerrojos
  m1.unlock();
  m2.unlock();
  m3.unlock()
} 
\end{lstlisting}
\end{block}
\end{frame}

\begin{frame}[t,fragile]{Adquisición de múltiples \texttt{mutex}}
\begin{itemize}
  \item Especialmente útil en cooperación con \cppid{unique\_lock}
\end{itemize}

\mode<presentation>{\vfill}
\begin{block}{Adquisición múltiple automática}
\begin{lstlisting}
void f() {
  std::unique_lock l1{m1, std::defer_lock};
  std::unique_lock l2{m2, std::defer_lock};
  std::unique_lock l3{m3, std::defer_lock};
  lock(l1,l2,l3);
  // Acceso a datos compartidos

} // Liberación automática
\end{lstlisting}
\end{block}
\end{frame}
