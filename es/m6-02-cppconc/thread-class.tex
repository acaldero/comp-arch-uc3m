\section{La clase \texttt{thread}}

\begin{frame}{La clase \texttt{thread}}
\begin{itemize}
  \item La abstracción de hilo representada mediante clase \cppid{thread}.
  \item Correspondencia uno-a-uno con los hilos del sistema operativo.
  \item Todos los hilos de una aplicación se ejecutan en el mismo espacio de direcciones.
  \item Cada hilo tiene su propia pila.
  \item Peligros:
    \begin{itemize}
      \item Pasar un puntero o una referencia no constante a otro hilo.
      \item Paso de referenica a través de captura en expresiones lambda.
    \end{itemize}
  \item \cppid{thread} representa un enlace a un hilo del sistema.
    \begin{itemize}
      \item No se pueden copiar.
      \item Si se pueden mover.
    \end{itemize}
\end{itemize}
\end{frame}

\begin{frame}[fragile]{Construcción de hilos}
\begin{itemize}
  \item Un hilo se construye con una función y los argumentos que se debe pasar a la función.
    \begin{itemize}
      \item Plantilla con número variable de argumentos.
      \item Seguro en tipos.
    \end{itemize}
\end{itemize}
\begin{block}{Ejemplo}
\begin{lstlisting}
void f();
void g(int, double);

std::thread t1{f}; // OK
std::thread t2{f, 1}; // Error: demasiados argumentos.

std::thread t3{g, 1, 0.5}; // OK
std::thread t4{g}; // Error: faltan argumentos.
std::thread t5{g, 1, "Hola"}; // Error: tipos incorrectos
\end{lstlisting}
\end{block}
\end{frame}

\begin{frame}[fragile]{Construcción y referencias}
\begin{itemize}
  \item El constructor de \cppid{thread} es una plantilla con argumentos variables.
\begin{lstlisting}
template <class F, class ...Args> 
explicit thread(F&& f, Args&&... args);
\end{lstlisting}
  \item El paso de parámetros a un hilo es por valor.
  \item Para forzar el paso de parámetros por referencia:
    \begin{itemize}
      \item Usar una función de ayuda para \cppid{reference\_wrapper}.
      \item Usar lambdas y capturas por referencia.
    \end{itemize}
\end{itemize}
\begin{block}{}
\begin{lstlisting}[basicstyle=\tiny]
void f(registro & r);

void g(registro & s) {
  std::thread t1{f,s}; // Error si f toma parámetro por referencia
  std::thread t2{f, std::ref(s)}; // OK. Referencia a s
  std::thread t3{[&] { f(s); }}; // Referencia a s
}
\end{lstlisting}
\end{block}
\end{frame}

\begin{frame}[fragile]{Construcción en dos etapas}
\begin{itemize}
  \item La construcción incluye el inicio de la ejecución del hilo.
    \begin{itemize}
      \item No hay operación separada para \emph{iniciar} la ejecución.
    \end{itemize}
\end{itemize}
\begin{columns}

\begin{column}{.5\textwidth}
\begin{block}{Productor/Consumidor}
\begin{lstlisting}
struct productor {
  productor(cola<peticiones> & c);
  void operator()();
  // ...
};

struct consumidor {
  consumidor(cola<peticiones> & c);
  void operator()();
  // ...
};
\end{lstlisting}
\end{block}
\end{column}

\begin{column}{.5\textwidth}
\begin{block}{Etapas}
\begin{lstlisting}
void f() {
  // Etapa 1: Construcción
  cola<peticiones> c;
  productor prod{c};
  consumidor cons{c};

  // Etapa 2: Arranque
  std::thread tp{prod};
  std::thread tc{cons};

  //...
\end{lstlisting}
\end{block}
\end{column}

\end{columns}
\end{frame}

\begin{frame}[fragile]{Hilo vacío}
\begin{itemize}
  \item El constructor por defecto crea un hilo sin tarea de ejecución asociada.
\begin{lstlisting}
thread() noexcept;
\end{lstlisting}
  \item Útil en combinación con el constructor de movimiento.
\begin{lstlisting}
thread(thread &&) noexcept;
\end{lstlisting}
  \item Se puede mover una tarea de ejecución de un hilo a otro.
    \begin{itemize}
      \item El hilo original se queda sin tarea de ejecución asociada.
    \end{itemize}
\begin{lstlisting}
std::thread crea_tarea();
std::thread t1 = crea_tarea();
std::thread t2 = std::move(t1); // t1 está vacío ahora
\end{lstlisting}
\end{itemize}
\end{frame}

\begin{frame}{Identidad de un hilo}
\begin{itemize}
  \item Cada hilo tiene un identificador único.
    \begin{itemize}
      \item Tipo \cppid{thread::id}.
      \item Si el \cppid{thread} no está asociado con un hilo \cppid{get\_id()} devuelve \cppid{id\{\}}.
      \item El identificador del hilo actual se obtiene con \cppid{this\_thread::get\_id()}.
    \end{itemize}
  \vfill
  \item \cppid{t.get\_id()} devuelve \cppid{id\{\}} si:
    \begin{itemize}
      \item No se le ha asignado una tarea de ejecución.
      \item Ha finalizado.
      \item La tarea se ha movido a otro \cppid{thread}.
      \item Se desasociado (\cppid{detach()}).
    \end{itemize}
\end{itemize}
\end{frame}

\begin{frame}[fragile]{Operaciones sobre \texttt{thread::id}}
\begin{itemize}
  \item Es un tipo dependiente de la implementación, pero debe permitir:
    \begin{itemize}
      \item Copia.
      \item Operadores de comparación (\cppid{==}, \cppid{<}, ...).
      \item Salida mediante el operador \cppid{<<}.
      \item Transformación \emph{hash} mediante la especialización \cppid{hash<thread::id>}.
    \end{itemize}
\end{itemize}
\begin{block}{Ejemplo}
\begin{lstlisting}
void imprime_id(std::thread & t) {
  if (t.get_id() == std::thread::id{}) 
    cout << "Hilo no válido" << endl;
  else {
    cout << "Hilo actual: " << std::this_thread::get_id() << "\n";
    cout << "Hilo recibido: " << t.get_id() << "\n";
  }
}
\end{lstlisting}
\end{block}
\end{frame}

\begin{frame}[fragile]{Unión de tareas}
\begin{itemize}
  \item Cuando un hilo desea esperar a que otro hilo finalice puede usar la operación \cppid{join()}.
    \begin{itemize}
      \item \cppid{t.join()} $\rightarrow$ espera a que t haya finalizado.
    \end{itemize}
\end{itemize}
\begin{block}{Ejemplo}
\begin{lstlisting}
void f() {
  std::vector<std::thread> vt;
  for (int i=0; i< 8; ++i) {
    vt.push_back(std::thread(f,i));
  }

  for (auto & t : vt) { // Espera a que todos los hilos terminen
    t.join();
  }
}
\end{lstlisting}
\end{block}
\end{frame}

\begin{frame}[fragile]{Tareas periódicas}
\begin{block}{Idea inicial}
\begin{lstlisting}
void actualiza_barra() {
  while (!tarea_terminada()) {
    using namespace std::literals;
    std::this_thread::sleep_for(1s)
    update_progress();
  }
}

void f() {
  std::thread t{actualiza_barra};
  t.join();
}
\end{lstlisting}
\end{block}
\begin{itemize}
  \item \textbad{¿Problemas?}
\end{itemize}
\end{frame}

\begin{frame}[fragile]{¿Qué pasa si se olvida \texttt{join}?}
\begin{itemize}
  \item Si se sale del alcance donde se define el hilo, se invoca al destructor.
  \item \alert{Problema}: Se podría perder el vínculo con un hilo del sistema que se seguiría
        ejecutando y al que no se podría acceder.
    \begin{itemize}
      \item Si no se ha hecho \cppid{join()} el destructor llama a \cppid{terminate()}.
    \end{itemize}
\end{itemize}

\begin{columns}

\begin{column}{.5\textwidth}
\begin{block}{Ejemplo}
\begin{lstlisting}[basicstyle=\tiny]
void actualiza() {
  for (;;) {
    using namespace std::literals;
    muestra_reloj(steady_clock::now());
    std::this_thread::sleep_for(1s);
  }
}

void f() {
  std::thread t{actualiza};
}
\end{lstlisting}
\end{block}
\end{column}

\begin{column}{.5\textwidth}
\begin{itemize}
  \item Se ejecuta \cppid{terminate()} al abandonar \cppid{f()}.
\end{itemize}
\end{column}
\end{columns}
\end{frame}

\begin{frame}[fragile]{Destrucción}
\begin{itemize}
  \item Objetivo: Evitar que un hilo sobreviva a su objeto \cppid{thread}.
  \item Solución: Si un \cppid{thread} es \emph{unible} su destructor invoca \cppid{terminate()}.
    \begin{itemize}
      \item Un \cppid{thread} es unible si está asociado a un hilo del sistema.
    \end{itemize}
\end{itemize}
\begin{block}{Ejemplo}
\begin{lstlisting}[basicstyle=\tiny]
void comprueba() {
  for (;;) {
    using namespace std::literals;
    comprueba_estado();
    std::this_thread::sleep_for(10s);
  }
}

void f() {
  std::thread t{comprueba};
} // Destrucción sin join() -> Invoca a terminate()
\end{lstlisting}
\end{block}
\end{frame}

\begin{frame}[fragile]{Problemas con la destrucción}
\begin{columns}

\begin{column}{.65\textwidth}
\begin{block}{Ejemplo}
\begin{lstlisting}[basicstyle=\tiny]
void f();
void g();

void ejemplo() {
  std::thread t1{f}; // Hilo que ejecuta la tarea f
  std::thread t2; // Hilo vacío

  if (modo == modo1) {
    std::thread tg {g}; 
    // ...
    t2 = std::move(tg); // tg vacía, t2 asociada a g()
  }

  std::vector<int> v{10000}; // Posibles excepciones
  t1.join();
  t2.join();
}
\end{lstlisting}
\end{block}
\end{column}

\begin{column}{.35\textwidth}
\begin{itemize}
  \item ¿Qué ocurre si el constructor de \cppid{v} lanza una excepción.
  \vspace{2em}
  \item ¿Qué ocurre si se llega al final con \cppid{modo==modo1}?
\end{itemize}
\end{column}

\end{columns}
\end{frame}

\begin{frame}[fragile]{Hilo automático}
\begin{itemize}
  \item Se puede usar el patrón RAII.
    \begin{itemize}
      \item Resource Acquisition Is Initialization.
    \end{itemize}
\end{itemize}
\begin{block}{Un hilo que hace join}
\begin{lstlisting}
struct hilo_automatico : std::thread {
  using std::thread::thread; // Todos los constructores de thread
  ~hilo_automatico() { 
    if (joinable()) join(); 
  }
};
\end{lstlisting}
\end{block}
\begin{itemize}
  \item El constructor adquire el recurso.
  \item El destructor libera el recurso.
  \item Evita el goteo de recursos.
\end{itemize}
\end{frame}

\begin{frame}[fragile]{Simplificación con RAII}
\begin{itemize}
  \item Simplificación de código y seguridad.
\end{itemize}
\begin{block}{Ejemplo}
\begin{lstlisting}
void ejemplo() {
  hilo_automatico t1{f}; // Hilo que ejecuta la tarea f
  hilo_automatico t2; // Hilo vacío.

  if (modo == modo1) {
    hilo_automatico tg {g}; 
    // ...
    t2 = std::move(tg); // tg vacía, t2 asociada a g()
  }

  std::vector<int> v{10000}; // Podría lanzar excepciones
}
\end{lstlisting}
\end{block}
\end{frame}

\begin{frame}[fragile]{Hilos no asociados}
\begin{itemize}
  \item Se puede indicar que un hilo sigue ejecutando después de que el destructor se ejecute con \cppid{detach()}.
  \item Útil para tareas que se ejecutan como demonios.
\end{itemize}
\begin{block}{Ejemplo}
\begin{lstlisting}[basicstyle=\tiny]
void actualiza() {
  for (;;) {
    using namespace std::literals;
    muestra_reloj(std::stead_clock::now());
    std::this_thread::sleep_for(1s);
  }
}

void f() {
  std::thread t{actualiza};
  t.detach();
}
\end{lstlisting}
\end{block}
\end{frame}

\begin{frame}[fragile]{Problemas con hilos no asociados}
\begin{itemize}
  \item Incovenientes:
    \begin{itemize}
      \item Se pierde el control de qué hilos están activos.
      \item No se sabe si se puede usar el resultado generado por un hilo.
      \item No se sabe si un hilo ha liberado sus recursos.
      \item Se podría acabar accediendo a objetos que han sido destruidos.
    \end{itemize}
  \vfill
  \item Recomendaciones:
    \begin{itemize}
      \item Evitar el uso de hilos no asociados.
      \item Mover los hilos a otro alcance (via valor de retorno).
      \item Mover los hilos a un contenedor en un alcance mayor.
    \end{itemize}
\end{itemize}
\end{frame}

\begin{frame}[fragile]{Un bug difícil de cazar}
\begin{itemize}
  \item \alert{Problema}: Cuidado con el acceso a variables locales desde un hilo no asociado despues de su destrucción.
\end{itemize}
\begin{block}{Ejemplo}
\begin{lstlisting}
void g() {
  double x = 0;
  std::thread t{[&x]{ f1(); x = f2();}}; // Si g ha terminado -> Problema
  t.detach();
}
\end{lstlisting}
\end{block}
\end{frame}

\begin{frame}{Operaciones sobre el hilo actual}
\begin{itemize}
  \item Operaciones sobre el hilo actual como funciones globales en sub-espacio de nombres \cppid{this\_thread}.
    \begin{itemize}
      \item \cppid{get\_id()}: Obtiene el identificador del hilo actual.
      \item \cppid{yield()}: Permite que potencialmente se seleccione otro hilo para ejecución.
      \item \cppid{sleep\_until(t)}: Espera hasta un determinado instante de tiempo.
      \item \cppid{sleep\_for(d)}: Espera durante una duración determinada de tiempo.
    \end{itemize}
  \item Esperas con tiempo:
    \begin{itemize}
      \item Si se modifica el reloj, \cppid{wait\_until()} se ve afectado.
      \item Si se modifica el reloj, \cppid{wait\_for()} \textbf{no} se ve afectado.
    \end{itemize}
\end{itemize}
\end{frame}
