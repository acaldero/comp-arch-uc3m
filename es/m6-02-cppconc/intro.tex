\section{Introducción a la concurrencia en C++}

\begin{frame}{Motivación}
\begin{itemize}
  \item C++11 (ISO/IEC 14882:2011) ofrece un modelo de concurrencia propio.
    \begin{itemize}
      \item Revisiones posteriores en C++14, C++17 y C++20.
    \end{itemize}

  \mode<presentation>{\vfill\pause}
  \item Cualquier implementación que cumple el estándar debe proporcionarlo.
    \begin{itemize}
      \item Resuelve problemas inherentes a \textmark{PThreads}.
      \item Portabilidad de código concurrente: \textmark{Windows}, \textmark{POSIX}, \ldots
    \end{itemize}

  \mode<presentation>{\vfill\pause}
  \item \textgood{Implicaciones}:
    \begin{itemize}
      \item Modificaciones en el \textmark{lenguaje}.
      \item Modificaciones en la \textmark{biblioteca estándar}.
    \end{itemize}

  \mode<presentation>{\vfill\pause}
  \item Influencia sobre C11 (ISO/IEC 9899:2011).

  \mode<presentation>{\vfill\pause}
  \item \textmark{Importante}: Concurrencia y paralelismo son conceptos 
        \textgood{relacionados} pero \textbad{distintos}.
\end{itemize}
\end{frame}

\begin{frame}{Estructura}
\begin{itemize}
  \item El lenguaje \textgood{ofrece}:
    \begin{itemize}
      \item Un \textmark{modelo de memoria}.
      \item Variables \cppkey{thread\_local}.
    \end{itemize}

  \mode<presentation>{\vfill\pause}
  \item La \textmark{biblioteca estándar} \textgood{ofrece}:
    \begin{itemize}

      \mode<presentation>{\vfill\pause}
      \item \textgood{Tipos atómicos}.
        \begin{itemize}
          \item Útiles para programación \textmark{libre de cerrojos} de forma portable.
        \end{itemize}

      \mode<presentation>{\vfill\pause}
      \item \textgood{Abstracciones portables} para la concurrencia.
        \begin{itemize}
          \item \textmark{Hilos}: \cppid{thread}, \cppid{jthread}.
          \item \textmark{Exclusión mutua}: \cppid{mutex}, \cppid{*\_mutex}, \cppid{\ldots},
          \item \textmark{Gestores de cerrojos}: \cppid{lock}, \cppid{*\_lock}, \cppid{\ldots},
          \item \textmark{Variables condición}: \cppid{condition\_variable}, \cppid{condition\_variable\_any}.
          \item \textmark{Semáforos}: \cppid{counting\_semaphore}, \cppid{binary\_semaphore}.
          \item \textmark{Comunicación}: \cppid{packaged\_task}, \cppid{future}, \cppid{promise}.
        \end{itemize}
    \end{itemize}
\end{itemize}
\end{frame}
