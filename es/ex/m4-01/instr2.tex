\begin{acexercise}\end{acexercise}
\label{ex:m4-01:instr-02}

Dado el siguiente fragmento de código:

\lstinputlisting[language=generalasm2]{int/ex/m4-01/instr2-question.asm}

Se considera la ejecución en un pipeline de 5 etapas con hardarware envío adelantado
(o \emph{forwarding}). Las lecturas a la caché de instrucciones requieren un ciclo de
memoria. Las lecturas a la caché de datos requieren dos ciclos de memoria y las
escrituras requieren 3 ciclos de memoria. Para las instrucciones de bifurcación
tanto la dirección de salto como el sentido del salto se conocen al final de la
etapa de ejecución y se predicen todos los saltos siempre a no tomados.

Presente la temporización (cronograma) de la primera iteración completa (incluyendo la captación o fetch
de la primera instrucción de la siguiente iteración).

\begin{acsolution}\end{acsolution}

En el código estudiado se identifican las siguientes dependencias RAW:
\begin{itemize}
  \item \asmreg{t4}: \asmlabel{I1} $\rightarrow$ \asmlabel{I3}.
  \item \asmreg{t5}: \asmlabel{I2} $\rightarrow$ \asmlabel{I3}.
  \item \asmreg{t5}: \asmlabel{I3} $\rightarrow$ \asmlabel{I4}.
  \item \asmreg{t3}: \asmlabel{I7} $\rightarrow$ \asmlabel{I8}.
\end{itemize}


La solución se presenta en la tabla~\ref{ex:m4-01:instr-02:chrono:forward}.

Observaciones:

\begin{enumerate}

\item La etapa \textmark{M} de la instrucción \asmlabel{I1}
      ocupa dos ciclos por tratarse de una lectura de memoria.

\item Se produce una detención en la instrucción \asmlabel{I2}, 
      por estar ocupada la etapa \textmark{M}.

\item La etapa \textmark{M} de la instrucción \asmlabel{I2}
      ocupa dos ciclos por tratarse de una lectura de memoria.

\item Se produce una detención en la instrucción \asmlabel{I3},
      hasta que se dispone del nuevo valor de \asmreg{t5}.
      El valor se pasa a la etapa \textmark{EX},
      mediante envío adelantado una vez que se ha obtenido 
      en la etapa \textmark{M} de la instrucción \asmlabel{I2}.

\item Se produce una detención en la instrucción \asmlabel{I4},
      por estar ocupada la etapa \textmark{ID}.

\item La etapa \textmark{M} ocupa de la instrucción \asmlabel{I4}
      tres ciclos por tratarse de una escritura de memoria.

\item Se produce una detención en la instrucción \asmlabel{I5},
      por estar ocupada la etapa \textmark{M}.

\item Se produce una detención en la instrucción \asmlabel{I6},
      por estar ocupada la etapa \textmark{EX}.

\item Se produce una detención en la instrucción \asmlabel{I7},
      por estar ocupada la etapa \textmark{ID}.

\item La instrucción \asmlabel{I8} no puede empezar hasta el ciclo
      13 por estar ocupada la etapa \textmark{IF}.

\item Se capta la instrucción \asmlabel{I9} (siguiente instrucción)
      en el ciclo 14, al predecirse el salto a no tomado.

\item Al final del ciclo 15, se determina que el salto debe tomarse
      por lo que se retira del \emph{pipeline} la instrucción \asmlabel{I9}
      y se capta la instrucción \asmlabel{I1}.

\end{enumerate}

\begin{table}[htb]
\begin{tabular}{|l|l|*{15}{>{\footnotesize}c|}}
\hline
\# &
\textbf{Instr.} &
1 & 2 & 3 & 4 & 5 &
6 & 7 & 8 & 9 & 10 &
11 & 12 & 13 & 14 & 15 
\\
\hline
\hline

I1 &
lw t4, 0(t1)
& IF & ID & EX & M & M & WB
\\
\hline

I2 &
lw t5, 0(t2)
& 
& IF & ID & EX & -- & M & M & WB
\\
\hline

I3 &
mul t5, t4, t5
& &
& IF & ID & -- & -- & -- & EX & M & WB
\\
\hline

I4 &
sw t5, 0(t3)
& & &
& IF & -- & -- & -- & ID & EX & M & M & M & WB
\\
\hline

I5 &
addi t1, t1, 4
& & & & & & &
& IF & ID & EX & -- & -- & M & WB
\\
\hline

I6 &
addi t2, t2, 4
& & & & & & & &
& IF & ID & -- & --& EX & M & WB
\\
\hline

I7 &
addi t3, t3, 4
& & & & & & & & &
& IF & -- & -- & ID & EX & M
\\
\hline

I8 &
bne t3, zero, loop
& & & & & & & & & & & &
& IF & ID & EX
\\
\hline

I9 & 
--
& & & & & & & & & & & & &
& IF & ID
\\
\hline

I10 & 
--
& & & & & & & & & & & & &
& IF
\\
\hline


\end{tabular}

\vspace{1em}

\begin{tabular}{|l|l|*{7}{>{\footnotesize}c|}}
\hline
\# &
\textbf{Instr.} &
16 & 17 & 18 & 19 & 21 & 22
\\
\hline
\hline

I7 &
addi t3, t3, 4
& WB
\\
\hline

I8 &
bne t3, zero, loop
& M & WB
\\
\hline

I9 &
--
& --
\\
\hline

I1 &
lw t4, 0(t1)
& IF & ID & EX & M & M & WB
\\
\hline

\end{tabular}

\caption{Diagrama de tiempos del ejercicio~\ref{ex:m4-01:instr-02} con envío adelantado.}
\label{ex:m4-01:instr-02:chrono:forward}
\end{table}
