\begin{acexercise}\end{acexercise}

Dada una arquitectura con dos niveles caché con las siguientes características:

\begin{tabular}{|l|r|r|}
\hline
Memoria &
Tiempo de acceso (ns) &
Tasa de aciertos
\\
\hline
\hline

L1 &
2 &
0.8
\\
\hline

L2 &
8 &
0.9
\\
\hline

RAM &
100 &
1
\\
\hline

\end{tabular}

El ordenador ejecuta un programa que reside completamente en memoria (no hay
accesos a disco). Se pide:

\begin{enumerate}

  \item Asumiendo que el 100\% de los accesos a memoria son operaciones de
escritura, calcular de forma justificada el tiempo medio de acceso a memoria
para (a) una política de escritura inmediata (write-through) y (b) una política
de post-escritura (write-back) en las memorias caché L1 y L2.

  \item Considerando los dos niveles de memoria caché L1 y L2 como una única
caché global se pide calcular el tiempo medio de acceso y la tasa de aciertos
de esta caché global.

  \item Dado el siguiente código:
\begin{lstlisting}
for (i=0;i<1000;i=i+32){
  a[i]=a[i+8]+a[i+16];
}
\end{lstlisting}

El tamaño de cada entrada de a es de 8 bytes y el de bloque de 64 bytes. El
índice del bucle se almacena en un registro del procesador.  Se pide comentar
razonablemente el efecto en el rendimiento del empleo de una técnica de caché
multibanco que emplea 4 bancos. ¿Qué repercusión tendría el empleo de esta
técnica en el tiempo de cada acceso y en el ancho de banda de la caché para
este código?

\end{enumerate}

\begin{acsolution}\end{acsolution}

\paragraph{Apartado 1}

En el caso de escritura inmediata, como el 100\% de los accesos son operaciones
de escritura, todos ellos se realizan en memoria principal. Por este motivo, el
tiempo de acceso será en todos los casos:

\[
t = 2 + 8 + 100 = 110 ns
\]

En el caso de post escritura,si se produce un acierto caché no es necesario escribir en memoria. De este modo el tiempo de acceso será:

\[
t= 2 + (1 - 0.8) \cdot (8 + (1-0.9) \cdot 100) =
2 + 0.2 \cdot (8 + 0.1 \cdot 100) =
2 + 0.2 \cdot 18 = 
2 + 3.6 = 5.6ns
\]

\paragraph{Apartado 2}
El tiempo medio de acceso es: 

\[
t = 2 + (1 - 0.8) \cdot 8 =
2 + 0.2 \cdot 8 =
2 + 1.6 =
3.6ns
\]

Y la tasa media de aciertos: 
\[
h =
1 - (1-0.8) \cdot (1-0.9) =
1-0.2 \cdot 0.1 = 
1 - 0.02 =
0.98.
\]

\paragraph{Apartado 3}

En cada iteración del bucle se acceden a tres bloques distintos que están
ubicados en posiciones consecutivas de memoria. Mediante el empleo de una
memoria caché no multibanco, se producirían 3 fallos caché que tendrían que ser
atendidos secuencialmente. En el caso de una caché multibanco, sería posible
acceder de forma paralela a estos tres bloques.

El tiempo de acceso sería el mismo que antes, debido a que esta optimización no
mejora el tiempo de acceso a la memoria caché.

El ancho de banda sería el agregado, ya que se estarían haciendo accesos en
paralelo. En este caso, serían 3 accesos simultáneos por lo que el ancho de
banda sería el triple.

