\begin{acexercise}\end{acexercise}

Un computador tiene un valor para CPI de 1.0 en condiciones ideales
(cuando todos los accesos a memoria son aciertos).

El 25\% de las instrucciones son de tipo carga/almacenamiento.
No hay ningún otro tipo de instrucción que acceda a la memoria.
La penalización por fallo es de 50 ciclos y la tase de fallos 
es del 5\%.

¿Cuál sería la aceleración si no hubiese fallos con respecto al
caso en que si los hay?

\begin{acsolution}\end{acsolution}

\[
t_{cpu} = (ciclos_{cpu} + ciclos_{\text{detención}}) \times t_{ciclo}
\]

Para el caso ideal, no hay detenciones de memoria y por tanto:

\[
t_{cpu} = IC \times CPI \times t_{ciclo} =
IC \times t_{ciclo}
\]

Para el caso con fallos, se tiene:

\[
ciclos_{\text{detención}} =
IC \times accesos_{instr} \times (1 - h) \times \text{penalización}_{fallo}
\]

Los accesos por instrucción serán $0.25$ puesto que
en el 25\% de las instrucciones se accede a un dato. En cuanto a la tasa de
aciertos $h$ será $0.95$. La penalización por fallo es de $50$ ciclos.

\[
ciclos_{\text{detención}} =
IC \times 0.25 \times 0.05 \times 50 =
0.625 IC
\]

Y por tanto el tiempo de CPU será:

\[
t_{cpu} = (IC \times 1.0 + 0.625 \times IC) \times t_{ciclo} =
1.625 \times IC \times t_{ciclo}
\]

Con esto la aceleración, será:

\[
S = \frac{1.625 \times IC \times t_{ciclo}}{IC \times t_{ciclo}} = 1.625
\]
