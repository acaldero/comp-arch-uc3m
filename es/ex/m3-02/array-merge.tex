\begin{acexercise}\end{acexercise}

Sea la siguiente definición de variables globales:
\begin{lstlisting}
const unsigned int max = 1024 * 1024;
double x[max];
double y[max];
double z[max];
double vx[max];
double vy[max];
double vz[max];
\end{lstlisting}

Y la siguiente función:

\begin{lstlisting}
void actualiza_posiciones(double dt) {
  for (unsinged int i=0; i<max; ++i) {
    x[i] = vx[i] * dt + x[i];
    y[i] = vy[i] * dt + y[i];
    z[i] = vz[i] * dt + z[i];
  }
}
\end{lstlisting}

Suponga que se dispone de una caché L1 asociativa por conjuntos de 4 vías con
un tamaño de 32 KB y un tamaño de línea de 64 bytes. La caché L2 es asociativa
por conjuntos de 8 vías con un tamaño de 1 MB y un tamaño de línea de 64 bytes.
La política de remplazo es LRU. Todas las cachés se encuentran inicialmente 
vacías.

Los arrays se almacenan de forma consecutiva en memoria y el primero ellos se
encuentra en una dirección múltiplo de 1024. Asuma que el argumento de función
\cppid{dt} y las variable índice \cppid{i} se encuentran asignadas a registros.

\begin{enumerate}
  \item Determine la tasa de aciertos de las caché L1 y L2 para la ejecución de
	la función \cppid{actualiza\_posiciones()}. ¿Cuál será la tasa global
        de aciertos?

  \item Modifique el código aplicando la optimización de fusión de arrays.

  \item Repita los cálculos del primer apartado para el código resultante del
        apartado segundo.

  \item Si se asume que un acierto en la caché L1 requiere 4 ciclos, y en la
	caché L2 requiere 14 ciclos y la penalización por traer un bloque de
	memoria principal a la caché de nivel 2 es de 80 ciclos ¿cuál será el
	tiempo medio de acceso en cada uno de los dos casos?  

\end{enumerate}


\begin{acsolution}\end{acsolution}

\paragraph{Apartado 1}

El patrón de accesos del bucle es:

\begin{lstlisting}
vx[i], x[i], x[i], vy[i], y[i], y[i], vz[i], z[i], z[i], ...
\end{lstlisting}

Cada uno de los arrays tiene $2^{20}$ posiciones de 8 bytes cada una. Lo que da un
total de 8 MB por array. Cada línea de la caché tiene 64 bytes, lo que permite
alojar 8 elementos del array.

La caché L1 tiene 4 vías, por lo que cada vía es de 8KB ($2^{13}$ bytes), lo
que da para $\frac{2^{13}}{2^6}$ = $2^7$ conjuntos.

Como cada array tiene $2^{23}$ bytes (que es múltiplo del tamaño de vía), cada
posición \cppid{i} de todos los arrays se corresponde con el mismo conjunto de la
caché. La secuencia de aciertos y fallos en la caché será:

\begin{itemize}
  \item \cppid{VX[0]} $\rightarrow$ Fallo. Selección de la vía 0.
  \item \cppid{X[0]} $\rightarrow$ Fallo. Selección de la vía 1.
  \item \cppid{X[0]} $\rightarrow$ Acierto.
  \item \cppid{VY[0]} $\rightarrow$ Fallo. Selección de la vía 2.
  \item \cppid{Y[0]} $\rightarrow$ Fallo. Selección de la vía 3.
  \item \cppid{Y[0]} $\rightarrow$ Acierto.
  \item \cppid{VZ[0]} $\rightarrow$ Fallo. Selección de la vía 0. Se expulsa \cppid{vx}.
  \item \cppid{Z[0]} $\rightarrow$ Fallo. Selección de la vía 1. Se expulsa \cppid{x}.
  \item \cppid{Z[0]} $\rightarrow$ Acierto.
\end{itemize}

Al pasar a la posición 1, se repite la misma sucesión de fallos y aciertos. Por
lo que en total se tienen:

\[
h_{L1} = \frac{3}{9} = \frac{1}{3}
\]

Para la caché de nivel 2, se producen solamente accesos para los casos en que
hay fallo en la caché L1. Para la posición 0, se producen 6 fallos, pero para
las posiciones de la 1 a la 7 se producen 6 aciertos por cada posición.

\[
h(L2) = 
\frac{42}{48} = 
\frac{7}{8}
\]

La tasa global de aciertos h es:

\[
H = 1 - M
\]

Donde:

\[
M = 
m_{h1} \cdot m_{h2} = 
\frac{2}{3} \cdot \frac{1}{8} = 
\frac{1}{12}
\]

\[
H = \frac{11}{12}
\]


\paragraph{Apartado 2}

Aplicando la optimización requerida el código queda:

\begin{lstlisting}
const unsigned int max = 1024 * 1024;
struct part {
  double x, y, z, vx, vy, vz;
};
part vec[max];

void actualiza_posiciones(double dt) {
  for (unsinged int i=0; i<max; ++i) {
    vec[i].x = vec[i].vx * dt + vec[i].x;
    vec[i].y = vec[i].vy * dt + vec[i].y;
    vec[i].z = vec[i].vz * dt + vec[i].z;
  }
}
\end{lstlisting}

\paragraph{Apartado 3}

El patrón de accesos en este caso es:

\begin{lstlisting}
vec[i].vx, vec[i].x, vec[i].x, vec[i].vy, vec[i].y, vec[i].y, vec[i]. vz, vec[i].z, vec[i]. z, ...
\end{lstlisting}

Como cada posición del array ocupa 6 posiciones y una entrada de la caché ocupa
8 valores, en 3 líneas de la caché cabrán 4 posiciones completas del array.

\begin{itemize}
  \item La posición 0 produce 1 fallo y 8 aciertos.
  \item La posición 1 produce 2 aciertos, 1 fallo y 6 aciertos.
  \item La posición 2 produce 4 aciertos, 1 fallo y 4 aciertos.
  \item La posición 3 produce 9 aciertos.
\end{itemize}

Este patrón se repite a lo largo de todo el array. Por tanto:

\[
h_{L1} = 
\frac{33}{36} = \frac{11}{12}
\]

Para la caché L2 se tiene ahora que todos los fallos en la caché de nivel 1,
son también fallos en la caché de nivel 2. Por tanto:

\[
h_{L2} = 0
\]

Y la tasa global:

\[
H = 1 - M
\]

\[
M = m_{h1} \cdot m_{h2} = 
\frac{1}{12}
\]

\[
H = \frac{11}{12}
\]

\paragraph{Apartado 4}

Para el caso original se tiene

\[
T = 
4 + (\frac{2}{3}) \cdot (14 + (\frac{1}{8}) \cdot 80) = 
4 + (\frac{2}{3}) \cdot 24 = 
4 +16 = 
20
\]

Para el segundo caso se tiene

\[
T = 
4 + (\frac{1}{12}) \cdot (14 + 1 \cdot 80) = 
4 + (\frac{1}{12}) \cdot 94 = 
4 + 7.83 = 
11.83
\]

