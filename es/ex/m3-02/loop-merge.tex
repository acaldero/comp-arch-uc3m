\begin{acexercise}\end{acexercise}

Sea el siguiente fragmento de código:
\begin{lstlisting}
double a[256][256], b[256][256], c[256][256], d[256][256];
//...
for (int i=0;i<256;++i)
  for (int j=0;j<256;++j) {
    a[i][j] = b[i][j] + c[i][j]
  }
}
for (int i=0;i<256;++i)
  for (int j=0;j<256;++j) {
    d[i][j] = b[i][j] - c[i][j]
  }
}
\end{lstlisting}

Se desea ejecutar este código en un computador que tiene una caché de nivel 1
totalmente asociativa de 16KB y política de sustitución LRU con tamaño de línea
de 64 bytes. Los fallos de caché de nivel 1 requieren 16 ciclos de reloj. Por
otra parte la caché de nivel 2 siempre genera aciertos en este código.

Asuma que los fallos de escritura en la caché de nivel 1 se envían directamente
un búfer de escritura y no generan ningún ciclo de espera.

Se pide

\begin{enumerate}
  \item Determine la tasa de aciertos del segmento de código, asumiendo que las
variables \cppid{i} y \cppid{j} se asignan a registros del procesador y que las
matrices \cppid{a}, \cppid{b}, \cppid{c} y \cppid{d} se encuentran totalmente
en la caché de nivel 2.

  \item Determine el tiempo acceso a memoria asumiendo que los accesos a la
caché de nivel 1 requieren un ciclo de reloj.

  \item Proponga una transformación de código de las que puede generar un
compilador para mejorar la tasa de aciertos, mostrando el código resultante en
lenguaje C++.

  \item Determine la nueva tasa de aciertos y el tiempo medio de acceso a
memoria resultantes.

\end{enumerate}


\begin{acsolution}\end{acsolution}

\paragraph{Apartado 1}

Como el tamaño de línea es de 64 bytes y todos los arrays contienen valores de 
tipo \cppkey{double} cada línea de la memoria caché puede almacenar 8 valores. 
Como la memoria L1 tiene un tamaño de 16 KB, puede almacenar $\frac{2^{14}}{2^6} = 2^8$ líneas.

Cada array almacena $2^8 \dot 2^8 = 2^{16}$ valores, lo que requiere $\frac{2^{16}}{2^3} = 2^{13}$
entradas de la memoria caché.

En el primer bucle el patrón de accesos es:

\begin{itemize}
  \item \cppid{b[0][0]}, \cppid{c[0][0]}, \cppid{a[0][0]}
  \item \cppid{b[0][1]}, \cppid{c[0][1]}, \cppid{a[0][1]}
  \item \ldots
  \item \cppid{b[0][7]}, \cppid{c[0][7]}, \cppid{a[0][7]}
  \item \cppid{b[0][8]}, \cppid{c[0][8]}, \cppid{a[0][8]}
  \item \ldots
  \item \cppid{b[0][255]}, \cppid{c[0][255]}, \cppid{a[0][255]}
  \item \cppid{b[1][0]}, \cppid{c[1][0]}, \cppid{a[1][0]}
  \item \ldots
\end{itemize}

Por tanto por cada 8 iteraciones del bucle interior se producen:

\begin{itemize}
  \item 16 lecturas (8 para \cppid{b} y 8 para \cppid{c}). De estas 2
        son fallos y las otras 14 son aciertos.
  \item 8 escrituras en \cppid{a}. Sin embargo al enviarse las escrituras
        al búfer de escrituras, éstas no generan fallos.
\end{itemize}

Cuando se inicia el segundo bucle, las entradas de la caché conteniendo los valores de los arrays
\cppid{b} y \cppid{c} han sido expulsadas de la caché, por lo que la proporciones de fallos y
aciertos es la misma.

En el caso de las escrituras, todas la escrituras generan fallo de caché. Por tanto:

\[
h_{escr} = 0
\]

En el caso de las lecturas, la tasa de aciertos es:

\[
h_{lect}  = \frac{14}{16} = \frac{7}{8}
\]

\paragraph{Apartado 2}

Para los accesos de escritura, todos los accesos son un fallo de escritura, y se tratan en el
búfer de escritura, requiriendo un ciclo de reloj. En total se realizan $2^8 \cdot 2^8$ accesos
en cada uno de los dos bucles, lo que da un total de $2 \cdot 2^{16} = 2^{17}$ accesos de escritura.

Para los accesos de lectura, el tiempo medio de acceso viene dado por:

\[
t_{lect} = t_a + (1 - h_{lect}) \cdot t_f
\]

El tiempo de acceso $t_a$ es 1 ciclo. La penalización por fallo de lectura es de 16 ciclos.
Por tanto se tiene que el tiempo medio de lectura es:

\[
t_{lect} =
1 + \frac{1}{8} \cdot 16 = 
3
\]

\paragraph{Apartado 3}

Se puede aplicar la técnica de fusión de bucles:

\begin{lstlisting}
for (int i=0;i<256;i++) {
  for (int j=0;j<256;j++) {
    a[i][j] = b[i][j] + c[i][j];
    d[i][j] = b[i][j] - c[i][j];
  }
}
\end{lstlisting}

\paragraph{Apartado 4}

En este caso los segundos accesos a \cppid{b} y \cppid{c} siempre generan acierto.

Ahora por cada 8 iteraciones del bucle interior se producen:

\begin{itemize}
  \item 32 lecturas. De éstas, 30 son aciertos y dos son fallos de lectura.
  \item 16 escrituras.
\end{itemize}

Al igual que en el caso anterior la tasa de aciertos de escritura ($h_{escr}$) sigue siendo 0.
En este caso, la tasa de aciertos de lectura pasa a ser:

\[
h_{lect} = \frac{30}{32} = \frac{15}{16}
\]

Para el caso de los accesos de lectura, usando la nueva tasa de aciertos, se tiene:

\[
t_{lect} =
1 + \frac{1}{16} \cdot 16 = 
2
\]

