\begin{acexercise}\end{acexercise}

Se dispone de un computador con un solo núcleo que ejecuta una aplicación de
evaluación de riesgos financieros. Esta aplicación es intensiva en cálculo, a
lo que dedica el 90\% del tiempo. El 10\% restante lo dedica a esperar en
operaciones de entrada/salida a disco.

Del tiempo que la aplicación pasa ejecutando instrucciones de cálculo un 75\%
del tiempo lo pasa ejecutando operaciones en coma flotante y un 25\% lo pasa
ejecutando otras instrucciones. La ejecución de una instrucción de coma
flotante requiere como promedio 12 CPI. El resto de instrucciones requieren
como promedio 4 CPI.

Se está valorando la migración de esta aplicación a las siguientes
alternativas, que no incorporan ninguna mejora para el tiempo de las
operaciones de entrada/salida a disco:

\begin{itemize}
  \item \textbf{Alternativa A}: Un procesador con un solo núcleo y con una
frecuencia de reloj un 50\% más alta que la de la máquina original en el que las
instrucciones de coma flotante requieren un 10\% más de ciclos por instrucción y
el resto de instrucciones requieren un 25\% más de ciclos por instrucción.

  \item \textbf{Alternativa B}: Un procesador con cuatro núcleos y con una
frecuencia de reloj un 50\% más baja que la de la máquina original, en el que
las instrucciones de coma flotante requieren un 20\% menos de ciclos de reloj y
el resto de instrucciones los mismos ciclos de reloj.  
\end{itemize}

Se pide responder de forma justificada a las siguientes cuestiones:
\begin{enumerate}

  \item ¿Cuál será la aceleración/deceleración global de la aplicación en el
caso A? 

  \item ¿Cuál será la aceleración/deceleración global de la aplicación en el
caso B si se asume que la parte de cálculo es totalmente paralelizable mientras
la entrada/salida no admite ningún tipo de paralelización?  \end{enumerate}


\begin{acsolution}
\end{acsolution}

El tiempo dedicado a la ejecución de instrucciones en el computador original será:

\begin{equation}
T_{orig} = 
0.75 \times 12 \times IC \times P + 0.25 \times 4 \times IC \times P = 
(9+1) \times IC \times P
\end{equation}

\paragraph{Alternativa A}

El tiempo dedicado a la ejecución de instrucciones en el computador A será:

\begin{equation}
T_{A} =
(0.75 \times (1,1 \times 12) + 0.25 \times (1.25 \times 4)) \times IC \times \frac{P}{1.5} = 
\frac{(9.9 + 1.25) \times IC  \times P}{1.5} = 
\frac{11.15}{1.5} \times IC * P 
\end{equation}

\begin{equation}
T_{A} = 7.433 \times IC \times P
\end{equation}

El Speedup debido a instrucciones será:

\begin{equation}
S_{A}^{I} = 
\frac{T_{orig}}{T_{A}} = 
\frac{10}{7.433} = 
1.345
\end{equation}

Aplicando la Ley de Amdahl el speedup global sera:

\begin{equation}
S_{A} = \frac{1}{0.1 + \frac{0.9}{1.345}} = 1.3
\end{equation}

\paragraph{Alternativa B}

En este caso, al asumirse paralelización completa de la parte de cálculo se
puede considerar que el número de instrucciones a ejecutar en cada núcleo es la
cuarta parte del original.

\begin{equation}
T_{B} = 
(0.75 \times 0.8 \times 12 + 0.25 \times 4) \times \frac{IC}{4} \times \frac{P}{0.5} = 
( 7.2 + 1) \times \frac{2}{4} \times IC \times P = 
\end{equation}

\begin{equation}
T_{B} = 4.1 \times IC \times P
\end{equation}

El Speedup debido a instrucciones sera:

\begin{equation}
S^{I}_{B} = \frac{T_{orig}}{T_{B}} = \frac{10}{4.1} = 2.439
\end{equation}

Aplicando la Ley de Amdahl el speedup global sera:

\begin{equation}
S_{B} = \frac{1}{0.1 + \frac{0.9}{2.439}} = 2.132
\end{equation}

