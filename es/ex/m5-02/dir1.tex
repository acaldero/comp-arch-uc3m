\begin{acexercise}
Con contribuciones de: David Expósito.
\end{acexercise}

Una máquina de memoria compartida dispone de cuatro nodos NUMA con las siguientes características:

\begin{itemize}
  \item Cada nodo tiene un único procesador.
  \item El procesador tiene un único nivel de caché de 64KB que es asociativa por conjuntos de 8 vías con tamaño de
línea de 64 bytes.
  \item La memoria RAM de cada nodo es de 64 GB.
  \item El directorio se encuentra distribuido entre los nodos.
\end{itemize}

Conteste a las siguientes preguntas:

\begin{enumerate}

  \item ¿Cuántas máquinas de estado MSI distintas hay en total en esta máquina?

  \item ¿Cuál es el espacio de memoria que puede direccionar el nodo número 2?

  \item Complete la siguiente tabla, para la entrada de caché donde reside la variable x, en la que se deben
        especificar los estados en las cachés de los nodos de cómputo 
        (M $\rightarrow$ Modificado, S $\rightarrow$ Compartido, I $\rightarrow$ Inválido), 
        así como el tráfico de red para cada nodo. 
        La columna \emph{Instrucción} indica el nodo en el que se ejecuta la instrucción
        (P1: \asminst{lw}\asmreg{ t1},\asmlabel{ x} significa ejecutar 
        \asminst{lw}\asmreg{ t1},\asmlabel{ x} en el nodo P1).

\end{enumerate}

\begin{tabular}{|>{\footnotesize}l|*{8}{>{\footnotesize}c|}}

\hline
& \multicolumn{2}{c}{P1}
& \multicolumn{2}{c}{P2}
& \multicolumn{2}{c}{P3}
& \multicolumn{2}{c|}{P4}
\\
\hline
Instrucción &
Estado & Tráfico &
Estado & Tráfico &
Estado & Tráfico &
Estado & Tráfico 
\\
\hline

Inicial &
I & -- & I & -- & I & --& I & --\\
\hline

P1: \asminst{lw} \asmreg{ t1}, \asmlabel{ x} &
&&&&&&&\\
\hline

P1: \asminst{lw} \asmreg{ t2}, \asmlabel{ x} &
&&&&&&&\\
\hline

P2: \asminst{lw} \asmreg{ t2}, \asmlabel{ x} &
&&&&&&&\\
\hline

P3: \asminst{lw} \asmreg{ t2}, \asmlabel{ x} &
&&&&&&&\\
\hline

P3: \asminst{sw} \asmreg{ t2}, \asmlabel{ x} &
&&&&&&&\\
\hline

P1: \asminst{lw} \asmreg{ t1}, \asmlabel{ x} &
&&&&&&&\\
\hline

P4: \asminst{sw} \asmreg{ t1}, \asmlabel{ x} &
&&&&&&&\\
\hline

P4: \asminst{lw} \asmreg{ t1}, \asmlabel{ x} &
&&&&&&&\\
\hline

\end{tabular}

Así mismo, complete la siguiente tabla en la que se deben especificar el estado en el directorio 
(U $\rightarrow$ No Cacheado, S $\rightarrow$ Compartido, E $\rightarrow$ Exclusivo), 
así como la lista de nodos en el directorio y la acciones del directorio.

\begin{tabular}{|l|c|l|l|}
\hline
Instrucción & Estado directorio & Lista directorio & Acción directorio
\\
\hline

Inicial &
U & {} & --
\\
\hline

P1: \asminst{lw} \asmreg{ t1}, \asmlabel{ x} &
&&\\
\hline

P1: \asminst{lw} \asmreg{ t2}, \asmlabel{ x} &
&&\\
\hline

P2: \asminst{lw} \asmreg{ t2}, \asmlabel{ x} &
&&\\
\hline

P3: \asminst{lw} \asmreg{ t2}, \asmlabel{ x} &
&&\\
\hline

P3: \asminst{sw} \asmreg{ t2}, \asmlabel{ x} &
&&\\
\hline

P1: \asminst{lw} \asmreg{ t1}, \asmlabel{ x} &
&&\\
\hline

P4: \asminst{sw} \asmreg{ t1}, \asmlabel{ x} &
&&\\
\hline

P4: \asminst{lw} \asmreg{ t1}, \asmlabel{ x} &
&&\\
\hline

\end{tabular}

\begin{acsolution}\end{acsolution}

\paragraph{Apartado 1}

Cada línea de caché contiene una máquina de estados MSI independiente con tres
posibles estados (modificado, compartido o inválido). El tamaño de la caché
local de cada procesador es 64 KiB y el tamaño de bloque es 64 B por lo que hay
1024 líneas en una caché. Como hay cuatro procesadores el número total de
máquinas de estado es 4 $\times$ 1024 = 4096.

\paragraph{Apartado 2}

Cada nodo tiene 64 GiB de RAM, pero puede acceder a la memoria del resto de nodos. Por tanto el procesador
número dos puede acceder a 4 * 64 GiB = 256 GiB de memoria.

\paragraph{Apartado 3}
A continuación se muestran las tablas resultado:

\begin{tabular}{|>{\footnotesize}l|*{8}{>{\footnotesize}c|}}

\hline
& \multicolumn{2}{c}{P1}
& \multicolumn{2}{c}{P2}
& \multicolumn{2}{c}{P3}
& \multicolumn{2}{c|}{P4}
\\
\hline
Instrucción &
Estado & Tráfico &
Estado & Tráfico &
Estado & Tráfico &
Estado & Tráfico 
\\
\hline

Inicial &
I & -- & I & -- & I & --& I & --\\
\hline

P1: \asminst{lw} \asmreg{ t1}, \asmlabel{ x} &
S & Fallo lect. & I & -- & I & -- & I & -- \\
\hline

P1: \asminst{lw} \asmreg{ t2}, \asmlabel{ x} &
S & -- & I & -- & I & -- & I & --\\
\hline

P2: \asminst{lw} \asmreg{ t2}, \asmlabel{ x} &
S & -- & S & Fallo lect. & I & -- & I & -- \\
\hline

P3: \asminst{lw} \asmreg{ t2}, \asmlabel{ x} &
S & -- & S & -- & S & Fallo lect. & I & -- \\
\hline

P3: \asminst{sw} \asmreg{ t2}, \asmlabel{ x} &
I & -- & I & -- & M & Invalid. & I & -- \\
\hline

P1: \asminst{lw} \asmreg{ t1}, \asmlabel{ x} &
S & Fallo lect. & I & -- & S & Post-escr. & I & -- \\
\hline

P4: \asminst{sw} \asmreg{ t1}, \asmlabel{ x} &
I & -- & I & -- & I & -- & M & Fallo escr.\\
\hline

P4: \asminst{lw} \asmreg{ t1}, \asmlabel{ x} &
I & -- & I & -- & I & -- & M & --\\
\hline

\end{tabular}

\vspace{2em}

\begin{tabular}{|l|c|l|l|}
\hline
Instrucción & Estado directorio & Lista directorio & Acción directorio
\\
\hline

Inicial &
U & \{\} & --
\\
\hline

P1: \asminst{lw} \asmreg{ t1}, \asmlabel{ x} &
S & \{ P1 \} & Resp. dato\\
\hline

P1: \asminst{lw} \asmreg{ t2}, \asmlabel{ x} &
S & \{ P1 \}& --\\
\hline

P2: \asminst{lw} \asmreg{ t2}, \asmlabel{ x} &
S & \{ P1, P2 \} & Resp. dato\\
\hline

P3: \asminst{lw} \asmreg{ t2}, \asmlabel{ x} &
S & \{ P1, P2, P3 \} & Resp. dato\\
\hline

P3: \asminst{sw} \asmreg{ t2}, \asmlabel{ x} &
E & \{ P3 \} & Invalidar\\
\hline

P1: \asminst{lw} \asmreg{ t1}, \asmlabel{ x} &
S & \{ P1, P3 \} & Captar. Resp. dato\\
\hline

P4: \asminst{sw} \asmreg{ t1}, \asmlabel{ x} &
E & \{ P4 \} & Invalidar. Resp. dato\\
\hline

P4: \asminst{lw} \asmreg{ t1}, \asmlabel{ x} &
E & \{ P4 \} & -- \\
\hline

\end{tabular}
