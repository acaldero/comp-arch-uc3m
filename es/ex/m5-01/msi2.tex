\begin{acexercise}
Con contribuciones de: David Expósito.
\end{acexercise}

Sea un multiprocesador con arquitectura de memoria compartida simétrica basado
en bus con protocolo de espionaje o snooping. Cada procesador tiene una caché
privada cuya coherencia se mantiene usando el protocolo MSI. Cada bloque caché
tiene una única palabra.  

La siguiente tabla muestra el estado inicial de cuatro variables distintas en
cada una de las cachés. 

\medskip

\begin{tabular}{|c|c|c|c|c|}
\hline
&
\multicolumn{4}{|c|}{Estado inicial de las variables}
\\
\hline

\textbf{Procesador} &
\textbf{A} &
\textbf{B} &
\textbf{C} &
\textbf{D}
\\
\hline
\hline

\textbf{P0} & 
Shared & Exclusive & Shared & Shared
\\
\hline

\textbf{P1} &
Invalid & Invalid & Invalid & Shared
\\
\hline

\textbf{P2} &
Invalid & Invalid & Shared & Shared
\\
\hline

\end{tabular}

\medskip

La siguiente tabla muestra el estado final de estas variables tras realizar una
serie de accesos a memoria. 

\medskip

\begin{tabular}{|c|c|c|c|c|}
\hline
&
\multicolumn{4}{|c|}{Estado final de las variables}
\\
\hline

\textbf{Procesador} &
\textbf{A} &
\textbf{B} &
\textbf{C} &
\textbf{D}
\\
\hline
\hline

\textbf{P0} & 
Invalid & Invalid & Invalid & Shared
\\
\hline

\textbf{P1} &
Invalid & Invalid & Shared & Exclusive
\\
\hline

\textbf{P2} &
Exclusive & Exclusive & Invalid & Shared
\\
\hline

\end{tabular}

\medskip

Se pide:
\begin{itemize}

  \item Para cada variable (A, B, C y D) describir de forma justificada el/los
accesos realizados y el/los procesos involucrados que  han permitido alcanzar
el estado final. 

    \begin{itemize}

      \item \textbf{Nota1}: para alcanzar el estado final puede ser suficiente
un solo acceso o una secuencia de accesos. 

      \item \textbf{Nota2}: puede existir un estado final
que sea inalcanzable (es decir, que no tiene solución).

    \end{itemize}

  \item Para cada caso anterior describir el tráfico de bus generado

\end{itemize}

\begin{acsolution}\end{acsolution}


\begin{itemize}

\item Variable \cppid{A}:
  \begin{itemize}
    \item \textbf{\textbf{P2}}  escribe \cppid{A} (estado a \emph{exclusive}) invalidando la copia en la caché de \textbf{P0}.
    \item \textbf{P2} genera un fallo de escritura en el bus que lo captura \textbf{P0} el cual no genera tráfico.
  \end{itemize}

\item Variable \cppid{B}:
  \begin{itemize}
    \item \textbf{P2}  escribe \cppid{B}  (estado a exclusive) invalidando la copia en la caché de \textbf{P0}
    \item \textbf{P2} genera un write miss en el bus que lo captura \textbf{P0} el cual realiza un write-back del bloque a la caché de \textbf{P2}
  \end{itemize}

\item Variable \cppid{C}:
  \begin{itemize}
    \item \textbf{P1} escribe \cppid{B} (estado a exclusive) invalidando las otras dos copias.
    \item \textbf{P1} genera un write miss en el bus que lo captura \textbf{P0} y \textbf{P1} los cuales no generan tráfico.
    \item \textbf{P1} lee o escribe \cppid{D} otra variable cuyo bloque ocupa la misma línea caché que \cppid{C} generando un fallo de conflicto. Esto hace que el bloque de \cppid{C} se reemplace y como ha sido modificado se escribe en memoria. El bloque de \cppid{C} no está presente en caché (equivalente a estado \emph{invalid}).
    \item Esta acción de \textbf{P1} genera de tráfico realizando el \emph{write-back} del bloque en memoria.
    \item \textbf{P1} lee \cppid{C}. Genera un fallo de escritura y pasa a estado compartido. 
  \end{itemize}

\item Variable \cppid{D}:
  \begin{itemize}
    \item El estado final no se puede alcanzar porque un bloque no puede estar exclusivo y compartido a la vez.
  \end{itemize}

\end{itemize}

