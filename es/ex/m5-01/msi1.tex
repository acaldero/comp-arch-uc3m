\begin{acexercise}\end{acexercise}
\label{ex:m5-01:msi-01}

Sea un multiprocesador con arquitectura de memoria compartida simétrica basado
en bus con protocolo de espionaje o snooping. Cada procesador tiene una caché
privada cuya coherencia se mantiene usando el protocolo MSI. Cada caché utiliza
correspondencia directa y tiene cuatro bloques cada uno con dos palabras. Esta
caché utiliza como campo de etiqueta la dirección de memoria completa.

Las siguientes tablas muestran el estado de cada memoria, con la palabra menos
significativa a la izquierda.

\medskip

\begin{tabular}{|l|l|l|l|}
\hline
\multicolumn{4}{|c|}{Procesador P0}\\
\hline
\textbf{Bloque} &
\textbf{Estado} &
\textbf{Etiqueta} &
\textbf{Datos}
\\
\hline
\hline

B0 & I & 0x00100700 & 0x00000000 0x7FAABB11\\
\hline

B1 & S & 0x00100708 & 0x00000000 0x00001234\\
\hline

B2 & M & 0x00100710 & 0x00000000 0x0077AABB\\
\hline

B3 & I & 0x00100718 & 0x00000000 0x7FAABB11\\
\hline
\end{tabular}

\medskip

\begin{tabular}{|l|l|l|l|}
\hline
\multicolumn{4}{|c|}{Procesador P1}\\
\hline
\textbf{Bloque} &
\textbf{Estado} &
\textbf{Etiqueta} &
\textbf{Datos}
\\
\hline
\hline

B0 & I & 0x00100700 & 0x00000000 0x7FAABB11\\
\hline
B1 & M & 0x00100728 & 0x00000000 0xFF000000\\
\hline
B2 & I & 0x00100710 & 0x00000000 0xEEEE7777\\
\hline
B3 & S & 0x00100718 & 0x00000000 0x7FAABB11\\
\hline
\end{tabular}

\medskip

\begin{tabular}{|l|l|l|l|}
\hline
\multicolumn{4}{|c|}{Procesador P2}\\
\hline
\textbf{Bloque} &
\textbf{Estado} &
\textbf{Etiqueta} &
\textbf{Datos}
\\
\hline
\hline

B0 & S & 0x00100720 & 0x00000000 0x1111AAAA\\
\hline
B1 & S & 0x00100708 & 0x00000000 0X00001234\\
\hline
B2 & I & 0x00100710 & 0x00000000 0x7FAABB11\\
\hline
B3 & I & 0x00100718 & 0x00001234 0x1111AABB\\
\hline

\end{tabular}

\medskip

Para cada uno de los apartados que a continuación se presentan, parta de la
situación inicial del problema, sin tener en cuenta los cambios de los
apartados anteriores. Indique los cambios que se producen en las cachés. En el
caso de las lecturas, indique además cuál es el valor efectivamente leído.

\begin{enumerate}
  \item P2: write 0x00100708, 0xFFFFFFFF
  \item P2: read 0x00100708
  \item P2: read 0x00100718
\end{enumerate}

En cada apartado deberá rellenar una tabla con el siguiente formato,
justificando la respuesta: 

\begin{tabular}{|l|l|l|l|l|l|p{2cm}|p{2cm}|}

\hline
\textbf{Procesador} &
\textbf{Bloque} &
\textbf{Estado} &
\textbf{Estado} &
\textbf{Etiqueta} &
\multicolumn{2}{|c|}{\textbf{Datos}}
\\

&
&
\textbf{anterior} &
\textbf{actual} &
&
\multicolumn{2}{|c|}{}
\\
\hline
\hline

& & & & & \multicolumn{1}{|p{2cm}|}{} & \\
\hline
& & & & & \multicolumn{1}{|p{2cm}|}{} & \\
\hline
& & & & & \multicolumn{1}{|p{2cm}|}{} & \\
\hline
& & & & & \multicolumn{1}{|p{2cm}|}{} & \\
\hline
\end{tabular}

\begin{acsolution}\end{acsolution}

\paragraph{Apartado 1}

Se produce una escritura en el bloque B1 de la caché de P2 que se encuentra en
estado S. Esto produce los siguientes cambios en P2:

\begin{itemize}

\item Se pasa a estado exclusivo (M).

\item Se coloca una invalidación en el bus.

\end{itemize}

La invalidación es ignorada por el procesador P1, pero es tratada por el
procesador P0.

En P0 se producen los siguientes cambios:

\begin{itemize}

\item Se pasa el estado a inválido (I).

\end{itemize}

En este caso no se producen modificaciones en la memoria principal.

Los estados en los distintos procesadores se muestran en la Tabla~\ref{ex:m5-01:msi-01:tabl:1}.

\begin{table}[htbp]

\begin{tabular}{|l|l|l|l|l|}

\hline
\multicolumn{5}{|c|}{\textbf{Procesador P0}}\\
\hline
Bloque & Estado & Etiqueta & \multicolumn{2}{c|}{Datos}\\
\hline
\hline
B0 & I & 0x00100700 & 0x00000000 & 0x7FAABB11\\
\hline
B1 & I & 0x00100708 & 0x00000000 & 0x00001234\\
\hline
B2 & M & 0x00100710 & 0x00000000 & 0x0077AABB\\
\hline
B3 & I & 0x00100718 & 0x00000000 & 0x7FAABB11\\
\hline

\hline
\multicolumn{5}{|c|}{\textbf{Procesador P1}}\\
\hline
Bloque & Estado & Etiqueta & \multicolumn{2}{c|}{Datos}\\
\hline
\hline

B0 & I & 0x00100700 & 0x00000000 & 0x7FAABB11\\
\hline
B1 & M & 0x00100728 & 0x00000000 & 0xFF000000\\
\hline
B2 & I & 0x00100710 & 0x00000000 & 0xEEEE7777\\
\hline
B3 & S & 0x00100718 & 0x00000000 & 0x7FAABB11\\
\hline


\hline
\multicolumn{5}{|c|}{\textbf{Procesador P2}}\\
\hline
Bloque & Estado & Etiqueta & \multicolumn{2}{c|}{Datos}\\
\hline
\hline

B0 & S & 0x00100720 & 0x00000000 & 0x1111AAAA\\
\hline
B1 & {\color{red}M} & 0x00100708 & {\color{red}0xFFFFFFFF} & 0X00001234\\
\hline
B2 & I & 0x00100710 & 0x00000000 & 0x7FAABB11\\
\hline
B3 & I & 0x00100718 & 0x00001234 & 0x1111AABB\\
\hline


\end{tabular}

\caption{Estados de caché correspondientes al apartado 1 del ejercicio~\ref{ex:m5-01:msi-01}.}
\label{ex:m5-01:msi-01:tabl:1}

\end{table}

Las transiciones de estados y las correspondientes modificaciones se muestran en la Tabla~\ref{ex:m5-01:msi-01:tabl:2}.

\begin{table}[htbp]

\begin{tabular}{|l|l|l|l|l|l|l|}

\hline
Procesador 	& Bloque 	& Estado 	& 		&  		& \multicolumn{2}{c|}{}\\
		&		& Anterior	& Estado Nuevo	& Etiqueta	& \multicolumn{2}{c|}{Datos}\\
\hline
\hline
P2 & B1 & I & M & 0X00100708 & 0XFFFFFFFF & 0X00001234\\
\hline
P0 & B1 & S & I & 0X00100708 & 0X00000000 & 0X00001234\\
\hline

\end{tabular}

\caption{Transiciones de estados y modificaciones correspondientes al apartado 1 del ejercicio~\ref{ex:m5-01:msi-01}.}
\label{ex:m5-01:msi-01:tabl:2}
\end{table}

\paragraph{Apartado 2}

Se produce una lectura en el bloque B1 de la caché P2 que se encuentra en el
estado S. Se trata de un acierto y no se produce ningún cambio en las cachés o
en memoria.

El valor leído es: 0x00000000

\paragraph{Apartado 3}

Se produce un fallo de lectura la caché de P2 para el bloque B3 que está en
estado inválido (I). Por tanto se coloca el fallo de lectura en el bus y se
pasa a estado compartido (S).

El procesador P0, tiene este bloque en estado inválido e ignora el fallo de lectura.

El procesador P1, tiene este bloque en estado compartido (S) y continúa en este estado.

El valor leído es 0x00000000

Los estados en los distintos procesadores se muestran en la Tabla~\ref{ex:m5-01:msi-01:tabl:3}.

\begin{table}[htbp]

\begin{tabular}{|l|l|l|l|l|}

\hline
\multicolumn{5}{|c|}{\textbf{Procesador P0}}\\
\hline
Bloque & Estado & Etiqueta & \multicolumn{2}{c|}{Datos}\\
\hline
\hline

B0 & I & 0x00100700 & 0x00000000 & 0x7FAABB11\\
\hline
B1 & S & 0x00100708 & 0x00000000 & 0x00001234\\
\hline
B2 & M & 0x00100710 & 0x00000000 & 0x0077AABB\\
\hline
B3 & I & 0x00100718 & 0x00000000 & 0x7FAABB11\\
\hline

\hline
\multicolumn{5}{|c|}{\textbf{Procesador P1}}\\
\hline
Bloque & Estado & Etiqueta & \multicolumn{2}{c|}{Datos}\\
\hline
\hline

B0 & I & 0x00100700 & 0x00000000 & 0x7FAABB11\\
\hline
B1 & M & 0x00100728 & 0x00000000 & 0xFF000000\\
\hline
B2 & I & 0x00100710 & 0x00000000 & 0xEEEE7777\\
\hline
B3 & S & 0x00100718 & 0x00000000 & 0x7FAABB11\\
\hline


\hline
\multicolumn{5}{|c|}{\textbf{Procesador P2}}\\
\hline
Bloque & Estado & Etiqueta & \multicolumn{2}{c|}{Datos}\\
\hline
\hline

B0 & S & 0x00100720 & 0x00000000 & 0x1111AAAA\\
\hline
B1 & S & 0x00100708 & 0x00000000 & 0X00001234\\
\hline
B2 & I & 0x00100710 & 0x00000000 & 0x7FAABB11\\
\hline
B3 & S & 0x00100718 & {\color{red}0x00000000} & {\color{red}0x7FAABB11}\\
\hline

\end{tabular}

\caption{Estados de caché correspondientes al apartado 3 del ejercicio~\ref{ex:m5-01:msi-01}.}
\label{ex:m5-01:msi-01:tabl:3}

\end{table}

Las transiciones de estados y las correspondientes modificaciones se muestran en la Tabla~\ref{ex:m5-01:msi-01:tabl:4}.

\begin{table}[htbp]

\begin{tabular}{|l|l|l|l|l|l|l|}

\hline
Procesador 	& Bloque 	& Estado 	& 		&  		& \multicolumn{2}{c|}{}\\
		&		& Anterior	& Estado Nuevo	& Etiqueta	& \multicolumn{2}{c|}{Datos}\\
\hline
\hline

P2 & B3 & I & S & 0X00100718 & 0X00000000 & 0X7FAABB11\\
\hline

\end{tabular}

\caption{Transiciones de estados y modificaciones correspondientes al apartado 3 del ejercicio~\ref{ex:m5-01:msi-01}.}
\label{ex:m5-01:msi-01:tabl:4}
\end{table}

