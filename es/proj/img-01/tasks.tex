\section{Tareas}

\subsection{Desarrollo del software}


\subsubsection{Versiones del programa}
\label{sec:tasks:versions}

Esta tarea consiste en el desarrollo de la versión secuencial de la
aplicación descrita en C++20. Se prepararán dos programas ejecutables:
\cppkey{image-soa} y \cppkey{image-aos} que implementarán dos
estrategias:

\begin{itemize}
  \item \textmark{SOA -- Structure of Arrays}:
        Se representará una imagen mediante tres vectores independientes de valores
        entre \cppid{0} y \cppid{255}.
  \item \textmark{AOS -- Array of Structures}:
        Se representará una imagen con un único vector de valores de tipos pixel.
        Un pixel representa una estructura con tres valores entre \cppid{0} y
        \cppid{255}.
\end{itemize}

\subsubsection{Componentes}

Deberá desarrollar los siguientes componentes:

\begin{itemize}
  \item \cppid{common}: Biblioteca con los archivos fuente comunes
        a las dos versiones del programa.

  \item \cppid{soa}: Biblioteca con los archivos fuente específicos 
        a loscomponentes de la versión \textmark{AOS}.

  \item \cppid{aos}: Biblioteca con los archivos fuente específicos 
        a loscomponentes de la versión \textmark{AOS}.

  \item \cppid{image-soa}: Ejecutable con el programa para la versión \textmark{SOA}.

  \item \cppid{image-aos}: Ejecutable con el programa para la versión \textmark{AOS}.

\end{itemize}

A continuación se describe con más detalle cada uno de los cinco componentes mencionados:

\paragraph{common}: Componentes comunes.

Contendrá como mínimo los siguientes archivos fuente:
\begin{itemize}
  \item \cppid{progargs.cpp}: Tratamiento de los argumentos del
        programa recibidos por \cppid{main}.
\end{itemize}

\paragraph{soa}: Biblioteca de componentes \textmark{SOA}.

Contendrá como mínimo los siguientes archivos fuente:
\begin{itemize}
  \item \cppid{imagesoa.cpp}: Tratamiento de imágenes
        con una representación \textmark{SOA}.
\end{itemize}

\paragraph{aos}: Biblioteca de componentes \textmark{AOS}.

Contendrá como mínimo los siguientes archivos fuente:

\begin{itemize}
  \item \cppid{imageaos.cpp}: Tratamiento de imágenes
        con una representación \textmark{AOS}.
\end{itemize}

\paragraph{image-soa}: Programa para la versión \textmark{SOA}.

Contendrá un único archivo fuente:
\begin{itemize}
  \item \cppid{imgsoa}: Contendrá exclusivamente la función
        \cppid{main()} para esta versión.
        No podrá incluir ninguna función ni tipo adicional.
\end{itemize}

\paragraph{image-aos}: Programa para la versión \textmark{AOS}.

Contendrá un único archivo fuente:
\begin{itemize}
  \item \cppid{imgaos}: Contendrá exclusivamente la función
        \cppid{main()} para esta versión.
        No podrá incluir ninguna función ni tipo adicional.
\end{itemize}

\subsubsection{Compilación del proyecto}

Todos sus archivos fuente deben compilar sin problemas y no deben emitir
ninguna advertencia del compilador. 

Deberá incluirse un archivo de configuración
CMake con las siguientes opciones:

\begin{lstlisting}[title={CmakeLists.txt},frame=single]
cmake_minimum_required(VERSION 3.23)
project(image LANGUAGES CXX)

set(CMAKE_CXX_STANDARD 20)
set(CMAKE_CXX_STANDARD_REQUIRED ON)
set(CMAKE_CXX_EXTENSIONS_OFF)
add_compile_options(-Wall -Wextra -Werror -pedantic -pedantic-errors)

set(CMAKE_CXX_FLAGS_RELEASE "-march=native")

add_library(common file1-common.cpp file2-common.cpp file3-common.cpp)
add_library(aos file1-aos.cpp file2-aos.cpp)
add_library(soa file1-soa.cpp file2-soa.cpp)

add_executable(image-aos imgaos.cpp)
target_link_libraries(imgaos common aos)
add_executable(imgsoa image-soa.cpp)
target_link_libraries(imgsoa common soa)
\end{lstlisting}


Tenga también en cuenta que deberá realizar todas las evaluaciones con las
optimizaciones del compilador activadas con la opción de CMake
\cppid{-DCMAKE\_BUILD\_TYPE=Release}.


\textbad{Importante}: No se podrá utilizar ninguna biblioteca externa a la 
biblioteca estándar de C++.

\subsubsection{Normas de cálidad del código}

El código fuente debe estar bien estructurado y organizado, así como
apropiadamente documentado. Se recomienda, pero no es de obligado
cumplimiento, hacer uso de las \textemph{C++ Core Guidelines}
(\url{http://isocpp.github.io/CppCoreGuidelines/CppCoreGuidelines}).

En cualquier caso, las siguientes reglas si son de obligado y
estricto cumplimiento:

\begin{itemize}
  \item No se podrán utilizar variables globales que no sean constantes.
  \item No está permitido el uso del patrón \emph{Singleton}.
  \item No se podrá pasar arrays a una función como parámetros de tipo
        puntero.
  \item Ninguna función ni función miembro podrá tener más de cuatro parámetros.
  \item Ninguna función podrá tener más de 25 líneas estando apropiadamente
        formateadas.
  \item Todos los parámetros se pasarán a las funciones
        por valor, por referencia o por referencia constante.
  \item No se podrá usar explicitamente las funciones \cppid{malloc()} o
        \cppid{free()} ni los operadores \cppkey{new} o \cppkey{delete}.
  \item No se permite el uso de macros, excepto para la definición de
        guardas de inclusión.
  \item No se permite ningún cast excepto \cppkey{static\_cast},
        \cppkey{dynamic\_cast}, \cppkey{const\_cast} o \cppkey{reinterpret\_cast}.
\end{itemize}


\subsubsection{Pruebas unitarias}

Debe definir y entregar un conjunto de pruebas unitarias.
Se recomienda el uso de GoogleTest (\url{https://github.com/google/googletest}),
aunque no es imprescindible.

En cualquier caso debe aportar evidencias de haber realizado suficientes
pruebas unitarias.


\subsection{Evaluación del rendimiento y energía}

Esta tarea consiste en realizar una evaluación comparativa del rendimiento
de las dos estrategias de implementación \cppid{image-aos} e \cppid{image-soa}.

Para evaluar el rendimiento debe medir el tiempo de ejecución de la
aplicación. También debe medirse el consumo energético de la aplicación
y derivar el uso de potencia.

Todas las evaluaciones de rendimiento se realizarán en un nodo del
clúster \cppid{avignon}.

Represente en una gráfica todos los tiempos totales de ejecución, uso de energía y potencia
obtenidos para imágenes de distinto tamaño. A tal efecto
se suministrarán imágenes para la evaluación.

\textbf{Incluya en la memoria de esta práctica las conclusiones que pueda inferir de los resultados.}
No se limite simplemente a describir los datos. Debe buscar también una
explicación convincente de los resultados.

