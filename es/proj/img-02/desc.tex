\section{Descripción del proyecto}

En este proyecto se realizará la paralelización de una posible solución a
el proyecto práctico anterior. Para la paralelización su hará uso de
OpenMP.

\subsection{Visión general}

El proyecto que se suministra está organizado en varias carpetas que contienen
distintos componentes:

\begin{itemize}
  \item \textmark{Directorio raíz}: Contiene los programas principales y el archivo
        \cppkey{CMakeLists.txt}.
  \item \textmark{common}: Contiene la biblioteca común a las dos versiones.
  \item \textmark{soa}: Contiene la biblioteca específica de la solución basada en
        \emph{estructuras de arrays}.
  \item \textmark{aos}: Contiene la biblioteca específica de la solución basada en
        \emph{arrays de estructuras}.
  \item \textmark{test}: Contiene algunas pruebas generales.
  \item \textmark{utest}: Contiene las pruebas unitarias.
  \item \textmark{in}: Directorio con imagen de prueba que se usará para la evaluación.
\end{itemize}

Al compilar el programa en modo \emph{release} en una carpeta llamada \cppkey{release},
se generan los siguientes ejecutables:

\begin{itemize}
  \item \cppkey{release/img-aos}: Ejecutable para la versión \emph{arrays de estructuras}.
  \item \cppkey{release/img-soa}: Ejecutable para la versión \emph{estructuras arrays}.
  \item \cppkey{release/utest/utest}: Ejecutable para pruebas unitarias. Tenga en cuenta que este
        ejecutable asume que se ejecuta el programa desde el directorio \cppkey{release/utest},
        por lo que debe ejecutar desde el mismo.
\end{itemize}

