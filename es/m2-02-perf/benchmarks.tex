\subsection{Benchmarks}

\begin{frame}[t]{Carga de trabajo}
\begin{itemize}
  \item El \textmark{rendimiento} de un computador depende de 
        la \textmark{carga de trabajo} con la que se evalúa.
  \mode<presentation>{\pause\vfill}
  \item Computadores adaptados a cargas específicas:
    \begin{itemize}
      \item Servidores web.
      \item Servidores de bases de datos.
      \item Servidores de ficheros.
      \item Computadores personales.
      \item Multiprocesadores.
      \item Multicomputadores.
      \item \ldots
    \end{itemize}
\end{itemize}
\end{frame}

\begin{frame}[t]{Benchmarks}
\begin{itemize}
  \item Aplicación o conjunto de aplicaciones usadas para evaluar el rendimiento.
  \mode<presentation>{\pause\vfill}
  \item \textgood{Aproximaciones}:
    \begin{itemize}
      \item \textmark{Kernels}: Partes pequeñas de aplicaciones reales.
        \begin{itemize}
          \item \emph{Ejemplo}: FFT.
        \end{itemize}
      \item \textmark{Programas de juguete}: Programas cortos.
        \begin{itemize}
          \item \emph{Ejemplo}: Quicksort.
        \end{itemize}
      \item \textmark{Benchmarks sintéticos}: Inventados para representar aplicaciones reales.
        \begin{itemize}
          \item \emph{Ejemplo}: Dhrystone.
        \end{itemize}
    \end{itemize}
  \mode<presentation>{\pause\vfill}
  \item Todas malas aproximaciones:
    \begin{itemize}
      \item \alert{¡El arquitecto y el compilador pueden engañar!}
    \end{itemize}
\end{itemize}
\end{frame}

\begin{frame}[t]{Benchmarks}
\begin{itemize}
  \item \textgood{Empotrados}:
    \begin{itemize}
      \item Dhrystone (relevancia discutible).
      \item EEMBC (kernels).
    \end{itemize}
  \mode<presentation>{\pause\vfill}
  \item \textgood{Desktop}:
    \begin{itemize}
      \item SPEC2017 (mezcla de programas enteros y coma flotante).
    \end{itemize}
  \mode<presentation>{\pause\vfill}
  \item \textgood{Servidores}:
    \begin{itemize}
      \item SPECWeb, SPECSFS, SPECjbb, SPECvirt\_Sc2010.
      \item TPC
    \end{itemize}
\end{itemize}
\end{frame}

\begin{frame}[t]{Ejemplo: SPEC2017}
\begin{itemize}
  \item \textmark{SPECrate/SPECspeed 2017 Integer}: Aritmética entera.
    \begin{itemize}
      \item 20 programas: C (10), C++ (8), Fortran (2).
      \item Diversas áreas de aplicación:
        \begin{itemize}
          \item Lenguajes y compiladores, 
                planficiación de rutas,
                simulación de eventos,
                compresión de video,
                inteligencia artificial, \ldots
        \end{itemize}
    \end{itemize}
  \item \textmark{SPECrate 2017, SPECspeed 2017 Floating point}: Coma flotante
    \begin{itemize}
      \item 23 programas.
           Fortran (6),
           C (6),
           Fortran/C (5)
           C++ (2),
           C++/C (2),
           C++/C/Fortran (2).
      \item Diversas áreas de aplicación:
        \begin{itemize}
          \item Física, Química, Biología, Álgebra, \emph{Rendering} de imágenes,
                \ldots
        \end{itemize}
    \end{itemize}
\end{itemize}
\end{frame}
