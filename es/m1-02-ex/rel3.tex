\begin{acexercise}\end{acexercise}

Durante un año un computador ha estado funcionando todo el tiempo excepto
durante un único fallo que requirión el apagado del sistema durante un día.

Responda a las siguientes preguntas:

\begin{enumerate}

  \item Calcule la disponibilidad del sistema durante ese año.

  \item ¿Es posible que otro computador que ha tenido 4 fallos durante un año
        tenga el mismo valor de disponibilidad? Razona tu respuesta.

\end{enumerate}

\begin{acsolution}\end{acsolution}

\begin{enumerate}

\item

Durante un año, el sistema estuvo funcionando correctamente 364 días y estuvo
fuera de servicio durante un día.

\[
A = \frac{MTTF}{MTTF + MTTR} = \frac{364}{365} = 0.9973 \quad
\rightarrow \quad 99.73\%
\]

\item

Si. Si el tiempo de reparación de los 4 fallos suma 1 día, la disponibilidad sería la misma.

For ejemplo, si el sistema no está disponible durante un día completo
es equivalente a no estar disponible 6 horas en cuatro días diferentes.

\end{enumerate}
