\begin{acexercise}\end{acexercise}

Sea un computador funcionando continuamente y sin fallos hasta $t=20$ meses.

\begin{enumerate}

\item ¿Cuál es la fiabilidad de ese computador para $t=0$, $t24$ and $t=30$ (donde $t$ se expresa en meses)?

\item Si durante dos años de uso, el computador tiene dos fallos cuyos tiempos de reparación son 
      respectivamente $3.85$ y $4.15$ días ¿cuál sería su disponibilidad?

\end{enumerate}

\begin{acsolution}\end{acsolution}

La fiabilidad ($R$) es la probabilidad de que la vida del sistema ($X$) 
sea mayor que un valor dado del tiempo $t$. Esto es $P[X>t]$.

La fiabilidad es una función del tiempo y cumple las siguientes propiedades:

\[ R(t=0)=1 \]
\[ R(t=\infty)=0 \]
\[ R(0<t<\infty) \in [0,1] \]

En el ejemplo:

\[R(t=0)=1\] 
\[R(t=24)=0\] 
\[R(t=30)=0\]

La disponibilidad ($A$) es la fracción del tiempo que el sistema está funcionando correctamente,
o libre de errores.
Formalmente, la disponibilidad promedio será:

\[
A=\frac{MTTF}{MTTF+MTTR}
\]

Donde $MTTF$ es el tiempo medio entre fallos y $MTTR$ es el tiempo medio de reparación.

En consecuencia, la disponibilidad del sistema del ejemplo:

\[ MTTR=3.85+4.15=8 \]
\[ MTTF=365*2-8=722 \]
\[ A=\frac{722}{730}= 0,9890 \rightarrow 98.9\% \]
