\begin{acexercise}\end{acexercise}

Se dispone de un centro de datos con muchos servidores. El centro de datos opera
la mayor parte del tiempo al 50\% de su capacidad. No ostante, el 10\% del tiempo
opera a plena capacidad. Cuando los servidores operan al 50\% de su capacidad la
potencia consumida es del 90\%. Un servidor se puede poner en un modo de suspensión
en el que solamente consume el 20\% de la potencia.

Responda a las siguientes cuestiones:

\begin{enumerate}

\item ¿Cuál será el ahorro de potencia obtenido al suspender el 50\% de los
servidores comparado con apagarlos?

\item ¿Cuánto ahorro de potencia se puede obtener si se reduce el voltaje un
10\% y la frecuencia un 25\%?

\item ¿Cuál será el ahorro de potenica si se suspenden el 25\% de los servidores
y se apaga otro 25\%?

\end{enumerate}


\begin{acsolution}\end{acsolution}

\begin{enumerate}

\item Cuando el 50\% de los servidores se apagan el ahorro obtenido es del 50\%.

Por otra parte, si se supende el 50\% de los servidores se suspenden, la energía 
necesaria será:

\[P = 0.5 + 0.5 \cdot 0.2 = 0.75 \Rightarrow 75\%\]

\item Consideremos la razon entre potencias:

\[
\frac{P_{new}}{P_{old}} =
\frac{(0.9 \cdot V)^2 0.75 \cdot f}{V^2 \cdot f} =
0.9^2 \cdot 0.75 =
0.6075
\]

\item El 25\% de los servidores se apagan mientras que el otro 25\%
necesita uno 20\% de su potencia original.

\[
P = 0.5 + 0.25 \cdot 0.2 = 0.55
\]

\end{enumerate}

