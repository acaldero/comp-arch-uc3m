\section{Bases del protocolo de directorio}

\begin{frame}[t]{Directorio}
\begin{itemize}
  \item Operaciones básicas.
    \begin{itemize}
      \item Tratamiento de fallo de lectura.
      \item Tratamiento de escritura en un bloque compartido limpio.
    \end{itemize}

  \mode<presentation>{\vfill\pause}
  \item El directorio debe mantener el \textgood{estado de cada bloque}:
    \begin{itemize}
      \item \textmark{Compartido}: 
            Uno o más nodos tienen el bloque en caché y el valor en memoria está actualizado.
      \item \textmark{No cacheado}: 
            Ningún nodo tiene una copia del bloque.
      \item \textmark{Modificado}: 
            Solamente un nodo tiene copia del bloque en caché y lo ha escrito.
        \begin{itemize}
          \item Valor en memoria no actualizado.
        \end{itemize}
    \end{itemize}

  \mode<presentation>{\vfill\pause}
  \item \textgood{Además}:
    \begin{itemize}
      \item Mapa de bits con información de nodos que tienen copias del bloque.
    \end{itemize}
\end{itemize}
\end{frame}

\begin{frame}[t]{Mensajes}

{
\scriptsize
\begin{tabular}{|p{6.2em}|p{4em}|l|c|c|}

\hline
Mensaje & Fuente & Destino & Contenido & Función \\
\hline
\hline

Fallo & Caché & Directorio & P,A & P tiene fallo de lectura en A.\\
lectura & local & local & & Pedir dato y P compartidor.\pause\\
\hline

Fallo & Caché & Directorio & P,A & P tiene fallo de escritura en A.\\
escritura & local & local & & Pedir dato y P propietario.\pause\\
\hline

Invalidación & Caché & Directorio & A & Invalidar A en todas las cachés.\\
& local & local & & \pause \\
\hline

Invalidación & Caché & Directorio & A & Invalidar copia compartida.\\
& local & remota & & \pause \\
\hline

Captación & Directorio & Caché & A & Capta bloque.\\
& local & remota & & Estado a compartido.\pause\\
\hline

Captación/ & Directorio & Caché & A & Capta bloque.\\
Innvalidación & local & remota & & Invalida bloque.\pause\\
\hline

Respuesta & Directorio & Caché & D & Devolver valor a directorio.\\
valor dato & local & local & & \pause\\
\hline

Post-escritura & Caché & Directorio & A,D & Post-escritura de dato.\\
dato & remota & local & & \\
\hline
\end{tabular}
\mode<presentation>{\vfill}
\onslide<1->{
\textgood{P} $\rightarrow$ \textmark{Nodo}, 
\textgood{A} $\rightarrow$ \textmark{Dirección}, 
\textgood{D}$\rightarrow$ \textmark{Dato}}
}
\end{frame}
